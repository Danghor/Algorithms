
\section{Timsort}
Der Algorithmus ``\emph{Sortieren durch Mischen}'' ist in der Praxis der Algorithmus, der
am effizientesten arbeitet.  Dies schlie\3e ich daraus, dass dieser Algorithmus
beispielsweise sowohl in der Sprache \textsl{Python} als auch in \textsl{Java} in den
Bibliotheken zum Sortieren eingesetzt wird.
Bei einer praktischen Implementierung von Merge-Sort gibt es eine Reihe von Tricks, die
verwendet werden k\"onnen, um die Effizienz zu steigern.  Tim Peters hat eine Reihe solcher
Tricks zusammengestellt:
\\[0.2cm]
\hspace*{1.3cm}
\texttt{http://mail.python.org/pipermail/python-dev/2002-July/026837.html}
\\[0.2cm]
Der so verbesserte  Algorithmus ``\emph{Sortieren durch Mischen}'' wird als ``\emph{Timsort}''
bezeichnet.  In der neuesten Version der Sprache Java, die voraussichtlich im Sommer des Jahres 2011
erscheinen wird, ist die Methode \texttt{sort()} in der Klasse \texttt{java.util.Arrays}
mit Hilfe von \emph{Timsort} implementiert:
\\[0.2cm]
\hspace*{1.3cm}
\texttt{http://hg.openjdk.java.net/}
\\
\hspace*{2.0cm}
\texttt{jdk7/jdk7/jdk/file/jdk7-b76/src/share/classes/java/util/TimSort.java}
\\[0.2cm]
Ausgangspunkt der Entwicklung von \emph{Timsort} war die Tatsache, dass die zu
sortierenden Daten in der Praxis h\"aufig die folgenden Eigenschaften haben:
\begin{enumerate}
\item Oft sind Teilfelder bereits vorsortiert, allerdings nicht immer aufsteigend sondern genau so
      h\"aufig auch absteigend.
\item Die Daten innerhalb eines Feldes sind oft \emph{klumpig}: Damit ist gemeint, dass das Feld
      in Teilfelder aufgeteilt werden kann, in denen die Daten entweder alle relativ gro\3 oder klein
      sind.
\end{enumerate}
Aus diesem Grunde verwendet \emph{Timsort} die folgenden Tricks um ein Feld zu sortieren.
\begin{enumerate}
\item Erkennen vorsortierter Felder.

      In einem ersten Schritt unterteilen wir das zu sortierende Feld in Teilfelder, 
      die wahlweise aufsteigend oder absteigend sortiert sind.  Anschlie\3end wird ein absteigend
      sortiertes Teilfeld umgedreht, so dass es danach aufsteigend sortiert ist.
\item Verl\"angern zu kleiner Felder.

      ``\emph{Sortieren durch Mischen}'' hat nur dann eine Komplexit\"at von $\Oh(n \cdot \ln(n))$, wenn
      wir sicherstellen k\"onnen, dass die zu mischenden Teilfelder ann\"ahernd die gleiche Gr\"o\3e haben.
      Daher wird ein vorsortiertes Teilfeld, dessen L\"ange k\"urzer als eine gewisse Mindestl\"ange 
      ist, k\"unstlich auf eine vorgegebene Mindestl\"ange verl\"angert.  Als Mindestl\"ange wird ein Zahl zwischen
      32 und 63 gew\"ahlt. 

      Zum Verl\"angern der Teilfelder auf die Mindestl\"ange wird das Verfahren 
      ``\emph{Sortieren durch Einf\"ugen}'' benutzt, denn dieses Verfahren hat f\"ur Felder, die bereits
      teilweise vorsortiert sind, nur eine lineare Komplexit\"at.  Das Verfahren wird noch dadurch
      verbessert, dass beim Einf\"ugen eine bin\"are Suche verwendet wird.  Diese verbesserte Variante
      von ``\emph{Sortieren durch Einf\"ugen}'' bezeichnen wir als 
      ``\emph{bin\"ares Sortieren durch Einf\"ugen}''.
\item Verwaltung eines Stacks mit den zu sortierenden Teilfeldern.

      Wie bereits oben erw\"ahnt wurde, kann die Komplexit\"at von $\Oh(n \cdot \ln(n))$ nur dann
      sichergestellt werden, wenn die zu mischenden Teilfelder im wesentlichen dieselbe L\"ange
      haben.  Dies wird dadurch erreicht, dass die vorsortierten Teilfelder auf einem Stack
      verwaltet werden.  Dabei wird darauf geachtet, dass die zu mischenden Teilfelder im
      wesentlichen dieselbe Gr\"o\3e haben.
\item Verbesserungen des Algorithmus zum Mischen.

      Werden zwei Teilfelder gemischt bei denen alle Elemente des ersten Teilfeldes kleiner als alle
      Elemente des zweiten Teilfeldes sind, so w\"urde der konventionelle Algorithmus zum Mischen
      alle Elemente des ersten Teilfeldes mit dem ersten Element des zweiten Teilfeldes vergleichen
      und h\"atte daher eine lineare Komplexit\"at.  \emph{Timsort} erkennt, wenn zwei Teilfelder stark
      unterschiedlich sind und verwendet in diesem Fall \emph{exponentielle Suche}.  Dadurch hat 
      das Mischen zweier Teilfelder in vielen in der Praxis wichtigen Spezialf\"allen nur eine
      Komplexit\"at, die logarithmisch von der Gr\"o\3e der Teilfelder abh\"angt.
\end{enumerate}
Wir diskutieren nun eine vereinfachte Version des Algorithmus \emph{Timsort}.  Ausgangspunkt bei
dieser vereinfachten Version war die Implementierung von \emph{Timsort} in \textsl{Java 7}.
Die Orginal-Version in der JDK ist etwa doppelt so lang, so dass eine Diskussion der Orginal-Version
f\"ur die Vorlesung zu aufwendig w\"are.  Abbildung \ref{fig:TimSort.java} zeigt die Struktur der Klasse
\texttt{SimplifiedTimSort.java}:

\begin{figure}[!ht]
\centering
\begin{Verbatim}[ frame         = lines, 
                  framesep      = 0.3cm, 
                  firstnumber   = 1,
                  labelposition = bottomline,
                  numbers       = left,
                  numbersep     = -0.2cm,
                  xleftmargin   = 0.3cm,
                  xrightmargin  = 0.3cm,
                  commandchars  = \\\{\},
                ]
    public class SimplifiedTimSort \{
        private static final int MIN_MERGE  = 32;  
        private static final int MIN_GALLOP = 7;
    
        private Double[] mArray;      // the array to be sorted
        private Double[] mAux;        // an auxilliary array 
    
        private int   mStackSize = 0;  // number of pending runs on stack
        private int[] mRunBase;
        private int[] mRunLen;
    
        public SimplifiedTimSort(Double[] array) \{
            mArray   = array;
            mAux     = new Double[array.length];  
            mRunBase = new int[40];
            mRunLen  = new int[40];
        \}
    
        public void sort() \{ \(\cdots\) \}
    
        private void binarySort(int low, int high, int start) \{ \(\cdots\) \}
        private int  countRunAndMakeAscending(int low) \{ \(\cdots\) \}
        private void reverseRange(int low, int high) \{ \(\cdots\) \}
        private void pushRun(int runBase, int runLen) \{ \(\cdots\) \}
        private void mergeCollapse() \{ \(\cdots\) \}
        private void mergeForceCollapse() \{ \(\cdots\) \}
        private void mergeAt(int i) \{ \(\cdots\) \}
        private int  gallop(Double x, int base, int len) \{ \(\cdots\) \}
        private void merge(int base1, int len1, int base2, int len2) \{ \(\cdots\) \}
    \}
\end{Verbatim}
\vspace*{-0.3cm}
\caption{Struktur der Klasse \texttt{SimplifiedTimSort}.}
\label{fig:TimSort.java}
\end{figure}

\begin{enumerate}
\item Die Konstante \texttt{MIN\_MERGE} gibt die L\"ange an, die Teilfelder mindestens haben m\"ussen,
      bevor Sie mit anderen Teilfeldern gesmischt werden. In der tats\"achlichen Implementierung
      wird hier eine Zahl zwischen 32 und 63 gew\"ahlt, die aber noch von der L\"ange des zu
      sortierenden Feldes abh\"angt.  In der optimalen Implementierung ist das Ziel, diese Zahl so zu
      w\"ahlen, dass m\"oglichst alle zu mischenden Teilfelder die gleiche L\"ange haben.
\item Die Konstante \texttt{MIN\_GALLOP} legt fest, wann beim Mischen zweier Teilfelder eine
      \emph{exponentielle Suche} verwendet wird.  Diesen Begriff werden wir sp\"ater noch n\"aher erl\"autern.
\item \texttt{mArray} bezeichnet das zu sortierende Feld.
\item \texttt{mAux}   ist das Hilfsfeld, was wir zum Mischen ben\"otigen.
\item Die Klasse verwaltet intern einen Stack, auf dem zu mischende Teilfelder abgelegt werden.
      Dieser Stack wird durch drei Variablen implementiert:
      \begin{enumerate}
      \item \texttt{mStackSize} gibt die Anzahl der Teilfelder an, die auf dem Stack liegen und
            auf eine Sortierung warten.
      \item $\texttt{mRunBase}[i]$ ist der Index des ersten Elements des $i$-ten Teilfeldes.
      \item $\texttt{mRunLen}[i]$  gibt die Anzahl der Elemente des $i$-ten Teilfeldes an.
      \end{enumerate}
\item Der Konstruktor initialisiert die Member-Variablen der Klasse.  Wir werden sp\"ater sehen,
      dass der Stack, der die zu sortierenden Teilfelder enth\"alt, nie mehr als 40 Elemente enthalten
      kann, falls das zu sortierende Feld mit einem \textsl{Java} \texttt{int} indiziert werden kann.
\end{enumerate}
Wir diskutieren nun die verschiedenen Methoden der Klasse \texttt{SimplifiedTimSort}.
Wir beginnen mit der in Abbildung \ref{fig:TimSort.java:sort} gezeigten Methode $\textsl{sort}()$,
deren Aufgabe es ist, das Feld \texttt{mArray} zu sortieren.

\begin{figure}[!ht]
\centering
\begin{Verbatim}[ frame         = lines, 
                  framesep      = 0.3cm, 
                  firstnumber   = 1,
                  labelposition = bottomline,
                  numbers       = left,
                  numbersep     = -0.2cm,
                  xleftmargin   = 0.3cm,
                  xrightmargin  = 0.3cm,
                ]
    public void sort() {
        int low = 0;
        int nRemaining = mArray.length;
        if (nRemaining < 2) {
            return;  // Arrays of size 0 and 1 are always sorted
        }
        if (nRemaining < MIN_MERGE) {
            int initRunLen = countRunAndMakeAscending(low);
            binarySort(low, mArray.length, low + initRunLen);
            return;
        }
        do {
            int runLen = countRunAndMakeAscending(low);
            if (runLen < MIN_MERGE) {
                int force = nRemaining <= MIN_MERGE ? nRemaining : MIN_MERGE;
                binarySort(low, low + force, low + runLen);
                runLen = force;
            }
            pushRun(low, runLen);
            mergeCollapse();  // establish stack invariants
            low += runLen;    // Advance to find next run
            nRemaining -= runLen;
        } while (nRemaining != 0);
        mergeForceCollapse();
    }
\end{Verbatim}
\vspace*{-0.3cm}
\caption{Die Methode $\textsl{sort}()$}
\label{fig:TimSort.java:sort}
\end{figure}

\begin{enumerate}
\item Die Variable \texttt{low} ist der Index des ersten noch unsortierten Elements in dem Feld 
      \texttt{mArray}.  Diese Variable wird daher zun\"achst mit $0$ initialisiert.  
\item \texttt{nRemaining} ist die Anzahl der noch zu sortierenden Elemente.
\item Felder mit weniger als zwei Elementen sind bereits sortiert.
\item Kleine Felder, konkret solche Felder die weniger als \texttt{MIN\_MERGE} Elemente haben,
      werden mit Hilfe einer Variante des Algorithmus ``\emph{Sortieren durch Einf\"ugen}''
      sortiert.  Dazu sucht die Methode $\textsl{countRunAndMakeAscending}()$ zun\"achst das gr\"o\3te
      Teilfeld, das beginnend an dem Index \texttt{low} entweder aufsteigend oder absteigend
      sortiert ist.  Falls das Teilfeld absteigend sortiert ist, werden die Elemente innerhalb
      des Teilfeldes umgedreht, so dass das Teilfeld anschlie\3end auf jeden Fall aufsteigend
      sortiert ist.  Die Methode $\textsl{countRunAndMakeAscending}()$ gibt als Ergebnis die L\"ange
      des aufsteigend sortierten Teilfelds zur\"uck.  Wenn das Programm in Zeile 9 angekommen ist,
      dann wissen wir, dass das Teilfeld
      \\[0.2cm]
      \hspace*{1.3cm}
      \texttt{[ mArray[$\texttt{low} + i$] | $i \in [0, \cdots, \mathtt{initRunLen}-1]$ ]}
      \\[0.2cm]
      sortiert ist.  Die Elemente beginnend mit dem Index $\mathtt{low}+ \mathtt{initRunLen}$ m\"ussen
      noch in dieses Feld einsortiert werden.  
      Dies wird von der Methode $\textsl{binarySort}()$ in Zeile 9 geleistet.
\item Gro\3e Felder werden zun\"achst in Teilfelder, die bereits sortiert sind, aufgespalten.
      Dazu wird in Zeile 13 zun\"achst das l\"angste sortierte Teilfeld berechnet, dass an dem Index 
      \texttt{low} beginnt und das bereits sortiert ist.  Dann wird in Zeile 14 gepr\"uft, 
      ob dieses Teilfeld die Mindest-L\"ange \texttt{MIN\_MERGE} besitzt.  Falls nicht und wenn
      au\3erdem noch mehr als \texttt{MIN\_MERGE} Elemente vorhanden sind, dann wird dieses Teilfeld
      durch den Aufruf der Methode $\textsl{binarySort}()$ in Zeile 16 zu einem sortierten
      Teilfeld der L\"ange \texttt{force} verl\"angert.  Diese L\"ange ist \texttt{MIN\_MERGE},
      falls mehr als \texttt{MIN\_MERGE} Elemente \"ubrig sind, sonst ist diese L\"ange
      einfach die Anzahl aller noch unsortierten Elemente.
      Zum Sortieren wird wieder der Algorithmus ``\emph{Sortieren durch Einf\"ugen}'' verwendet.
\item Das sortierte Teilfeld  wird in Zeile 19 von der Methode $\textsl{pushRun}()$ auf den Stack der bereits
      sortierten Teilfelder gelegt.  Liegen schon mehrere Teilfelder auf dem Stack und sind die
      Gr\"o\3en dieser Teilfelder nicht zu stark unterschiedlich, so mischt die Methode
      $\textsl{mergeCollapse}()$ einige der auf dem Stack liegenden Teilfelder.  Dies
      werden wir sp\"ater noch im Detail analysieren, wenn wir die Methode $\textsl{mergeCollapse}()$
      besprechen.
\item Anschlie\3end wird in Zeile 21 der Index \texttt{low} um die bereits sortierten Elemente erh\"oht,
      und die Zahl der noch zu sortierenden Elemente wird entsprechend erniedrigt.
\item Zum Abschluss der Methode werden alle noch auf dem Stack verbliebenen Teilfelder so
      gemischt, dass das resultierende Feld insgesamt aufsteigend geordnet ist.
\end{enumerate}

\begin{figure}[!ht]
\centering
\begin{Verbatim}[ frame         = lines, 
                  framesep      = 0.3cm, 
                  firstnumber   = 1,
                  labelposition = bottomline,
                  numbers       = left,
                  numbersep     = -0.2cm,
                  xleftmargin   = 0.8cm,
                  xrightmargin  = 0.8cm,
                ]
    private void binarySort(int low, int high, int start) 
    {
        assert low < start && start <= high;
        for (int i = start; i < high; ++i) {
            Double next  = mArray[i];
            int    left  = low;
            int    right = i;
            assert left <= right;
            while (left < right) {
                int middle =  left + (right - left) / 2;
                if (next < mArray[middle]) {
                    right = middle;
                } else {
                    left = middle + 1;
                }
            }
            assert left == right;
            System.arraycopy(mArray, left, mArray, left + 1, i - left);
            mArray[left] = next;
        }
        assert isSorted(low, high): "binarySort: not sorted";
    }
\end{Verbatim}
\vspace*{-0.3cm}
\caption{Die Methode $\textsl{binarySort}()$}
\label{fig:TimSort.java:binarySort}
\end{figure}

Abbildung \ref{fig:TimSort.java:binarySort} zeigt die Methode $\textsl{binarySort}()$. Ein
Aufruf der Form
\\[0.2cm]
\hspace*{1.3cm}
$\textsl{binarySort}(\textsl{low}, \textsl{high}, \textsl{start})$
\\[0.2cm]
hat die Aufgabe, das Teilfeld
\\[0.2cm]
\hspace*{1.3cm}
$\mathtt{mArray}[\textsl{low}, \cdots, \textsl{high} - 1]$
\\[0.2cm]
zu sortieren.  Dabei darf vorausgesetzt werden, dass das Teilfeld
\\[0.2cm]
\hspace*{1.3cm}
$\mathtt{mArray}[\textsl{low}, \cdots, \textsl{start} - 1]$
\\[0.2cm]
bereits sortiert ist.  Es m\"ussen also lediglich die Elemente
$\mathtt{mArray}[\textsl{start}]$, $\cdots$, $\mathtt{mArray}[\textsl{high}-1]$, in das
bereits sortierte Teilfeld eingef\"ugt werden.  Dazu l\"auft die \texttt{for}-Schleife in Zeile 4 \"uber alle
Indizes $i$ aus dem Intervall $[\textsl{start}, \textsl{high}-1]$ und f\"ugt die Elemente
$\mathtt{mArray}[i]$ so in das schon sortierte Teilfeld ein, dass die Sortierung erhalten bleibt.
Die Invariante der \texttt{for}-Schleife ist also, dass das Teilfeld
\\[0.2cm]
\hspace*{1.3cm}
$\mathtt{mArray}[\textsl{low}, \cdots, i - 1]$
\\[0.2cm]
bereits sortiert ist und die Aufgabe des n\"achsten Schleifendurchlaufs ist es, f\"ur das Element
$\texttt{mArray}[i]$ eine Position $k \in \{\textsl{low}, \cdots, i \}$ zu suchen, an der es eingef\"ugt werden kann. 
F\"ur diesen Index $k$ soll gelten:
\begin{enumerate}
\item $\forall j \in \{\textsl{low}, \cdots, k-1\}: \mathtt{mArray}[j] \leq \mathtt{mArray}[i]$
      \quad und 
\item $\forall j \in \{k, \cdots, i-1\}: \mathtt{mArray}[j] > \mathtt{mArray}[i]$.
\end{enumerate}
Um den Index $k$ zu bestimmen, verwendet die Methode $\textsl{binarySort}()$ 
das Verfahren der Intervall-Halbierung.  Dazu wird die linke
Grenze \texttt{left} des Intervalls mit \texttt{low} initialisiert, die rechte Grenze
\texttt{right} wird mit $i$ initialisiert, denn das sind die beiden extremen Positionen,
die der Index $k$ annehmen kann:
\begin{enumerate}
\item Falls alle Elemente der Menge 
      $\bigl\{ \mathtt{mArray}[j] \mid j \in \{ \textsl{low}, \cdots, i-1 \}\bigr\}$
      gr\"o\3er als $\mathtt{mArray}[i]$ sind, so wird das Element an der Position \textsl{low}
      eingef\"ugt und die Elemente des Feldes \texttt{mArray} werden nach rechts verschoben.
\item Falls alle Elemente der Menge 
      $\bigl\{ \mathtt{mArray}[j] \mid j \in \{ \textsl{low}, \cdots, i-1 \}\bigr\}$
      kleiner-gleich $\mathtt{mArray}[i]$ sind, so wird das Element an der Position $i$
      eingef\"ugt und bleibt folglich da, wo es schon ist.
\end{enumerate}
Die \texttt{while}-Schleife in Zeile 9 hat die folgenden beiden Invarianten:
\begin{enumerate}
\item $\forall j \in \{ \textsl{low}, \textsl{left} - 1 \} : \mathtt{mArray}[j] \leq
  \mathtt{mArray}[i]$ \quad und
\item $\forall j \in \{ \textsl{right}, i - 1 \} : \mathtt{mArray}[i] < \mathtt{mArray}[j]$.
\end{enumerate}
Zu Beginn sind diese beiden Invarianten sicher erf\"ullt, denn da $\textsl{left} = \textsl{low}$ ist,
ist die Menge $\{ \textsl{low}, \textsl{left} - 1 \}$ leer und aus $\textsl{right} = i$ folgt
$\{ \textsl{right}, i - 1 \} = \{\}$, so dass beide Aussagen trivial sind.  Wir m\"ussen nun zeigen,
dass diese Invarianten bei jedem Schleifendurchlauf erhalten bleiben.
\begin{enumerate}
\item In Zeile 10 berechnen wir die Mitte \textsl{middle} des Intervalls $[\textsl{left}, \textsl{right}]$,
      wobei wir den Fall, dass $\textsl{right} = \textsl{left} + 1$ ist, sp\"ater noch genauer
      analysieren m\"ussen.

      Der Ausdruck zur Berechnung der Mitte des Intervalls ist komplizierter, als Sie es auf den
      ersten Blick erwarten w\"urden.  Das Problem ist, dass es bei dem einfacheren Ausdruck
      \\[0.2cm]
      \hspace*{1.3cm}
      \texttt{(left + middle) / 2}
      \\[0.2cm]
      zu einem Überlauf kommen kann.
\item Falls $\mathtt{mArray}[i] < \mathtt{mArray}[\textsl{middle}]$ ist, so sind alle Elemente
      rechts von dem Index \textsl{middle} sicher gr\"o\3er als das einzusortierende Element 
      $\textsl{mArray}[i]$ und damit gilt f\"ur den Index $k$, den wir suchen, die Ungleichung
      \\[0.2cm]
      \hspace*{1.3cm}
      $k \leq \textsl{middle}$.
      \\[0.2cm]
      Daher k\"onnen wir in diesem Fall die rechte Seite \textsl{right} des Intervalls zu \textsl{middle} verkleinern.
\item Falls $\mathtt{mArray}[\textsl{middle}] \leq \mathtt{mArray}[i]$ ist, so sind alle Elemente
      links von dem Index \textsl{middle} sicher kleiner-gleich dem einzusortierenden Element 
      $\textsl{mArray}[i]$. Da auch $\mathtt{mArray}[\textsl{middle}] \leq \mathtt{mArray}[i]$ gilt 
      \\[0.2cm]
      \hspace*{1.3cm}
      $k > \textsl{middle}$.
      \\[0.2cm]
      Daher k\"onnen wir in diesem Fall die linke Seite \textsl{left} des Intervalls zu $\textsl{middle}+1$
      vergr\"o\3ern, wobei die Invariante erhalten bleibt.
\end{enumerate}
Falls nun die \texttt{while}-Schleife abbricht, muss danach $\textsl{left} = \textsl{right}$ gelten
und damit ist \textsl{left} (oder genausogut \textsl{right}) der gesuchte Index $k$.  Wir verschieben
dann die Elemente des Teilfeldes 
\\[0.2cm]
\hspace*{1.3cm}
$[\mathtt{mArray}[\textsl{left}, \cdots, i-1]$ 
\\[0.2cm]
um eine Position
nach rechts.  Dieses Teilfeld hat $(i-1) - \textsl{left} + 1 = i - \textsl{left}$ Elemente.
Anschlie\3end kopieren wir das Element $\texttt{mArray}[i]$ an die nun freie Position $k = \textsl{left}$.

Es bleibt noch zu zeigen, dass die \texttt{while}-Schleife in Zeile 9 tats\"achlich abbricht.
Das Problem ist, dass das Intervall $[\textsl{left}, \textsl{right}]$ nur solange tats\"achlich
kleiner wird, solange \textsl{left} und \textsl{right} sich um mehr als 1 unterscheiden, denn nur
dann ist \textsl{middle} zwischen \textsl{left} und \textsl{right}.  Falls nun
\\[0.2cm]
\hspace*{1.3cm}
$\textsl{right} = \textsl{left} + 1$
\\[0.2cm]
ist, liefert die Berechnung von \textsl{middle} auf Grund der Ganzzahl-Division den Wert \textsl{left}:
\\[0.2cm]
\hspace*{1.3cm}
$\textsl{middle} = (\textsl{left} + \textsl{left} + 1) / 2 = (2 \cdot\textsl{left} + 1) / 2 = \textsl{left}$.
\\[0.2cm]
Abh\"angig von dem Test in Zeile 11 gibt es nun zwei F\"alle:
\begin{enumerate}
\item $\mathtt{mArray}[i] < \mathtt{mArray}[\textsl{left}]$
  
      In diesem Fall wird $\textsl{right} := \textsl{middle} = \textsl{left}$ gesetzt,
      so dass neue Intervall jetzt die Form $[\textsl{left}, \textsl{left}]$ hat, so dass die
      Schleife abbricht, weil die linke Grenze mit der rechten Grenze \"ubereinstimmt.
\item $\mathtt{mArray}[i] \geq \mathtt{mArray}[\textsl{left}$

      Jetzt haben wir
      \\[0.2cm]
      \hspace*{1.3cm}
      $\textsl{left} := \textsl{left} + 1 = \textsl{right}$
      \\[0.2cm]
      gesetzt, so dass das neue Intervall  $[\textsl{left}, \textsl{right}]$ ebenfalls die L\"ange 0
      hat, so dass die Schleife auch in diesem Fall abbricht.
\end{enumerate}


\begin{figure}[!ht]
\centering
\begin{Verbatim}[ frame         = lines, 
                  framesep      = 0.3cm, 
                  firstnumber   = 1,
                  labelposition = bottomline,
                  numbers       = left,
                  numbersep     = -0.2cm,
                  xleftmargin   = 0.0cm,
                  xrightmargin  = 0.0cm,
                ]
    private int countRunAndMakeAscending(int low)
    {
        int high    = mArray.length;
        int runHigh = low + 1;
        if (runHigh == high) {
            return 1;
        }
        if (mArray[runHigh] < mArray[low]) {                                  
            ++runHigh;
            while (runHigh < high && mArray[runHigh] < mArray[runHigh - 1]) { 
                ++runHigh;
            }
            reverseRange(low, runHigh);  // reverse it
        } else {                                                                 
            ++runHigh;
            while (runHigh < high && mArray[runHigh - 1] <= mArray[runHigh] ) {  
                ++runHigh;
            }
        }
        assert isSorted(low, runHigh): "run not sorted ";
        return runHigh - low;   // return length of actual run
    }
\end{Verbatim}
\vspace*{-0.3cm}
\caption{Die Methode $\textsl{countRunAndMakeAscending}()$}
\label{fig:TimSort.java:countRunAndMakeAscending}
\end{figure}

\noindent
In der Praxis zeigt sich, dass ein zu sortierendes Feld
oft Teilfelder enth\"alt, die bereits sortiert sind.  Es ist sinnvoll, solche Felder vorab zu
identifizieren.  Die in Abbildung \ref{fig:TimSort.java:countRunAndMakeAscending} gezeigte
 Methode $\textsl{countRunAndMakeAscending}(\textsl{low})$ hat die Aufgabe,
innerhalb des Feldes \texttt{mArray}  startend an der Position \textsl{low} das l\"angste Teilfeld zu
suchen, das bereits aufsteigend oder absteigend sortiert ist.   Falls dieses Teilfeld absteigend
sortiert ist, so wird es umgedreht.  Die Methode $\textsl{countRunAndMakeAscending}$ arbeitet
wie folgt:
\begin{enumerate}
\item Der Index \textsl{high} ist eine echte obere Schranke f\"ur den oberen Index des bereits 
      sortierten Teilfelds.  Im g\"unstigsten Fall geht dies bis zum Ende des Feldes,
      daher wird \textsl{high} mit \texttt{mArray.length} initialisiert.
\item Der Index \textsl{runHigh} zeigt auf den letzten Index des sortierten Feldes.
      Wir initialisieren \textsl{runHigh} mit $\textsl{low} + 1$, denn ein Feld der L\"ange 2
      ist immer sortiert:  Falls
      \\[0.2cm]
      \hspace*{1.3cm}
      $\mathtt{mArray}[\textsl{low}] \leq \mathtt{mArray}[\textsl{low} + 1]$
      \\[0.2cm]
      gilt, ist das Teilfeld 
      \\[0.2cm]
      \hspace*{1.3cm}
      $\bigl[\mathtt{mArray}[\textsl{low}], \mathtt{mArray}[\textsl{low} + 1]\bigr]$
      \\[0.2cm]
      aufsteigend sortiert und wenn statt dessen \\[0.2cm]
      \hspace*{1.3cm}
      $\mathtt{mArray}[\textsl{low}] > \mathtt{mArray}[\textsl{low} + 1]$
      \\[0.2cm]
      gilt, dann ist dieses Teilfeld absteigend sortiert.
\item Die beiden F\"alle, dass das Teilfeld aufsteigend oder absteigend sortiert ist, werden nun 
      getrennt betrachtet.  Falls der Test 
      \\[0.2cm]
      \hspace*{1.3cm}
      $\mathtt{mArray}[\textsl{runHigh}] < \mathtt{mArray}[\textsl{low}]$
      \\[0.2cm]
      in Zeile 8 erfolgreich ist, ist die Teilfolge, die wir suchen, absteigend sortiert.
      Die Invariante der \texttt{while}-Schleife in Zeile 10 lautet:
      \\[0.2cm]
      \hspace*{1.3cm}
      $\bigl[ \mathtt{mArray}[low], \cdots, \mathtt{mArray}[\textsl{runHigh}-1] \bigr]$
      ist absteigend sortiert.  
      \\[0.2cm]
      Daher wird die Variable \textsl{runHigh} so lange inkrementiert, solange der n\"achste Wert
      kleiner als der vorhergehende Wert ist.  Abschlie\3end dreht die Methode
      $\textsl{reverseRange}$
      die Elemente des Teilfeldes
      \\[0.2cm]
      \hspace*{1.3cm}
      $\bigl[ \mathtt{mArray}[low], \cdots, \mathtt{mArray}[\textsl{runHigh}-1] \bigr]$
      \\[0.2cm]
      so um, dass anschlie\3end dieses Teilfeld aufsteigend sortiert ist. 
\item Falls das Teilfeld aufsteigend sortiert ist, wird das Teilfeld in analoger Weise solange nach
      oben erweitert, solange die neu hinzugef\"ugten Elemente gr\"o\3er als die bereits vorhandenen sind.
\item Wenn am Schluss die Differenz $\textsl{runHigh} - \textsl{low}$ zur\"uck gegeben wird, ist dies
      genau die Zahl der Elemente des Teilfeldes.
\end{enumerate}

\noindent
Die in Abbildung \ref{fig:TimSort.java:reverseRange} gezeigte Methode
$\textsl{reverseRange}(\textsl{low}, \textsl{high})$ hat die Aufgabe, das Teilfeld
\\[0.2cm]
\hspace*{1.3cm}
$\bigl[ \mathtt{mArray}[\textsl{low}], \cdots, \mathtt{mArray}[\textsl{high}-1] \bigr]$
\\[0.2cm]
umzudrehen.  Dazu verwaltet diese Methode zwei Indizes $l$ und $h$: $l$ startet am linken Rand des
Feldes und $h$ am rechten Rand.  In den Zeilen 5 -- 7 werden die Werte, auf die $l$ und $h$ zeigen,
vertauscht.  Anschlie\3end wird der linke Index inkrementiert und der rechte wird dekrementiert.
Dies geschieht solange, bis sich die Indizes kreuzen.  In diesem Fall bricht die Schleife ab.


\begin{figure}[!ht]
\centering
\begin{Verbatim}[ frame         = lines, 
                  framesep      = 0.3cm, 
                  firstnumber   = 1,
                  labelposition = bottomline,
                  numbers       = left,
                  numbersep     = -0.2cm,
                  xleftmargin   = 0.8cm,
                  xrightmargin  = 0.8cm,
                ]
    private void reverseRange(int low, int high) {
        int l = low;
        int h = high - 1;
        while (l < h) {
            Double t  = mArray[l];
            mArray[l] = mArray[h];
            mArray[h] = t;
            ++l; --h;
        }
        assert isSorted(low, high - 1): "not sorted after reverse";
    }
\end{Verbatim}
\vspace*{-0.3cm}
\caption{Die Methode $\textsl{reverseRange}()$.}
\label{fig:TimSort.java:reverseRange}
\end{figure}

Die in Abbildung \ref{fig:TimSort.java:pushRun} gezeigte Methode $\textsl{pushRun}()$ hat die
Aufgabe, ein Teilfeld, von dem wir bereits wissen, dass es aufsteigend sortiert ist,
abzuspeichern. Hierzu reicht es aus, den Start-Index des Teilfeldes sowie die L\"ange des Feldes
abzuspeichern.   Hierzu werden die globalen Felder \texttt{mRunBase} und \texttt{mRunLen} als Stacks
verwendet.  Die globale Variable \textsl{mStackSize} gibt dabei an, wieviele Teilfelder bereits
gespeichert sind.

\begin{figure}[!ht]
\centering
\begin{Verbatim}[ frame         = lines, 
                  framesep      = 0.3cm, 
                  firstnumber   = 1,
                  labelposition = bottomline,
                  numbers       = left,
                  numbersep     = -0.2cm,
                  xleftmargin   = 0.8cm,
                  xrightmargin  = 0.8cm,
                ]
    private void pushRun(int runBase, int runLen) {
        mRunBase[mStackSize] = runBase;
        mRunLen [mStackSize] = runLen;
        mStackSize++;
    }
\end{Verbatim}
\vspace*{-0.3cm}
\caption{Die Methode $\textsl{pushRun}()$.}
\label{fig:TimSort.java:pushRun}
\end{figure}

Abbildung \ref{fig:TimSort.java:mergeCollapse} zeigt die Methode $\textsl{mergeCollapse}()$.  Diese
Methode hat die Aufgabe daf\"ur zu sorgen, dass der Stack, auf dem die bereits sortierten Teilfelder
abgespeichert sind, nicht zu gro\3 wird.  Dies wir durch zwei Invarianten sichergestellt:
\begin{enumerate}
\item Einerseits fordern wir, dass die L\"angen der Teilfelder, die auf dem Stack liegen, absteigend
      sind, es soll also gelten
      \\[0.2cm]
      \hspace*{1.3cm}
      $\forall i \in \{0, \cdots, \textsl{mStackSize} - 1\}: \mathtt{mRunLen}[i-1] > \mathtt{mRunLen}[i]$.
\item Zust\"atzlich fordern wir, dass die L\"angen der Teilfelder, wenn wir den Stack von oben nach
      unten durchgehen, mindenstens so schnell wachsen wie die Fibonacci-Zahlen:  
      \\[0.2cm]
      \hspace*{1.3cm}
      $\forall i \in \{0, \cdots, \textsl{mStackSize} - 1\}: 
        \mathtt{mRunLen}[i-2] > \mathtt{mRunLen}[i-1] + \mathtt{mRunLen}[i]$.
      \\[0.2cm]
      Die zweite Bedingung stellt sicher, dass wir mit einem Stack der Gr\"o\3e 40 auskommen, denn f\"ur die
      Summe der Fibonacci-Zahlen
      \\[0.2cm]
      \hspace*{1.3cm}
      $S_n := \sum\limits_{i=0}^n F_i$
      \\[0.2cm]
      kann gezeigt werden, dass
      \\[0.2cm]
      \hspace*{1.3cm}
      $\sum\limits_{i=0}^n F_i = F_{n+2} - 1$
      \\[0.2cm]
      gilt und  $F_{41}$ hat den Wert $267\,914\,296$.  Da jedes der Teilfelder mindestens eine
      L\"ange von 32 hat, reicht der Stack f\"ur Felder bis zur Gr\"o\3e 
      $32 \cdot 267\,914\,291 = 8\,573\,257\,312$ auf jeden Fall aus.  Ein Feld, das mit ganzen
      Zahlen indiziert wird, hat maximal ein Gr\"o\3e von $2^{31} = 2\,147\,483\,648$,
      so dass ein Stack der G\"o\3e 40 sicher ausreicht.
\end{enumerate}

\begin{figure}[!ht]
\centering
\begin{Verbatim}[ frame         = lines, 
                  framesep      = 0.3cm, 
                  firstnumber   = 1,
                  labelposition = bottomline,
                  numbers       = left,
                  numbersep     = -0.2cm,
                  xleftmargin   = 0.8cm,
                  xrightmargin  = 0.8cm,
                ]
    private void mergeCollapse() {
        while (mStackSize > 1) {
            int n = mStackSize - 2;
            if (n > 0 && mRunLen[n-1] <= mRunLen[n] + mRunLen[n+1]) {
                if (mRunLen[n - 1] < mRunLen[n + 1]) {
                    --n;
                }
                mergeAt(n);
            } else if (mRunLen[n] <= mRunLen[n + 1]) {
                mergeAt(n);
            } else {
                break; // invariant is established
            }
        }
    }
\end{Verbatim}
\vspace*{-0.3cm}
\caption{Die Methode $\textsl{mergeCollapse}()$.}
\label{fig:TimSort.java:mergeCollapse}
\end{figure}

Die Implementierung von $\textsl{mergeCollapse}()$ stellt diese Invarianten sicher.  Voraussetzung
ist, dass die Invarianten bereits f\"ur alle Teilfelder mit eventueller Ausnahme des zuletzt auf den
Stack gelegten Teilfeldes erf\"ullt sind.  
\begin{enumerate}
\item Falls die Fibonacci-Invariante
      \\[0.2cm]
      \hspace*{1.3cm}
      $\mathtt{mRunLen}[i-2] > \mathtt{mRunLen}[i-1] + \mathtt{mRunLen}[i]$
      \\[0.2cm]
      an der Spitze des Stacks verletzt ist, so werden entweder die beiden Teilfelder
      \\[0.2cm]
      \hspace*{1.3cm}
      $i-1$ und $i$
      \\[0.2cm]
      zu einem neuen Teilfeld gemischt, oder die beiden Teilfelder
      \\[0.2cm]
      \hspace*{1.3cm}
      $i$ und $i+1$.
      \\[0.2cm]
      werden gemischt. Falls das Teilfeld $i+1$ l\"anger ist als das Teilfeld $i-1$, so werden die 
      beiden k\"urzeren Teilfelder $i-1$ und $i$ gemischt, andernfalls werden $i$ und $i+1$ gemischt.
\item Falls die Fibonacci-Invariante erf\"ullt ist, aber das neu auf dem Stack liegende Feld gr\"o\3er
      ist als das Feld darunter, so werden die beiden oben auf dem Stack liegenden Felder gemischt.
\item Da durch das Mischen der Stack verk\"urzt wird, sind die Invarianten eventuell wieder an der
      Spitze des Stacks verletzt.  Daher  m\"ussen wir das Mischen solange weiterf\"uhren, bis
      die Invarianten erf\"ullt sind.
\end{enumerate}

\begin{figure}[!ht]
\centering
\begin{Verbatim}[ frame         = lines, 
                  framesep      = 0.3cm, 
                  firstnumber   = 1,
                  labelposition = bottomline,
                  numbers       = left,
                  numbersep     = -0.2cm,
                  xleftmargin   = 0.8cm,
                  xrightmargin  = 0.8cm,
                ]
    private void mergeForceCollapse() {
        while (mStackSize > 1) {
            mergeAt(mStackSize - 2);
        }
    }
\end{Verbatim}
\vspace*{-0.3cm}
\caption{Die Methode $\textsl{mergeForceCollapse}()$}
\label{fig:TimSort.java:mergeForceCollapse}
\end{figure}

Nachdem alle bereits sortierten Teilfelder auf den Stack gelegt worden sind, m\"ussen wir diese
solange mischen, bis nur noch ein einziges Teilfeld auf dem Stack liegen bleibt.  Dieses Teilfeld
ist dann das aufsteigend sortierte Feld.  Die in Abbildung \ref{fig:TimSort.java:mergeForceCollapse}
gezeigte Methode $\textsl{mergeForceCollapse}()$ leistet dies: Solange noch mindestens zwei
Teilfelder auf dem Stack liegen, werden diese gemischt.  

\begin{figure}[!ht]
\centering
\begin{Verbatim}[ frame         = lines, 
                  framesep      = 0.3cm, 
                  firstnumber   = 1,
                  labelposition = bottomline,
                  numbers       = left,
                  numbersep     = -0.2cm,
                  xleftmargin   = 0.8cm,
                  xrightmargin  = 0.8cm,
                ]
    private void mergeAt(int i) {
        int base1 = mRunBase[i];      // start of first run
        int len1  = mRunLen[i];
        int base2 = mRunBase[i + 1];  // start of second run
        int len2  = mRunLen[i + 1];
        mRunLen[i] = len1 + len2;
        if (i == mStackSize - 3) {
            mRunBase[i + 1] = mRunBase[i + 2];   // slide over last run
            mRunLen [i + 1] = mRunLen [i + 2];    
        }
        --mStackSize;
        merge(base1, len1, base2, len2);
        assert isSorted(base1, base2 + len2);
    }
\end{Verbatim}
\vspace*{-0.3cm}
\caption{The methode $\textsl{mergeAt}()$.}
\label{fig:TimSort.java:mergeAt}
\end{figure}

Abbildung \ref{fig:TimSort.java:mergeAt} zeigt die Methode $\textsl{mergeAt}(i)$, welche die Aufgabe
hat, die beiden Teilfelder, die auf dem Stack an den Positionen $i$ und $i+1$ liegen, zu mischen.
Dabei wird vorausgesetzt, dass 
\\[0.2cm]
\hspace*{1.3cm}
$ i \in \{  \mathtt{mStackSize} - 2, \mathtt{mStackSize} - 3 \}$
\\[0.2cm] 
gilt, es werden also entweder die beiden vorletzten odr die beiden letzten Teilfelder gemischt.

Das erste Teilfeld beginnt an der Position $\mathtt{mRunBase}[i]$ und besteht aus
$\mathtt{mRunLen}[i]$ Elementen,
das zweite Teilfeld beginnt entsprechend an der Position $\mathtt{mRunBase}[i+1]$ und besteht aus
$\mathtt{mRunLen}[i+1]$ Elementen.  Die beiden Teilfelder folgen unmittelbar aufeinander, es gilt also
\\[0.2cm]
\hspace*{1.3cm}
$\mathtt{mRunBase}[i] + \mathtt{mRunLen}[i] = \mathtt{mRunBase}[i+1]$.
\\[0.2cm]
Diese beiden Teilfelder werden durch den Aufruf der Methode $\textsl{merge}()$ in Zeile 12 gemischt.
Das dabei neue entstehende Teilfeld ersetzt die beiden urspr\"unglichen Teilfelder.  Falls \"uber den
beiden Teilfeldern noch ein weiteres Teilfeld liegt, wird dieses nun in den Zeilen 8 und 9 im Stack
an die Position $i+1$ geschoben.  Da die Methode nur aufgerufen wird, wenn entweder die letzten oder
die vorletzten beiden Teilfelder auf dem Stack gemischt werden, reicht dies aus um den Stack zu verwalten.

\begin{figure}[!ht]
\centering
\begin{Verbatim}[ frame         = lines, 
                  framesep      = 0.3cm, 
                  firstnumber   = 1,
                  labelposition = bottomline,
                  numbers       = left,
                  numbersep     = -0.2cm,
                  xleftmargin   = 0.8cm,
                  xrightmargin  = 0.8cm,
                ]
    private int gallop(Double x, int b, int l) {
        if (x < mAux[b]) {
            return 0;
        } 
        int lastK = 0;
        int k     = 1;
        while (k < l && mAux[b + k] <= x) {
            lastK = k;
            k     = 2 * k + 1;
            if (k < 0) {
                k = l;
            }
        }
        if (k > l) {
            k = l;
        }
        while (lastK < k) {
            int m = lastK + (k - lastK) / 2;
            if (mAux[b + m] <= x) {
                lastK = m + 1;  
            } else {
                k = m;          
            }
        }
        return k;             
    }
\end{Verbatim}
\vspace*{-0.3cm}
\caption{Die Methode $\textsl{gallop}()$.}
\label{fig:TimSort.java:gallop}
\end{figure}

Die in Abbildung \ref{fig:TimSort.java:gallop} gezeigte Methode $\textsl{gallop}()$ implementiert
das Verfahren der \emph{exponentiellen Suche} um die Position zu bestimmen, an der das erste
Argument $x$ in dem bereits sortierten Teilfeld, das an der Position $b$ beginnt,
eingeordnet werden muss.  Genauer hat der Aufruf
\\[0.2cm]
\hspace*{1.3cm}
$\textsl{gallop}(x, b, l)$
\\[0.2cm]
die Aufgabe, in dem Feld \texttt{mAux} innerhalb des Intervalls $[b, \cdots, b + (l-1)]$ 
die Position $k$ zu finden, f\"ur die folgendes gilt:
\begin{enumerate}
\item Entweder haben wir
      \\[0.2cm]
      \hspace*{1.3cm}
      $\forall i \in \{ b, \cdots, b + l - 1 \}: \mathtt{mAux}[i] \leq x$.
      \\[0.2cm]
      Dann gilt $k = \textsl{gallop}(x, b, l) = l$, denn in diesem Fall soll $x$ hinter allen
      Elementen des Teilfelds eingef\"ugt werden.
\item Andernfalls bestimmen wir $k = \textsl{gallop}(x, b, l)$ so, dass folgendes gilt:
      \begin{enumerate}
      \item $\forall i \in \{ b, \cdots, b + k - 1 \}: \mathtt{mAux}[i] \leq x$,
      \item $\forall i \in \{ b + k, \cdots, b + (l - 1) \}: x < \mathtt{mAux}[i]$.
      \end{enumerate}
\end{enumerate}
Bei der Implementierung k\"onnen wir voraussetzen, dass das Teilfeld 
\\[0.2cm]
\hspace*{1.3cm}
$\bigl[\mathtt{mAux}[b], \cdots, \mathtt{mAux}[b + (l-1)]\bigr]$
\\[0.2cm]
aufsteigend sortiert ist.
Wir diskutieren nun die Details der in Abbildung \ref{fig:TimSort.java:gallop} gezeigten Implementierung.
\begin{enumerate}
\item Falls das einzuf\"ugende Element $x$ kleiner ist als das erste Element des Teilfeldes,
      soll $x$ am Anfang des Teilfeldes einsortiert werden und wir geben in Zeile
      3 den Index 0 zur\"uck.
\item Ansonsten speichern wir in \textsl{lastK} den letzten Wert von $k$, den wir schon
      (erfolglos) ausprobiert haben und initialisieren $k$ mit $1$, den wir wissen ja schon,
      dass $k$ gr\"o\3er als $0$ sein muss, denn das erste Element des Teilfeldes ist ja kleiner-gleich
      $x$.
\item Solange $k$ noch nicht \"uber den rechten Rand des Teilfeldes herauszeigt und solange au\3erdem
      das Element an der Stelle $\mathtt{mAux}[b + k]$ kleiner-gleich $x$ ist, m\"ussen wir $k$ vergr\"o\3ern.
      Wir inkrementieren $k$ aber nicht blo\3 um 1, was zu einer linearen Suche f\"uhren w\"urde, sondern
      vergr\"o\3ern $k$ in Zeile 9 nach der Formel
      \\[0.2cm]
      \hspace*{1.3cm}
      $k := 2 * k + 1$.
      \\[0.2cm]
      Dabei merken wir uns jedesmal den alten Wert von $k$ in der Variablen \textsl{lastK}.
\item Bei sehr gro\3en Feldern kann es bei der Berechnung von $2 * k + 1$ zu einem
      Überlauf kommen.  Einen Überlauf k\"onnen wir daran erkennen, dass $2 * k + 1$ negativ wird.
      In diesem Fall setzen wir $k$ auf den maximal zul\"assigen Wert $l$.
\item Wenn die \texttt{while}-Schleife in Zeile 7 abbricht, kann es passieren, dass $k$ \"uber die
      rechte Intervall-Grenze 
      hinauszeigt.  In diesem Fall setzen wir in Zeile 15 den Index $k$ auf den maximal zul\"assigen Wert $l$.
\item Wenn das Programm in Zeile 17 ankommt, dann wissen wir, dass der gesuchte Wert von $k$ sich
      innerhalb des Intervalls $[\textsl{lastK}+ 1, \cdots, k]$ befinden muss.  Die genaue Position
      von $k$ wird in der \texttt{while}-Schleife in Zeile 17 nun durch Intervall-Halbierung
      bestimmt:
      \begin{enumerate}
      \item Zun\"achst bestimmen wir in Zeile 18 die Mitte $m$ des Intevalls.
      \item Falls der Wert $\mathtt{mAux}[m] \leq x$ ist, muss $x$ rechts von $m$ liegen
            und wir k\"onnen die linke Intervall-Grenze auf $m+1$ erh\"ohen.
      \item Andernfalls muss $x$ in dem Intervall $[\textsl{lastK}, m]$ liegen und wir setzen die
            rechte Intervall-Grenze auf $m$.
      \end{enumerate}
      Die Invariante der \texttt{while}-Schleife ist
      \\[0.2cm]
      \hspace*{1.3cm}
      $\mathtt{mAux}[b + lastK] \leq x < \mathtt{mAux}[b + k]$.
      \\[0.2cm]
      Die Schleife verkleinert die Grenzen des Intervalls solange, bis $k = \textsl{lastK} + 1$
      gilt.  Wird dann $m$ berechnet, so gilt
      \\[0.2cm]
      \hspace*{1.3cm}
      $m = \textsl{lastK}$,
      \\[0.2cm]
      so dass der Test $\texttt{mAux}[b + m] \leq x$ erfolgreich ist und \textsl{lastK} auf 
      $\textsl{lastK} + 1 = k$ gesetzt wird.  Im n\"achsten Schritt hat das Intervall dann die L\"ange
      0 und die Schleife bricht ab.  Tats\"achlich sind dann alle Elemente des Teilfeldes links von
      $k$ kleiner-gleich $x$, der Wert an der Position $k$ ist aber echt gr\"o\3er als $x$, so dass $k$
      nun der gesuchte Index ist.
\end{enumerate}

\begin{figure}[!ht]
\centering
\begin{Verbatim}[ frame         = lines, 
                  framesep      = 0.3cm, 
                  firstnumber   = 1,
                  labelposition = bottomline,
                  numbers       = left,
                  numbersep     = -0.2cm,
                  xleftmargin   = 0.8cm,
                  xrightmargin  = 0.8cm,
                ]
    private void merge(int b1, int l1, int b2, int l2) {
        System.arraycopy(mArray, b1, mAux, b1, l1);
        System.arraycopy(mArray, b2, mAux, b2, l2);
        int c1 = b1;   // indexes into first  run
        int c2 = b2;   // indexes into second run 
        int d  = b1;   // destination, index where to write next element
    outer:
        while (true) {
            int n1 = 0; // Number of times in a row that first  run won
            int n2 = 0; // Number of times in a row that second run won
            do {
                if (mAux[c2] < mAux[c1]) {
                    mArray[d] = mAux[c2];
                    ++d; ++c2; --l2;
                    if (l2 == 0) { 
                        break outer;
                    }
                    ++n2; n1 = 0;
                } else { // mArray[c1] <= mAux[c2]
                    mArray[d] = mAux[c1];
                    ++d; ++c1; --l1;
                    if (l1 == 0) { break outer; }
                    n1++; n2 = 0;
                }
            } while (n1 + n2 < MIN_GALLOP);
            do {
                n1 = gallop(mAux[c2], c1, l1);
                if (n1 != 0) {
                    System.arraycopy(mAux, c1, mArray, d, n1);
                    d += n1; c1 += n1; l1 -= n1;
                    if (l1 == 0) { break outer; }
                }
                n2 = gallop(mAux[c1], c2, l2);
                if (n2 != 0) {
                    System.arraycopy(mAux, c2, mArray, d, n2);
                    d += n2; c2 += n2; l2 -= n2;
                    if (l2 == 0) { break outer; }
                }
            } while (n1 + n2 >= MIN_GALLOP);
        }  // end of "outer" loop
        if (l1 == 0) {
            System.arraycopy(mArray, c2, mArray, d, l2);
        } else { // l2 == 0
            System.arraycopy(mAux, c1, mArray, d, l1);
        }
    }
\end{Verbatim}
\vspace*{-0.3cm}
\caption{The method $\textsl{merge}()$}
\label{fig:TimSort.java:merge}
\end{figure}

Abbildung \ref{fig:TimSort.java:merge} zeigt die Implementierung der Methode $\textsl{merge}()$.
Der Aufruf $\textsl{merge}(b_1, l_1, b_2, l_2)$ hat die Aufgabe, die beiden aufsteigend sortierten Teilfelder
\\[0.2cm]
\hspace*{0.8cm}
$\bigl[ \texttt{mArray}[b_1], \cdots, \texttt{mArray}[b_1 + (l_1 - 1)]\bigr]$ \quad und \quad
$\bigl[ \texttt{mArray}[b_2], \cdots, \texttt{mArray}[b_2 + (l_1 - 1)]\bigr]$ 
\\[0.2cm]
so zu mischen, dass das resultierende Teilfeld wiederum aufsteigend sortiert ist.  Dabei ist
zus\"atzlich vorausgesetzt, dass das zweite Teilfeld dort beginnt, wo das erste Teilfeld endet, es
gilt also
\\[0.2cm]
\hspace*{1.3cm}
$b_1 + l_1 = b_2$.
\\[0.2cm]
Zun\"achst werden beide Teilfelder in das Hilfsfeld \texttt{mAux} kopiert.  An dieser Stelle ist die
gezeigte Implementierung noch verbesserungsf\"ahig, in dem Orginal von Tim Peters wird nur das
kleinere Teilfeld in das Hilfsfeld kopiert.  Das f\"uhrt aber zu einer un\"ubersichtlicheren
Implementierung.

Der Index $c_1$ iteriert nun \"uber das erste Teilfeld, w\"ahrend $c_2$ \"uber das zweite Teilfeld l\"auft.
Der Index $d$ gibt an, wohin das n\"achste Element geschrieben werden soll.  Die innere
\texttt{do}-\texttt{while}-Schleife, die in Zeile 11 beginnt, mischt die beiden Teilfelder auf
konventionelle Weise.  Gleichzeitig z\"ahlen wir mit, wie oft das n\"achste Element, das in das
Ergebnis-Feld eingef\"ugt \texttt{mAux} wird, hintereinander aus dem ersten bzw.~dem zweiten Teilfeld
kommt.   Falls wir feststellen, dass ein zusammenh\"angender Block, der  \texttt{MIN\_GALLOP} oder mehr
Elemente enth\"alt aus dem ersten oder zweiten Teilfeld kopiert wird, dann wird die erste
\texttt{do}-\texttt{while}-Schleife verlassen und das Programm  geht in Zeile 26 in den
\emph{Gallop}, sprich exponentielle Suche, \"uber.  Der Hintergrund ist hier folgender:  In der Praxis
sind zu sortierende Felder oft \emph{klumpig}, d.h.~das Feld enth\"alt Teilfelder der Art, dass
beispielsweise alle Elemente des ersten Teilfeldes gr\"o\3er sind als die Elemente des zweiten
Teilfeldes.  Werden zwei Teilfelder dieser Art auf konventionelle Art gemischt, so werden der Reihe
nach alle Elemente des zweiten Teilfeldes mit dem ersten Element des ersten Teilfeldes verglichen.
Dies ist zu aufwendig, denn wenn wir beispeilsweise feststellen, dass das letzte Element des
zweiten Teilfeldes kleiner ist als das erste Element des ersten Teilfeldes, dann ist kein weiterer
Vergleich mehr notwendig, da wir das zweite Teilfeld vor das erste Teilfeld h\"angen k\"onnen.   Im
Allgemeinen ist die Situation, wenn wir in Zeile 26 ankommen wie folgt: Wir wollen die Teilfelder 
\\[0.2cm]
\hspace*{1.3cm}
$\bigl[ \texttt{mAux}[c_1], \cdots, \texttt{mAux}[c_1 + (l_1 - 1)]\bigr]$ \quad und \quad
$\bigl[ \texttt{mAux}[c_2], \cdots, \texttt{mAux}[c_2 + (l_2 - 1)]\bigr]$ 
\\[0.2cm]
mischen.  Dazu bestimmen wir zun\"achst ein Position $n_1$ innerhalb des ersten Teilfeldes, so dass
alle Elemente links von $c_1 + n_1$ kleiner-gleich dem ersten Element $\mathtt{mAux}[c_2]$ 
des zweiten Teilfeldes sind,
\\[0.2cm]
\hspace*{1.3cm}
$\forall j \in \{ c_1, \cdots, c_1 + (n_1 - 1) \} : \mathtt{mAux}[j] \leq \mathtt{mAux}[c_2]$,
\\[0.2cm]
w\"ahrend das Element an der Position $c_1 + n_1$ gr\"o\3er als $\mathtt{mAux}[c_2]$ ist:
\\[0.2cm]
\hspace*{1.3cm}
$\mathtt{mAux}[c_1 + n_1] > \mathtt{mAux}[c_2]$.
\\[0.2cm]
Diese Position bestimmen wir durch exponentielle Suche.  Dann wissen wir, dass alle die Elemente mit
Indizes aus der Menge $\{ c_1, \cdots, c_1 + (n_1 - 1) \}$ sich bereits an der richtigen Position
befinden.  Analog bestimmen wir anschlie\3end eine Position $c_2 + n_2$, so dass alle Elemente aus
dem zweiten Teilfeld, die links von dieser Position stehen, kleiner-gleich dem ersten Element
$\mathtt{mAux}[c_1]$ des verbliebenen ersten Teilfeldes sind.   Diese Elemente k\"onnen dann in einem
Block in das Feld \texttt{mArray} kopiert werden.  Werden bei diesem Verfahren die als Block
kopierten Bereiche zu klein, so wechselt der Algorithmus in die konventionelle Methode zum Mischen zur\"uck.
Am Ende der \"au\3eren \texttt{while}-Schleife sind eventuell noch Elemente in einem der beiden
Teilfelder vorhanden, w\"ahrend das andere Teilfeld leer ist.  Diese werden dann in Zeile 42 bzw.~44
in das Feld \texttt{mArray} kopiert.

\subsection{Bewertung}
Bei Feldern, die bereits weitgehend vorsortiert sind, hat \emph{TimSort}, \"ahnlich wie
``\emph{Sortieren durch Einf\"ugen}'', eine lineare Komplexit\"at.  Da in vielen in der Praxis
auftretenden Sortierproblemen die zu sortierenden Daten zumindest teilweise vorsortiert sind, ist
\emph{TimSort} den anderen Sortier-Algorithmen \"uberlegen.

\paragraph{Historisches}  
Der Quick-Sort-Algorithmus wurde von Charles Antony Richard Hoare \cite{hoare:61}
entwickelt, der Merge-Sort-Algorithmus geht auf John von Neumann zur\"uck.

%%% Local Variables: 
%%% mode: latex
%%% TeX-master: "algorithmen"
%%% End: 
