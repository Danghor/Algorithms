\chapter{Die Monte-Carlo-Methode}
Bestimmte Probleme sind so komplex, dass es mit vertretbarem Aufwand nicht m\"oglich ist, eine exakte L\"osung zu
berechnen.  Oft l\"asst sich jedoch mit Hilfe einer Simulation das Problem zumindest n\"aherungsweise l\"osen.  
\begin{enumerate}
\item Das Problem der Berechnung der Volumina von K\"orpern, die eine gro{\ss}e Zahl von Begrenzungsfl\"achen haben,
      l\"asst sich auf die Berechnung mehrdiminsionalee Integrale zur\"uckf\"uhren.  In der Regel k\"onnen diese
      Integrationen aber nicht analytisch ausgef\"uhrt werden.   Mit der Monte-Carlo-Methode l\"asst sich hier
      zumindest ein N\"aherungswert bestimmen.
\item Die Gesetzm\"a{\ss}igkeiten des Verhaltens komplexer Systeme, die zuf\"alligen Einfl\"ussen einer Umgebung
      ausgesetzt sind, k\"onnen oft nur durch Simulationen bestimmt werden.  Wird beispielsweise ein neues
      U-Bahn-System geplant, so wird Kapazit\"at eines projektierten Systems durch Simulationen ermittelt.
\item Bei Gl\"uckspielen ist die exakte Berechnung bestimmter Wahrscheinlichkeiten oft nicht m\"oglich.
      Mit Hilfe von Simulationen lassen sich aber gute N\"aherungswerte bestimmen.  
\end{enumerate}
Die obige Liste k\"onnte leicht fortgesetzt werden.  In diesem Kapitel werden wir zwei Beispiele betrachten.
\begin{enumerate}
\item Als einf\"uhrendes Beispiel zeigen wir, wie sich mit Hilfe der Monte-Carlo-Methode Fl\"acheninhalte bestimmen
      lassen.  Konkret berechnen wir den Fl\"acheninhalt eines Kreises und bestimmen auf diese Weise
      die Zahl $\pi$.
\item Als zweites Beispiel zeigen wir, wie sich Karten zuf\"allig mischen lassen.
      Damit kann beispielsweise die Wahrscheinlichkeit daf\"ur berechnet werden, dass im Texas Hold'em Poker eine
      gegebene Hand gegen eine zuf\"allige Hand gewinnt.
\end{enumerate}

\section{Berechnung der Kreiszahl $\pi$}
Eine sehr einfache Methode zur Berechnung einer Approximation der Zahl $\pi$ funktioniert wie folgt.
Wir betrachten in der reellen Ebene den Einheits-Kreis $E$, der als die Menge
\[ E = \{ \pair(x,y) \in \mathbb{R}^2 \mid x^2 + y^2 \leq 1 \} \]
definiert ist.  Hier gibt der Ausdruck $x^2 + y^2$ nach dem Satz des Pythagoras gerade den Abstand an,
den der Punkt $\pair(x,y)$ vom Koordinatenursprung $\pair(0,0)$ hat.  Der Einheits-Kreis hat offenbar den
Radius $r = 1$ damit gilt f\"ur die Fl\"ache dieses Kreises
\[ \textsl{Fl\"ache}(E) = \pi \cdot r^2 = \pi. \]
Wenn es uns gelingt, diese Fl\"ache zu berechnen, dann haben wir also $\pi$ bestimmt.  Eine experimentelle
Methode zur Bestimmung dieser Fl\"ache besteht darin, dass wir in das Einheits-Quadrat $Q$, dass durch
\[ Q = \{ \pair(x,y) \in \mathbb{R} \mid -1 \leq x \leq 1 \;\wedge\; -1 \leq y \leq 1 \} \]
definiert ist, zuf\"allig eine gro{\ss}e Zahl $n$ von Sandk\"ornern werfen.  Wir notieren uns dabei die Zahl $k$ 
der Sandk\"orner, die in den Einheits-Kreis fallen.  Die Wahrscheinlichkeit $p$ daf\"ur, dass ein Sandkorn in den
Einheits-Kreis f\"allt, wird nun proportional zur Fl\"ache des Einheits-Kreises sein:
\[ p = \frac{\textsl{Fl\"ache}(E)}{\textsl{Fl\"ache}(Q)}. \]
Das Einheits-Quadrat die Seitenl\"ange 2 hat, gilt f\"ur die Fl\"ache des Einheits-Quadrats $Q$ 
\[ \textsl{Fl\"ache}(Q) = 2^2 = 4. \]
Auf der anderen Seite wird bei einer hohen Anzahl von Sandk\"ornern das Verh\"altnis $\frac{k}{n}$ gegen diese
Wahrscheinlichkeit $p$ streben, so dass wir insgesamt
\[ \frac{k}{n} \approx \frac{\pi}{4} \]
haben, woraus sich f\"ur $\pi$ die N\"aherungsformel
\[ \pi \approx 4 \cdot \frac{k}{n} \]
ergibt.  W\"ahrend die alten Ägypter bei dieser historischen Methode zur Berechung von $\pi$ noch Tonnen von
Sand\footnote{So ist die Sahara entstanden.}
ben\"otigten,  k\"onnen wir dieses Experiment heute einfacher mit Hilfe eines Computers durchf\"uhren.

\begin{figure}[!ht]
\centering
\begin{Verbatim}[ frame         = lines, 
                  framesep      = 0.3cm, 
                  labelposition = bottomline,
                  numbers       = left,
                  numbersep     = -0.2cm,
                  xleftmargin   = 0.8cm,
                  xrightmargin  = 0.8cm,
                ]
    import java.util.*;
    
    public class CalculatePi {
        private Random mRandom;
        
        public CalculatePi() {
            mRandom = new Random();
        }    
        public double calcPi(long n) {
            long k = 0;  // number of points inside the circle
            for (long i = 0; i < n; ++i) {
                double x = 2 * mRandom.nextDouble() - 1;
                double y = 2 * mRandom.nextDouble() - 1;
                double r = x * x + y * y;
                if (r <= 1) { ++k; }
            }
            return 4.0 * k / n;
        }    
        public static void main(String[] args) {
            CalculatePi c = new CalculatePi();
            for (long n = 10; n < 1000000000000L; n = n * 10) {
                double pi = c.calcPi(n);
                System.out.printf("n = %14d: %g,  Fehler: %+f\n", 
                                   n, pi, pi - Math.PI);
            }
        }
    }
\end{Verbatim}
\vspace*{-0.3cm}
\caption{Experimentelle Bestimmung von $\pi$ mit der Monte-Carlo-Methode.}
\label{fig:CalculatePi.java}
\end{figure}

Abbildung \ref{fig:CalculatePi.java} zeigt die Implementierung der Klasse \texttt{CalculatePi}, die das
Verfahren implementiert. 
\begin{enumerate}
\item Die Klasse hat eine Member-Variable \texttt{mRandom}.  Hierbei handelt es sich um einen
      Zufallszahlen-Generator, der von der Klasse \texttt{java.util.Random} zur Verf\"ugung gestellt wird.
\item Die Methode $\textsl{calcPi}(n)$ f\"uhrt $n$ Versuche zur Berechnung der Zahl $\pi$ durch.
      Dazu werden mit Hilfe der Methode $\textsl{nextDouble}()$ zun\"achst Zufallszahlen erzeugt, die in dem
      Intervall $[0,1]$ liegen.  Mit Hilfe der Transformation 
      \[ t \mapsto 2 \cdot t - 1  \]
      wird das Intervall $[0,1]$ in das Intervall $[-1, 1]$ transformiert, so dass die in den Zeilen 12 und 13 
      berechneten Koordinaten \texttt{x} und \texttt{y} dann einen zuf\"allig in das Einheits-Quadrat $Q$ 
      geworfenes Sandkorn beschreiben.  Wir berechnen in Zeile 14 den Abstand dieses Sandkorns vom
      Koordinatenursprung und \"uberpr\"ufen in Zeile 15, ob das Sandkorn innerhalb des Kreises liegt.
\end{enumerate}

\begin{table}[htbp]
  \label{tab:pi}
  \centering
  \begin{tabular}[t]{|r|r|r|}
  \hline
  n & N\"aherung f\"ur $\pi$ & Fehler der N\"aherung \\
  \hline
  \hline
               $10$ & 2.40000 & -0.741593 \\
\hline
              $100$ & 3.28000 & +0.138407 \\
\hline
           $1\,000$ & 3.21600 & +0.074407 \\
\hline
          $10\,000$ & 3.13080 & -0.010793 \\
\hline
         $100\,000$ & 3.13832 & -0.003273 \\
\hline
      $1\,000\,000$ & 3.13933 & -0.002261 \\
\hline
     $10\,000\,000$ & 3.14095 & -0.000645 \\
\hline
    $100\,000\,000$ & 3.14155 & -0.000042 \\
\hline
 $1\,000\,000\,000$ & 3.14160 & +0.000011 \\
\hline
  \end{tabular}
  \caption{Ergebnisse bei der Bestimmung von $\pi$ mit der Monte-Carlo-Methode}
\end{table}

Lassen wir das Progamm laufen, so erhalten wir die in Tabelle \ref{tab:pi} gezeigten Ergebnisse.
Wir sehen, dass wir zur Berechnung auf eine Genauigkeit von zwei Stellen hinter dem Komma etwa $100\,000$
Versuche brauchen, was angesichts der Rechenleistung heutiger Computer kein Problem darstellt.  Die Berechnung
weiterer Stellen gestaltet sich jedoch sehr aufwendig:  Die Berechnung der dritten Stelle hinter dem Komma
erfordert  $100\,000\,000$ Versuche. Grob gesch\"atzt k\"onnen wir
sagen, dass sich der Aufwand bei der Berechnung einer weiteren Stelle verhundertfacht!  
Wir halten folgende Beobachtung fest:

\begin{center}
  \framebox{
  \framebox{
  \begin{minipage}[t]{12cm}
  Die Monte-Carlo-Methode ist gut geeignet, um einfache Absch\"atzungen zu berechnen, wird aber sehr aufwendig,
  wenn eine hohe Genauigkeit gefordert ist.
  \end{minipage}
  }}
\end{center}

\section{Theoretischer Hintergrund}
Wir diskutieren nun den theoretischen Hintergrund der Monte-Carlo-Methode.  Da im zweiten Semester noch
keine detailierteren Kenntnisse aus der Wahrscheinlichkeitsrechnung vorhanden sind, beschr\"anken wir
uns darauf, die wesentlichen Ergebnisse anzugeben.  Eine Begr\"undung dieser Ergebnisse erfolgt dann
in der Mathematik-Vorlesung im vierten Semester. 

Bei der Monte-Carlo-Methode wird ein Zufalls-Experiment, im gerade diskutierten Beispiel war es das Werfen
eines Sandkorns, sehr oft wiederholt.  F\"ur den Ausgang dieses Zufalls-Experiments gibt es dabei zwei
M\"oglichkeiten:  Es ist entweder erfolgreich (im obigen Beispiel landet das Sandkorn im Kreis) oder nicht
erfolgreich.  Ein solches Experiment bezeichnen wir als \emph{Bernoulli-Experiment}.
 Hat die Wahrscheinlichkeit, dass das Experiment erfolgreich ist, den Wert $p$ und wird das
Experiment $n$ mal ausgef\"uhrt, so ist die Wahrscheinlichkeit, dass genau $k$ dieser Versuche erfolgreich sind,
durch die Formel
\[ P(k) = \frac{n!}{k! \cdot (n-k)!} \cdot p^k \cdot (1-p)^{n-k} \]
gegeben, die auch als \emph{Binomial-Verteilung} bekannt ist.  F\"ur gro{\ss}e Werte von $n$ ist die obige Formel
sehr unhandlich, kann aber gut durch die \emph{Gau{\ss}-Verteilung} approximiert werden, es gilt
\[ \frac{n!}{k! \cdot (n-k)!} \cdot p^k \cdot (1-p)^{n-k} \approx  
   \frac{1}{\sqrt{2\cdot \pi \cdot n \cdot p \cdot (1-p)\;}} \cdot 
   \exp\left(-\frac{(k - n \cdot p)^2}{2 \cdot n \cdot p \cdot (1 - p)}\right)
\]
Wird das Experiment $n$ mal durchgef\"uhrt, so erwarten wir im Durchschnitt nat\"urlich, dass $n \cdot p$ der
Versuche erfolgreich sein werden.  Darauf basiert unsere Sch\"atzung f\"ur den Wert von $p$, denn wir approximieren
$p$ durch die Formel
\[ p \approx \frac{k}{n}, \]
wobei $k$ die Anzahl der erfolgreichen Experimente bezeichnet.  Nun werden in der Regel nicht genau $n \cdot p$
Versuche erfolgreich sein sondern zufallsbedingt werden ein Paar mehr oder ein Paar weniger Versuche
erfolgreich sein.  Das f\"uhrt dazu, dass unsere Sch\"atzung von $p$ eine Ungenauigkeit aufweist, deren ungef\"ahre
Gr\"o{\ss}e wir irgendwie absch\"atzen m\"ussen um unsere Ergebnisse beurteilen zu k\"onnen.

Um eine Idee davon zu bekommen, wie sehr die Anzahl der erfolgreichen Versuche von dem Wert $\frac{k}{n}$
abweicht,  f\"uhren wir den Begriff der \emph{Streuung} $\sigma$ ein, die f\"ur eine Gau{\ss}-verteilte Zufallsgr\"o{\ss}e
durch die Formel
\[ \sigma = \sqrt{n \cdot p \cdot (1 - p)} \]
gegeben ist.  Die Streuung gibt ein Ma{\ss} daf\"ur, wie stark der gemessene Wert von $k$ von dem im Mittel
erwarteten Wert $p \cdot n$ abweicht.  Es kann gezeigt werden, dass die Wahrscheinlichkeit, dass $k$ au{\ss}erhalb
des Intervalls
\[ [ p \cdot n - 3 \cdot \sigma,\, p \cdot n + 3 \cdot \sigma ] \]
liegt, also um mehr als das dreifache von dem erwarteten Wert abweicht, kleiner als $0.27 \%$ ist.  F\"ur
 die Genauigkeit unserer Sch\"atzung $p \approx \frac{k}{n}$ hei{\ss}t das, dass dieser Sch\"atzwert mit hoher
 Wahrscheinlichkeit ($99.73\%$) in dem Intervall
\[  \left[ \frac{p \cdot n - 3 \cdot \sigma}{n},\, \frac{p \cdot n + 3 \cdot \sigma}{n} \right] 
  = \left[ p - 3 \cdot \frac{\sigma}{n},\, p + 3 \cdot \frac{\sigma}{n} \right]
\] 
liegt.  Die Genauigkeit $\varepsilon(n)$ ist durch die halbe L\"ange dieses Intervalls gegeben und hat daher den
Wert 
\[ \varepsilon(n) = 3 \cdot \frac{\sigma}{n} = 3 \cdot \sqrt{\frac{p \cdot (1 - p)}{n}}. \]
Wir erkennen hier, dass zur Erh\"ohung der Genauigkeit den Faktor 10 die Zahl der Versuche um den Faktor 100
vergr\"o{\ss}ert werden muss.

Wenden wir die obige Formel auf die im letzen Abschnitt durchgef\"uhrte Berechnung der Zahl $\pi$ an, so erhalten
wir wegen $p = \frac{\pi}{4}$ die in Abbildung \ref{tab:Precision.java} gezeigten Ergebnisse.

\begin{table}[htbp]
  \centering
  \begin{tabular}[t]{|r|r|}
\hline
Anzahl Versuche $n$ & Genauigkeit $\varepsilon(n)$ \\
\hline
\hline
                $10$ & 0.389478 \\
\hline
               $100$ & 0.123164 \\
\hline
            $1\,000$ & 0.0389478 \\
\hline
           $10\,000$ & 0.0123164 \\
\hline
          $100\,000$ & 0.00389478 \\
\hline
       $1\,000\,000$ & 0.00123164 \\
\hline
      $10\,000\,000$ & 0.000389478 \\
\hline
     $100\,000\,000$ & 0.000123164 \\
\hline
  $1\,000\,000\,000$ & 3.89478e-05 \\
\hline
 $10\,000\,000\,000$ & 1.23164e-05 \\
\hline
$100\,000\,000\,000$ & 3.89478e-06 \\
\hline

  \end{tabular}
  \caption{Genauigkeit der Bestimung von $\pi$ bei einer Sicherheit von $99,73\%$.}
  \label{tab:Precision.java}
\end{table}

\vspace*{0.3cm}

\exercise
Wieviel Tonnen Sand ben\"otigte Pharao Ramses \texttt{II}, als er $\pi$ mit der Monte-Carlo-Methode auf 6
Stellen hinter dem Komma berechnet hat?
\pagebreak

\noindent
\textbf{Hinweise}: 
\begin{enumerate}
\item Ein Sandkorn wiegt im Durchschnitt etwa $\frac{1}{1000}$ Gramm.
\item Um eine Genauigkeit von 6 Stellen hinter dem Komma zu haben, sollte der Fehler durch $10^{-7}$
      abgesch\"atzt werden.  Das Ergebnis soll mit einer Wahrscheinlichkeit von $99,7\%$ korrekt sein.
\end{enumerate}

\solution
Nach dem Hinweis soll 
\[ \varepsilon(n) = 10^{-7} \]
gelten. Setzen wir hier die Formel f\"ur $\varepsilon(n)$ ein, so erhalten wir
\[  
\begin{array}[t]{llcl}
                & 3 \cdot \sqrt{\frac{p \cdot (1 - p)}{n}} & = & 10^{-7}   \\[0.3cm]
\Leftrightarrow & 9 \cdot        \frac{p \cdot (1 - p)}{n} & = & 10^{-14}      \\[0.3cm]
\Leftrightarrow & 9 \cdot    p \cdot (1 - p) \cdot 10^{14} & = & n 
\end{array}
\]
Um an dieser Stelle weitermachen zu k\"onnen, ben\"otigen wir den Wert der Wahrscheinlichkeit $p$.  Der korrekte
Wert von $p$ ist f\"ur unser Experiment durch $\frac{\pi}{4}$ gegeben.  Da wir $\pi$ ja erst berechnen wollen,
nehmen wir als Absch\"atzung von $\pi$ den Wert $3$, so dass $p$ den Wert $\frac{3}{4}$ hat.  Da eine Tonne
insgesamt $10^9$ Sandk\"orner enth\"alt, bekommen wir f\"ur das Gewicht $g$ das Ergebnis
\begin{eqnarray*}
g & \approx & 9 \cdot \frac{3}{4} \cdot \frac{1}{4} \cdot 10^5\; \mbox{Tonnen} \\[0.2cm]
  & \approx & 168\,750 \; \mbox{Tonnen} 
\end{eqnarray*}
Die Cheops-Pyramide ist 35 mal so schwer. \qed

\section{Erzeugung zuf\"alliger Permutationen}
In diesem Abschnitt lernen wir ein Verfahren kennen, mit dem es m\"oglich ist, eine gegebene
Liste zuf\"allig zu permutieren.  Anschaulich kann ein solches Verfahren mit dem Mischen von
Karten verglichen werden.  Das Verfahren wird auch tats\"achlich genau dazu eingesetzt:
Bei der Berechnung von Gewinn-Wahrscheinlichkeiten bei Kartenspielen wie Poker wird das
Mischen der Karten durch den gleich vorgestellten Algorithmus erledigt.  

Um eine $n$-elementige Liste $L = [x_1,x_2, \cdots, x_n]$ zuf\"allig zu permutieren,
unterscheiden wir zwei F\"alle:
\begin{enumerate}
\item Die Liste $L$ hat die L\"ange 1 und besteht folglich nur aus einem Element, $L = [x]$.
      In diesem Fall gibt die Funktion $\textsl{permute}(L)$ die Liste unver\"andert zur\"uck:
      \[ \#L = 1 \rightarrow \textsl{permute}(L) = L \]
\item Die Liste $L$ hat eine L\"ange, die gr\"o{\ss}er als $1$ ist.  In diesem Fall w\"ahlen wir
      zuf\"allig ein Element aus, das hinterher  in der zu erzeugenden Permutation an der letzten Stelle
      stehen soll.  Wir entfernen dieses Element aus der Liste und permutieren
      anschlie{\ss}end die verbleibende Liste.  An die dabei erhaltene Permutation h\"angen wir
      noch das anfangs ausgew\"ahlte Element an.  Haben wir eine Funktion
      \[ \textsl{random}: \mathbb{N} \rightarrow \mathbb{N}, \]
      so dass der Aufruf $\textsl{random}(n)$ zuf\"allig eine Zahl aus der Menge $\{1,\cdots,n\}$
      liefert, so k\"onnen wir diese Überlegung wie folgt formalisieren:
      \\[0.2cm]
      \hspace*{1.3cm}
      $ \#L = n \wedge n > 1 \wedge \textsl{random}(n) = k \rightarrow 
         \textsl{permute}(L) = \textsl{permute}\bigl(\textsl{delete}(L,k)\bigr) + \bigl[ L(k)
         \bigr]
      $.
\\[0.2cm]
      Der  Funktionsaufruf $\textsl{delete}(L,k)$ k\"oscht dabei das $k$-te Element aus der Liste $L$,
      wir k\"onnten also schreiben
      \\[0.2cm]
      \hspace*{1.3cm}
      $\textsl{delete}(L,k) = L(1\,..\,k-1) + L(k+1\,..\,\#L)$.
\end{enumerate}

\begin{figure}[!ht]
\centering
\begin{Verbatim}[ frame         = lines, 
                  framesep      = 0.3cm, 
                  labelposition = bottomline,
                  numbers       = left,
                  numbersep     = -0.2cm,
                  xleftmargin   = 0.8cm,
                  xrightmargin  = 0.8cm,
                ]
    public static void permute(int[] array) {
        Random random = new Random(0);
        int    n      = array.length;
        for (int i = n; i > 1; --i) {
            int  k = random.nextInt(i);
            swap(array, i - 1, k);
        }
    }
    private static void swap(int[] array, int i, int j) {
        int t = array[i];
        array[i] = array[j];
        array[j] = t;
    }
\end{Verbatim}
\vspace*{-0.3cm}
\caption{Berechnung zuf\"alliger Permutationen eines Feldes}
\label{fig:RandomPermutation.java}
\end{figure}

Abbildung \ref{fig:RandomPermutation.java} zeigt die Umsetzung dieser Idee in \textsl{Java}.
Die dort gezeigte Methode \textsl{permute} erzeugt eine zuf\"allige Permutation des Feldes \texttt{array}, das
dieser Methode als Argument \"ubergeben wird.
\begin{enumerate}
\item Zun\"achst wird in Zeile 2 ein Zufallszahlen-Generator erzeugt.
\item Die Schleife in Zeile 4 l\"auft nun r\"uckw\"arts \"uber das Feld, denn wir wollen ja zun\"achst 
      das letzte Element des Feldes bestimmen.  Beim $i$-ten Durchlauf gehen wir davon aus,
      dass die Positionen $\mathtt{array}[i, \cdots, n-1]$ bereits bestimmt sind
      und dass nur noch die Positionen $\mathtt{array}[0, \cdots, i-1]$ permutiert werden m\"ussen.
\item Der Aufruf $\texttt{random.nextInt}(i)$ liefert eine Zufallszahl $k$ aus der Menge 
      $\{0, \cdots, i-1)$.  Diese Zahl gibt an, welches Element an die $(i-1)$-te Position der
      erzeugten Permutation gesetzt wird. Dieses Element wird dann durch den Aufruf von
      \texttt{swap} an die Position $i-1$ gesetzt und das Element, das vorher an der Position $i-1$
      stand, wird an der Position $k$ gespeichert.
\end{enumerate}
Es kann gezeigt werden, dass der oben vorgestellte Algorithmus tats\"achlich alle Permutationen einer
gegebenen Liste mit derselben Wahrscheinlichkeit erzeugt.  Einen Beweis dieser Behauptung
finden Sie beispielsweise in \cite{cormen:01}.

%%% LOCAL Variables: 
%%% mode: latex
%%% TeX-master: "algorithmen"
%%% End: 
