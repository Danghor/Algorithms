\chapter{The Programming Language \textsl{Python}}
\section{Motivation}
In order to present algorithms and actually test them, we need a programming language.  I have chosen
\href{http://www.python.org}{\textsl{Python}} for this purpose for the following reasons:
\begin{enumerate}
\item \textsl{Python} is easy to learn.
\item Usage of \textsl{Python} is widespread.  The \href{https://www.tiobe.com/tiobe-index/}{Tiobe index}
      of September 2022 lists it as the programming language that is currently most wide spread.
\item It has a number of advanced features like garbage collection and matching that make it very easy to
      to implement the algorithms that we are going to discuss in these lectures.
\item Furthermore, \textsl{Python} has a huge number of openly available libraries for almost anything.
      As of September 2022, the \textsl{Python} package index \href{https://pypi.org}{PyPi} contains over
      $400,000$ packages.  Other indices also rank \textsl{Python} very high.  For example,
      the \href{https://redmonk.com/sogrady/2022/03/28/language-rankings-1-22/}{RedMonk programming language ranking}
      ranks \textsl{Python} on the second place.  The first place is taken by
      \href{https://www.javascript.com}{\textsl{JavaScript}}.
\item According to a report
      \href{https://cacm.acm.org/blogs/blog-cacm/176450-python-is-now-the-most-popular-introductory-teaching-language-at-top-u-s-universities/fulltext}{report} 
      of 2014, 8 of the top 10 US universities teach \textsl{Python} in their introductory computer science courses.
\end{enumerate}
The rest of this chapter is structured as follows:
\begin{enumerate}[(a)]
\item We begin by discussing the installation of \textsl{Python}.
\item After that, I present a \textsl{Jupyter notebook} that introduces those features of \textsl{Python} that
      we will use in this lecture.
\item Finally, I discuss books and tutorials that can be used to learn \textsl{Python}.  
\end{enumerate}

\section{Installation}
Although most operating systems in current use, i.e.~Windows, MacOs, and Linux all have a \textsl{Python}
interpreter included, we should install \textsl{Python}, since we need som additional libraries and it is best
not to mess with the \textsl{Python} interpreter that is used by the operating systems, because adding
libraries to it might lead to inconsistencies.

The easiest way to install python and its libraries is via \href{https://www.anaconda.com/download/}{Anaconda}.
\index{Anaconda}
The installation proceeds in the following steps.
\begin{enumerate}[(a)]
\item We download the appropriate version of the Anaconda software from
      \\[0.2cm]
      \hspace*{1.3cm}
      \href{https://www.anaconda.com/products/distribution}{https://www.anaconda.com/products/distribution}.
\item Next, we double-click the downloaded file to start the installation.  During the installation we have to
      agree to the end user license and a choose a directory where the software is installed.
      questions.  
\item Then we open a terminal to create a \blue{virtual environment} that we will use for this lecture.
      In order to create this virtual environment, we enter the following commands:
      \begin{enumerate}
      \item \texttt{conda create -n algo python=3.10 jupyter notebook}

            This command creates a virtual environment with the name \texttt{algo}.  Furthermore, we specify
            that we are going to use \textsl{Python} version \texttt{3.10} and thatwe want to install
            jupyter notebooks.
      \item \texttt{conda activate algo}

            Every time we want to use the newly created virtual environment that we have just created we have
            activate it with this command.
      \item \texttt{conda install -c anaconda graphviz}

            This command installs the package \texttt{graphviz}.  This package contains a program that
            can be used to visualize graphs.
      \item \texttt{conda install -c conda-forge python-graphviz ipycanvas matplotlib seaborn}

            This command installs additional libraries that will be needed later.
      \item In order to test whether everything has been installed correctly, we run the following command:
            \\[0.2cm]
            \hspace*{1.3cm}
            \texttt{jupyter notebook}
            \\[0.2cm]
            This command should open a browser window.  We can start to write \textsl{Python} programs in this window.
      \end{enumerate}
\end{enumerate}
You do not have to install \textsl{Python} on your own computer to be able to program in \textsl{Python},
because you can use the cloud service \href{https://deepnote.com}{\texttt{deepnote.com}} instead.  
In this lecture, we will be using the version 3.10 of \textsl{Python}.

\section{An Introduction to \textsl{Python}}
For technical reasons\footnote{\textsl{Jupyter} notebooks can be exported in {\LaTeX} format.  However, the 
{\LaTeX} that is produced by \textsl{Jupyter} notebooks needs a number of packages that conflict with other
packages that are needed for these lecture notes.}
I have outsourced the introduction to \textsl{Python} into a separate document, which can be found
here.

\section{Other References}
For reasons of time and space, this lecture has just scratched the surface of what is possible with
\textsl{Python}.  If you want to obtain a deeper understanding of \textsl{Python}, here are three places that 
I would recommend:
\begin{enumerate}
\item First, there is the official \textsl{Python} tutorial, which is available at
      \\[0.2cm]
      \hspace*{1.3cm}
      \href{https://docs.python.org/3.10/tutorial/index.html}{\texttt{https://docs.python.org/3.10/tutorial/index.html}}.

      Furthermore, there are a number of good books available.  I would like to suggest the following 
      book, which is available electronically in our library:
\item \emph{The Quick Python Book} written by Naomi R.~Ceder \cite{ceder:2018} is up to date and gives a
      concise introduction to \textsl{Python}.  The book assumes that the reader has some prior programming
      experience.  I would assume that most of our students have the necessary background to feel comfortable
      with this book.
\item The website
      \href{https://www.w3schools.com/python/default.asp}{https://www.w3schools.com/python/default.asp}
      offers an interactive \textsl{Python} course, which is free.
      The \textsl{Python} course offered by \href{https://programiz.com}{https://programiz.com} 
      can also be recommended.  
\end{enumerate}
Since \textsl{Python} is \textbf{not} the primary objective of these lecture notes, there is no requirement to read
either the \textsl{Python} tutorial or the book mentioned above.  The primary objective of these
lecture notes is to introduce a number of algorithms.
\textsl{Python} is merely used to implement these algorithms.  You should
be able to pick up enough knowledge of \textsl{Python} by closely inspecting the \textsl{Python} programs
discussed in these lecture notes.  
\pagebreak

\section{Reflection}
After having studied the notebook
\href{https://github.com/karlstroetmann/Algorithms}{\texttt{Introduction.ipynb}}, you should be able to answer the following questions.
\begin{enumerate}[(a)]
\item Which \textsl{Python} data types have been introduced in this notebook?
\item How can you build lists via \emph{compressions}? 
\item How can lists be defined in \textsl{Python}?
\item How does \emph{list indexing} work?
\item What type of control structures are supported in \textsl{Python}?
\item How do we invoke mathematical functions in \textsl{Python}?
\item How does the algorithm of Eratosthenes work?  
\end{enumerate}

%%% Local Variables: 
%%% mode: latex
%%% TeX-master: "algorithms"
%%% End: 

