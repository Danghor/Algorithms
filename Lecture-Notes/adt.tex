\chapter{Abstract Data Types}
In the same way as the notion of an algorithm abstracts from the details of a concrete
implementation of this algorithm, the notion of an \emph{abstract data type} abstracts from concrete data
structures.  The notion enables us to separate algorithms from the data
structures used in these algorithms.  The next section gives a formal definition of
``\emph{abstract data types}''.  As an example, we introduce the abstract data type of \emph{stacks}.
The second section shows how abstract data types are supported in \textsc{SetlX}.  Finally, we show how stacks can be used
to evaluate arithmetic expressions.

\section[Formal Definition]{A Formal Definition of Abstract Data Types}
Formally, an \emph{abstract data type} $\mathcal{D}$ is defined as a 5-tupel of the form
\\[0.2cm]
\hspace*{1.3cm}
 $\mathcal{D} = \langle N, P, Fs, Ts, Ax \rangle$.
\\[0.2cm] 
where the meaning of the components is as follows:
\begin{enumerate}
\item $N$ is the \emph{name} of the abstract data type.
\item $P$ is the set of \emph{type parameters}.   Here, a type parameter is just a string.
      This string is interpreted as a type variable.  The idea is that we can later substitute 
      a concrete data type for this string.
\item $\textsl{Fs}$ is the set of \emph{function symbols}.  These function symbols denote the 
      operations that are supported by this abstract data type.
\item $\textsl{Ts}$ is a set of \emph{type specifications}.  For every function symbol
      $f \in \textsl{Fs}$
      the set $Ts$ contains a \emph{type specifications} of the form 
      \\[0.2cm]
      \hspace*{1.3cm} 
      $f: T_1 \times \cdots \times T_n \rightarrow S$. 
      \\[0.2cm]
      Here,  $T_1$, $\cdots$, $T_n$ and $S$ are names of data types.  There are three cases for
      these data types: 
      \begin{enumerate}
      \item We can have concrete data types like, e.~g.~``\texttt{int}'' or ``\texttt{String}''.
      \item Furthermore, these can be the names of abstract data types.
      \item Finally,  $T_1$, $\cdots$, $T_n$ and $S$ can be type parameters from the set $P$.
      \end{enumerate}
      The type specification $f: T_1 \times \cdots \times T_n \rightarrow S$ expresses the fact that
      the function $f$ has to be called as \\[0.2cm] 
      \hspace*{1.3cm}
      $f(t_1,\cdots,t_n)$ 
      \\[0.2cm]
      where for all $i \in \{1,\cdots,n\}$ the argument $t_i$ has type 
      $T_i$.  Furthermore, the result of the function $f$ is of type $S$.

      Additionally, we must have either $T_1 = T$ or $S = T$.  Therefore, either
      the first argument of $f$ has to be of type $T$ or the result of $f$ has to be of type 
      $T$.  If we have  $T_1 \not= T$ and, therefore, $S = T$,
      then $f$ is called a \emph{constructor} of the data type $T$.  Otherwise,
      $f$ is called a  \emph{method}.
\item $Ax$ is a set of mathematical formul\ae.   These formul\ae\ 
      specify the behaviour of the abstract data type and are therefore called
      the \emph{axioms} of $\mathcal{D}$.
\end{enumerate}
The notion of an \underline{a}bstract \underline{d}ata \underline{t}ype is often abbreviated as \textsc{Adt}.

Next, we provide a simple example of an abstract data type, the \emph{stack}.
Informally, a stack can be viewed as a pile of objects that are put on top of each other, so that
only the element on top of the pile is accessible.  An ostensive example of a stack is a pile of
plates that can be found in a canteen.  Usually, the clean plates are placed on top of each other
and only the plate on top is accessible.  Formally, we define the data type \textsl{Stack} as follows:
\begin{enumerate}
\item The name of the data type is \textsl{Stack}.
\item The set of type parameters is $\{ \textsl{Element} \}$.
\item The set of function symbols is \\[0.2cm]
      \hspace*{1.3cm} 
      $\bigl\{ \textsl{Stack}, \textsl{push}, \textsl{pop}, \textsl{top}, \textsl{isEmpty} \bigr\}$.
\item The type specifications of these function symbols are given as follows:
      \begin{enumerate}
      \item $\textsl{stack}: \textsl{Stack}$

            The function $\textsl{stack}$ takes no arguments and produces an empty stack.
            Therefore, this function is a constructor.  Intuitively, the function call $\textsl{stack}()$ 
            creates an empty stack.
      \item $\textsl{push}: \textsl{Stack} \times \textsl{Element} \rightarrow \textsl{Stack}$

            The function call $\textsl{push}(S,x)$ puts the element $x$ on top of the stack $S$.  In
            the following, we will use an object oriented notation and write 
            $S.\textsl{push}(x)$, instead of $\textsl{push}(S,x)$.
      \item $\textsl{pop}: \textsl{Stack}  \rightarrow \textsl{Stack}$

            The function call $S.\textsl{pop}()$ removes the first element from the stack $S$.
      \item $\textsl{top}: \textsl{Stack} \rightarrow \textsl{Element}$

            The function call $S.\textsl{top}()$ returns the element that is on top of the stack $S$. 
      \item $\textsl{isEmpty}: \textsl{Stack} \rightarrow \mathbb{B}$

            The function call $S.\textsl{isEmpty}()$ checks whether the stack $S$ is empty.
      \end{enumerate}
\end{enumerate}
The intuition that we have of a stack is captured by the following axioms.
\begin{enumerate}
\item $\textsl{stack}().\textsl{top}() = \Omega$

      Here, $\Omega$ denotes the undefined value\footnote{
       Some philosophers are concerned that it is not possible to define an undefined value.
       They argue that if an undefined value could be defined, it would be no longer undefined
       and hence it can not be defined.  However, that is precisely the point of the undefined 
       value: As it cannot be defined, it is undefined. \raisebox{-0.1cm}{$\Large\smiley$}}.  
      The expression $\textsl{stack}()$
      creates an empty stack.  Therefore, the given axiom expresses the fact that there is no
      element on top of the empty stack.
\item $S.\textsl{push}(x).\textsl{top}() = x$

      If we have a stack $S$ and then push an element $x$ on top of $S$, then the element on top
      of the resulting stack is, obviously, $x$.
\item $\textsl{stack}().\textsl{pop}() = \Omega$

      Trying to remove an element from the empty stack yields an undefined result.
\item $S.\textsl{push}(x).\textsl{pop}() = S$

      If we have a stack $S$, then push an element $x$ of top of $S$, and finally remove the element
      on top of the resulting stack, then we are back at our stack $S$.
    
\item $\textsl{stack}().\textsl{isEmpty}() = \mathtt{true}$

      This axiom expresses the fact that the stack created by the function call $\textsl{stack}()$
      is empty.
\item $S.\textsl{push}(x).\textsl{isEmpty}() = \mathtt{false}$

      If we push an element $x$ on top of a stack $S$, then the resulting stack cannot be empty.
\end{enumerate}
When contemplating the axioms given above we can recognize some structure.  If we denote the
functions \texttt{stack} and \texttt{push} as \emph{generators},  then the axioms specify the
behavior of the remaining functions on the stacks created by the generators.

The data type of a stack has many applications in computer science.  To give just one example, the
implementation of the \textsl{Java virtual machine} is based on a stack.  Furthermore,  we will
later see how,  using three stacks, arithmetic expressions can be evaluated.



\section[Implementation]{Implementing Abstract Data Types in \textsc{SetlX}}
In an object oriented programming language, abstract data types are conveniently implemented via a
class.  In a typed object oriented programming language like \textsl{Java}, the usual way to proceed
is to create an interface describing the signatures of the abstract data type and then to implement
the abstract data type as a class.  Instead of an interface, we can also use an abstract class
to describe the signatures.  In an untyped language like \textsc{SetlX} there is no way to
neatly capture the signatures.  Therefore, the implementation of an abstract data type in
\textsc{SetlX} merely consists of a class.
At this point we note that classes are discussed in depth in chapter 7 of the \textsc{SetlX} 
\href{http://wwwlehre.dhbw-stuttgart.de/~stroetma/SetlX/tutorial.pdf}{tutorial}.  If the reader
hasn't encountered classes in \textsc{SetlX}, she is advised to consult this chapter before reading
any further. 

\begin{figure}[!h]
  \centering
\begin{Verbatim}[ frame         = lines, 
                  framesep      = 0.3cm, 
                  labelposition = bottomline,
                  numbers       = left,
                  numbersep     = -0.2cm,
                  xleftmargin   = 0.0cm,
                  xrightmargin  = 0.0cm
                ]
    class stack() {
        mStackElements := [];
    
      static {  
        push := procedure(e) {
            this.mStackElements += [e];
        };
        pop := procedure() {
            assert(#mStackElements > 0, "popping empty stack");
            this.mStackElements := mStackElements[1 .. #mStackElements - 1];
        };
        top := procedure() {
            assert(#mStackElements > 0, "top of empty stack");
            return mStackElements[#mStackElements];
        };
        isEmpty := procedure() {
            return mStackElements == [];
        };
        f_str := procedure() {
             copy   := this;
             result := convert(copy);
             dashes := "\n";
             for (i in {1 .. #result}) {
                  dashes += "-";
             }
             return dashes + "\n" + result + dashes + "\n";
        };
        convert := procedure(s) {
            if (s.isEmpty()) {
                return "|";
            } 
            top := s.top();
            s.pop();
            return convert(s) + " " + top + " |";
        };
      }
    }
    
    createStack := procedure(l) {
        result := stack();
        n := #l;
        for (i in [n, n-1 .. 1]) {
            result.push(l[i]);
        }
        return result;
    };
\end{Verbatim}
\vspace*{-0.3cm}
  \caption{An array based implementation of the ADT \textsl{Stack} in \textsc{SetlX}.}
  \label{fig:stack-array.stlx}
\end{figure} 

Figure \ref{fig:stack-array.stlx} shows an implementation of the ADT \textsl{Stack} that is 
discussed next.
\begin{enumerate}
\item The definition of the \textsc{Adt} \textsl{Stack} starts with the keyword \texttt{class}
      in line 1.
      After the keyword \texttt{class}, the name of the class has to be given.  In Figure
      \ref{fig:stack-array.stlx} this name is \texttt{stack}.

      In \textsc{SetlX}, every class also defines a constructor wich has the same name as the class.
      Since in line 1 the name \texttt{class} is followed by ``\texttt{()}'', the constructor
      \texttt{stack} does not take any arguments.  In order to use it, we can write
      \\[0.2cm]
      \hspace*{1.3cm}
      \texttt{s := stack();}
      \\[0.2cm]
      This assignment creates an empty stack and assigns this stack to the variable \texttt{s}.
\item Line 2 defines the first (and in this case only) member variable of the class \texttt{stack}.
      Therefore, every object $o$ of class stack will have a \emph{member variable} called
      \texttt{mStackElements}. 
      We will use this list to store the elements of the stack.  To retrieve this member variable from
      the object $o$ we can use the following expression:
      \\[0.2cm]
      \hspace*{1.3cm}
      $o$\texttt{.mStackElements}
      \\[0.2cm]
      The implementation of stacks shown
      in Figure \ref{fig:stack-array.stlx} is based on storing the elements of the stack in a
      \textsc{SetlX} list.  In \textsc{SetlX}, lists are internally implemented as arrays.  However,
      this is not the only way to implement a stack: A stack can also be implemented as a linked list.

      One word on naming convention.  It is my convention to start all member variables of a class
      with the lower case letter ``\texttt{m}'' followed by a descriptive name.
\item The rest of the definition of the class \texttt{stack} is enclosed in one big static
      block that starts with the keyword \texttt{static} in line 4 and ends with the closing brace 
      ``\texttt{\}}'' in line 36.  The static block is not part of the constructor.
      Therefore, the only thing that the constructor class
      \texttt{stack()} does, is to initialize the member variable \texttt{mStackElements}.

      The static block itself contains a number of procedure definitions.  These procedures are
      called \emph{methods}.  As these methods are
      defined inside the static block, they are not considered to be defined in the constructor
      and, therefore, are not member variables but are, instead, class variables.  Hence, 
      they are available in the class \texttt{stack}.  For example,
      \\[0.2cm]
      \hspace*{1.3cm}
      \texttt{stack.push}
      \\[0.2cm]
      refers to the method \texttt{push} defined in line 5 to 7.  Of course, every object of class
      \texttt{stack} will also have access to these 
      methods.  For example, if $s$ is an object of class \texttt{stack}, then we can invoke the
      method \texttt{push} by writing:
      \\[0.2cm]
      \hspace*{1.3cm}
      $s$\texttt{.push($x$)}
      \\[0.2cm]
      The reason the methods in class \texttt{stack} are all inside the static block is the fact
      that these methods work the same way for all stacks.  The only case where it is not a good 
      idea to declare a method as static is when this method needs to be defined on a
      per-object-basis: If the code of the method differs for every object so that it is necessary
      to construct the method for every object differently, then the definition of the method has to
      be part of the constructor and not part of the static block.
\item Line 5 starts the definition of the method \texttt{push}.  This method is called with one
      argument $e$, where $e$ is the element that is to be pushed on the stack.  In the array based
      implementation, this is achieved by appending $e$ to the list \texttt{mStackElements}.

      If you are new to object-oriented programming, then at this point you might wonder why we do
      not need to specify the stack onto which the element $e$ is pushed.  However, we do have to
      specify the stack by prefixing it to the method invocation.  That is, if $s$ is a stack and we
      want to push $e$ onto this stack, then we can do this by writing the following:
      \\[0.2cm]
      \hspace*{1.3cm}
      $s\mathtt{.push}(e)$

      There is another subtle point that needs to be discussed: When referring to the member
      variable \texttt{mStackElements} we had to prefix this variable with the string
      ``\texttt{this.}''.  In the case that we just want to read a variable, it is not necessary to
      prefix the variable with ``\texttt{this.}''.  However, once we want to change a member
      variable, the prefix ``\texttt{this.}'' is mandatory.  If we just had written
      \\[0.2cm]
      \hspace*{1.3cm}
      \texttt{mStackElements += [e];}
      \\[0.2cm]
      then we would have created a new local variable with the name \texttt{mStackElements} and we
      would have changed this local variable rather than the member variable \texttt{mStackElements}.      
\item Line 8 starts the implementation of the method \texttt{pop}, which has the task to remove 
      one element from the stack.  Of course, it would not make sense to remove an element from the
      stack if the stack is empty.  Therefore, the \texttt{assert} statement in line 9 checks
      whether the number of elements of the list \texttt{mStackElements} is bigger than $0$.
      If this condition is satisfied, the last element of the list \texttt{mStackElements} is removed.
\item Line 12 starts the definition of the method \texttt{top}.  First, it is checked that the stack
      is non-empty.  Then, the element at the end of the list \texttt{mStackElements} is returned.
\item Line 16 defines the method \texttt{isEmpty}.  This method checks whether the list
      \texttt{mStackElements} is empty.
\item Line 19 defines the method \texttt{f\_str}.  This method serves a similar purpose as the method
      \texttt{toString} in a \textsl{Java} program:  If an object of class \texttt{stack} needs to
      be converted into a string, then the method \texttt{f\_str} is invoked automatically to
      perform this conversion.

      In order to understand the implementation of \texttt{f\_str} we look at an example and type
      \\[0.2cm]
      \hspace*{1.3cm}
      \texttt{s := stack(); s.push(1); s.push(2); s.push(3); print(s);}
      \\[0.2cm]
      at the prompt of the interpreter.  These commands create an empty stack and push the numbers 1, 2, and 3 
      onto this stack.  Finally, the resulting stack is printed.  The string that is then printed
      is the result of calling \texttt{f\_str} and has the following form:
      \begin{verbatim}
      -------------
      | 1 | 2 | 3 |
      -------------
      \end{verbatim}
      \vspace*{-0.5cm}
      Hence, the topmost element of the stack is printed last.

      The implementation of the method \texttt{f\_str} works as follows.
      \begin{enumerate}
      \item First, line 20 creates a copy of the stack.  The reason for working with a copy is
            that all the methods that are available on a stack only permit us to look at the 
            topmost element of the stack.  In order to inspect the second element we first have to
            remove the first element.  However, the method \texttt{f\_str} is not supposed to change
            the stack.  Therefore, we have to create a copy of the stack first.
      \item Next, we use the auxiliary method \texttt{convert}.  This method computes a string of
            the form \\[0.2cm]
            \hspace*{1.3cm} \texttt{| 1 | 2 | 3 |}. 
            \\[0.2cm]
            The implementation of \texttt{convert} is done via a case distinction:
            If the given stack s is empty, the result of \texttt{convert} will be the string ``\texttt{|}''.  
            Otherwise we get the top element of the stack using the method \texttt{top()} and remove
            it using \texttt{pop()}.  Next, the remaining stack is converted to a string in
            line 34 and finally the element \texttt{top} is appended to this string.
      \item The method \texttt{f\_str} uses a \texttt{for} loop to create a line of dashes.
            The result of \texttt{convert} is then decorated with these dashes.
      \end{enumerate}
\item Note that there is no ``\texttt{;}'' at the end of the class definition in line 37.
      In contrast to a procedure definition, a class definition must \underline{not} be terminated 
      by the character ``\texttt{;}''.
\end{enumerate}
You should note that we were able to implement  the method \texttt{f\_str} without knowing anything
about the internal representation of the stack.  In order to implement \texttt{f\_str} we only used
the methods \texttt{top}, \texttt{pop}, and \texttt{isEmpty}.  This is one of the main advantages of
an abstract data type: An abstract data type abstracts from the concrete data structures that
implement it.  If an abstract data type is done right, it can be used without knowing how the data
that are administered by the abstract data type are actually represented.


\section{Evaluation of Arithmetic Expressions}
Next, in order to demonstrate the usefulness of stacks, we show how \emph{arithmetic expressions} can be
evaluated using stacks.  An \emph{arithmetic expression} is a string that is made up of numbers and
the operator symbols ``\texttt{+}'', ``\texttt{-}'', ``\texttt{*}'', ``\texttt{/}'',
``\texttt{\symbol{37}}'', and ``\texttt{**}''. Here $x \;\texttt{\symbol{37}}\; y$ denotes the remainder when doing
integer division and
$x\;\texttt{**}\;y$ denotes the power $x^y$.    Furthermore, arithmetic expressions can use
the parentheses ``\texttt{(}'' and ``\texttt{)}''.
  
The set of arithmetic expressions can be defined inductively as
follows:
\begin{enumerate}
\item Every number $n \in \mathbb{N}$ is an arithmetic expression.
\item If $s$ and $t$ are arithmetic expressions, then
      \\[0.2cm]
      \hspace*{1.3cm}
      $s + t$, \quad $s - t$, \quad $s * t$, \quad
      $s \,/\, t$, \quad $s \;\texttt{\symbol{37}}\; t$, \quad and \quad $s \,\texttt{**}\, t$
      \\[0.2cm]
      are arithmetic expressions.
\item If $s$ is an arithmetic expression, then $(s)$ is an arithmetic expression.
\end{enumerate}
If we have been given a string that is an arithmetic expression, then in order to evaluate this
arithmetic expression we need to know the precedence and the associativity of the operators.
In mathematics the operators ``\texttt{*}'', ``\texttt{/}'' and ``\texttt{\symbol{37}}'' have a
higher precedence than the operators ``\texttt{+}'' and ``\texttt{-}''.  Furthermore, the operator
  ``\texttt{**}'' has a precedence that is higher than the precedence
 of any other operators.  The operators
``\texttt{+}'', ``\texttt{-}'', ``\texttt{*}'', ``\texttt{/}'', and ``\texttt{\symbol{37}}''
associate to the left:  An expression of the form 
\\[0.2cm]
\hspace*{1.3cm} 
\texttt{1 - 2 - 3} \quad is interpreted as \quad \texttt{(1 - 2) - 3}.
 \\[0.2cm]
Finally, the operator ``\texttt{**}'' associates to the right:
The arithmetic expression \\[0.2cm]
\hspace*{1.3cm} 
\texttt{2 \texttt{**} 3 \texttt{**}  2} \quad is interpreted as \quad 
\texttt{2 \texttt{**} (3 \texttt{**} 2)}. 
\\[0.2cm]
Our goal is to implement a program that evaluates an arithmetic expression.


\subsection{A First Example}
In order to explain the algorithm that is used to evaluate an arithmetic expression, we present a
simple example first.  Consider the arithmetic expression 
\\[0.2cm]
\hspace*{1.3cm} 
\texttt{1 + 2 * 3 - 4}. 
\\[0.2cm]
This string is processed from left to right, one \emph{token}\footnote{A \emph{token} is either a
  number, an operator symbol, or a parenthesis.}
 at a time.  In order to process the string we use three stacks.
\begin{enumerate}
\item The  \emph{token stack} contains all tokens of the arithmetic expression. 
      The first token of the arithmetic expression is on top of this stack.
\item The  \emph{argument stack} contains numbers.
\item The \emph{operator stack} contains operator symbols and the left parenthesis ``\texttt{(}''.
\end{enumerate}
The evaluation of \texttt{1 + 2 * 3 - 4} proceeds as follows:
\begin{enumerate}
\item In the beginning, the token stack contains the arithmetic expression and the other two stacks
      are empty: \\[0.2cm]
      \hspace*{1.3cm} 
      \texttt{mTokens \ \ \ = [ 4, \symbol{34}-\symbol{34}, 3,
        \symbol{34}*\symbol{34}, 2, \symbol{34}+\symbol{34}, 1 ]}, 
      \\[0.2cm]
      Note that the number that is at the beginning of the arithmetic expression is on top of the
      stack.  \\[0.2cm]
      \hspace*{1.3cm} \texttt{mArguments = []}, \\[0.2cm]
      \hspace*{1.3cm} \texttt{mOperators = []}. 
\item The number \texttt{1} is removed from the token stack and is put onto the argument stack
      instead.  The three stacks are now as follows: \\[0.2cm]
      \hspace*{1.3cm} \texttt{mTokens \ \ \ = [ 4, \symbol{34}-\symbol{34}, 3, \symbol{34}*\symbol{34}, 2, \symbol{34}+\symbol{34} ]}, \\[0.2cm]
      \hspace*{1.3cm} \texttt{mArguments = [ 1 ]}, \\[0.2cm]
      \hspace*{1.3cm} \texttt{mOperators = []}. 
\item Next, the operator \texttt{\symbol{34}+\symbol{34}} is removed from the token stack and is put
      onto the operator stack.  Then we have: \\[0.2cm]
      \hspace*{1.3cm} \texttt{mTokens \ \ \ = [ 4, \symbol{34}-\symbol{34}, 3, \symbol{34}*\symbol{34}, 2 ]}, \\[0.2cm]
      \hspace*{1.3cm} \texttt{mArguments = [ 1 ]} \\[0.2cm]
      \hspace*{1.3cm} \texttt{mOperators = [ \symbol{34}+\symbol{34} ]}. 
\item Now, we remove the number \texttt{2} from the  token stack and put it onto the argument stack.
      We have: \\[0.2cm]
      \hspace*{1.3cm} \texttt{mTokens \ \ \ = [ 4, \symbol{34}-\symbol{34}, 3, \symbol{34}*\symbol{34} ]}, \\[0.2cm]
      \hspace*{1.3cm} \texttt{mArguments = [ 1, 2 ]}, \\[0.2cm]
      \hspace*{1.3cm} \texttt{mOperators = [ \symbol{34}+\symbol{34} ]}. 
\item We remove the operator \texttt{\symbol{34}*\symbol{34}} from the  token stack and compare the
      precedence of the operator with the precedence of the operator \texttt{\symbol{34}+\symbol{34}},
      which is on top of the operator stack.  Since the precedence of the operator 
      \texttt{\symbol{34}*\symbol{34}} is greater than the precedence of the operator 
      \texttt{\symbol{34}+\symbol{34}}, the operator \texttt{\symbol{34}*\symbol{34}} is put onto
      the operator stack.  The reason is that we have to evaluate this operator before we can
      evaluate the operator \texttt{\symbol{34}+\symbol{34}}.  Then we have: 
      \\[0.2cm]
      \hspace*{1.3cm} \texttt{mTokens \ \ \ = [ 4, \symbol{34}-\symbol{34}, 3 ]}, \\[0.2cm]
      \hspace*{1.3cm} \texttt{mArguments = [ 1, 2 ]}, \\[0.2cm]
      \hspace*{1.3cm} \texttt{mOperators = [ \symbol{34}+\symbol{34}, \symbol{34}*\symbol{34}]}. 
\item We remove the number \texttt{3} from the  token stack and put it onto the argument stack.
      \\[0.2cm]
      \hspace*{1.3cm} \texttt{mTokens \ \ \ = [ 4, \symbol{34}-\symbol{34} ]}, \\[0.2cm]
      \hspace*{1.3cm} \texttt{mArguments = [ 1, 2, 3 ]}, \\[0.2cm]
      \hspace*{1.3cm} \texttt{mOperators = [ \symbol{34}+\symbol{34}, \symbol{34}*\symbol{34} ]}. 
\item We remove the operator \texttt{\symbol{34}-\symbol{34}} from the token stack and
      compare this operator with the operator \texttt{\symbol{34}*\symbol{34}}, which is on top of
      the operator stack.  As the precedence of the  operator \texttt{\symbol{34}*\symbol{34}} is
      higher as the precedence of the operator \texttt{\symbol{34}-\symbol{34}},
      we have to evaluate the operator \texttt{\symbol{34}*\symbol{34}}.  In order to do so, we
      remove the arguments 3 and 2 from the argument stack, remove the operator
      \texttt{\symbol{34}*\symbol{34}} from the operator stack and compute the product of the two
      arguments.  This product is then put back on the 
      argument stack.  The operator \texttt{\symbol{34}-\symbol{34}} is put back on the token stack
      since it has not been used.  Hence, the stacks look as shown below: \\[0.2cm]
      \hspace*{1.3cm} \texttt{mTokens \ \ \ = [ 4, \symbol{34}-\symbol{34} ]}, \\[0.2cm]
      \hspace*{1.3cm} \texttt{mArguments = [ 1, 6 ]}, \\[0.2cm]
      \hspace*{1.3cm} \texttt{mOperators = [ \symbol{34}+\symbol{34} ]}. 
\item Again, we take the operator \texttt{\symbol{34}-\symbol{34}} from the token stack and
      compare it with the operator \texttt{\symbol{34}+\symbol{34}} that is now on top of the
      operator stack.  Since both operators have the same precedence, the operator
      \texttt{\symbol{34}+\symbol{34}} is evaluated:  We remove two arguments from the argument
      stack, remove the operator
      \texttt{\symbol{34}+\symbol{34}} from the operator stack  and compute the sum of the
      arguments.  The result is put back on the argument stack.  Furthermore, the operator
      \texttt{\symbol{34}-\symbol{34}} is put back on the token stack.
      Then we have: \\[0.2cm]
      \hspace*{1.3cm} \texttt{mTokens \ \ \ = [ 4, \symbol{34}-\symbol{34} ]}, \\[0.2cm]
      \hspace*{1.3cm} \texttt{mArguments = [ 7 ]}, \\[0.2cm]
      \hspace*{1.3cm} \texttt{mOperators = []}. 
\item Next, the operator \texttt{\symbol{34}-\symbol{34}} is removed from the token stack and is now
      put on the operator stack.  We have: \\[0.2cm]
      \hspace*{1.3cm} \texttt{mTokens \ \ \ = [ 4 ]}, \\[0.2cm]
      \hspace*{1.3cm} \texttt{mArguments = [ 7 ]}, \\[0.2cm]
      \hspace*{1.3cm} \texttt{mOperators = [ \symbol{34}-\symbol{34} ]}. 
\item The number \texttt{4} is removed from the token stack and put onto the argument stack. We have: \\[0.2cm]
      \hspace*{1.3cm} \texttt{mTokens \ \ \ = []}, \\[0.2cm]
      \hspace*{1.3cm} \texttt{mArguments = [ 7, 4 ]}, \\[0.2cm]
      \hspace*{1.3cm} \texttt{mOperators = [ \symbol{34}-\symbol{34} ]}. 
\item Now the input has been consumed completely.
      Hence, the operator \texttt{\symbol{34}-\symbol{34}} is removed from the  operator stack and
      furthermore, the arguments of this operator are removed from the argument stack.  Then, the
      operator \texttt{\symbol{34}-\symbol{34}} is evaluated and the result is put onto the argument
      stack.  We have: \\[0.2cm]
      \hspace*{1.3cm} \texttt{mTokens \ \ \ = []}, \\[0.2cm]
      \hspace*{1.3cm} \texttt{mArguments = [ 3 ]}, \\[0.2cm]
      \hspace*{1.3cm} \texttt{mOperators = []}. \\[0.2cm]
      Therefore, the result of evaluating the arithmetic expression ``\texttt{1+2*3-4}'' is the
      number 3.
\end{enumerate}

\subsection{The Shunting-Yard-Algorithm \label{algo-arith}}
The algorithm introduced in the last example is known as the 
\href{http://en.wikipedia.org/wiki/Shunting-yard_algorithm}{\emph{shunting-yard-algorithm}}.
It was discovered by  \href{http://en.wikipedia.org/wiki/Edsger_Dijkstra}{Edsger Dijkstra} in 1961.
We give a detailed presentation of this algorithm next.  To begin with, we fix the data structures
that are needed for this algorithm.
\begin{enumerate}
\item \texttt{mTokens} is a stack of input tokens.  The operator symbols and parentheses are
      represented as strings, while the numbers are represented as rational numbers.
\item \texttt{mArguments} is a stack of rational numbers.  
\item \texttt{mOperators} is the operator stack.
\end{enumerate}
Considering the previous example we realize that the numbers are always put onto the argument stack,
while there are two cases for operator symbols that are removed from the token stack:
\begin{enumerate}
\item We have to put the operator onto the operator stack in all of the following cases:
      \begin{enumerate}
      \item The operator stack is empty.
      \item The operator on top of the operator stack is an opening parenthesis \texttt{\symbol{34}(\symbol{34}}. 
      \item The operator has a higher precedence than the operator that is currently on top of the
            operator stack.
      \item The operator is the same as the operator that is on top of the operator stack and, furthermore,
            the operator associates to the right.
      \end{enumerate}
\item In all other cases, the operator that has been taken from the token stack is put back onto the
      token stack.  In this case, the operator on top of the operator stack is removed from the
      operator stack and, furthermore, the arguments of this operator are removed from the argument
      stack.  Next, this operator is evaluated and the resulting number is put onto the argument stack.
\end{enumerate}


\begin{figure}[!b]
  \centering
\begin{Verbatim}[ frame         = lines, 
                  framesep      = 0.3cm, 
                  labelposition = bottomline,
                  numbers       = left,
                  numbersep     = -0.2cm,
                  xleftmargin   = 0.0cm,
                  xrightmargin  = 0.0cm,
                ]
    class calculator(s) {
        mTokenStack := createStack(extractTokens(s));
        mArguments  := stack();
        mOperators  := stack();
    
      static {
        evaluate := procedure() {
            while (!mTokenStack.isEmpty()) {
                if (isInteger(mTokenStack.top())) {
                    number := mTokenStack.top(); mTokenStack.pop();
                    mArguments.push(number);
                    continue;
                } 
                nextOp := mTokenStack.top(); mTokenStack.pop();
                if (mOperators.isEmpty() || nextOp == "(") {
                    mOperators.push(nextOp);
                    continue;
                }
                stackOp := mOperators.top();
                if (stackOp == "(" && nextOp == ")") {
                    mOperators.pop();
                } else if (nextOp == ")") {
                    popAndEvaluate();
                    mTokenStack.push(nextOp);
                } else if (evalBefore(stackOp, nextOp)) {
                    popAndEvaluate();
                    mTokenStack.push(nextOp);
                } else {
                    mOperators.push(nextOp);
                }
            }
            while (!mOperators.isEmpty()) { popAndEvaluate(); }
            return mArguments.top();
        };
        evalBefore := procedure(stackOp, nextOp) {
            prec := {["+",1],["-",1],["*",2],["/",2],["%",2],["**",3]};
            if (stackOp == "(") { return false;  }
            if (prec[stackOp] > prec[nextOp]) {
                return true;
            } else if (prec[stackOp] == prec[nextOp]) {
                if (stackOp == nextOp) {
                    return stackOp in { "+", "-", "*", "/", "%" };
                }
                return true;
            } 
            return false;
        };
\end{Verbatim}
\vspace*{-0.3cm}
  \caption{The class \texttt{calculator}, part 1.}
  \label{fig:calculator.setlx-1}
\end{figure} 

\begin{figure}[!htb]
  \centering
\begin{Verbatim}[ frame         = lines, 
                  framesep      = 0.3cm, 
                  labelposition = bottomline,
                  firstnumber   = last,
                  numbers       = left,
                  numbersep     = -0.2cm,
                  xleftmargin   = 0.8cm,
                  xrightmargin  = 0.8cm,
                ]
        popAndEvaluate := procedure() {
            rhs := mArguments.top(); mArguments.pop();
            lhs := mArguments.top(); mArguments.pop();
            op  := mOperators.top(); mOperators.pop();
            match (op) {
                case "+" : result := lhs + rhs;
                case "-" : result := lhs - rhs;
                case "*" : result := lhs * rhs;
                case "/" : result := lhs / rhs;
                case "%" : result := lhs % rhs;
                case "**": result := lhs ** rhs;       
                default: abort("ERROR: *** Unknown Operator *** $op$");
            }
            mArguments.push(result);
        };    
      }
    }
    
    c := calculator("1+2*3**4-5/2");
    c.evaluate();
\end{Verbatim}
\vspace*{-0.3cm}
  \caption{The class \texttt{calculator}, part 2.}
  \label{fig:calculator.setlx-2}
\end{figure} 

\noindent
An implementation of this algorithm in \textsc{SetlX} is shown in the Figures
\ref{fig:calculator.setlx-1} and \ref{fig:calculator.setlx-2} on the following pages.
We start our discussion of the class \texttt{calculator} by inspecting the method
\texttt{evalBefore} that is defined in line 35.
This method takes two operators \texttt{stackOp} and \texttt{nextOp} and decides
whether \texttt{stackOp} should be evaluated before \texttt{nextOp}.  Of course, \texttt{stackOp} is
intended to be the operator on top of the operator stack, while \texttt{nextOp} is an operator that
is on top of the token stack.
In order to decide whether the operator \texttt{stackOp} should be evaluated before the operator
\texttt{nextOp},  we first have to know the \emph{precedences} of these operators.  Here, a
\emph{precedence} is a natural number that specifies how strong the operator binds to its
arguments. Table \ref{tab:predence} on page \pageref{tab:predence} lists the precedences of our
operators.  This table is coded as the binary relation \texttt{prec} in line 36.

\begin{table}[!h]
  \centering
\framebox{
  \begin{tabular}{|r|r|}
\hline
   Operator             & Precedence  \\
\hline
\hline
   \texttt{\symbol{34}+\symbol{34}}, \texttt{\symbol{34}-\symbol{34}}  & 1   \\
\hline
   \texttt{\symbol{34}*\symbol{34}}, \texttt{\symbol{34}/\symbol{34}}, \texttt{\symbol{34}\symbol{37}\symbol{34}}  & 2   \\
\hline
   \texttt{\symbol{34}**\symbol{34}}  & 3   \\
\hline
  \end{tabular}}
  \caption{Precedences of the operators.}
  \label{tab:predence}
\end{table}

If the precedence of \texttt{stackOp} is bigger than the precedence of \texttt{nextOp}, then we have
to evaluate \texttt{stackOp} before we evaluate \texttt{nextOp}.  On the other hand, if the precedence
of \texttt{stackOp} is smaller than the precedence of \texttt{nextOp}, then we have to push
\texttt{nextOp} onto the operator stack as we have to evaluate this operator before we evaluate \texttt{stackOp}.
If \texttt{stackOp} and \texttt{nextOp} have the same precedence, there are two cases:
\begin{enumerate}
\item $\mathtt{stackOp} \not= \mathtt{nextOp}$.

      Let us consider an example:  The arithmetic expression 
      \\[0.2cm]
      \hspace*{1.3cm} 
      \texttt{2 + 3 - 4} \quad is processed as \quad \texttt{(2 + 3) - 4}. 
      \\[0.2cm]
      Therefore, in this case we have to evaluate \texttt{stackOp} first.
\item $\mathtt{op1} = \mathtt{op2}$.

      In this case we have to consider the \emph{associativity} of the operator.
      Let us consider two examples: \\[0.2cm]
      \hspace*{1.3cm} \texttt{2 + 3 + 4} \quad is interpreted as \quad \texttt{(2 + 3) + 4}. \\[0.2cm]
      The reason is that the operator ``\texttt{+}'' \emph{associates to the left}.
      On the other hand, \\[0.2cm]
      \hspace*{1.3cm} 
      \texttt{2 \texttt{**} 3 \texttt{**} 4}  \quad is interpreted as \quad \texttt{2 \texttt{**} (3 \texttt{**} 4)}
      \\[0.2cm]
      because the operator ``\texttt{**}'' \emph{associates to the right}.

      The operators ``\texttt{+}'', ``\texttt{-}'', ``\texttt{*}'', ``\texttt{/}'' and 
      ``\texttt{\symbol{37}}'' are all left associative.  Hence, in this case \texttt{stackOp} is
      evaluated before \texttt{nextOp}.
      The operator ``\texttt{**}'' associates to the right. Therefore, if
      the operator on top of the operator stack is the operator ``\texttt{**}'' and then this operator
      is read again, then we have to push the  operator ``\texttt{**}'' on the operator stack.
\end{enumerate}
Now we can understand the implementation of $\texttt{evalBefore}(\mathtt{stackOp},\mathtt{nextOp})$.
\begin{enumerate}
\item If \texttt{stackOp} is the opening parenthesis \texttt{\symbol{34}(\symbol{34}}, we have to put
      \texttt{nextOp} onto the operator stack.  The reason is that \texttt{\symbol{34}(\symbol{34}}
      is no operator that can be evaluated.  Hence, we return \texttt{false} in line 37.
\item If the precedence of \texttt{stackOp} is higher than the precedence of \texttt{nextOp}, we
      return \texttt{true} in line 39.
\item If the precedences of \texttt{stackOp} and \texttt{nextOp} are identical, there are two cases:
      \begin{enumerate}
      \item If both operators are equal, then the result of
            \texttt{evalBefore(stackOp,nextOp)} is \texttt{true} if and only if
            this operator associates to the left.  The operators that associate to the left are
            listed in the set in line 42.
      \item Otherwise, if \texttt{stackOp} is different from \texttt{nextOp}, then \\
            \texttt{evalBefore(stackOp,nextOp)} returns \texttt{true}. 
      \end{enumerate}
\item If the precedence of \texttt{stackOp} is less than the precedence of \texttt{nextOp}, then 
      $\texttt{evalBefore(stackOp,nextOp})$ returns \texttt{false}.
\end{enumerate}
Figure  \ref{fig:calculator.setlx-2} on page \pageref{fig:calculator.setlx-2} shows the
implementation of the method \texttt{popAndEvaluate}.
This method works as follows: 
\begin{enumerate}
\item It takes an operator from the operator stack (line 51), 
\item it fetches the arguments of this operator from the argument stack (line 49 and line 50),
\item it evaluates the operator,  and
\item finally puts the result back on top of the argument stack.
\end{enumerate}
Finally, we are ready to discuss the implementation of the method \texttt{evaluate} in line 7 of
Figure \ref{fig:calculator.setlx-1}.  

\begin{enumerate}
\item First, as long as the token stack is non-empty we take a token from the token stack.
\item If this token is a number, then we put it on the argument stack and continue to read the next token.
  
      In the following code of the \texttt{while} loop that starts at line 14, we can assume that
      the last token that has been read is either an operator symbol
      or one of the parentheses \texttt{\symbol{34}(\symbol{34}} or \texttt{\symbol{34})\symbol{34}}. 
\item If the operator stack is empty or if the token that has been read is an opening parenthesis
      \texttt{\symbol{34}(\symbol{34}}, the operator or parenthesis is pushed onto the operator stack.
\item If the  token that has been read as \texttt{nextOp} is a closing parenthesis 
      \texttt{\symbol{34})\symbol{34}} and, furthermore, the operator on top of the operator stack is
      an opening parenthesis \texttt{\symbol{34}(\symbol{34}}, then this parenthesis is removed from the
      operator stack.
\item If now in line 22 the token \texttt{nextOp} is a closing parenthesis
      \texttt{\symbol{34})\symbol{34}}, then we know that the token on the operator stack
      can't be an opening parenthesis but rather has to be an operator.
      This operator is then evaluated using the method \texttt{popAndEvaluate()}.
      Furthermore, the closing parenthesis \texttt{nextOp} is pushed back onto the
      token stack as we have not yet found the matching open parenthesis.

      After pushing the closing parenthesis back onto the token stack, we return to the beginning of
      the \texttt{while} loop in line 8.  Hence, in this case we keep evaluating operators
      on the operator stack until we hit an opening parenthesis on the operator stack.

      In the part of the \texttt{while} loop following line 24 we may assume that 
      \texttt{nextOp} is not a parenthesis, since the other case  has been dealt with.

\item If the operator \texttt{stackOp} on top of the operator stack needs to be evaluated before the operator
      \texttt{nextOp}, we evaluate \texttt{stackOp} using the method \texttt{popAndEvaluate()}.
      Furthermore, the operator \texttt{nextOp} is put back on the token stack 
      as it has not been consumed.
\item Otherwise, \texttt{nextOp} is put on the operator stack.

      The \texttt{while} loop ends when the token stack gets empty.
\item Finally, the operators remaining on the operator stack are evaluated using
      \texttt{popAndEvaluate}.  If the input has been a syntactically correct arithmetic expression,
      then at the end of the computation there should be one number left on the argument stack.
      This number is the result of the evaluation and hence it is returned.
\end{enumerate}


\section[Benefits]{Benefits of Using Abstract Data Types}
We finish this chapter with a short discussion of the benefits of abstract data types.
 \begin{enumerate}
 \item The use of abstract data types separates an algorithm from the data structures that
       are used to implement this algorithm.

       When we implemented the algorithm to evaluate arithmetic expressions we did not need to know
       how the data type \emph{stack} that we used was implemented.  It was sufficient for us to know 
       \begin{enumerate}
       \item the signatures of its functions and
       \item the axioms describing the behaviour of these functions.
       \end{enumerate}
       Therefore, an abstract data type can be seen as an interface that shields the user of the
       abstract data type from the peculiarities of an actual implementation of the data type.
       Hence it is possible that different groups of people develop the algorithm and the
       concrete implementation of the abstract data types used by the algorithm.  

       Today, many software systems have sizes that can only be described as gigantic.  No single
       person is able to understand every single aspect of these systems.  It is therefore important
       that these systems are structured in a way such that different groups of developers can work
       simultaneously on these systems without interfering with the work done by other groups.
 \item Abstract data types are \emph{reusable}.

       Our definition of stacks was very general.  Therefore, stacks can be used in many different
       places:  For example, we will see later how stacks can be used to traverse a directed graph.

       Modern industrial strength programming languages like \texttt{C++} or \textsl{Java} contain
       huge libraries containing the implementation of many abstract data types.  This fact reduces
       the cost of software development substantially.     
 \item Abstract data types are \emph{exchangeable}.

       In our program for evaluating  arithmetic expressions it is trivial to substitute the given
       array based implementation of stacks with an implementation that is based on linked list.  In general,
       this enables the following methodology for developing software:  
       \begin{enumerate}
       \item First, an algorithm is implemented using abstract data types.
       \item The initial implementation of these abstract data may be quite crude and inefficient.
       \item Next, detailed performance tests spot those data types that are performance 
             bottlenecks.
       \item Finally, the implementations of data types that have been identified as bottlenecks are optimized.
       \end{enumerate}
       The reason this approach works is the 
       \href{http://en.wikipedia.org/wiki/Pareto_principle#In_software}{80-20 rule}:  
       80 percent of the running time of most programs is spent in 20 percent of the code.  It is
       therefore sufficient to optimize the 
       implementation of those data structures that really are performance bottlenecks.  If,
       instead, we would try to optimize everything we would only achieve the following:
       \begin{enumerate}
       \item We would waste of time.  There is no point optimizing some function to make it 10 times
             faster if the program spends less than a millisecond in this function anyway but the
             overall running time is several minutes.
       \item The resulting program would be considerably bigger and therefore more difficult to 
             maintain.
       \end{enumerate}
 \end{enumerate}




%%% Local Variables: 
%%% mode: latex
%%% TeX-master: "algorithms"
%%% End: 
