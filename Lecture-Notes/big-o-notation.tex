\chapter{Big $\mathcal{O}$ Notation} 
This chapter introduces both the
\href{http://en.wikipedia.org/wiki/O_notation}{\emph{big $O$ notation}} and the
\emph{tilde notation} advocated by Sedgewick \cite{sedgewick:11}.
These two notions are needed to analyse the running time of algorithms.  In order to illustrate
the application of these notions, we show how to implement the computation of powers efficiently,
i.e.~we discuss how to evaluate the expression $a^b$ for given $a,b \in \mathbb{N}$ in a way that
is significantly faster than the naive approach. 

\section{Motivation}
Sometimes it is necessary to have a precise understanding of the complexity of an algorithm.  
In order to obtain this understanding we could proceed as follows:  
\begin{enumerate}
\item We implement the algorithm in a given programming language.
\item We count how many additions, multiplications, assignments, etc.~are needed
      for an input of a given length.
\item We read the processor handbook to look up the amount of time that is needed for the different operations.
\item Using the information discovered in the previous two steps we can then predict the running
      time of our algorithm for given input.
\end{enumerate}
This approach is problematic for a number of reasons.
\begin{enumerate}
\item It is very complicated.
\item The execution time of the basic operations is highly dependent on the memory hierarchy of the
      computer system:  For many modern computer architectures, adding two numbers that happen to be
      in a register is more than ten times faster than adding two numbers that reside in main memory.
      Unless we peek into the machine code generated by our compiler, it is very difficult
      to predict whether a variable will be stored in memory or in a register.  Even if a variable
      is stored in main memory, we still might get lucky if the variable is also stored in a cache.
\item If we would later code the algorithm in a different programming language or if we would port
      the program to a computer with a different processor we would have to redo most of the
      computation. 
\end{enumerate}
The final reason shows that the approach sketched above is not well suited to measure the complexity of
an algorithm: After all, the notion of an algorithm is more abstract than the notion of a program
and we really need a notion measuring the complexity of an algorithm that is more abstract than the
notion of the running time of a program.  This notion of complexity should satisfy the following
specification: 
\begin{itemize}
\item The notion of complexity should abstract from constant factors.  After all, according to 
      \href{http://en.wikipedia.org/wiki/Moore's_law}{\emph{Moore's law}}, 
      computers hitting the market 18 month from now will be about twice as powerful as today's computers.

\item The notion should abstract from \emph{insignificant terms}.

      Assume you have written a program that  multiplies two $n \times n$ matrices.  Assume,
      furthermore, that you have computed the running time $T(n)$ of this program as a function 
      of the size $n$ of the matrix as
      \\[0.2cm]
      \hspace*{1.3cm} $T(n) = 3 \cdot n^3 + 2 \cdot n^2 + 7$. 
      \\[0.2cm]
      When compared with the total running time, the portion of running time that is due to the term 
      $2\cdot n^2 + 7$ will decrease with increasing value of $n$.  To see this, consider the
      following table:
      \\[0.3cm]
      \hspace*{1.3cm} 
      \begin{tabular}{|r|r|}
        \hline
        $n$  & \rule{0pt}{16pt} $\bruch{2 \cdot n^2 + 7}{3 \cdot n^3 + 2 \cdot n^2 + 7}$ \\[0.3cm]
        \hline
        \hline
        1       &  0.75000000000000  \\
        10      &  0.06454630495800  \\
        100     &  0.00662481908150  \\
        1000    &  0.00066622484855  \\
        10\,000 &  6.6662224852\,e\,-05  \\
       \hline
      \end{tabular}

      This table clearly shows that, for large values of $n$, the term $2 \cdot n^2 + 7$ can be
      neglected. 
\item The notion of complexity should describe how the running time increases
      when the size of the input increases:  For small inputs, the running time is not very
      important but the question is how the running time grows when the size of the input is increased. 
      Therefore the notion of complexity should capture the \emph{growth} of the running time.
\end{itemize}
Let us denote the set of all positive real numbers as $\R_+$, i.e.~let us define 
\\[0.2cm]
\hspace*{1.3cm}
$\R_+ := \{ x \in \R \mid x > 0 \}$. 
\\[0.2cm]
Furthermore, the set of all functions defined on  $\N$ yielding a positive real number is defined
as: 
\\[0.2cm]
\hspace*{1.3cm} 
$\R_+^{\;\N} = \bigl\{ f \mid \mbox{$f$ is a function of the form $f: \N \rightarrow \R_+$} \}$.

\begin{Definition}[$\Oh(g)$] 
Assume $g \in \R_+^{\N}$ is given.   Let us define the set of all functions that \emph{grow at most as fast}
  as the function $g$ as follows:
  \\[0.2cm]
  \hspace*{0.5cm} 
  \colorbox{orange}{
  $ \Oh(g) \;:=\; \left\{ f \in \R_+^{\;\N} \mid \exists k \in \N \colon 
    \exists c \in \R_+\colon \forall n \in \N \colon \bigl(n \geq k \rightarrow f(n) \leq c \cdot g(n)\bigr) \right\}$.}
  \eox
\end{Definition}

The definition of $\Oh(g)$ contains three nested quantifiers and may be difficult to understand
when first encountered.  Therefore, let us analyse this definition carefully.  Consider a given
function $g$ and try to understand what $f \in \Oh(g)$ means.  Informally, this means that the
function $f$ does not grow faster than the function $g$. 
\begin{enumerate}
\item The fact that $f \in \Oh(g)$ holds does not impose any restriction on small values of $n$.
      After all, the condition
      \\[0.2cm]
      \hspace*{1.3cm}
      $f(n) \leq c \cdot g(n)$
      \\[0.2cm]
      is only required for those values of $n$ that are bigger than or equal to $k$ and the value
      $k$ can be \underline{an}y\hspace*{-0.1cm}\underline{\ } suitable natural number.

      This property shows that the big $\Oh$ notation captures the \colorbox{orange}{growth rate} of functions.
\item Furthermore, $f(n)$ can be bigger than $g(n)$ even for arbitrary values of $n$ 
      but it can only be bigger by a constant factor:  There must be some fixed constant $c$
      such that 
      \\[0.2cm]
      \hspace*{1.3cm}
      $f(n) \leq c \cdot g(n)$
      \\[0.2cm]
      holds for all values of $n$ that are sufficiently big.  This implies that if $f \in \Oh(g)$
      holds then, for example, the function $2 \cdot f$ will also be in $\Oh(g)$.

      This last property shows that the big $\Oh$ notation \colorbox{orange}{abstracts from constant factors}.
\end{enumerate}
I have borrowed Figure \ref{fig:big-o.eps} from the \href{http://www.wikipedia.org}{Wikipedia} article on 
\href{http://en.wikipedia.org/wiki/Asymptotic_notation}{asymptotic notation}.  It shows two functions 
$f(x)$ and $c \cdot g(x)$ such that $f \in \Oh(g)$.  Note that the function $f(x)$, which is drawn
in red, is less or equal than $c \cdot g(x)$ for all values of $x$ such that $x \geq k$.  In the
figure, we have $k=5$, since the condition $f(x) \leq g(x)$ is satisfied for $x \geq 5$. For
values of $x$ that are less than $k = 5$, sometimes $f(x)$ is bigger than $c \cdot g(x)$ but that does
not matter.  In Figure \ref{fig:big-o.eps} the functions $f(x)$ and $g(x)$ are drawn as if they were
functions defined for all positive real numbers.  However, this is only done to support the
visualization of these functions.  In reality, the functions $f$ and $g$ are only defined for
natural numbers.


\begin{figure}[!ht]
  \centering
  \framebox{\epsfig{file=Abbildungen/big-o.eps, scale=0.50}} 
  \caption{Example for $f \in \Oh(g)$.}
  \label{fig:big-o.eps}
\end{figure}


\noindent
We discuss some concrete examples in order to further clarify the notion $f \in \Oh(g)$.

\example
We claim that the following holds:
\\[0.2cm]
\hspace*{1.3cm}
$3 \cdot n^3 + 2 \cdot n^2 + 7 \in \Oh(n^3)$. 
\\[0.2cm]
\textbf{Proof}:  We have to  provide a constant $c$ and another constant $k$ such that for all $n\in
\N$ satisfying
$n \geq k$ the inequality
\\[0.2cm]
\hspace*{1.3cm} 
$3 \cdot n^3 + 2 \cdot n^2 + 7 \leq c \cdot n^3$
\\[0.2cm]
holds.  Let us define  $k := 1$ and $c := 12$.  Then we may assume that 
\begin{equation}
  \label{eq:u1}
  1\leq n  
\end{equation}
holds and we have to show that this implies 
\begin{equation}
  \label{eq:u2}
  3 \cdot n^3 + 2 \cdot n^2 + 7 \leq 12 \cdot n^3.
\end{equation}
If we take the third power of both sides of the inequality (\ref{eq:u1}) then we see that
\begin{equation}
  \label{eq:u3pre}
  1 \leq n^3  
\end{equation}
holds.  Let us multiply both sides of this inequality with $7$.  We get: 
\begin{equation}
  \label{eq:u3}
  7 \leq 7 \cdot n^3
\end{equation}
Furthermore, let us multiply the inequality (\ref{eq:u1}) with the term $2\cdot n^2$.  This yields
\begin{equation}
  \label{eq:u4}
  2 \cdot n^2 \leq 2 \cdot n^3  
\end{equation}
Finally, we obviously have
\begin{equation}
  \label{eq:u5}
  3 \cdot n^3 \leq 3 \cdot n^3
\end{equation}
Adding up the inequalities (\ref{eq:u3}), (\ref{eq:u4}), and (\ref{eq:u5}) gives \\[0.2cm]
\hspace*{1.3cm} $3 \cdot n^3 + 2 \cdot n^2 + 7 \leq 12 \cdot n^3$ \\[0.2cm]
and therefore the proof is complete. \qed

\example
We have  $n \in \Oh(2^n)$. 

\proof
We have to provide a constant $c$ and a constant $k$
such that 
\\[0.2cm]
\hspace*{1.3cm}
$ n \leq c \cdot 2^n$ 
\\[0.2cm]
holds for all $n \geq k$.  Let us define $k := 0$ and $c := 1$.  We will then have to show that \\[0.2cm]
\hspace*{1.3cm} $n \leq 2^n$ \quad holds for all $n \in \N$.
\\[0.2cm]
We prove this claim by induction on $n$.
\begin{enumerate}
\item \textbf{Base case}: $n = 0$

      Obviously, $n = 0 \leq 1 = 2^0 = 2^n$ holds.  
\item \textbf{Induction step}: $n \mapsto n + 1$

      By the induction hypothesis we have 
      \\[0.2cm]
      \hspace*{1.3cm}
      $n \leq 2^n$.    
      \\[0.2cm]
      Furthermore, a trivial induction shows that
      \\[0.2cm]
      \hspace*{1.3cm}
      $1 \leq 2^n$.
      \\[0.2cm]
      Adding these two inequalities yields
      \\[0.2cm]
      \hspace*{1.3cm} $n+1 \leq 2^n + 2^n = 2^{n+1}$. \qed
\end{enumerate}

\exercise
\begin{enumerate}[(a)]
\item Prove that \\[0.2cm]
      \hspace*{1.3cm}
      $n^2 \in \Oh(2^n)$. 
\item Prove that \\[0.2cm]
      \hspace*{1.3cm}
      $n^3 \in \Oh(2^n)$. 
\item Prove that for every $\alpha \in \mathbb{N}$ we have  \\[0.2cm]
      \hspace*{1.3cm}
      $n^\alpha \in \Oh(2^n)$. 
      
      \hint
      \begin{enumerate}[1.]
      \item Try to prove the following claim by induction on $\alpha$:  For every $\alpha \in \mathbb{N}$
            there exists a number $c(\alpha)$ such that
            \\[0.2cm]
            \hspace*{1.3cm}
            $n^\alpha \leq c(\alpha) \cdot 2^n$ \quad holds for all $n \in \mathbb{N}$.
            \\[0.2cm]
            In the induction step you will have to prove that there is a number $c(\alpha+1)$ such that
            \\[0.2cm]
            \hspace*{1.3cm}
            $n^{\alpha+1} \leq c(\alpha+1) \cdot 2^n$ \hspace*{\fill} $(*)$
            \\[0.2cm]
            holds.  When proving this claim you may assume by induction hypothesis that for all 
            $k \leq \alpha+1$ we have numbers $c(k)$ such that
            \\[0.2cm]
            \hspace*{1.3cm}
            $n^{k} \leq c(k) \cdot 2^n$ \quad holds for all $n \in \mathbb{N}$.
            \\[0.2cm]
            You can prove the claim $(*)$ via a side induction on $n$.
      \item The \href{https://en.wikipedia.org/wiki/Binomial_theorem}{\emph{binomial theorem}} tells
            us that for all $n \in \mathbb{N}$ and all $a,b \in \mathbb{R}$ the equation
            \\[0.2cm]
            \hspace*{1.3cm}
            $\ds (a+b)^n = \sum\limits_{k=0}^n {n \choose k} \cdot a^k \cdot b^{n-k}$
            \\[0.2cm]
            holds.  Here, the expression ${n \choose k}$ is read as \emph{$n$ choose $k$} and is
            defined for all $n \in \mathbb{N}$ and all $k \in \{0, 1, 2, \cdots, n\}$ via
            \\[0.2cm]
            \hspace*{1.3cm}
            $\ds {n \choose k} = \frac{n!}{k! \cdot (n-k)!}$.
            \\[0.2cm]
            The binomial theorem should be used to expand the term $(n+1)^{\alpha + 1}$ that occurs
            in the induction step of the side induction.  \eox
      \end{enumerate}
\end{enumerate}



It would be very tedious if we would have to use induction every time we need to prove that 
$f \in \Oh(g)$ holds for some functions $f$ and $g$.   Therefore, we show a number of properties of
the big $\Oh$ notation next.  These properties will later enable us to prove a claim of the form 
$f \in \Oh(g)$ much quicker than by induction.



\begin{Proposition}[Reflexivity]
  For all functions $f\colon \N \rightarrow {\R_+}$ we have that
  \\[0.2cm]
  \hspace*{1.3cm} $f \in \Oh(f)$ \quad holds. 
\end{Proposition}

\proof
Let us define $k:=0$ and $c:=1$.  Then our claim  follows immediately from the inequality 
\\[0.2cm]
\hspace*{1.3cm}
$\forall n \in \N\colon f(n) \leq f(n)$. \qed

\begin{Proposition}[Multiplication with Constants] \hspace*{\fill} \linebreak
Assume that we have functions  $f,g\colon \N \rightarrow {\R_+}$ and a number $d \in \R_+$.  Then we
have 
\\[0.2cm]
\hspace*{1.3cm}
 $g \in \Oh(f) \rightarrow d \cdot g \in \Oh(f)$.
\end{Proposition}

\proof
The premiss $g \in \Oh(f)$ implies that there are constants $c'\in \R_+$ and $k'\in \N$ 
such that \\[0.2cm]
\hspace*{1.3cm} 
$\forall n \in \N \colon \bigl(n \geq k'  \rightarrow g(n) \leq c' \cdot f(n)\bigr)$ 
\\[0.2cm]
holds.  If we multiply the inequality involving $g(n)$ with $d$, we get \\[0.2cm]
\hspace*{1.3cm} 
$\forall n \in \N \colon\bigl( n \geq k'  \rightarrow d \cdot g(n) \leq d \cdot c' \cdot f(n)\bigr)$ 
\\[0.2cm]
Let us therefore define $k:=k'$ and $c := d \cdot c'$.  Then we have \\[0.2cm]
\hspace*{1.3cm}
$\forall n \in \N \colon \bigl(n \geq k  \rightarrow d \cdot g(n) \leq c \cdot f(n)\bigr)$ 
\\[0.2cm]
and by definition this implies $d \cdot g \in \Oh(f)$. \qed
\vspace*{0.3cm}

\remark 
The previous proposition shows that the big $\Oh$ notation does indeed abstract
from constant factors.

\begin{Proposition}[Addition]
Assume that $f,g,h \colon \N \rightarrow {\R_+}$. Then we have 
\\[0.2cm]
\hspace*{1.3cm}
$f \in \Oh(h) \wedge g \in \Oh(h) \,\rightarrow\, f + g \in \Oh(h)$.
\end{Proposition}

\proof
The preconditions $f \in \Oh(h)$ and $g \in \Oh(h)$ imply that there are constants $k_1,k_2\in \N$
and $c_1,c_2\in \R$ such that both \\[0.2cm]
\hspace*{1.3cm} 
$\forall n \in \N \colon \bigl( n \geq k_1 \rightarrow f(n) \leq c_1 \cdot h(n)\bigr)$ 
\quad and
\\[0.2cm]
\hspace*{1.3cm} 
$\forall n \in \N \colon\bigl( n \geq k_2 \rightarrow g(n) \leq c_2 \cdot h(n)\bigr)$
\\[0.2cm]
holds.  Let us define $k := \max(k_1,k_2)$ and $c:= c_1 + c_2$.  For all $n \in \mathbb{N}$ such
that $n \geq k$ it then follows that both
\\[0.2cm]
\hspace*{1.3cm}
 $f(n) \leq c_1 \cdot h(n)$ \quad and \quad  $g(n) \leq c_2 \cdot h(n)$
\\[0.2cm]
holds.  Adding these inequalities we conclude that 
\\[0.2cm]
\hspace*{1.3cm} $f(n) + g(n) \leq (c_1 + c_2) \cdot h(n) = c \cdot h(n)$
\\[0.2cm]
holds for all $n \geq k$.
\qed

\exercise
Assume that $f_1,f_2,h_1,h_2 \colon \N \rightarrow {\R_+}$.  Prove that 
\\[0.2cm]
\hspace*{1.3cm}
$f_1 \in \Oh(h_1) \wedge f_2 \in \Oh(h_2) \,\rightarrow\, f_1 \cdot f_2 \in \Oh(h_1 \cdot h_2)$
\quad holds.  \eox


\exercise
Assume that $f_1,f_2,h_1,h_2 \colon \N \rightarrow {\R_+}$.  Prove or refute the claim that 
\\[0.2cm]
\hspace*{1.3cm}
$f_1 \in \Oh(h_1) \wedge f_2 \in \Oh(h_2) \,\rightarrow\, f_1 / f_2 \in \Oh(h_1 / h_2)$
\quad holds.  \eox

\begin{Proposition}[Transitivity]
Assume $f,g,h \colon \N \rightarrow {\R_+}$. Then we have \\[0.2cm]
\hspace*{1.3cm}
 $f \in \Oh(g) \wedge g \in \Oh(h) \,\rightarrow\, f \in \Oh(h)$.

\end{Proposition}

\proof
The precondition $f \in \Oh(g)$ implies that there exists a $k_1 \in \N$ and a number $c_1 \in \R$
such that 
\\[0.2cm]
\hspace*{1.3cm} 
$\forall n \in \N \colon\bigl( n \geq k_1 \rightarrow f(n) \leq c_1 \cdot g(n)\bigr)$ 
\\[0.2cm]
holds, while the precondition $g \in \Oh(h)$ implies the existence of $k_2 \in \N$ and $c_2 \in \R$
such that \\[0.2cm]
\hspace*{1.3cm} 
$\forall n \in \N \colon\bigl( n \geq k_2 \rightarrow g(n) \leq c_2 \cdot h(n)\bigr)$ 
\\[0.2cm]
holds.  Let us define $k:= \max(k_1,k_2)$ and $c := c_1 \cdot c_2$.  Then for all $n \in \mathbb{N}$
such that $n \geq k$ we have the following:
\\[0.2cm]
\hspace*{1.3cm}
$f(n) \leq c_1\cdot g(n)$ \quad and \quad $g(n) \leq c_2 \cdot h(n)$. 
\\[0.2cm]
Let us multiply the second of these inequalities with $c_1$.  Keeping the first inequality this yields
\\[0.2cm]
\hspace*{1.3cm}
$f(n) \leq c_1\cdot g(n)$  \quad and \quad $c_1\cdot g(n) \leq c_1\cdot c_2 \cdot h(n)$. 
\\[0.2cm]
However, this immediately implies $f(n) \leq c \cdot h(n)$ and our claim has been proven. 
\qed

\begin{Proposition}[Limit Proposition] \label{limit}
  Assume that $f,g \colon \N \rightarrow {\R_+}$.   Furthermore, assume that the limit 
  \\[0.2cm]
  \hspace*{3.3cm}
 $\lim\limits_{n \rightarrow \infty} \bruch{\,f(n)\,}{g(n)}$
  \\[0.2cm]
  exists.  Then we have $f \in \Oh(g)$. 
\end{Proposition}

\proof
Define \\[0.2cm]
\hspace*{1.3cm}
$\lambda := \lim\limits_{n \rightarrow \infty} \bruch{f(n)}{g(n)}$.  
\\[0.2cm]
Since the limit exists by our assumption, we know that
\\[0.2cm]
\hspace*{1.3cm}
$\ds\forall \varepsilon \in \mathbb{R}_+:\exists k \in \mathbb{R}: \forall n \in \mathbb{N}: 
 \biggl(n \geq k \rightarrow \Bigl| \frac{f(n)}{g(n)} - \lambda \Bigr| < \varepsilon \biggr)
$
\\[0.2cm]
Since this is valid for all positive values of $\varepsilon$, let us define $\varepsilon := 1$. 
Then there exists a number $k \in \N$ such that
for all $n\in \N$ satisfying $n \geq k$ the inequality 
\\[0.2cm]
\hspace*{1.3cm}
$\left| \bruch{f(n)}{g(n)} - \lambda \right| \leq 1$ 
\\[0.2cm]
holds.  Let us multiply this inequality with $g(n)$.  As $g(n)$ is positive, this yields
\\[0.2cm]
\hspace*{1.3cm}
$|f(n) - \lambda \cdot g(n)| \leq g(n)$. 
\\[0.2cm]
The triangle inequality $|a + b| \leq |a| + |b|$ for real numbers tells us that 
\\[0.2cm]
\hspace*{1.3cm}
$f(n) \leq \bigl|f(n) - \lambda \cdot g(n)\bigr| + \lambda \cdot g(n)$ 
\\[0.2cm]
holds.  Combining the previous two inequalities yields
\\[0.2cm]
\hspace*{1.3cm}
$f(n) \leq g(n) + \lambda \cdot g(n) = (1 + \lambda) \cdot g(n)$. \\[0.2cm]
Therefore, we define
\\[0.2cm]
\hspace*{1.3cm}
 $c := 1 +  \lambda$
\\[0.2cm] 
and have shown that $f(n) \leq c \cdot g(n)$ holds for all $n \geq k$. \qed
\vspace*{0.3cm}

\noindent
The following examples show how to put the previous propositions to good use.

\example
Assume $k \in \N$.  Then we have
\\[0.2cm]
\hspace*{1.3cm}
 $n^k \in \Oh(n^{k+1})$.

\proof
We have \\[0.2cm]
\hspace*{1.3cm} 
$\lim\limits_{n \rightarrow \infty} \bruch{n^{k}}{n^{k+1}} = \lim\limits_{n \rightarrow   \infty} \bruch{1}{n} = 0$.
\\[0.2cm]
Therefore, the claim follows from the limit proposition. 
\qed


\example
Assume $k \in \N$ and $\lambda \in \R$ where $\lambda > 1$.  Then we have \\[0.2cm]
\hspace*{1.3cm} $n^k \in \Oh(\lambda^n)$.



\proof  We will show that 
\hspace*{1.3cm} 
\begin{equation}
  \label{eq:star}
  \lim\limits_{n \rightarrow \infty} \bruch{n^{k}}{\lambda^n} = 0  
\end{equation}
is true.  Then the claim is an immediate consequence of the limit proposition. 
According to \href{https://en.wikipedia.org/wiki/L%27H�pital%27s_rule}{L'H\^opital's rule}\footnote{Basically, L'H\^opital's rule states that provided the
  limit $\ds \lim\limits_{x \rightarrow \infty} \frac{f'(x)}{g'(x)}$ exists and $g'(x) \not= 0$, we have
\\
\hspace*{1.3cm}
$\ds\lim\limits_{x \rightarrow \infty} \frac{f(x)}{g(x)} = \lim\limits_{x \rightarrow \infty} \frac{f'(x)}{g'(x)}$.
\\[0.2cm]
Here $f'$ and $g'$ denote the derivatives of $f$ and $g$. 
  We will discuss L'H\^opital's rule in the lectures on analysis.
},  
the limit can be computed as follows: \\[0.2cm]
\hspace*{1.3cm} 
$\displaystyle \lim\limits_{n \rightarrow \infty} \bruch{n^{k}}{\lambda^n} =
\lim\limits_{x \rightarrow \infty} \bruch{x^{k}}{\lambda^x} =
\lim\limits_{x \rightarrow \infty} \bruch{\;\frac{d\,x^{k}}{dx}\;}{\frac{d\,\lambda^x}{dx}}$.
\\[0.2cm]
The derivatives can be computed as follows: \\[0.2cm]
\hspace*{1.3cm}
 $\displaystyle \frac{d\,x^{k}}{dx} = k \cdot x^{k-1}$ \quad and \quad 
 $\displaystyle \frac{d\,\lambda^{x}}{dx} = \ln(\lambda) \cdot \lambda^x$. \\[0.2cm]
We compute the second derivative and get \\[0.2cm]
\hspace*{1.3cm}  
$\displaystyle \frac{d^{2}\,x^{k}}{dx^2} = k \cdot (k-1) \cdot x^{k-2}$ \quad and \quad 
 $\displaystyle \frac{d^2\,\lambda^{x}}{dx^2} = \ln(\lambda)^2 \cdot \lambda^x$. \\[0.2cm]
In the same manner, we compute the $k$-th order derivative and find \\[0.2cm]
\hspace*{1.3cm} 
$\displaystyle \frac{d^{k}\,x^{k}}{dx^k} = k \cdot (k-1) \cdot \cdots \cdot 1 \cdot x^{0} = k!$ \quad and \quad 
 $\displaystyle \frac{d^k\,\lambda^{x}}{dx^k} = \ln(\lambda)^k \cdot \lambda^x$. \\[0.2cm]
After $k$ applications of L'H\^opital's rule we arrive at the following chain of equations:
\\[0.2cm]
\hspace*{1.3cm} 
$\ds 
\lim\limits_{x \rightarrow \infty} \frac{x^{k}}{\lambda^x} =
\lim\limits_{x \rightarrow \infty} \frac{\ds\;\frac{d\,x^{k}}{dx}\;}{\ds\frac{d\,\lambda^x}{dx}} =
\lim\limits_{x \rightarrow \infty} \frac{\ds\;\frac{d^2\,x^{k}}{dx^2}\;}{\ds\frac{d^2\,\lambda^x}{dx^2}} =
\cdots$
\\[0.3cm]
\hspace*{2.8cm}
$\ds = 
\lim\limits_{x \rightarrow \infty} \frac{\ds\;\frac{d^k\,x^{k}}{dx^k}\;}{\ds\frac{d^k\,\lambda^x}{dx^k}} =
\lim\limits_{x \rightarrow \infty} \frac{k!}{\ln(\lambda)^k \lambda^x} = 0$.
\\[0.2cm] 
Therefore the limit exists and the claim follows from the limit proposition.
\qed

\example
We have $\ln(n) \in \Oh(n)$.

\proof
This claim is again a simple consequence of the limit proposition.  We will use L'H\^opital's rule
to show that we have
\\[0.4cm]
\hspace*{1.3cm} 
$\lim\limits_{n \rightarrow \infty} \bruch{\;\ln(n)\;}{n} = 0$.
\\[0.2cm]
In the lecture on analysis we will see that \\[0.2cm]
\hspace*{1.3cm} $\displaystyle \frac{d\, \ln(x)}{dx} = \frac{1}{x}$ 
\quad and \quad
 $\displaystyle \frac{d\, x}{dx} = 1$. \\[0.2cm]
Therefore, we have \\[0.2cm]
\hspace*{1.3cm} 
$\displaystyle \lim\limits_{n \rightarrow \infty} \bruch{\;\ln(n)\;}{n} = 
\lim\limits_{x \rightarrow \infty} \bruch{\rule[-10pt]{0pt}{10pt}\;\frac{1}{x}\;}{1} = 
\lim\limits_{x \rightarrow \infty} \bruch{1}{x} = 0$. \qed


\exercise
Prove that $\sqrt{n} \in \Oh(n)$ holds.  \eox

\exercise
Assume $\varepsilon \in \mathbb{R}$ and $\varepsilon > 0$.  Prove that $n \cdot \ln(n) \in \Oh\bigl(n^{1 + \varepsilon}\bigr)$
holds. \eox

\example
We have $2^n \in \Oh(3^n)$, but  $3^n \notin \Oh(2^n)$.
\eox

\proof
 First, we have \\[0.2cm]
\hspace*{1.3cm} 
$\displaystyle\lim\limits_{n \rightarrow \infty} \bruch{2^n}{3^n} = 
 \lim\limits_{n \rightarrow \infty} \left(\bruch{2}{3}\right)^n = 0$
\\[0.2cm]
and therefore we have $2^n \in \Oh\bigl(3^n\bigr)$.  The proof of $3^n \notin \Oh(2^n)$ is a proof
by contradiction.  Assume that 
$3^n \in \Oh(2^n)$ holds.  Then, there must be numbers $c$ and $k$ such that 
\\[0.2cm]
\hspace*{1.3cm}
$3^n \leq c \cdot 2^n$ \quad holds for $n \geq k$. 
\\[0.2cm]
Taking the logarithm of both sides of this inequality we find 
\\[0.2cm]
\hspace*{1.3cm}
$
\begin{array}{llcl}
                & \ln(3^n) & \leq & \ln(c \cdot 2^n) \\[0.2cm]
\Leftrightarrow\quad &  n \cdot \ln(3) & \leq & \ln(c) + n \cdot \ln(2) \\[0.2cm]
\Leftrightarrow &  n \cdot \bigl(\ln(3) - \ln(2)\bigr) & \leq & \ln(c)  \\[0.2cm]
\Leftrightarrow &  n  & \leq & \bruch{\ln(c)}{\ln(3) - \ln(2)}          \\[0.2cm]
\end{array}
$
\\[0.2cm]
The last inequality would have to hold for all natural numbers $n$ that are bigger than $k$.  Obviously,
this is not possible as, no matter what value $c$ takes, there are natural numbers $n$ that are bigger than 
\\[0.2cm]
\hspace*{1.3cm}
$\bruch{\ln(c)}{\ln(3) - \ln(2)}$.
\qed

\exercise
\begin{enumerate}[(a)]
\item Assume that $b > 1$.  Prove that $\log_{b}(n) \in \Oh(\ln(n))$.

      \solution
      By the definition of the natural logarithm we have for any positive number $n$ we have that
      \\[0.2cm]
      \hspace*{1.3cm}
      $n = \mathrm{e}^{\ln(n)}$, \quad where \href{http://en.wikipedia.org/wiki/E_(mathematical_constant)}{$\mathrm{e}$} denotes Euler's number.
      \\[0.2cm]
      Therefore, we can rewrite the expression $\log_b(n)$
      as follows:
      \\[0.2cm]
      \hspace*{1.3cm}
      $
      \begin{array}[t]{lcl}
        \log_b(n) & = & \log_b\bigl(\mathrm{e}^{\ln(n)}\bigr) \\[0.2cm]
                  & = & \ln(n) \cdot \log_b(\mathrm{e})      \\[0.2cm]
                  & = & \log_b(\mathrm{e}) \cdot \ln(n)
      \end{array}
      $
      \\[0.2cm]
      This shows that the logarithm with respect to some base $b$ and the natural logarithm 
      only differ by a constant factor, namely $\log_b(\mathrm{e})$.  Since the big $\mathcal{O}$ notation abstracts from
      constant factors, we conclude that
      \\[0.2cm]
      \hspace*{1.3cm}
      $\log_{b}(n) \in \Oh(\ln(n))$
      \\[0.2cm]
      holds. \qed

      \remark
      The previous exercise shows that, with respect to the big $\mathcal{O}$ notation, the base of
      a logarithm is not important.  The reason is that if $b > 1$ and $c > 1$ are both used as
      bases for the logarithm, we have 
      \\[0.2cm]
      \hspace*{1.3cm}
      $\log_b(n) = \log_b(\mathrm{e}) \cdot \ln(n)$ \quad and \quad $\log_c(n) = \log_c(\mathrm{e}) \cdot \ln(n)$.
      \\[0.2cm]
      Solving these equations for $\ln(n)$ yields
      \\[0.4cm]
      \hspace*{1.3cm}
      $\bruch{\log_b(n)}{\log_b(\mathrm{e})} = \ln(n)$ \quad and \quad $\bruch{\log_c(n)}{\log_c(\mathrm{e})} = \ln(n)$.
      \\[0.2cm]
      This shows that
      \\[0.4cm]
      \hspace*{1.3cm}
      $\bruch{\log_b(n)}{\log_b(\mathrm{e})} = \bruch{\log_c(n)}{\log_c(\mathrm{e})}$
      \\[0.2cm]
      holds.  This can be rewritten as
      \\[0.2cm]
      \hspace*{1.3cm}
      $\log_b(n) = \bruch{\log_b(\mathrm{e})}{\log_c(\mathrm{e})} \cdot \log_c(n)$,
      \\[0.2cm]
      showing that $\log_b(n)$ and $\log_c(n)$ differ only by a constant factor. 
      \eox
\item Prove $3 \cdot n^2 + 5 \cdot n + \sqrt{n} \in \Oh(n^2)$.
\item Prove $7 \cdot n + \bigl(\log_2(n)\bigr)^2 \in \Oh(n)$.
\item Prove  $\sqrt{n} + \log_2(n) \in \Oh\left(\sqrt{n}\right)$.
\item Assume that $f, g \in \mathbb{R}_+^\mathbb{N}$ and that, furthermore, $f \in \Oh(g)$.  
      Proof or refute the claim that this implies 
      \\[0.2cm]
      \hspace*{1.3cm}
      $\displaystyle 2^{f(n)} \in \Oh\bigl(2^{g(n)}\bigr)$.
\item Assume that $f, g \in \mathbb{R}_+^\mathbb{N}$ and that, furthermore, 
      \\[0.4cm]
      \hspace*{1.3cm}
      $\lim\limits_{n \rightarrow \infty} \bruch{f(n)}{g(n)} = 0$.  
      \\[0.2cm]
      Proof or refute the claim that this implies
      \\[0.2cm]
      \hspace*{1.3cm}
      $\displaystyle 2^{f(n)} \in \Oh\bigl(2^{g(n)}\bigr)$.
\item Prove $n^n \in \mathcal{O}\bigl(2^{2^n}\bigr)$.
      \eox
\end{enumerate}

\section{A Remark on Notation}
Technically, for some function $g:\mathbb{N} \rightarrow \mathbb{R}_+$ the expression $\Oh(g)$
denotes a set.  Therefore, for a given function $f:\mathbb{N} \rightarrow \mathbb{R}_+$ we can
either have
\\[0.2cm]
\hspace*{1.3cm}
$f \in \Oh(g)$ \quad or \quad $f \not\in \Oh(g)$,
\\[0.2cm]
we can never have $f = \Oh(g)$.  Nevertheless, in the literature it has become common to abuse the
notation and write
\\[0.2cm]
\hspace*{1.3cm}
$f = \Oh(g)$ \quad instead of \quad $f \in \Oh(g)$.
\\[0.2cm]
Where convenient, we will also use this notation.  However, you have to be aware of the fact that
this is quite dangerous.  For example, if we have two different functions $f_1$ and
$f_2$ such that both
\\[0.2cm]
\hspace*{1.3cm}
$f_1 \in \Oh(g)$ \quad and \quad $f_2 \in \Oh(g)$
\\[0.2cm]
holds, when we write this as
\\[0.2cm]
\hspace*{1.3cm}
$f_1 = \Oh(g)$ \quad and \quad $f_2 = \Oh(g)$,
\\[0.2cm]
then we must not conclude that $f_1 = f_2$ as the functions $f_1$ and $f_2$ are merely members of
the same set $\Oh(g)$ and are not necessarily equal.  For example, $n \in \Oh(n)$ and 
$2 \cdot n \in \Oh(n)$,  but $n \not= 2 \cdot n$.
\vspace*{0.3cm}

Furthermore, for given functions $f$, $g$, and $h$ we write
\\[0.2cm]
\hspace*{1.3cm}
$f = g + \Oh(h)$
\\[0.2cm]
to express the fact that  $(f - g) \in \Oh(h)$.  For example, we have 
\\[0.2cm]
\hspace*{1.3cm}
$n^2 + \frac{1}{2} \cdot n \cdot \log_2(n) + 3 \cdot n = n^2 + \Oh\bigl(n \cdot \log_2(n)\bigr)$.
\\[0.2cm]
This is true because
\\[0.2cm]
\hspace*{1.3cm}
$\frac{1}{2} \cdot n \cdot \log_2(n) + 3 \cdot n \in \Oh\bigl(n \cdot \log_2(n)\bigr)$.
\\[0.2cm] 
The notation $f = g + \Oh(h)$ is useful because it is more precise than the pure big $\Oh$
notation.  For example, assume we have two algorithms $A$ and $B$ for sorting a list of length
$n$.  Assume further that the number $\textsl{count}_A(n)$ of comparisons used by algorithm $A$ to sort a list of
length $n$ is given as
\\[0.2cm]
\hspace*{1.3cm}
$\textsl{count}_A(n) = n \cdot \log_2(n) + 7 \cdot n$,
\\[0.2cm]
while for algorithm $B$ the corresponding number of comparisons is given as
\\[0.2cm]
\hspace*{1.3cm}
$\textsl{count}_B(n) = \frac{3}{2} \cdot n \cdot \log_2(n) + 4 \cdot n$.
\\[0.2cm]
Then the big $\Oh$ notation is not able to distinguish between the complexity of algorithm $A$ and
algorithm $B$ since we have
\\[0.2cm]
\hspace*{1.3cm}
$\textsl{count}_A(n) \in \Oh\bigl(n \cdot \log_2(n)\bigr)$ \quad as well as \quad
$\textsl{count}_B(n) \in \Oh\bigl(n \cdot \log_2(n)\bigr)$.
\\[0.2cm]
However, by writing
\\[0.2cm]
\hspace*{1.3cm}
$\textsl{count}_A(n) = n \cdot \log_2(n) + \Oh(n)$ \quad and \quad
$\textsl{count}_B(n) = \frac{3}{2} \cdot n \cdot \log_2(n) + \Oh(n)$
\\[0.2cm]
we can abstract from lower order terms while still retaining the leading coefficient of the term
determining the complexity.  

\section[Computation of Powers]{Case Study:  Efficient Computation of Powers}
Let us study an example to clarify the notions introduced so far.  
Consider the program shown in Figure \ref{fig:power-naive.stlx}.  Given an integer $m$ and a
natural number $n$, $\mathtt{power}(m, n)$ computes $m^n$.
The basic idea is to compute the value of $m^n$ according to the formula \\[0.2cm]
\hspace*{1.3cm} 
$m^n = \underbrace{m \cdot {\dots} \cdot m}_n$. 


\begin{figure}[!h]
  \centering
\begin{Verbatim}[ frame         = lines, 
                  framesep      = 0.3cm, 
                  labelposition = bottomline,
                  numbers       = left,
                  numbersep     = -0.2cm,
                  xleftmargin   = 0.8cm,
                  xrightmargin  = 0.8cm
                ]
    power := procedure(m, n) {
        r := m;
        for (i in {2 .. n}) {
            r := r * m;
        }
        return r;
    };
\end{Verbatim}
\vspace*{-0.3cm}
  \caption{Naive computation of $m^n$ for  $m,n \in \N$.}
  \label{fig:power-naive.stlx}
\end{figure} 

This program is obviously correct.  The computation of $m^n$ requires $n-1$ multiplications if the
function \texttt{power}  is implemented as shown in Figure \ref{fig:power-naive.stlx}.
Fortunately, there is an algorithm for computing $m^n$ that is much more efficient.
Consider we have to evaluate $m^4$.  We have
 \\[0.2cm]
\hspace*{1.3cm} 
$m^4 = (m \cdot m) \cdot (m \cdot m)$.
\\[0.2cm]
If the expression $m\cdot m$ is computed just once, the computation of
$m^4$ needs only two multiplications while the naive approach would already need 3 multiplications.
In order to compute $m^8$ we can proceed according to the following formula: \\[0.2cm]
\hspace*{1.3cm} 
$m^8 = \bigl( (m \cdot m) \cdot (m \cdot m) \bigr) \cdot \bigl( (m \cdot m) \cdot (m \cdot m)
\bigr)$. 
\\[0.2cm]
If the expression $(m \cdot m) \cdot (m \cdot m)$ is computed only once, then we need just 3 multiplications
in order to compute $m^8$.   On the other hand, the naive approach would take 7 multiplications to
compute $m^8$.  The general case is implemented in the program shown in Figure \ref{fig:power.stlx}.  
In this program, the value of $m^n$ is computed according to the 
\href{http://en.wikipedia.org/wiki/Divide_and_conquer_algorithm}{\emph{divide and conquer} paradigm}.
The basic idea that makes this program work is captured by the following formula: 
\\[0.2cm] 
\hspace*{1.3cm} 
$m^n = 
\left\{\begin{array}{ll}
m^{n/2} \cdot m^{n/2}          & \mbox{if $n$ is even};    \\
m^{n/2} \cdot m^{n/2} \cdot m  & \mbox{if $n$ is odd}.
\end{array}
\right.
$
\vspace*{0.3cm}

\begin{figure}[!h]
  \centering
\begin{Verbatim}[ frame         = lines, 
                  framesep      = 0.3cm, 
                  labelposition = bottomline,
                  numbers       = left,
                  numbersep     = -0.2cm,
                  xleftmargin   = 0.8cm,
                  xrightmargin  = 0.8cm
                ]
    power := procedure(m, n) {
        if (n == 0) {
            return 1;
        }
        p := power(m, n \ 2);
        if (n % 2 == 0) {
            return p * p;
        } else {
            return p * p * m;
        }
    };
\end{Verbatim}
\vspace*{-0.3cm}
  \caption{Computation of $m^n$ for $m,n \in \N$.}
  \label{fig:power.stlx}
\end{figure} 

It is by no means obvious that the program shown in \ref{fig:power.stlx} does compute
$m^n$.  We prove this claim by  \emph{computational induction}.
Computational induction is an induction on the number of recursive invocations.
This method is the method of choice to prove the correctness of a recursive procedure.
The method of computational induction consists of two steps:
\begin{enumerate}
\item The \emph{base case}.

      In the base case we have to show that the procedure is correct in all those cases were it
      does not invoke itself recursively.
\item The \emph{induction step}.

      In the induction step we have to prove that the method works in all those cases were
      it does invoke itself recursively.  In order to prove the correctness of these cases we may
      assume that the recursive invocations work correctly.  This assumption is called the
      \emph{induction hypotheses}.
\end{enumerate}
Let us prove the claim 
\\[0.2cm]
\hspace*{1.3cm}
 $\mathtt{power}(m,n) \leadsto m^n$
\\[0.2cm] 
by computational induction.
\begin{enumerate}
\item \textbf{Base case}:

      The only case where \texttt{power} does not invoke itself recursively is the case $n = 0$.  
      In this case, we have
      \\[0.2cm]
      \hspace*{1.3cm} 
      $\mathtt{power}(m,0) \leadsto 1 =  m^0$.
\item \textbf{Induction step}:

      The recursive invocation of $\mathtt{power}$ has the form
      $\mathtt{power}(m,n \backslash 2)$.  By the induction hypotheses we know that 
      \\[0.2cm]
      \hspace*{1.3cm}
      $\displaystyle \mathtt{power}(m,n \backslash 2) \leadsto m^{n \backslash 2}$ 
      \\[0.2cm]
      holds.  After the recursive invocation there are two different cases:
      \begin{enumerate}
      \item $n \;\texttt{\%}\; 2 = 0$, therefore $n$ is even.

            Then there exists a number $k \in \N$ such that $n = 2 \cdot k$ and therefore
            $n / 2 = k$.
            Then, we have the following:
            \\[0.2cm]
            \hspace*{1.3cm}
           $ 
            \begin{array}{lcl}
            \mathtt{power}(m,n) & \leadsto & \mathtt{power}(m,k) \cdot \mathtt{power}(m,k) \\[0.2cm]
                                & \stackrel{I.V.}{\leadsto} & m^k \cdot m^k  \\[0.2cm]
                                & = & m^{2\cdot k} \\[0.2cm]
                                & = & m^{n}.
            \end{array}
            $            
            \\[0.2cm]
      \item $n \;\texttt{\%}\; 2 = 1$, therefore $n$ is odd.

            Then there exists a number $k \in \N$ such that $n = 2 \cdot k + 1$ and we have
            $n \backslash 2 = k$, where $n \backslash 2$ denotes integer division of $n$ by $2$.
            In this case we have:
            \[ 
            \begin{array}{lcl}
            \mathtt{power}(m,n) & \leadsto & \mathtt{power}(m,k) \cdot \mathtt{power}(m,k) \cdot m  \\[0.2cm]
                                & \stackrel{I.V.}{\leadsto} & m^k \cdot m^k \cdot m  \\[0.2cm]
                                & = & m^{2\cdot k+1} \\[0.2cm]
                                & = & m^{n}.
            \end{array}
            \]
      \end{enumerate}
      As we have $\mathtt{power}(m,n) = m^n$ in both cases, the proof is finished.
      \qed
\end{enumerate}
Next, we want to investigate the computational complexity of this implementation of \texttt{power}.
To this end, let us compute the number of multiplications that are done when
$\mathtt{power}(m,n)$ is called.  If the number $n$ is odd there will be more multiplications than
in the case when $n$ is even.  Let us first investigate the \emph{worst case}.  
The worst case happens if there is an $l\in \N$ such that 
\\[0.2cm]
\hspace*{1.3cm}
$n = 2^l - 1$ 
\\[0.2cm]
because then we have 
\\[0.2cm]
\hspace*{1.3cm} $n \backslash 2 = 2^{l-1} - 1$ \quad and \quad $n \,\texttt{\symbol{37}}\, 2 = 1$, \\[0.2cm]
because in that case we have 
\\[0.2cm]
\hspace*{1.3cm}
$2 \cdot(n \backslash 2) + n \,\texttt{\symbol{37}}\, 2 = 2 \cdot (2^{l-1} - 1) + 1 = 2^l - 1 = n$.
\\[0.2cm]
Therefore, if $n = 2^l - 1$ the exponent $n$ will be odd on every recursive call.
Therefore, let us assume $n = 2^l - 1$ and let us compute the number $a_n$ of multiplications that
are done when $\mathtt{power}(m,n)$ is evaluated. 

First, we have $a_0 = 0$, because if we have $n = 2^0 - 1 = 0$, then the evaluation of 
$\mathtt{power}(m, n)$ does not require a single multiplication.
Otherwise, we have in line 9 two multiplications that have to be added to those multiplications
that are performed in the recursive call in line 5.  Therefore, we get the following
\href{http://en.wikipedia.org/wiki/Recurrence_relation}{\emph{recurrence relation}}:
\\[0.2cm]
\hspace*{1.3cm}
$a_n = a_{n \backslash 2} + 2$ \qquad for all $n \in \left\{2^l - 1 \mid l \in \N\right\}$\quad and $a_0 = 0$. 
\\[0.2cm]
In order to solve this recurrence relation, let us define $b_l := a_{2^l-1}$.  Then, the sequence
$(b_l)_l$ 
satisfies the recurrence relation
 \\[0.2cm]
\hspace*{1.3cm} 
$b_l = a_{2^l-1} = a_{(2^l-1) \backslash 2} + 2 = a_{2^{l-1}-1} + 2 = b_{l-1} +2$ \qquad for all $l\in\N$
\\[0.2cm]
and the initial term $b_0$ satisfies $b_0 = a_{2^0-1} = a_0 = 0$.  It is quite obvious that the
solution of this recurrence relation is given by
\\[0.2cm]
\hspace*{1.3cm} $b_l = 2 \cdot l$ \qquad for all $l \in \N$. 
\\[0.2cm] 
This claim is readily established via a trivial induction.  Plugging in the definition $b_l = a_{2^l-1}$ we
see that the sequence $a_n$ satisfies \\[0.2cm]
\hspace*{1.3cm} $a_{2^l-1} = 2 \cdot l$. 
\\[0.2cm]
Let us solve the equation $n = 2^l - 1$ for $l$.  This yields
 $l =
\log_2(n+1)$.  Substituting this expression in the formula above gives \\[0.2cm]
\hspace*{1.3cm} $a_n = 2 \cdot \log_2(n+1) \in \Oh\bigl(\log_2(n)\bigr)$.
\vspace*{0.3cm}

Next, we consider the best case.  The computation of
$\mathtt{power}(m,n)$ needs the least number of multiplications if the test 
\texttt{n \% 2 == 0} always evaluates as true.  In this case, $n$ must be a power of $2$.  
Hence there must exist an $l\in \N$ such that we have
 \\[0.2cm]
\hspace*{1.3cm} $n = 2^l$.
 \\[0.2cm]
Therefore, let us now assume $n = 2^l$ and let us again compute the number $a_n$ of multiplications
that are needed to compute $\mathtt{power}(m,n)$. 

First, we have $a_{2^0} = a_1 = 2$, because if $n = 1$, the test \texttt{n \% 2 == 0} fails and in
this case line 9 yields 2 multiplications.  Furthermore, in this case
line 5 does not add any multiplications since the call $\mathtt{power}(m,0)$ immediately returns its
result.

Now, if $n = 2^l$ and $n > 1$ then line 7 yields one multiplication that 
has to be added to those multiplications that are done during the recursive invocation of
\texttt{power} in line 5.  Therefore, we have the following recurrence relation:
 \\[0.2cm]
\hspace*{1.3cm} $a_n = a_{n \backslash 2} + 1$ \qquad for all $n \in \left\{2^l \mid l \in \N\right\}$\quad and
$a_1 = 2$. 
\\[0.2cm]
Let us define $b_l := a_{2^l}$.  Then the sequence $(b_l)_l$ satisfies the recurrence relation
 \\[0.2cm]
\hspace*{1.3cm} 
$b_l = a_{2^l} = a_{(2^l) \backslash 2} + 1 = a_{2^{l-1}} + 1 = b_{l-1} + 1$ \qquad for all $l\in\N$, \\[0.2cm]
and the initial value is given as $b_0 = a_{2^0} = a_1 = 2$.
Therefore, we have to solve the recurrence relation 
\\[0.2cm]
\hspace*{1.3cm}
 $b_{l+1} = b_l + 1$ \qquad for all $l \in \N$ \quad with $b_0 = 2$.\\[0.2cm]
Obviously, the solution is \\[0.2cm]
\hspace*{1.3cm} $b_l = 2 + l$ \qquad for all $l\in\N$.
\\[0.2cm]
If we substitute this into the definition of $b_l$ in terms of $a_l$ we have: \\[0.2cm]
\hspace*{1.3cm}
$a_{2^l} = 2 + l$. 
\\[0.2cm]
If we solve the equation $n = 2^l$ for  $l$ we get $l =
\log_2(n)$. Substituting this value leads to
\\[0.2cm]
\hspace*{1.3cm}
 $a_n = 2 + \log_2(n) \in \Oh\bigl(\log_2(n)\bigr)$.
\vspace*{0.3cm}

Since we have gotten the same result both in the worst case and in the best case we may conclude
that in general the number $a_n$ of multiplications satisfies 
\\[0.2cm]
\hspace*{1.3cm} 
$a_n \in \Oh\bigl(\log_2(n)\bigr)$. \qed

\remark
In reality, we are not interested in the number of multiplications but we are rather interested
in the amount of computation time needed by the algorithm given above.
However, this computation would be much more tedious because then we would have to take into account
that the time needed to multiply two numbers depends on the size of these numbers.

\exercise
Implement a procedure $\mathtt{prod}$ that multiplies two numbers:
For given natural numbers $m$ and $n$, the expression $\mathtt{prod}(m, n)$  should compute the product
$m\cdot n$.  Of course, your implementation must not use the multiplication operator ``\texttt{*}''.
However, you may use the operators ``\texttt{\symbol{92}}'' and ``\texttt{\%}'' provided
the second argument of these
operators is the number $2$.  The reason it that division by $2$ can be implemented by a simple
shift, while $n\, \texttt{\symbol{37}}\, 2$ is just the last bit of $n$.

In your implementation, you should use the
divide and conquer paradigm.  Furthermore, you should use computational induction to prove the
correctness of your implementation.  Finally, you should provide an estimate for the number of
additions needed to compute $\mathtt{prod}(m,n)$.  This estimate should make use of the big $\Oh$
notation. 
\eox


\section{The Master Theorem}
In order to analyse the complexity of  the procedure $\textsl{power}()$,
we have first computed a  recurrence relation, then we have solved this recurrence and, finally,  
we have approximated the result using the big $\Oh$ notation.  In many cases we are only interested in this
last approximation and then it is not necessary to actually solve the recurrence relation.  
Instead, we can use the 
\href{http://en.wikipedia.org/wiki/Master_theorem#Generic_form}{\emph{master theorem}} to short
circuit the procedure for computing the complexity of an algorithm. 
We present a simplified version of the master theorem next.
\pagebreak

\begin{Theorem}[Master Theorem] 
  Assume that 
  \begin{enumerate}
  \item $\alpha,\beta \in \mathbb{N}$ such that  $\beta \geq 2$,
  \item $\delta \in \mathbb{R}$ and $\delta \geq 0$,
  \item the function $f:\N \rightarrow \R_+$ satisfies the recurrence relation
        \\[0.2cm]
        \hspace*{1.3cm}
        $f(n) = \alpha \cdot f\left(n \backslash \beta\right) + \Oh(n^\delta)$,
        \\[0.2cm]
        where $n \backslash \beta$ denotes \emph{integer division}\footnote{
          For given integers $a, b \in \mathbb{N}$, the \emph{integer division} $a \backslash b$
          is defined as the biggest number $q \in \mathbb{N}$ such that $q \cdot b \leq a$.  It can
          be implemented via the formula $a \backslash b = \textsl{floor}(a/b)$, where
          $\textsl{floor}(x)$ rounds $x$ down to the nearest integer.  In \textsc{SetlX}, integer
          division is available via the backslash operator ``\texttt{\symbol{92}}''.
        }  
        of $n$ by $\beta$.
  \end{enumerate}
  Then we have the following:
  \begin{enumerate}
  \item $\alpha < \beta^\delta\rightarrow f \in \Oh\bigl(n^{\delta}\bigr)$,
  \item $\alpha = \beta^\delta\rightarrow f \in \Oh\bigl(\log_\beta(n) \cdot n^{\delta}\bigr)$,
  \item $\alpha > \beta^{\delta} \rightarrow f \in \Oh\bigl(n^{\log_{\beta}(\alpha)}\bigr)$. 
  \end{enumerate}
\end{Theorem}

\proof
We will compute an upper bound for the expression $f(n)$ but in order to keep our exposition clear
and simple we will only discuss the case where $n$ is a power of $\beta$, that is $n$ has the form
\\[0.2cm]
\hspace*{1.3cm}
$n = \beta^k$ \quad for some $k \in \mathbb{N}$.
\\[0.2cm]
The general case is similar, but is technically much more involved.
Observe that the equation $n = \beta^k$ implies $k = \log_{\beta}(n)$.  We will need this equation later. 
Furthermore, in order to simplify our exposition even further, we assume that the recurrence
relation for $f$ has the form 
\\[0.2cm]
\hspace*{1.3cm}
 $f(n) = \alpha \cdot f\left(n \backslash \beta\right) + n^\delta$,
\\[0.2cm]
i.e.~instead of adding the term $\Oh\bigl(n^\delta\bigr)$ we just add $n^\delta$.  These
simplifications do not change the proof idea.  We start the proof by defining
\\[0.2cm]
\hspace*{1.3cm}
 $a_k := f(n) = f\bigl(\beta^k\bigr)$.  
\\[0.2cm]
Then the recurrence relation for the function $f$
is transformed into a recurrence relation for the sequence $a_k$ as follows:
\\[0.2cm]
\hspace*{1.3cm}
$
\begin{array}[t]{lcl}
a_k & = & f\bigl(\beta^{k}\bigr)                                             \\[0.2cm]
    & = & \alpha \cdot f\bigl(\beta^{k} \backslash \beta\bigr) + \bigl(\beta^{k}\bigr)^\delta    \\[0.2cm]
    & = & \alpha \cdot f\bigl(\beta^{k-1}\bigr) + \beta^{k \cdot \delta}    \\[0.2cm]
    & = & \alpha \cdot a_{k-1} + \beta^{k \cdot \delta}    \\[0.2cm]
    & = & \alpha \cdot a_{k-1} + \bigl(\beta^{\delta}\bigr)^k    \\[0.2cm]
\end{array}
$
\\[0.2cm]
In order to simplify this recurrence relation, let us define
\\[0.2cm]
\hspace*{1.3cm}
$\gamma := \beta^{\delta}$.
\\[0.2cm]
Then, the recurrence relation for the sequence $a_k$ can be written as
\\[0.2cm]
\hspace*{1.3cm}
$a_k = \alpha \cdot a_{k-1} + \gamma^k$.
\\[0.2cm]
Let us substitute $k-1$ for $k$ in this equation.  This yields
\\[0.2cm]
\hspace*{1.3cm}
$a_{k-1} = \alpha \cdot a_{k-2} + \gamma^{k-1}$.
\\[0.2cm]
Next, we plug the value of $a_{k-1}$ into the equation for $a_k$.  This yields
\\[0.2cm]
\hspace*{1.3cm}
$
\begin{array}[t]{lcl}
  a_k & = & \alpha \cdot a_{k-1} + \gamma^k  \\[0.2cm]
      & = & \alpha \cdot \bigl(\alpha \cdot a_{k-2} + \gamma^{k-1}\bigr) + \gamma^k \\[0.2cm]
      & = & \alpha^2 \cdot a_{k-2} + \alpha \cdot \gamma^{k-1} + \gamma^k. \\[0.2cm]
\end{array}
$
\\[0.2cm]
We observe that 
\\[0.2cm]
\hspace*{1.3cm}
$a_{k-2} = \alpha \cdot a_{k-3} + \gamma^{k-2}$
\\[0.2cm]
holds and substitute the right hand side of this equation into the previous equation.  This yields
\\[0.2cm]
\hspace*{1.3cm}
$
\begin{array}[t]{lcl}
  a_k & = & \alpha^2 \cdot a_{k-2} + \alpha \cdot \gamma^{k-1} + \gamma^k \\[0.2cm]
      & = & \alpha^2 \cdot \bigl(\alpha \cdot a_{k-3} + \gamma^{k-2}\bigr) + \alpha \cdot \gamma^{k-1} + \gamma^k \\[0.2cm]
      & = & \alpha^3 \cdot a_{k-3} + \alpha^2 \cdot \gamma^{k-2} + \alpha \cdot \gamma^{k-1} + \alpha^0 \cdot \gamma^k. \\[0.2cm]
\end{array}
$
\\[0.2cm]
Proceeding in this way we arrive at the general formula
\\[0.2cm]
\hspace*{1.3cm}
$
\begin{array}[b]{lcl}
a_k & = & \alpha^{i+1} \cdot a_{k-(i+1)} + \alpha^{i} \cdot \gamma^{k-i} + \alpha^{i-1} \cdot \gamma^{k-(i-1)} + \cdots + \alpha^0 \cdot \gamma^k \\[0.2cm]
    & = & \alpha^{i+1} \cdot a_{k-(i+1)} + \sum\limits_{j=0}^i \alpha^{j} \cdot \gamma^{k-j}. \\[0.2cm]
\end{array}
$
\\[0.2cm]
If we take this formula and substitute $i + 1 := k$, i.e.~$i := k - 1$, then we conclude
\\[0.2cm]
\hspace*{1.3cm}
$\displaystyle 
\begin{array}[b]{lcl}
a_k & = & \ds\alpha^{k} \cdot a_{0} + \sum\limits_{j=0}^{k-1} \alpha^{j} \cdot \gamma^{k-j} \\[0.5cm]
    & = & \ds\alpha^{k} \cdot a_{0} + \gamma^k \cdot \sum\limits_{j=0}^{k-1} \Bigl(\displaystyle\frac{\,\alpha\,}{\gamma}\Bigr)^j.  \\[0.2cm]
\end{array}
$
\\[0.2cm]
At this point we have to remember the formula for the geometric series.  This formula reads
\\[0.2cm]
\hspace*{1.3cm}
$\displaystyle\sum\limits_{j=0}^n q^j = \bruch{q^{n+1} - 1}{q - 1}$ \quad provided $q \not= 1$, while
\\[0.2cm]
\hspace*{1.3cm}
$\displaystyle\sum\limits_{j=0}^n q^j = n+1$ \quad if $q = 1$.
\\[0.2cm]
For the geometric series given above, $q = \displaystyle\frac{\,\alpha\,}{\gamma}$.  
In order to proceed, we have to perform a case distinction:
\begin{enumerate}
\item Case: $\alpha < \gamma$, i.e.~$\alpha < \beta^\delta$.

      In this case, the series $\sum\limits_{j=0}^{k-1} \bigl(\frac{\,\alpha\,}{\gamma}\bigr)^j$ is bounded 
      by the value
      \\[0.2cm]
      \hspace*{1.3cm}
      $\displaystyle \sum\limits_{j=0}^{\infty} \Bigl(\frac{\,\alpha\,}{\gamma}\Bigr)^j = \bruch{1}{1 - \frac{\,\alpha\,}{\gamma}.}$
      \\[0.2cm]
      Since this value does not depend on $k$ and the big $\Oh$ notation abstracts from constant factors,
      we are able to drop the sum.  Therefore, we have
      \\[0.2cm]
      \hspace*{1.3cm}
      $a_k = \alpha^k \cdot a_0 + \Oh\bigl(\gamma^k\bigr)$.
      \\[0.2cm] 
      Furthermore, let us observe that, since $\alpha < \gamma$ we have that 
      \\[0.2cm]
      \hspace*{1.3cm}
      $\alpha^k \cdot a_0 \in \Oh\bigl(\gamma^k\bigr)$.
      \\[0.2cm]
      Therefore, the term $\alpha^k \cdot a_0$ is subsumed by $\Oh\bigl(\gamma^k\bigr)$ and we have shown that
      \\[0.2cm]
      \hspace*{1.3cm}
      $a_k \in \Oh\bigl(\gamma^k\bigr)$.
      \\[0.2cm]
      The variable $\gamma$ was defined as $\gamma = \beta^\delta$.  Furthermore, by definition of $k$ and $a_k$
      we have 
      \\[0.2cm]
      \hspace*{1.3cm}
      $k = \log_{\beta}(n)$ \quad and \quad $f(n) = a_k$.  
      \\[0.2cm]
      Therefore we have 
      \\[0.2cm]
      \hspace*{1.3cm}
      $f(n) \in \Oh\Bigl(\bigl(\beta^\delta\bigr)^{\log_{\beta}(n)}\Bigr) = \Oh\Bigl(\bigl(\beta^{\log_{\beta}(n)}\bigr)^\delta\Bigr) = \Oh\bigl(n^\delta\bigr)$.
      \\[0.2cm] 
      Thus we have shown the following:
      \\[0.2cm]
      \hspace*{1.3cm}
      $\alpha < \beta^\delta \rightarrow f(n) \in \Oh\bigl(n^\delta\bigr)$.
\item Case: $\alpha = \gamma$, i.e.~$\alpha = \beta^\delta$.

      In this case, all terms in the series $\sum\limits_{j=0}^{k-1} \bigl(\frac{\,\alpha\,}{\gamma}\bigr)^j$ 
      have the value $1$ and therefore we have
      \\[0.2cm]
      \hspace*{1.3cm}
      $\sum\limits_{j=0}^{k-1} \bigl(\frac{\,\alpha\,}{\gamma}\bigr)^j = \sum\limits_{j=0}^{k-1} 1 = k$.
      \\[0.2cm]
      Therefore, we have
      \\[0.2cm]
      \hspace*{1.3cm}
      $a_k = \alpha^k \cdot a_0 + \Oh\bigl(k \cdot \gamma^k\bigr)$.
      \\[0.2cm] 
      Furthermore, let us observe that, since $\alpha = \gamma$ we have that 
      \\[0.2cm]
      \hspace*{1.3cm}
      $\alpha^k \cdot a_0 \in \Oh\bigl(k \cdot \gamma^k\bigr)$.
      \\[0.2cm]
      Therefore, the term $\alpha^k \cdot a_0$ is subsumed by $\Oh\bigl(k \cdot \gamma^k\bigr)$ and we have shown that
      \\[0.2cm]
      \hspace*{1.3cm}
      $a_k \in \Oh\bigl(k \cdot \gamma^k\bigr)$.
      \\[0.2cm]
      We have $\gamma = \beta^\delta$,  $k = \log_{\beta}(n)$, and $f(n) = a_k$.  
      Therefore,
      \\[0.2cm]
      \hspace*{1.3cm}
      $f(n) \in \Oh\Bigl(\log_{\beta}(n) \cdot \bigl(\beta^\delta\bigr)^{\log_{\beta}(n)}\Bigr) = \Oh\bigl(\log_{\beta}(n) \cdot n^\delta\bigr)$.
      \\[0.2cm] 
      Thus we have shown the following:
      \\[0.2cm]
      \hspace*{1.3cm}
      $\alpha = \beta^\delta \rightarrow f(n) \in \Oh\bigl(\log_{\beta}(n) \cdot n^\delta\bigr)$.
\item Case: $\alpha > \gamma$, i.e.~$\alpha > \beta^\delta$.

      In this case we have
      \\[0.2cm]
      \hspace*{1.3cm}
      $\ds\sum\limits_{j=0}^{k-1} \Bigl(\frac{\,\alpha\,}{\gamma}\Bigr)^j = \bruch{\bigl(\frac{\alpha}{\gamma})^k - 1}{\frac{\alpha}{\gamma} - 1} \in \Oh\biggl(\Bigl(\frac{\alpha}{\gamma}\Bigr)^k\biggr)$.
      \\[0.2cm]
      Therefore, we have
      \\[0.2cm]
      \hspace*{1.3cm}
      $a_k = \alpha^k\cdot a_0 + \gamma^k \cdot \bigl(\frac{\alpha}{\gamma}\bigr)^k = \alpha^k\cdot a_0 + \Oh\bigl(\alpha^k\bigr)$.
      \\[0.2cm] 
      Since $\alpha^k \cdot a_0 \in \Oh\bigl(\alpha^k\bigr)$, we have shown that
      \\[0.2cm]
      \hspace*{1.3cm}
      $a_k \in  \Oh\bigl(\alpha^k\bigr)$.
      \\[0.2cm]
      Since  $k = \log_{\beta}(n)$ and $f(n) = a_k$ we have
      \\[0.2cm]
      \hspace*{1.3cm}
      $f(n) \in \Oh\Bigl(\alpha^{\log_{\beta}(n)}\Bigr)$.
      \\[0.2cm] 
      Next, we observe that 
      \\[0.2cm]
      \hspace*{1.3cm}
      $\alpha^{\log_{\beta}(n)} = n^{\log_{\beta}(\alpha)}$
      \\[0.2cm]
      holds.   This equation is easily proven by taking the logarithm with base $\beta$ on both sides
      of the equation.  Using this equation we conclude that
      \\[0.2cm]
      \hspace*{1.3cm}
      $\alpha > \beta^\delta \rightarrow f(n) \in \Oh\bigl(n^{\log_{\beta}(\alpha)}\bigr)$
      \\[0.2cm]
      holds.  \qed
\end{enumerate}


\example
Assume that $f$ satisfies the recurrence relation
\\[0.2cm]
\hspace*{1.3cm}
$f(n) = 9 \cdot f(n \backslash 3) + n$.
\\[0.2cm] 
Define $\alpha := 9$, $\beta := 3$, and $\delta := 1$.
Then we have
\\[0.2cm]
\hspace*{1.3cm}
$\alpha = 9 > 3^1 = \beta^\delta$.
\\[0.2cm] 
This is the last case of the master theorem and, since 
\\[0.2cm]
\hspace*{1.3cm}
$\log_{\beta}(\alpha) = \log_3(9) = 2$,
\\[0.2cm]
we conclude that
\\[0.2cm]
\hspace*{1.3cm}
$f(n) \in \Oh(n^2)$ \quad holds.
 \qed


\example
Assume that the function $f(n)$ satisfies the recurrence relation
\\[0.2cm]
\hspace*{1.3cm}
$f(n) = f(n \backslash 2) + 2$.
\\[0.2cm]
We want to analyse the asymptotic growth of $f$ with the help of the master theorem.
Defining $\alpha := 1$, $\beta := 2$,  $\delta = 0$ and noting that $2 \in \Oh(n^0)$ we see that
the recurrence relation for $f$ can be written as
\\[0.2cm]
\hspace*{1.3cm}
$f(n) = \alpha \cdot f(n \backslash \beta) + \Oh(n^\delta)$.
\\[0.2cm]
Furthermore, we have
\\[0.2cm]
\hspace*{1.3cm}
$\alpha = 1 = 2^0 = \beta^\delta$.
\\[0.2cm]
Therefore, the second case of the master theorem tells us that
\\[0.2cm]
\hspace*{1.3cm}
$f(n) \in \Oh\bigl(\log_\beta(n) \cdot n^\delta\bigr) = \Oh\bigl(\log_2(n) \cdot n^0\bigr) = \Oh\bigl(\log_2(n)\bigr)$.
\qed

\example
This time, $f$ satisfies the recurrence relation 
\\[0.2cm]
\hspace*{1.3cm}
$f(n) = 3 \cdot f(n \backslash 4) + n^2$.
\\[0.2cm]
Define $\alpha := 3$, $\beta := 4$, and $\delta := 2$.  Then we have
\\[0.2cm]
\hspace*{1.3cm}
$f(n) = \alpha \cdot f(n \backslash \beta) + \Oh(n^\delta)$.
\\[0.2cm]
Since this time we have
\\[0.2cm]
\hspace*{1.3cm}
$\alpha = 3 < 16 = \beta^\delta$
\\[0.2cm]
the first case of the master theorem tells us that
\\[0.2cm]
\hspace*{1.3cm}
$f(n) \in \Oh\bigl(n^2\bigr)$.  \qed

\example
This next example is a slight variation of the previous example.  Assume $f$ satisfies the recurrence relation 
\\[0.2cm]
\hspace*{1.3cm}
$f(n) = 3 \cdot f(n \backslash 4) + n \cdot \log_2(n)$.
\\[0.2cm]
Again, define $\alpha := 3$ and $\beta := 4$.  This time we define $\delta := 1 + \varepsilon$ where
$\varepsilon$ is some small positive number that will be defined later.  You can think of $\varepsilon$ being $\frac{1}{2}$ or
$\frac{1}{5}$ or even $\frac{1}{42}$.  Since the logarithm of $n$
grows slower than any positive power of $n$ we have
\\[0.2cm]
\hspace*{1.3cm}
$\log_2(n) \in \Oh\bigl(n^\varepsilon\bigr)$.
\\[0.2cm]
We conclude that
\\[0.2cm]
\hspace*{1.3cm}
$n \cdot \log_2(n) \in \Oh\bigl(n \cdot n^\varepsilon\bigr) = \Oh\bigl(n^{1+\varepsilon}\bigr) = \Oh\bigl(n^\delta\bigr)$.
\\[0.2cm]
Therefore, we have
\\[0.2cm]
\hspace*{1.3cm}
$f(n) = \alpha \cdot f(n \backslash \beta) + \Oh(n^\delta)$.
\\[0.2cm]
Furthermore, we have
\\[0.2cm]
\hspace*{1.3cm}
$\alpha = 3 < 4 < 4^\delta = \beta^\delta$.
\\[0.2cm]
Therefore, the first case of the master theorem tells us that
\\[0.2cm]
\hspace*{1.3cm}
$f(n) \in \Oh\bigl(n^{1+\varepsilon}\bigr)$ \quad holds for all $\varepsilon > 0$.
\\[0.2cm]
Hence, we have shown that
\\[0.2cm]
\hspace*{1.3cm}
$f(n) \in \Oh\bigl(n^{1+\frac{1}{2}}\bigr)$, \quad
$f(n) \in \Oh\bigl(n^{1+\frac{1}{5}}\bigr)$, \quad and even \quad
$f(n) \in \Oh\bigl(n^{1+\frac{1}{42}}\bigr)$
\\[0.2cm]
holds.  Using a stronger form of the master theorem it can be shown that
\\[0.2cm]
\hspace*{1.3cm}
$f(n) \in \Oh\bigl(n \cdot \log_2(n)\bigr)$
\\[0.2cm]
holds.  This example shows that the master theorem, as given in these lecture notes, does not always
produce the most precise estimate for the asymptotic growth of a function.
\qed



\exercise
For each of the following recurrence relations, use the master theorem to give estimates of the
growth of the function $f$. 
\begin{enumerate}
\item $f(n) = 4 \cdot f(n \backslash 2) + 2 \cdot n + 3$.
\item $f(n) = 4 \cdot f(n \backslash 2) + n^2$.
\item $f(n) = 3 \cdot f(n \backslash 2) + n^3$.  \eox
\end{enumerate}



\exercise
Consider the recurrence relation
\\[0.2cm]
\hspace*{1.3cm}
$f(n) = 2 \cdot f(n \backslash 2) + n \cdot \log_2(n)$.
\\[0.2cm] 
How can you bound the growth of $f$ using the master theorem? 
\vspace*{0.1cm}

\noindent
\textbf{Optional}: Assume that $n$ has the form $n = 2^k$ for some natural number $k$.
Furthermore, you are told that $f(1) = 1$.  Solve the recurrence relation in this case.
\eox

\section{Variants of Big $\Oh$ Notation}
The big $\Oh$ notation is useful if we want to express that some function $f$ does not grow faster
than another function $g$.  Therefore, when stating the running time of the worst case of some algorithm,
big $\Oh$ notation is the right tool to use.  However, sometimes we want to state a lower bound for
the complexity of a problem.  For example, it can be shown that every comparison based sort algorithm needs at least
$n \cdot \log_2(n)$ comparisons to sort a list of length $n$.  In order to be able to express lower
bounds concisely, we introduce the big $\Omega$ notation next.

\begin{Definition}[$\Omega(g)$] 
  Assume $g \in \R_+^{\N}$ is given.   Let us define the set of all functions that grow 
  \underline{at least as fast as} the function $g$ as follows:
  \\[0.2cm]
  \hspace*{0.5cm} 
  \colorbox{orange}{
  $ \Omega(g) \;:=\; \Bigl\{ f \in \R_+^{\;\N} \mid \exists k \in \N \colon 
    \exists c \in \R_+\colon \forall n \in \N \colon\bigl(n \geq k \rightarrow c \cdot g(n) \leq f(n)\bigr)\Bigr\}$.}
  \eox
\end{Definition}

It is not difficult to show that
\\[0.2cm]
\hspace*{1.3cm}
 $f \in \Omega(g)$ \quad if and only if \quad $g \in \Oh(f)$.
\\[0.2cm]
Finally, we introduce big $\Theta$ notation.  The idea is that $f \in \Theta(g)$ if 
$f$ and $g$ have the same asymptotic growth rate. 

\begin{Definition}[$\Theta(g)$]
  Assume $g \in \R_+^{\N}$ is given.   The set of functions that have the same asymptotic growth rate
  as the function $g$ is defined as
  \\[0.2cm]
  \hspace*{0.5cm} 
  \colorbox{orange}{
  $ \Theta(g) \;:=\; \Oh(g) \cap \Omega(g)$.}
  \qed 
\end{Definition}

\noindent
It can be shown that $f \in \Theta(g)$ if and only if the limit
\\[0.4cm]
\hspace*{1.3cm}
$\lim\limits_{n \rightarrow \infty} \bruch{f(n)}{g(n)}$
\\[0.2cm]
exists and is greater than $0$.
\vspace*{0.3cm}

Sedgewick \cite{sedgewick:11} claims that the $\Theta$ notation is too imprecise and advocates the
tilde notation instead.  For two functions $f,g : \mathbb{N} \rightarrow \mathbb{R}_+$ he defines
\\[0.3cm]
\hspace*{1.3cm}
$f \sim g$ \quad iff \quad $\lim\limits_{n \rightarrow \infty} \bruch{f(n)}{g(n)} = 1$.
\\[0.2cm]
To see why this is more precise, let us consider the case of two algorithms $A$ and $B$ for sorting a list of length
$n$.   Assume that the number $\textsl{count}_A(n)$ of comparisons used by algorithm $A$ to sort a list of
length $n$ is given as
\\[0.2cm]
\hspace*{1.3cm}
$\textsl{count}_A(n) = n \cdot \log_2(n) + 7 \cdot n$,
\\[0.2cm]
while for algorithm $B$ the corresponding number of comparisons is given as
\\[0.2cm]
\hspace*{1.3cm}
$\textsl{count}_B(n) = \frac{3}{2} \cdot n \cdot \log_2(n) + 4 \cdot n$.
\\[0.2cm]
Clearly, if $n$ is big then algorithm $A$ is better than algorithm $B$ but as we have pointed out in a previous
section, the big $\Oh$ notation is not able to distinguish between the complexity of algorithm $A$ and
algorithm $B$.  However we have that
\\[0.2cm]
\hspace*{1.3cm}
$\frac{3}{2} \cdot \textsl{count}_A(n) \sim \textsl{count}_B(n)$
\\[0.2cm]
and this clearly shows that for big values of $n$, algorithm $A$ is faster than algorithm $B$ by a
factor of $\frac{3}{2}$.



\section{Further Reading}
Chapter 3 of the book ``\emph{Introduction to Algorithms}'' by Cormen et.~al.~\cite{cormen:09}
contains a detailed description of several variants of the big $\Oh$ notation, while chapter 4 gives a more
general version of the master theorem together with a detailed proof.

%%% Local Variables: 
%%% mode: latex
%%% TeX-master: "algorithms"
%%% End: 
