
\section[LZW Algorithm$^*$]{The Algorithm  of Lempel, Ziv, and Welch$^*$}
The algorithm developed by Abraham Lempel, Jacob Ziv \cite{ziv:77,ziv:78} and Terry A.~Welch
\cite{welch:84}, which is also known as the \blue{LZW algorithm}, is based on the idea that in most
texts certain combinations of letters are quite frequent.  Therefore, it should pay of to view
these combinations of letters as new letters and insert them into the alphabet.  This is the main
idea of the LZW algorithm.  However, since counting the occurrences of all words would be too time
consuming, the LZW algorithm works with a \blue{dynamic} coding dictionary.  Initially, this dictionary
contains only the \textsc{Ascii} characters.  Then, the idea is to extend this dictionary
dynamically: Every time a new string is encountered, it is entered into the dictionary and a code is
assigned to the corresponding string.  However, since it would not make sense to add arbitrary
strings to the dictionary, a new string $s$ of length $n=\#s$ is only added to the dictionary if
\begin{enumerate}
\item $s$ is a substring of the string that is encoded and
\item the substring $s[1..n-1]$ has already been entered into the dictionary.  
\end{enumerate} 
The algorithm is best
explained via an example.  The basic working of the algorithm is explained with the help of four
variables:
\begin{enumerate}
\item $\alpha$ is the last substring that has been encoded.  Initially, this is the empty string
      $\varepsilon$.
  
      The encoding of a string $s$ by the LZW algorithm works by encoding substrings of $s$ as
      numbers and $\alpha$ denotes the last of theses substrings.
\item $c$ is the next character of the string that is inspected.  This is also known as the
      \blue{look-ahead character}.
\item \textsl{d} is the dictionary mapping strings to numbers.  Initially, \textsl{d} maps all
      \textsc{Ascii} characters to their respective \textsc{Ascii} codes.
\item \texttt{nextCode} is the number assigned as code to the next string that is entered into
      the dictionary $d$.  Since the \textsc{Ascii} codes are the numbers from 0 up to 127,
      initially \texttt{nextCode} is equal to $128$.
\end{enumerate}
To describe the working of the algorithm, let us encode the string ``\texttt{maumau}''.
\begin{enumerate}
\item Initially, we have
      \\[0.2cm]
      \hspace*{1.3cm}
      $\alpha = \varepsilon$ \quad and \quad $c = \texttt{m}$.
      \\[0.2cm]
      Since the \textsc{Ascii} code of the character ``\texttt{m}'' is $109$, we output this number.
\item After reading the next character ``\texttt{a}'' we have
      \\[0.2cm]
      \hspace*{1.3cm}
      $\alpha = \texttt{m}$ \quad and \quad $c = \texttt{a}$.
      \\[0.2cm]
      Now, the substring $\alpha c$, which is ``\texttt{ma}'', is entered into the dictionary and
      assigned to the code $128$:
      \\[0.2cm]
      \hspace*{1.3cm}
      $d = d \cup \{\langle \texttt{ma}, 128 \rangle\}$.
      \\[0.2cm]
      Furthermore, we output the \textsc{Ascii} code of ``\texttt{a}'', which is $97$.
\item After reading the next character ``\texttt{u}'' we have
      \\[0.2cm]
      \hspace*{1.3cm}
      $\alpha = \texttt{a}$ \quad and \quad $c = \texttt{u}$.
      \\[0.2cm]
      Now, the substring $\alpha c$, which is ``\texttt{au}'', is entered into the dictionary and
      assigned to the next available code, which is $129$:
      \\[0.2cm]
      \hspace*{1.3cm}
      $d = d \cup \{\langle \texttt{au}, 129 \rangle\}$.
      \\[0.2cm]
      Furthermore, we output the \textsc{Ascii} code of ``\texttt{u}'', which is $117$.
\item After reading the next character, which is the character ``\texttt{m}'', we have
      \\[0.2cm]
      \hspace*{1.3cm}
      $\alpha = \texttt{u}$ \quad and \quad $c = \texttt{m}$.
      \\[0.2cm]
      Next, the substring $\alpha c$, which is ``\texttt{um}'', is entered into the dictionary and
      assigned to the next available code, which is $130$:
      \\[0.2cm]
      \hspace*{1.3cm}
      $d = d \cup \{\langle \texttt{um}, 130 \rangle\}$.
      \\[0.2cm]
      Since our dictionary already contains the substring ``\texttt{ma}'' and the character
      ``\texttt{a}'' is indeed the character following the character ``\texttt{m}'', we output
      $128$, which is the code assigned to the string ``\texttt{ma}''.

\item The next character to be read is now the final character ``\texttt{u}''.  We have
      \\[0.2cm]
      \hspace*{1.3cm}
      $\alpha = \texttt{ma}$ \quad and \quad $c = \texttt{u}$.
      \\[0.2cm]
      Next, the substring $\alpha c$, which is ``\texttt{mau}'', is entered into the dictionary and
      assigned to the next available code, which is $131$:
      \\[0.2cm]
      \hspace*{1.3cm}
      $d = d \cup \{\langle \texttt{mau}, 131 \rangle\}$.
      \\[0.2cm]
      Furthermore, we output the \textsc{Ascii} code of ``\texttt{u}'', which is $117$.
\end{enumerate} 
Putting everything together, we have coded the string ``\texttt{maumau}'' as the list
\\[0.2cm]
\hspace*{1.3cm}
$[109,97,117,128,117]$
\\[0.2cm]
If we had encoded this string in \textsc{Ascii} we would have used $6 \cdot 7 = 42$ bits.  Since the
dictionary that we have built on the fly uses codes starting at 128 we now have to use 8 bits to
encode the numbers.  However, we have only used 5 numbers to encode the string ``\texttt{maumau}''.
Hence we have only used $5 \cdot 8 = 40$ bits.   Of course, in this tiny example the compression
factor is quite low.  However, for texts that are longer and have more repetitions, the compression
factor is usually higher: On average, the experience shows that text corresponding to natural
language is compressed by a factor that is slightly bigger than $2$.

If we use the LZW algorithm there is no need to add the dictionary to the encoded string.  The
reason is that the recipient of an encoded string can construct the dictionary using exactly the
algorithm that is used when encoding the string.

Let us summarize the algorithm seen in the previous example:
\begin{enumerate}
\item The dictionary is initialized to map all \textsc{Ascii} characters to their \textsc{Ascii} codes.
\item Next, we search for the longest prefix $\beta$ of $s$ that is in the dictionary.  This prefix
      is removed from $s$.
\item We emit the code stored for $\beta$ in the dictionary.
\item Let $\alpha$ be the string that has been encoded in the previous step.  Append the first
      character $c$ of $\beta$ to $\alpha$ and enter the resulting string $\alpha c$ to the
      dictionary.

      This step expands the dictionary dynamically.
\item Go to step 2 and repeat as long as the string $s$ is not empty.
\end{enumerate}
Decoding a list of numbers $l$ into a string $s$ is quite similar to the encoding and works as follows.
\begin{enumerate}
\item This time, the dictionary is initialized to map all \textsc{Ascii} codes to their corresponding
      \textsc{Ascii} characters.  Hence, the dictionary constructed in this step is just the inverse
      of the dictionary constructed when starting to encode the string.
\item We initialize $s$ as the 
      empty string, which is denoted as $\varepsilon$:
      \\[0.2cm]
      \hspace*{1.3cm}
      $s := \varepsilon$.
\item We remove the first number $n$ from the list $l$ and look up the corresponding
      string $\beta$ in the dictionary.  This string is appended to $s$.
\item Assume that $\alpha$ is the string decoded in the previous iteration and that $c$ is the first
      character of $\beta$.  Enter the resulting string $\alpha c$ into the dictionary.
\item Goto step 2 and repeat as long as the list $l$ is not empty.
\end{enumerate}
The third step of this algorithm needs to refined:  The problem is
that it might happen that the dictionary does not have an entry for the number $n$.  This can occur because
the encoder is one step ahead of the decoder: The encoder encodes a substring and enters a code
corresponding to the previous substring into the dictionary.  Now if the next substring is identical
to the substring just entered, the encoder will produce a code that is not yet in the dictionary of
the decoder when he tries to decode it.   The question then is: How do we decode a number that has
not yet been entered into the dictionary.  To answer this question, we can reason as follows:
If the encoder outputs a code that it has just entered into the dictionary, then the string that is
encoded starts with the string that has been output previously, followed by some character.  However,
this character must be the first character of the string encoded now.  The string encoded now
corresponds to the code and hence this string is the same as the string previously decoded plus one
character. Therefore, if the previous string is $\alpha$, then the string
corresponding to an unknown code must be $\alpha \alpha[1]$, i.e. $\alpha$ followed by the first
character of $\alpha$.



\subsection{Implementing the LZW algorithm in \textsc{SetlX}}
In order to gain a better understanding of a complex algorithm it is best to code this algorithm.
Then the resulting program can be run on several examples.  Since humans tend
to learn better from examples than from logical reasoning, inspecting these examples deepens
the understanding of the algorithm.  We proceed to discuss an implementation of the LZW
algorithm.  

\begin{figure}[!ht]
\centering
\begin{Verbatim}[ frame         = lines, 
                  framesep      = 0.3cm, 
                  firstnumber   = 1,
                  labelposition = bottomline,
                  numbers       = left,
                  numbersep     = -0.2cm,
                  xleftmargin   = 0.8cm,
                  xrightmargin  = 0.8cm,
                ]
    class lzw() {
        mDictionary := { [ char(i), i ] : i in [32 .. 127] };
        mInverse    := { [ i, char(i) ] : i in [32 .. 127] };
        mNextCode   := 128;
    
        static {
            compress      := procedure(s)     { ... };
            uncompress    := procedure(l)     { ... };
            longestPrefix := procedure(s, i)  { ... };
        }
    }
\end{Verbatim}
\vspace*{-0.3cm}
\caption{Outline of the class \texttt{lzw}.}
\label{fig:lzw.stlx-outline}
\end{figure}


Figure \ref{fig:lzw.stlx-outline} shows the outline of the class \texttt{lzw}.  This class contains both the
method \texttt{compress} that takes a string $s$ and encodes this string into a list of numbers
and the method \texttt{uncompress} that takes a list of numbers $l$ and decodes this list back into
a string $s$.  These methods are designed to satisfy the following specification:
\\[0.2cm]
\hspace*{1.3cm}
$l = \texttt{lzw().compress}(s_1) \wedge s_2 = \texttt{lzw().uncompress}(l) \rightarrow s_1 = s_2$.
\\[0.2cm]
Furthermore, the class \texttt{lzw} contains the auxiliary method \texttt{longestPrefix}, which will
be discussed later.  The class \texttt{lzw} contains 3 member variables:
\begin{enumerate}
\item \texttt{mDictionary} is the dictionary used when encoding a string.  It is initialized to map
      the \textsc{Ascii} characters to their codes.  Remember that for a given number $i$, the
      expression $\texttt{char}(i)$ returns the \textsc{Ascii} character with code $i$.
\item \texttt{mInverse} is a binary relation that associates the codes with the corresponding
      strings.  It is initialized to map every number in the set $\{ 0, 1, 2, \cdots, 127 \}$
      with the corresponding \textsc{Ascii} character.  The binary relation \texttt{mInverse} is the
      inverse of the relation \texttt{mDictionary}.
\item \texttt{mNextCode} gives the value of the next code used in the dictionary.  Since the codes
      up to and including $127$ are already used for the \textsc{Ascii} character, the next
      available code will be $128$.
\end{enumerate}

\begin{figure}[!ht]
\centering
\begin{Verbatim}[ frame         = lines, 
                  framesep      = 0.3cm, 
                  firstnumber   = 1,
                  labelposition = bottomline,
                  numbers       = left,
                  numbersep     = -0.2cm,
                  xleftmargin   = 0.8cm,
                  xrightmargin  = 0.8cm,
                ]
    compress := procedure(s) {
        result := [];
        idx    := 1;
        while (idx <= #s) {
            p := longestPrefix(s, idx);
            result += [ mDictionary[s[idx..p]] ];
            if (p < #s) {
                mDictionary[s[idx..p+1]] := mNextCode;
                this.mNextCode += 1;
            }
            idx := p + 1;
        }
        return result;
    };
\end{Verbatim}
\vspace*{-0.3cm}
\caption{The method \texttt{compress} encodes a string as a list of integers.}
\label{fig:lzw.stlx-compress}
\end{figure}
Figure \ref{fig:lzw.stlx-compress} shows the implementation of the method compress.  We discuss this
implementation line by line.
\begin{enumerate}
\item The variable \texttt{result} points to the list that encodes the string $s$ given as argument.
      Initially, this list is empty.  Every time a substring of $s$ is encoded, the corresponding code
      is appended to this list.
\item The variable \texttt{idx} is an index into the string $s$.  The idea is that the substring
      $s[1..\texttt{idx}-1]$ has been encoded and the corresponding codes have already been written
      to the list \texttt{result}, while the substring $s[\texttt{idx}..]$ is
      the part of $s$ that still needs to be encoded.
\item Hence, the \texttt{while}-loop runs as long as the index \texttt{idx} is less or equal than
      the length $\texttt{\#}s$ of the string $s$.
\item Next, the method \texttt{longestPrefix}  computes the index of longest prefix of the substring
      $s[\texttt{idx}..]$ that can be found in the dictionary \texttt{mDictionary}, i.e.~$p$ is the
      maximal number such that the expression \texttt{mDictionary[s[idx..p]]} is defined.
\item The code corresponding to this substring is looked up in \texttt{mDictionary}
      and is then appended to the list \texttt{result}.
\item Next, we take care to maintain the dictionary \texttt{mDictionary} and add the substring
      $s[\texttt{idx}..p+1]$ to the dictionary.  Of course, we can only do this if the upper index
      of this expression, which is $p+1$, is an index into the string $s$. 
      Therefore we have to check that $p < \texttt{\#}s$.
      Once we have entered the
      new string with its corresponding code into the dictionary, we have to make sure that the
      variable \texttt{mNextCode} is incremented so that every string is associated with a unique
      code.  
\item Since the code corresponding to the substring $s[\texttt{idx}..p]$ has been written to the
      list \texttt{result}, the index \texttt{idx} is set to $p+1$.
\item Once the while loop has terminated, the string $s$ has been completely encoded and the list
      containing the codes can be returned.
\end{enumerate}


\begin{figure}[!ht]
\centering
\begin{Verbatim}[ frame         = lines, 
                  framesep      = 0.3cm, 
                  firstnumber   = 1,
                  labelposition = bottomline,
                  numbers       = left,
                  numbersep     = -0.2cm,
                  xleftmargin   = 0.8cm,
                  xrightmargin  = 0.8cm,
                ]
    longestPrefix := procedure(s, i) {
       oldK := i;
       k    := i+1;
       while (k <= #s && mDictionary[s[i..k]] != om) {
           oldK := k;
           k    += 1;
       }
       return oldK;
    };
    incrementBitNumber := procedure() {
        if (2 ** mBitNumber <= mNextCode) {
            this.mBitNumber += 1;
        }
    };
\end{Verbatim}
\vspace*{-0.3cm}
\caption{Computing the longest prefix.}
\label{fig:lzw.stlx-longestPrefix}
\end{figure}
Figure \ref{fig:lzw.stlx-longestPrefix} show the implementation of the auxiliary function
\texttt{longestPrefix}.  
The function $\texttt{longestPrefix}(s, i)$ computes the maximum value of $k$ such that
\\[0.2cm]
\hspace*{1.3cm}
$i \leq k \wedge k \leq \texttt{\#}s \wedge \texttt{mDictionary}[s[i..k]] \not= \Omega$.
\\[0.2cm]
This value is well defined since the dictionary is initialized to contain all strings of
length 1.  Therefore, $\texttt{mDictionary}[s[i..i]]$ is known to be defined: It is the
\textsc{Ascii} code of the character $s[i]$.
      
The required value is computed by a simple \texttt{while}-loop that tests all possible values of $k$.
The loop exits once the value of $k$ is too big.  Then the previous value of $k$, which is
stored in the variable \texttt{oldK} is returned as the result.




\begin{figure}[!ht]
\centering
\begin{Verbatim}[ frame         = lines, 
                  framesep      = 0.3cm, 
                  firstnumber   = 1,
                  labelposition = bottomline,
                  numbers       = left,
                  numbersep     = -0.2cm,
                  xleftmargin   = 0.8cm,
                  xrightmargin  = 0.8cm,
                ]
    uncompress := procedure(l) {
        result := "";
        idx    := 1;
        code   := l[idx]; 
        old    := mInverse[code];
        idx    += 1;
        while (idx < #l) {
            result += old;
            code := l[idx];
            idx  += 1;
            next := mInverse[code];
            if (next == om) {
                next := old + old[1];
            }
            mInverse[mNextCode] := old + next[1];
            this.mNextCode += 1;
            old := next;
        }
        result += old;
        return result;
    };
\end{Verbatim}
\vspace*{-0.3cm}
\caption{The method \texttt{uncompress} to decode a list of integers into a string.}
\label{fig:lzw.stlx-uncompress}
\end{figure}
Figure \ref{fig:lzw.stlx-uncompress} shows the implementation of the method \texttt{uncompress} that
takes a list of numbers and decodes it into a string $s$.
\begin{enumerate}
\item The variable \texttt{result} contains the decoded string.  Initially, this variable is empty.
      Every time a code of the list $l$ is deciphered into some string, this string is added to
      \texttt{result}.
\item The variable \texttt{idx} is an index into the list $l$.  It points to the next code that
      needs to be deciphered.
\item The variable \texttt{code} contains the code in $l$ at position \texttt{idx}.  Therefore, we
      always have
      \\[0.2cm]
      \hspace*{1.3cm}
      $l[\texttt{idx}] = \texttt{code}$
\item The variable \texttt{old} contains the substring associated with \texttt{code}.  Therefore,
      the invariant
      \\[0.2cm]
      \hspace*{1.3cm}
      $\texttt{mInverse}[\texttt{code}] = \texttt{old}$
      \\[0.2cm]
      is maintained.
\item As long as the index \texttt{idx} still points inside the list, the substring 
      that has just been decoded is appended to the string \texttt{result}.
\item Then, an attempt is made to decode the next number in the list $l$ by looking up the code
      in the dictionary \texttt{mInverse}.  
      
      Now there is one subtle case: If the \texttt{code} has not yet been defined in the
      dictionary,  then we can conclude that this code has been created when coding the
      substring \texttt{old} followed by some character $c$.  However, as the next substring $\beta$
      corresponds to this code, the character $c$ must be the first
      character of this substring, i.e.~we have
      \\[0.2cm]
      \hspace*{1.3cm}
      $c = \beta[1]$.
      \\[0.2cm]
      On the other hand, we know that the substring $\beta$ has the form
      \\[0.2cm]
      \hspace*{1.3cm}
      $\beta = \texttt{old} + c$,
      \\[0.2cm]
      where the operator ``$+$'' denotes string concatenation.  But then the first character of this string
      must be the first character of \texttt{old}, i.e.~we have
      \\[0.2cm]
      \hspace*{1.3cm}
      $\beta[1] = \texttt{old}[1]$
      \\[0.2cm]
      and hence we have shown that
      \\[0.2cm]
      \hspace*{1.3cm}
      $c = \texttt{old}[1]$.
      \\[0.2cm]
      Therefore, we conclude
      \\[0.2cm]
      \hspace*{1.3cm}
      $\beta = \texttt{old} + \texttt{old}[1]$
      \\[0.2cm]
      and hence this is the string encoded by a code that is not yet defined in the dictionary
      \texttt{mInverse}.
\item Next, we need to maintain the dictionary \texttt{mInverse} in the same fashion as the
      dictionary \texttt{mDictionary} is maintained in the method \texttt{compress}:
      Hence we take the string previously decoded and concat the next character of the
      string decoded in the current step.  Of course, this string is
      \\[0.2cm]
      \hspace*{1.3cm}
      $\texttt{old} + \texttt{next}[1]$
      \\[0.2cm]
      and this string is then associated with the next available code value.
\item At the end of the loop, we need to set \texttt{old} to \texttt{next} so that \texttt{old}
      will always contain the string decoded in the previous step.
\item When the \texttt{while}-loop has terminated, we still need to append the final value of \texttt{old}
      to the variable \texttt{result}.
\end{enumerate}
Now that we have discussed the implementation of the \texttt{LZW} algorithm I would like to
encourage you to test it on several examples that are not too long.  Time does not permit me
to discuss examples of this kind in these lecture notes and, indeed, I do not think that discussing
these examples here would be as beneficial for the student as performing the algorithm on their own.

\exercise
\begin{enumerate}[(a)]
\item Use the LZW algorithm to encode the string ``\texttt{abcabcabcabc}''.  Compute the compression
      factor for this string.
\item For all $n \in \mathbb{N}$ with $n \geq 1$ the string $\alpha_n$ is defined inductively as
      follows:
      \\[0.2cm]
      \hspace*{1.3cm} $\alpha_1 := \texttt{a}$ \quad and \quad $\alpha_{n+1} = \alpha_n + \texttt{a}$.
      \\[0.2cm]
      Hence, the string $\alpha_n$ has the form $\underbrace{\texttt{a} \cdots \texttt{a}}_n$,
      i.e. it is the character \texttt{a} repeated $n$ times.
      Encode the string $\alpha_n$ using the LZW algorithm.  What is the compression rate?
\item Decode the list 
      \\[0.2cm]
      \hspace*{1.3cm}
      $[97, 98, 128, 130]$
      \\[0.2cm]
      using the LZW algorithm.  \eox
\end{enumerate}

%%% Local Variables:
%%% mode: latex
%%% TeX-master: "algorithms"
%%% End:
