\chapter{Recurrence Equations}
In some cases knowing the growth rate of a function is not sufficient.  Rather, we want to be able to compute
the growth rate exactly.  If the growth rate to be computed results from the complexity analysis of a recursive
algorithm, the growth rate is often defined via a
\href{https://en.wikipedia.org/wiki/Recurrence_relation}{recurrence equation}.  In the literature,
\emph{recurrence equations} are also known as \blue{recurrence relations} or
\blue{difference equations}\index{difference equation}.  I will call them both recurrence relations and
recurrence equations interchangeably.

\section{The Fibonacci Numbers}
In this section we discuss a simple program whose complexity analysis
leads to a recurrence equation for the  
number of additions.  We will then see how this recurrence equation can be solved explicitly.  The program that
we will analyse is shown in Figure \ref{fig:Fibonacci.ipynb}.  The function $\mytt{fib}(n)$ computes the
\href{https://en.wikipedia.org/wiki/Fibonacci_number}{Fibonacci number} \index{Fibonacci number} $F_n$.  These
numbers are defined inductively as follows:
\begin{enumerate}
\item $F_0 := 0$,
\item $F_1 := 1$, 
\item $F_{n+2} = F_{n+1} + F_n$.

  This last equation is an example of a \blue{recurrence equation}, while the first two equations are the
  \blue{initial conditions}. 
\end{enumerate}

\begin{figure}[!h]
  \centering
\begin{python3code} 
    def fib(n):
        if n <= 1:
            return n
        return fib(n-1) + fib(n-2)
\end{python3code}
\vspace*{-0.7cm}
  \caption{A \textsl{Python} program to compute the Fibonacci numbers.}
  \label{fig:Fibonacci.ipynb}
\end{figure} 

If we run the program shown in Figure \ref{fig:Fibonacci.ipynb} , we will find that the runtime grows very rapidly with
the input parameter $n$.  To analyse this phenomenon, we investigate the number of additions that are used in the
calculation of $\mytt{fib}(n)$ for a given $n \in \mathbb{N}$.  If we call this number $a_n$, we find
the following:
\begin{enumerate}
\item $a_0 = 0$,
\item $a_1 = 0$,
\item $n \geq 2 \rightarrow a_n = a_{n-1} + a_{n-2} + 1$,

      because in the recursive calls $\mytt{fib}(n-1)$ and $\mytt{fib}(n-2)$ we have, respectively, 
      $a_{n-1}$ and $a_{n-2}$ additions and, furthermore, the expression
      \mytt{fib(n-1) + fib(n-2)} adds another addition.
\end{enumerate}
We take the equation $a_n = a_{n-1} + a_{n-2} + 1$ and substitute $i+2$ for $n$.  This yields
\\[0.2cm]
\hspace*{1.3cm} $a_{i+2} = a_{i+1} + a_i + 1$ \hspace*{\fill} $(1)$
\\[0.2cm]
This type of equation is called a \blue{linear \green{inhomogeneous} recurrence equation}.  The expression $+ 1$ is
called the \blue{inhomogeneity} of this recurrence equation.
\index{linear inhomogeneous recurrence equation}
The \blue{associated homogeneous recurrence equation} \index{associated homogeneous recurrence equation}
results from dropping this term in equation $(1)$ and has the form
\\[0.2cm]
\hspace*{1.3cm}
$a_{i+2} = a_{i+1} + a_i$. \hspace*{\fill} $(2)$
\\[0.2cm]
We solve this equation with the following \href{https://en.oxforddictionaries.com/definition/ansatz}{ansatz}:
\\[0.2cm]
\hspace*{1.3cm} $a_i = \lambda^i$. \\[0.2cm]
Substituting this ansatz into equation $(2)$ leads to the equation 
\\[0.2cm]
\hspace*{1.3cm}
$\lambda^{i+2} = \lambda^{i+1} + \lambda^i$.
\\[0.2cm]
If we divide both sides of this equation by $\lambda^i$, we get the
\href{https://en.wikipedia.org/wiki/Quadratic_equation}{quadratic equation}
\\[0.2cm]
\hspace*{1.3cm}
$\lambda^2 = \lambda + 1$,
\\[0.2cm]
which we solve by \href{https://en.wikipedia.org/wiki/Completing_the_square}{completing the square}:
\\[0.2cm]
\hspace*{1.3cm}
$\begin{array}{lrcll}
                &  \lambda^2 & = & \lambda + 1 & |\;- \lambda \\[0.2cm]
\Leftrightarrow & \ds \lambda^2 - 2 \cdot \frac{1}{2} \cdot\lambda & = & 1 & |\;\ds+ \frac{1}{4} \\[0.4cm]
\Leftrightarrow & \ds \lambda^2 - 2 \cdot \frac{1}{2} \cdot\lambda + \Big(\frac{1}{2}\Big)^2 & = & \ds\frac{5}{4}
                & | \;\; \mbox{completing the square}
                  \\[0.4cm]
\Leftrightarrow & \ds \Big(\lambda -\frac{1}{2}\Big)^2 & = & \ds\frac{5}{4}         & | \;\;\sqrt{\;\;} \\[0.4cm]
\Leftrightarrow & \ds \lambda -\frac{1}{2} & = & \ds\pm\frac{\sqrt{5}}{2} & | \ds\;+ \frac{1}{2} \\[0.4cm]
\Leftrightarrow & \ds \lambda_{1/2} & = & \ds\frac{1}{2} \cdot \bigl(1 \pm \sqrt{5}\bigr) & 
 \end{array}
 $
\\[0.2cm]
We note that any \blue{linear combination} of the form
\\[0.2cm]
\hspace*{1.3cm}
$a_n = \alpha \cdot \lambda_1^n + \beta \cdot \lambda_2^n$
\\[0.2cm]
is a solution to the homogeneous recurrence equation $(2)$.
We also note that the following identities hold for the values $\lambda_1$ and $\lambda_2$:
\\[0.2cm]
\hspace*{1.3cm} 
$\lambda_1 - \lambda_2 = \sqrt{5}$ \quad and \quad $\lambda_1 + \lambda_2 = 1$. \hspace*{\fill} (3)
\\[0.2cm]
These equations will be needed later.
From the last equation it follows immediately that \\[0.2cm]
\hspace*{1.3cm} $1 - \lambda_1 = \lambda_2$ \quad and \quad $1 - \lambda_2 = \lambda_1$. \hspace*{\fill} (4)
\\[0.1cm]
To solve the original recurrence equation (1) we try the ansatz
$a_i = c$, where $c$ is an unknown constant that is to be determined.  This leads us to the equation \\[0.2cm] 
\hspace*{1.3cm}
$c = c + c + 1$,
\\[0.2cm]
which has the solution $c = -1$.  Hence
\\[0.2cm]
\hspace*{1.3cm}
$a_n = -1$ 
\\[0.2cm]
is a solution of the inhomogeneous recurrence equation $(1)$. We call this solution a \blue{special solution}.
\index{special solution}  Although this is a solution of the recurrence equation $a_{i+2} = a_{i+1} + a_i + 1$,
it does not satisfy the initial conditions $a_0 = 0$ and $a_1 = 0$.
The \blue{general solution} \index{general solution} of the recurrence equation (1) 
is the sum of the general solution of the homogeneous recurrence equation and the special
solution and therefore reads 
\[ a_i = \alpha \cdot \lambda_1^i + \beta \cdot \lambda_2^i - 1 \]
where $\lambda_1 = \frac{1}{2}\cdot (1 + \sqrt{5})$ and $\lambda_2 = \frac{1}{2}\cdot (1 - \sqrt{5})$.
The coefficients $\alpha$ and $\beta$ have to be determined so that the
initial conditions $a_0 = 0$ and $a_1 = 0$ are met.  This leads to the following
\blue{system of linear equations}: 
\\[0.2cm]
\hspace*{1.3cm}
$\begin{array}{lcl}
    0 & = & \alpha \cdot \lambda_1^0 + \beta \cdot \lambda_2^0 - 1 \\[0.1cm]
    0 & = & \alpha \cdot \lambda_1^1 + \beta \cdot \lambda_2^1 - 1 \\
  \end{array}
  $
\\[0.2cm]
Let's add 1 to both equations and simplify the powers $\lambda_i^0$ to $1$ for $i=1,2$ and
$\lambda_i^1$ to $\lambda_i$.  This results in the following system of linear equations:
\\[0.2cm]
\hspace*{1.3cm}
$ \begin{array}{lcl}
    1 & = & \alpha + \beta \\[0.1cm]
    1 & = & \alpha \cdot \lambda_1 + \beta \cdot \lambda_2 
  \end{array}
$
\\[0.2cm]
The first of these two equations yields
\\[0.2cm]
\hspace*{1.3cm}
$\alpha = 1 - \beta$.
\\[0.2cm]
If we insert this value of $\alpha$  into the second equation, we get 
\\[0.2cm]
\hspace*{1.3cm}
$
\begin{array}{clcl}
                      &  1 & = & (1 - \beta)\cdot  \lambda_1 + \beta \cdot \lambda_2 \\[0.2cm]
\Leftrightarrow\quad  &  1 & = & 
 \lambda_1  + \beta \cdot \bigl( \lambda_2 - \lambda_1\bigr) \\[0.2cm]
\Leftrightarrow\quad  &  1 - \lambda_1 & = & \beta \cdot \bigl(\lambda_2 - \lambda_1\bigr)  \\[0.2cm]
\Leftrightarrow\quad  &  \bruch{1 - \lambda_1}{\lambda_2 - \lambda_1} & = & \beta 
\end{array}
$
\\[0.2cm]
Because of $\alpha = 1 - \beta$ we find  \\[0.2cm]
\hspace*{1.3cm} $\alpha = - \bruch{1 - \lambda_2}{\lambda_2 - \lambda_1}$. \\[0.2cm]
If we use the equations $\lambda_1 - \lambda_2 = \sqrt{5}$ and $\lambda_1 + \lambda_2 = 1$ next, we arrive at
\\[0.2cm]
\hspace*{1.3cm} 
$\alpha = \bruch{\lambda_1}{\sqrt{5}} $.
\\[0.2cm]
As we have
\\[0.2cm]
\hspace*{1.3cm}
$\beta =  \bruch{1 - \lambda_1}{\lambda_2 - \lambda_1}$
\\[0.2cm]
we conclude that
\\[0.2cm]
\hspace*{1.3cm}
$\beta = -\bruch{\lambda_2}{\sqrt{5}}$. 
\\[0.2cm]
Therefore the \blue{solution} of our recurrence equation is: \\[0.2cm]
\hspace*{1.3cm}
  \colorbox{red}{\framebox{\colorbox{orange}{
        $\ds a_i = \bruch{1}{\sqrt{5}} \cdot \left( \lambda_1^{i+1} - \lambda_2^{i+1} \right) - 1$ \quad
        \mbox{for all $i\in\mathbb{N}$.}
  }}}
\\[0.2cm]
Because of $\lambda_1\approx 1.61803$ and $\lambda_2 \approx - 0.61803$ the first term
of this sum dominates the second term and the number of additions increases exponentially with the factor $\lambda_1$.
This explains the strong increase in computing time.
\vspace*{0.1cm}

\remark 
The number $\lambda_1$ is also called the \href{https://en.wikipedia.org/wiki/Golden_ratio}{golden ratio} 
and plays an important role in geometry, alchemy, and the dark arts. \eoxs

\section{Second Order Recurrence Equations: The General Case}
In the remainder of this section we will define the notion of a \blue{linear inhomogeneous recurrence equation}
of the second order and describe a general method to solve this kind of recurrence equations.

\begin{Definition}[Second Order Linear Inhomogeneous Recurrence Equation] \hspace*{\fill} \\
  A \blue{second order linear inhomogeneous recurrence equation}
  \index{linear inhomogeneous recurrence equation}
  has the form
  \\[0.2cm]
  \hspace*{1.3cm}
  $x_{n+2} = a \cdot x_{n+1} + b \cdot x_n + c$ \quad and \quad $x_0 = d$, $x_1 = e$
  \quad where $a, b, c, d, e \in \mathbb{R}$.
  \\[0.2cm]
  Here, $x_0 = d$, $x_1 = e$ are the \blue{initial conditions}.\index{initial conditions}  Taken together, the recurrence
  equation and the initial conditions define a sequence $(x_n)_{n\in\mathbb{N}}$ of real numbers.  \eox
\end{Definition}

\noindent
Our aim is to derive an expression $f(n)$ such that $f(n) = x_n$ for all $n\in\mathbb{N}$.  An expression of
this kind is then called a \blue{solution} of the given recurrence in \blue{closed terms}.  For example, it can
be shown, that the expression
\\[0.2cm]
\hspace*{1.3cm}
$\ds \bruch{1}{\sqrt{5}} \cdot \left(\biggl(\bruch{1+\sqrt{5}}{2}\biggr)^{n} - \biggl(\bruch{1-\sqrt{5}}{2}\biggr)^{n+1}\right)$ \quad
\\[0.2cm]
is a solution of the recurrence equation 
\\[0.2cm]
\hspace*{1.3cm}
$x_{n+2} = x_{n+1} + x_n$ \quad and \quad $x_0 = 0$, $x_1 = 0$.
\\[0.2cm]
Our method to \blue{solve} a second order linear inhomogeneous recurrence equation consists of three steps.
\begin{enumerate}
\item We use the ansatz $x_n = \lambda^n$ to solve the \blue{associated homogeneous recurrence equation}
      \\[0.2cm]
      \hspace*{1.3cm}
      $x_{n+2} = a \cdot x_{n+1} + b \cdot x_n$.
      \\[0.2cm]
      Substituting this ansatz into this homogeneous recurrence equation yields the equation
      \\[0.2cm]
      \hspace*{1.3cm}
      $\lambda^{n+2} = a \cdot \lambda^{n+1} + b \cdot \lambda^n$.
      \\[0.2cm]
      Dividing both sides of this equation by $\lambda^n$ yields the quadratic equation
      \\[0.2cm]
      \hspace*{1.3cm}
      $\lambda^{2} = a \cdot \lambda^{1} + b$,
      \\[0.2cm]
      which is the same as
      \\[0.2cm]
      \hspace*{1.3cm}
      $\lambda^{2} - a \cdot \lambda^{1} = b$.
      \\[0.2cm]
      We \href{https://en.wikipedia.org/wiki/Completing_the_square}{complete the square} by adding $(a/2)^2$ on
      both sides of this equation: 
      \\[0.2cm]
      \hspace*{1.3cm}
      $\ds \lambda^{2} - a \cdot \lambda^{1} + \Bigl(\frac{a}{2}\Bigr)^2 = b + \frac{a^2}{4}$.
      \\[0.2cm]
      Therefore, the solutions are
      \\[0.2cm]
      \hspace*{1.3cm}
      $\ds\lambda_1 = \frac{a}{2} + \frac{1}{2} \cdot \sqrt{4 \cdot b + a^2}$ \quad and \quad
      $\ds\lambda_2 = \frac{a}{2} - \frac{1}{2} \cdot \sqrt{4 \cdot b + a^2}$.
      \\[0.2cm]
      There is an exceptional case when $\lambda_1 = \lambda_2$.  This is called the 
      \blue{degenerated case}\index{degenerated case, recurrence equation}.
      This happens when $4 \cdot b + a^2 = 0$.  We will deal with the degenerated case later.  In the
      following, let us assume that $\lambda_1 \not= \lambda_2$.
\item In order to compute a \blue{special solution} \index{special solution} of the \blue{inhomogeneous} recurrence equation
      \\[0.2cm]
      \hspace*{1.3cm}
      $x_{n+2} = a \cdot x_{n+1} + b \cdot x_n + c$
      \\[0.2cm]
      we have to make a case distinction.
      \begin{enumerate}
      \item $a + b \not= 1$.

            In this case we can use the ansatz $x_n = \gamma$.  Substituting this ansatz into the
            inhomogeneous recurrence equation yields the equation
            \\[0.2cm]
            \hspace*{1.3cm}
            $\gamma = a \cdot \gamma + b \cdot \gamma + c$,
            \\[0.2cm]
            which is equivalent to the equation
            \\[0.2cm]
            \hspace*{1.3cm}
            $(1 - a - b) \cdot \gamma = c$.
            \\[0.2cm]
            Therefore we have
            \\[0.2cm]
            \hspace*{1.3cm}
            $\ds \gamma = \frac{c}{1 - a - b}$
            \\[0.2cm]
            and the special solution is
            \\[0.2cm]
            \hspace*{1.3cm}
            $\ds \gamma = \frac{c}{1 - a - b}$.
            \\[0.2cm]
            Of course, this only works out if $a + b \not= 1$.  This is the reason for our case distinction.
            
            Now the \blue{general solution} of the inhomogeneous recurrence equation is given as follows:
            \\[0.2cm]
            \hspace*{1.3cm}
            $x_n = \alpha \cdot \lambda_1^n + \beta \cdot \lambda_2^n + \gamma$.
            \\[0.2cm]
            We have to find $\alpha$ and $\beta$ so that we get the initial values $x_0 = d$ and $x_1 = e$ right.
            This leads to the following system of linear equations:
            \\[0.2cm]
            \hspace*{1.3cm}
            $d = \alpha + \beta + \gamma$ \quad and \quad $e = \alpha \cdot \lambda_1 + \beta \cdot \lambda_2 + \gamma$.
            \\[0.2cm]
            These are two linear equations for $\alpha$ and $\beta$.
            In order to access whether these equations are solvable, we rewrite these equations in matrix form:
            \\[0.2cm]
            \hspace*{1.3cm}
            $
            \left(\begin{array}{c}
                   d - \gamma \\
                   e - \gamma
                  \end{array}
            \right) =
            \left(
              \begin{array}{cc}
                1 & 1 \\
                \lambda_1 & \lambda_2
              \end{array}
            \right) \cdot
            \left(
              \begin{array}{c}
                \alpha \\
                \beta
              \end{array}
            \right)
            $
            \\[0.2cm]
            This equation has a unique solution if and only if the determinant of the matrix is different from
            $0$.  We have
            \\[0.2cm]
            \hspace*{1.3cm}
            $
            \texttt{det} \left(
              \begin{array}{cc}
                1 & 1 \\
                \lambda_1 & \lambda_2
              \end{array}
            \right) = \lambda_1 - \lambda_2 \not= 0 
            $
            \\[0.2cm]
            because $\lambda_1 \not= \lambda_2$.  Therefore we can solve the equations for $\alpha$ and
            $\beta$.  Then the solution of the recurrence equation is
            \\[0.2cm]
            \hspace*{1.3cm}
            $\ds x_{n} = \alpha \cdot \lambda_1^n + \beta \cdot \lambda_2^n + \gamma$.
      \item $a + b = 1$.

            In this case, the ansatz $x_n = \gamma$ does not work since we would have
            \\[0.2cm]
            \hspace*{1.3cm}
            $\ds \gamma = \frac{c}{1 - a - b} = \frac{\;c\;}{0}$
            \\[0.2cm]
            which is undefined.  Therefore, we have to use the ansatz
            \\[0.2cm]
            \hspace*{1.3cm}
            $x_n = \gamma \cdot n$           
            \\[0.2cm]
            instead of the ansatz $x_n = \gamma$.  The ansatz $x_n = \gamma \cdot n$ leads to the equation
            \\[0.2cm]
            \hspace*{1.3cm}
            $\gamma \cdot (n+2) = a \cdot \gamma \cdot (n+1) + b \cdot \gamma \cdot n + c$.
            \\[0.2cm]
            As $a + b = 1$, this equation can be rearranged into the equation
            \\[0.2cm]
            \hspace*{1.3cm}
            $2 \cdot \gamma = a \cdot \gamma + c$.
            \\[0.2cm]
            If $a = 2$ and $c \not= 0$, then our ansatz would fail.  Fortunately, this cannot happen as then
            $b = -1$ (remember we have $a + b = 1$) and then we would have $4 \cdot b + a^2 = 0$,
            which we have already ruled out because this would imply $\lambda_1 = \lambda_2$.

            As we have just seen that $a \not= 2$, we conclude
            \\[0.2cm]
            \hspace*{1.3cm}
            $\ds \gamma = \frac{c}{2 - a}$.
            \\[0.2cm]
            In this case the special solution is
            \\[0.2cm]
            \hspace*{1.3cm}
            $\ds x_n = \frac{c}{2 - a} \cdot n$.
            \\[0.2cm]
            Now the \blue{general solution} of the inhomogeneous recurrence equation is given as follows:
            \\[0.2cm]
            \hspace*{1.3cm}
            $x_n = \alpha \cdot \lambda_1^n + \beta \cdot \lambda_2^n + \gamma \cdot n$.
            \\[0.2cm]
            We have to find $\alpha$ and $\beta$ so that we get the initial values $x_0 = d$ and $x_1 = e$ right.
            This leads to the following system of linear equations:
            \\[0.2cm]
            \hspace*{1.3cm}
            $d = \alpha + \beta$ \quad and \quad $e = \alpha \cdot \lambda_1 + \beta \cdot \lambda_2 +
            \gamma$. \hspace*{\fill} $(*)$
            \\[0.2cm]
            In order to check whether this system of linear equations has a solution, we rewrite it as follows:
            $$
            \left(
              \begin{array}{c}
                d          \\
                e - \gamma
              \end{array}
            \right) =
            \left(
              \begin{array}{cc}
                1         &  1        \\
                \lambda_1 & \lambda_2 
              \end{array}
            \right) \cdot \left(
              \begin{array}{c}
                \alpha \\
                \beta
              \end{array}
              \right)
            $$
            This system of linear equations has a solution if and only if the
            \href{https://en.wikipedia.org/wiki/Determinant}{determinant} of the matrix on the right hand side
            is different from $0$, i.e.~we need to check that
            \\[0.2cm]
            \hspace*{1.3cm}
            $\texttt{det}\left(
              \begin{array}{cc}
                1         & 1        \\
                \lambda_1 & \lambda_2  
              \end{array}
               \right) = 1 \cdot \lambda_1 - 1 \cdot \lambda_2
            \not= 0$.
            \\[0.2cm]
            Fortunately, we have already made the assumption that $\lambda_1 \not= \lambda_2$ and this
            guarantees that the given system of equations has a solution.
            Therefore we can solve the equations $(*)$ for the unknowns $\alpha$ and $\beta$ and once we have
            done this, the solution of the recurrence equation is
            \\[0.2cm]
            \hspace*{1.3cm}
            $\ds x_{n} = \alpha \cdot \lambda_1^n + \beta \cdot \lambda_2^n + \gamma \cdot n$.
      \end{enumerate}
\end{enumerate}
If $\lambda_1 = \lambda_2 =: \lambda$  the procedure given above is changed as then the \blue{general solution} of
the inhomogeneous recurrence equation is given as either
\\[0.2cm]
\hspace*{1.3cm}
$x_n = \alpha \cdot \lambda^n + \beta \cdot n \cdot \lambda^n + \gamma$ \quad if $a + b \not= 1$
\\[0.2cm]
or
\\[0.2cm]
\hspace*{1.3cm}
$x_n = \alpha \cdot \lambda^n + \beta \cdot n \cdot \lambda^n + \gamma \cdot n$ \quad if $a + b = 1$.
\\[0.2cm]
We have to find $\alpha$ and $\beta$ so that we get the initial values $x_0 = d$ and $x_1 = e$ right.
Again, we need a case distinction:
\begin{enumerate}[(a)]
\item $a + b \not = 1$.

      In this case we have the following system of linear equations:
      \\[0.2cm]
      \hspace*{1.3cm}
      $d = \alpha + \gamma$ \quad and \quad $e = \alpha \cdot \lambda + \beta \cdot \lambda + \gamma$.
      \\[0.2cm]
      These are again two linear equations for $\alpha$ and $\beta$.
      If we rewrite these equations in matrix form, we get:
      \\[0.2cm]
      \hspace*{1.3cm}
      $
      \left(\begin{array}{c}
             d - \gamma \\
             e - \gamma
            \end{array}
      \right) =
      \left(
        \begin{array}{cc}
          1 & 0 \\
          \lambda & \lambda
        \end{array}
      \right) \cdot
      \left(
        \begin{array}{c}
          \alpha \\
          \beta
        \end{array}
      \right)
      $
      \\[0.2cm]
      This system of equations has a unique solution if and only if
      \\[0.2cm]
      \hspace*{1.3cm}
            $\texttt{det}\left(
              \begin{array}{cc}
                1       & 0        \\
                \lambda & \lambda
              \end{array}
               \right) = \lambda \not= 0$.
      \\[0.2cm]
      The case $\lambda = 0$ would imply $a = 0$ (because $\lambda = a/2$) and $b = 0$ (because $4 \cdot b + a^2 = 0$)
      and then the recurrence equations would have the form
      \\[0.2cm]
      \hspace*{1.3cm}
      $x_{n+2} = c$,
      \\[0.2cm]
      which is not a recurrence equation.  Therefore, this case is ruled out.
      
      Once we have solved these equations for $\alpha$ and
      $\beta$, the solution of the recurrence equation is now given as
      \\[0.2cm]
      \hspace*{1.3cm}
      $\ds x_{n} = \alpha \cdot \lambda^n + \beta \cdot n \cdot \lambda^n + \gamma$.
\item $a + b = 1$ \quad and $a \not= 2$.
    
      Now the system of linear equations is 
      \\[0.2cm]
      \hspace*{1.3cm}
      $d = \alpha$ \quad and \quad $e = \alpha \cdot \lambda + \beta \cdot \lambda + \gamma$. 
      \\[0.2cm]
      If we rewrite these equations in matrix form, we get:
      \\[0.2cm]
      \hspace*{1.3cm}
      $
      \left(\begin{array}{c}
             d \\
             e - \gamma
            \end{array}
      \right) =
      \left(
        \begin{array}{cc}
          1 & 0 \\
          \lambda & \lambda
        \end{array}
      \right) \cdot
      \left(
        \begin{array}{c}
          \alpha \\
          \beta
        \end{array}
      \right)
      $
      \\[0.2cm]      
      This system of equations has a unique solution if and only if
      \\[0.2cm]
      \hspace*{1.3cm}
            $\texttt{det}\left(
              \begin{array}{cc}
                1       & 0        \\
                \lambda & \lambda
              \end{array}
               \right) = \lambda \not= 0$.
      \\[0.2cm]
      This time $\lambda = 0$ is impossible because it would imply $a = 0$ and $b = 0$
      but then $a + b = 0 \not= 1$.  Hence we are sure that the system of equations has a solution.

      Once we have solved the system of equations for $\alpha$ and $\beta$, our solution is given as
      \\[0.2cm]
      \hspace*{1.3cm}
      $\ds x_{n} = \alpha \cdot \lambda^n + \beta \cdot n \cdot \lambda^n + \gamma \cdot n$. \qed
\item $a = 2$ and $b = -1$.

      In this case we have to solve the recurrence equation
      \\[0.2cm]
      \hspace*{1.3cm}
      $x_{n+2} = 2 \cdot x_{n+1} - x_n + c$ \quad with $x_0 = d$, $x_1=e$.
      \\[0.2cm]
      By subtracting $x_{n+1}$ on both sides of this equation we arrive at the equation
      \\[0.2cm]
      \hspace*{1.3cm}
      $x_{n+2} - x_{n+1} = x_{n+1} - x_n + c$.
      \\[0.2cm]
      We define $y_{n} := x_{n+1} - x_n$.  Then we have to solve the recurrence equation
      \\[0.2cm]
      \hspace*{1.3cm}
      $y_{n+1} = y_n + c$, \quad where $y_0 = e - d$.
      \\[0.2cm]
      This recurrence equation is equation to 
      \\[0.2cm]
      \hspace*{1.3cm}
      $y_n = y_{n-1} + c$
      \\[0.2cm]
      and has the solution 
      \\[0.2cm]
      \hspace*{1.3cm}
      $y_n = n \cdot c + e - d$.
      \\[0.2cm]
      Then we have
      \\[0.2cm]
      \hspace*{1.3cm}
      $
      \begin{array}{lcl}
        x_{n+1} & = & x_n + y_n \\
               & = & x_{n-1} + y_{n-1} + y_n \\
               & \vdots & \\
               & = & \ds x_0 + \sum\limits_{k=0}^n y_k \\[0.2cm]
               & = & \ds d + \sum\limits_{k=0}^n \bigl(c \cdot k + e - d\bigr) \\[0.4cm]
               & = & \ds d + c \sum\limits_{k=0}^n \cdot k + (n+1) \cdot (e - d) \\[0.5cm]
               & = & \ds d + \frac{\;c\;}{2} \cdot (n+1) \cdot n + (n+1) \cdot (e - d) 
      \end{array}
      $
      \\[0.2cm]
      The formula for $x_n$ is found by substituting $n-1$ for $n$ in the formula above:
      \\[0.2cm]
      \hspace*{1.3cm}
      $x_n = \ds \frac{\;c\;}{2} \cdot n \cdot (n-1) + n \cdot (e - d) + d$.
\end{enumerate}

\section{Summary}
We summarize the different cases that occur when solving the inhomogeneous recurrence equation
\\[0.2cm]
\hspace*{1.3cm}
$x_{n+2} = a \cdot x_{n+1} + b \cdot x_n + c$ \quad with the initial  conditions \quad $x_0 = d$, $x_1 = e$.
\\[0.2cm]
We use the ansatz $x_n = \lambda^n$ to solve the homogeneous recurrence equation
\\[0.2cm]
\hspace*{1.3cm}
$x_{n+2} = a \cdot x_{n+1} + b \cdot x_n$.
\\[0.2cm]
There are two cases:
\begin{enumerate}
\item Case: $4 \cdot b + a^2 \not= 0$.

      Then we define
      \\[0.2cm]
      \hspace*{1.3cm}
      $\ds\lambda_1 = \frac{a}{2} + \frac{1}{2} \cdot \sqrt{4 \cdot b + a^2}$ \quad and \quad
      $\ds\lambda_2 = \frac{a}{2} - \frac{1}{2} \cdot \sqrt{4 \cdot b + a^2}$.
      \\[0.2cm]
      In order to solve the inhomogeneous recurrence equation, we need another case distinction:
      \begin{enumerate}
      \item $a + b \not= 1$.
        
            Then a special solution can be found using the  \blue{ansatz} $x_n = \gamma$.
            and the general solution is
            \\[0.2cm]
            \hspace*{1.3cm}
            $x_n = \alpha \cdot \lambda_1^n + \beta \cdot \lambda_2^n + \gamma$.
      \item $a + b = 1$.

            Then the special solution is found using the ansatz $x_n = \gamma \cdot n$ and the general
            solution is
            \\[0.2cm]
            \hspace*{1.3cm}
            $x_n = \alpha \cdot \lambda_1^n + \beta \cdot \lambda_2^n + \gamma \cdot n$.
      \end{enumerate}
\item Case: $4 \cdot b + a^2 = 0$.

      Then we have 
      \\[0.2cm]
      \hspace*{1.3cm}
      $\ds \lambda = \frac{a}{2}$.
      \\[0.2cm]
      Again, we need another case distinction.
      \begin{enumerate}
      \item $a + b \not= 1$.

            Then the solution of the inhomogeneous recurrence equation has the form
            \\[0.2cm]
            \hspace*{1.3cm}
            $\ds x_{n} = \alpha \cdot \lambda^n + \beta \cdot n \cdot \lambda^n + \gamma$.
      \item $a + b = 1$.

            In this case the solution of the inhomogeneous recurrence equation has the form
            \\[0.2cm]
            \hspace*{1.3cm}
            $\ds x_{n} = \alpha \cdot \lambda^n + \beta \cdot n \cdot \lambda^n + \gamma \cdot n$.
      \item $a = 2$ and $b = -1$

            In this case the solution is the quadratic function
            \\[0.2cm]
            \hspace*{1.3cm}
            $x_n = \ds \frac{\;c\;}{2} \cdot n \cdot (n-1) + n \cdot (e - d) + d$.
      \end{enumerate}
\end{enumerate}
In all of these cases the parameters $\alpha$ and $\beta$ have to be determined so that the initial conditions
are satisfied.

\exercise
Solve the following second order recurrence equations:
\begin{enumerate}[(a)]
\item $x_{n+2} = x_{n+1} + x_n$, \quad where $x_0 = 0$ and $x_1 = 1$.  
\item $x_{n+2} = x_{n+1} + 2 \cdot x_n + 1$, \quad where $x_0 = 0$ and $x_1 = 0$.  
\item $x_{n+2} = 4 \cdot x_{n+1} - 4 \cdot x_n + 1$, \quad where $x_0 = 1$ and $x_1 = 3$.  \eox
\end{enumerate}

\section{Dealing with Complex Numbers}
It is interesting to consider those recurrence equations 
\\[0.2cm]
\hspace*{1.3cm}
$x_{n+2} = a \cdot x_{n+1} + b \cdot x_n + c$ \quad and \quad $x_0 = d$, $x_1 = e$
\\[0.2cm]
where $4 \cdot b + a^2 < 0$ because then according to the formulas
\\[0.2cm]
\hspace*{1.3cm}
$\ds\lambda_1 = \frac{a}{2} + \frac{1}{2} \cdot \sqrt{4 \cdot b + a^2}$ \quad and \quad
$\ds\lambda_2 = \frac{a}{2} - \frac{1}{2} \cdot \sqrt{4 \cdot b + a^2}$.
\\[0.2cm]
both $\lambda_1$ and $\lambda_2$ are complex numbers.  In this case our methodology to solve the
recurrence equation works in the same way as outline above, but the solution will exhibit an oscillatory
behaviour.  In order to understand what is happening, let us explore the solution of the homogeneous recurrence
equation 
\\[0.2cm]
\hspace*{1.3cm}
$\ds x_{n+2} = x_{n+1} - \frac{1}{2} \cdot x_n$ \quad where $x_0 = 0$ and $x_1 = 1$.
\\[0.2cm]
In this case we have $a = 1$ and $b = -\frac{1}{2}$ and therefore $4 \cdot b + a^2 = -2 + 1 = -1$ and hence we
have $\sqrt{4 \cdot b + a^2} = \sqrt{-1} = i$, where $i$ is the
\href{https://en.wikipedia.org/wiki/Imaginary_unit}{imaginary unit}.  \index{imaginary unit}
This implies
\\[0.2cm]
\hspace*{1.3cm}
$\ds\lambda_1 = \frac{1}{2} \cdot (1 + i)$ \quad and \quad $\ds\lambda_2 = \frac{1}{2} \cdot (1 - i)$.
\\[0.2cm]
The general solution of this recurrence equation is
\\[0.2cm]
\hspace*{1.3cm}
$\ds x_n = \alpha \cdot \lambda_1^n + \beta \cdot \lambda_2^n$
\\[0.2cm]
where $\alpha$ and $\beta$ are determined by the initial conditions $x_0 = 0$ and $x_1 = 1$.
This leads to two linear equations for $\alpha$ and $\beta$:
\\[0.2cm]
\hspace*{1.3cm}
$0 = \alpha + \beta$ \quad and \quad
$\ds 1 = \frac{1}{2} \cdot (1 + i) \cdot \alpha + \frac{1}{2} \cdot (1 + i) \cdot \beta$.
\\[0.2cm]
The first equation implies that $\beta = -\alpha$.  If we substitute this value into the second equation we have:
\\[0.2cm]
\hspace*{1.3cm}
$\ds 1 = \frac{1}{2} \cdot (1 + i) \cdot \alpha - \frac{1}{2} \cdot (1 + i) \cdot \alpha$.
\\[0.2cm]
This last equation yields $1 = i \cdot \alpha$ which has the solution $\alpha = -i$.  Therefore $\beta = i$
and hence we have found the solution
\\[0.2cm]
\hspace*{1.3cm}
$\ds x_n = -i \cdot \left(\frac{1}{2} \cdot (1 + i)\right)^n + i \cdot \left(\frac{1}{2} \cdot (1 - i)\right)^n$.
\\[0.2cm]
At this point we could stop.  However, it is possible to rewrite this solution in a way that eliminates the
occurrence of the imaginary unit.  To this end we have to use
\href{https://en.wikipedia.org/wiki/Euler%27s_formula}{Euler`s formula} \index{Euler's formula}
\\[0.2cm]
\hspace*{1.3cm}
$\ds \mathrm{e}^{i \cdot x} = \cos(x) + i \cdot \sin(x)$
\\[0.2cm]
to transform the numbers
$\frac{1}{2} \cdot (1 + i)$ and $\frac{1}{2} \cdot (1 - i)$ into
\href{https://en.wikipedia.org/wiki/Polar_coordinate_system#Complex_numbers}{polar coordinates}.  Since
\\[0.2cm]
\hspace*{1.3cm}
$\ds \cos(\pi/4) = \sin(\pi/4) = \frac{1}{\sqrt{2}}$
\\[0.2cm]
this yields
\\[0.2cm]
\hspace*{1.3cm}
$\ds \frac{1}{2} \cdot (1 + i) = \frac{1}{\sqrt{2}} \cdot \bigl(\cos(\pi/4) + i \cdot \sin(\pi/4) \bigr) =
 \frac{1}{\sqrt{2}} \cdot \mathrm{e}^{i \cdot\pi/4}$ \quad and 
\\[0.2cm]
\hspace*{1.3cm}
$\ds \frac{1}{2} \cdot (1 - i) = \frac{1}{\sqrt{2}} \cdot \bigl(\cos(\pi/4) - i \cdot \sin(\pi/4) \bigr) =
 \frac{1}{\sqrt{2}} \cdot \mathrm{e}^{-i \cdot\pi/4}$. 
\\[0.2cm]
As we have 
\\[0.2cm]
\hspace*{1.3cm}
$\ds i = \mathrm{e}^{i \cdot\pi/2}$  \quad and \quad $-i = \mathrm{e}^{-i \cdot\pi/2}$
\\[0.2cm]
the formula for $x_n$ can be rewritten as follows:
\\[0.2cm]
\hspace*{1.3cm}
$
\begin{array}{lcl}
  x_n & = & \ds  \mathrm{e}^{-i \cdot\pi/2} \cdot \left(\frac{1}{\sqrt{2}}\right)^n \cdot \mathrm{e}^{ i \cdot n \cdot\pi/4}
               + \mathrm{e}^{ i \cdot\pi/2} \cdot \left(\frac{1}{\sqrt{2}}\right)^n \cdot \mathrm{e}^{-i \cdot n \cdot\pi/4}
  \\[0.4cm]
  & = & \ds \left(\frac{1}{\sqrt{2}}\right)^n \cdot \bigl(
        \mathrm{e}^{i \cdot (n - 2) \cdot \pi / 4} + \mathrm{e}^{-i \cdot (n - 2) \cdot \pi / 4} \bigr) \\[0.4cm]
  & = & \ds \left(\frac{1}{\sqrt{2}}\right)^n \cdot 2 \cdot
        \cos\bigl((n - 2) \cdot \pi / 4\bigr) \qquad 
        \mbox{since \quad $2 \cdot \cos(x) = \mathrm{e}^{i\cdot x} + \mathrm{e}^{-i\cdot x}$}\\[0.4cm]
    & = & \ds \left(\frac{1}{\sqrt{2}}\right)^{n-2} \cdot \cos\bigl((n - 2) \cdot \pi / 4\bigr)
\end{array}
$
\\[0.2cm]
Note that this last formula does not contain the imaginary unit.  Furthermore, it shows that the solution
exhibits an oscillatory behaviour.
\vspace*{0.2cm}

\remark
It should be noted that recurrence equations that arise from the complexity analysis of an algorithm usually do
not exhibit oscillatory behaviour.  The reason is that that in most cases the running time of an algorithm increases
monotonely with the size of the input. \eoxs


\section{WolframAlpha}
In order to solve recurrence equations, the website \href{https://www.wolframalpha.com/}{WolframAlpha}
\index{WolframAlpha} is very useful.  For example, to solve the recurrence equation
\\[0.2cm]
\hspace*{1.3cm}
$x_{n+2} = x_{n+1} - 2 \cdot x_n$ \quad with $x_0 = 0$ and $x_1 = 1$ 
\\[0.2cm]
we can type the query
\\[0.2cm]
\hspace*{1.3cm}
\texttt{solve x(n+2) = x(n+1) - 2 * x(n), x(0) = 0, x(1) = 1}
\\[0.2cm]
into the search field of WolframAlpha.  After a few seconds, we will get back the answer
\\[0.2cm]
\hspace*{1.3cm}
$\ds x(n) = i \cdot \frac{\left(\frac{1}{2} - i \cdot\frac{\sqrt{7}}{2}\right)^n -
                          \left(\frac{1}{2} + i \cdot\frac{\sqrt{7}}{2}\right)^n}{\sqrt{3}}$
\\[0.2cm]
You can test this for yourself by clicking the following                        
\href{https://www.wolframalpha.com/input?i=solve+x%28n%2B2%29+%3D+x%28n%2B1%29+-+2+*+x%28n%29%2C+x%280%29+%3D+0%2C+x%281%29+%3D+1}{link}.


\section{Further Reading}
The book ``\blue{Calculus of Finite Differences and Difference Equations}'' by Murray R.~Spiegel
\cite{spiegel:1971} covers difference equations in much more detail than I was able to cover in my lecture.
Furthermore, this book contains a large number of both solved and unsolved exercises concerning
recurrence equations. 

\section{Check Your Understanding}
If you are able to answer the questions below confidently, then you should have mastered the concepts
introduced in this chapter.
\begin{enumerate}[(a)]
\item How are the Fibonacci numbers defined?
\item Can you derive an explicit formula for the Fibonacci numbers?
\item How is a second order linear inhomogeneous recurrence equations defined?  
\item What is the ansatz to solve a linear \blue{homogeneous} recurrence equation?
\item Describe the method to solve a linear \blue{inhomogeneous} recurrence equation!
\item Assume that we have to solve the recurrence equation $x_{n+2} = a \cdot x_{n+1} + b \cdot x_n$
      and we have $4 \cdot b + a^2 = 0$.  What is the general solution of the given homogeneous
      recurrence equation in this case?
\item Which special cases might occur when solving a linear recurrence equation?
\item How can we solve a recurrence equation of the form $x_{n+2} = 2 \cdot x_n - x_n + c$?
\item What happens if the numbers $\lambda_1$ and $\lambda_2$ that occur in the general solution
      of a recurrence equation are complex numbers?
\item What is the easiest method to check if your solution to a given recurrence equation is correct?  
\end{enumerate}


%%% Local Variables:
%%% mode: latex
%%% TeX-master: "algorithms"
%%% End:
