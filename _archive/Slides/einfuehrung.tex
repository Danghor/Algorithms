\documentclass{slides}
\usepackage{german}
\usepackage[latin1]{inputenc}
\usepackage{epsfig}
\usepackage{amssymb}
\usepackage{fancyvrb}

\pagestyle{empty}
\setlength{\textwidth}{17cm}
\setlength{\textheight}{24cm}
\setlength{\topmargin}{0cm}
\setlength{\headheight}{0cm}
\setlength{\headsep}{0cm}
\setlength{\topskip}{0cm}
\setlength{\oddsidemargin}{-0.5cm}
\setlength{\evensidemargin}{-0.5cm}


\newcounter{mypage}

\newcommand{\bruch}[2]{\frac{\displaystyle#1}{\displaystyle#2}}
\newcommand{\Oh}{\mathcal{O}}

\begin{document}

\begin{center}
Algorithmen und Daten--Strukturen
\end{center}


\rule{17cm}{1mm}

\footnotesize
Überblick 
\begin{enumerate}
\item Unl\"osbarkeit des Halte-Problems
\item Komplexit\"at von Algorithmen
      \begin{enumerate}
      \item Rekurrenz-Gleichungen
      \item $O$-Schreibweise
      \end{enumerate}
\item Abstrakte Daten-Typen und elementare Daten-Strukturen
  
      Stacks, sp\"ater Abbildungen, Priorit\"ats-Warteschlangen
\item Sortier-Algorithmen

      Minsort, Bubble-Sort, Quick-Sort, Merge-Sort, $\cdots$
\item Abbildungen
  \begin{enumerate}
  \item AVL-B\"aume
  \item Hashing
  \end{enumerate}
\item Graphen 

      Algorithmus von Dijkstra \\
      Berechnung k\"urzester Wege
\end{enumerate}

\begin{center}
\framebox{Ziel: selbst\"andig Algorithmen entwickeln und analysieren}
\end{center}

\vspace*{0.2cm}

\scriptsize

\vspace*{\fill}
\tiny \addtocounter{mypage}{1}
\rule{17cm}{1mm}
Algorithmen und Daten-Strukturen  \hspace*{\fill} Seite \arabic{mypage}

%%%%%%%%%%%%%%%%%%%%%%%%%%%%%%%%%%%%%%%%%%%%%%%%%%%%%%%%%%%%%%%%%%%%%%%%

\begin{slide}{}
\normalsize

\begin{center}
Algorithmen und Programme
\end{center}
\vspace*{0.5cm}

\footnotesize
\begin{enumerate}
\item \emph{Algorithmus}
  \begin{enumerate}
  \item abstrakt
  \item l\"asst vieles offen
  \item Beschreibung durch 
    \begin{enumerate}
    \item nat\"urlich sprachlichen Text
    \item mathematische Formeln
    \item Pseudo-Code, Programm
    \end{enumerate}
  \end{enumerate}
\item \emph{Programm}
  \begin{enumerate}
  \item konkret
  \item eindeutige Semantik
  \end{enumerate}
\end{enumerate}
Analogie: \emph{Idee} versus \emph{Text}
$$
\begin{array}{lcl}
  \emph{Idee} & \cong & \emph{Algorithmus} \\
  \emph{Text} & \cong & \emph{Programm} \\

\end{array}
$$
Beschreibung von Algorithmen in Vorlesung
\begin{enumerate}
\item mathematische Formeln (Sprache der Mengen-Lehre)
\item \textsl{Setl2}-Programme
\item \textsl{Java}-Programme (Übungen)
\end{enumerate}

\vspace*{\fill}
\tiny \addtocounter{mypage}{1}
\rule{17cm}{1mm}
Einf\"uhrung \hspace*{\fill} Seite \arabic{mypage}
\end{slide}

%%%%%%%%%%%%%%%%%%%%%%%%%%%%%%%%%%%%%%%%%%%%%%%%%%%%%%%%%%%%%%%%%%%%%%%%

\begin{slide}{}
\normalsize

\begin{center}
Literatur
\end{center}
\vspace*{0.5cm}

\footnotesize
\begin{enumerate}
\item \textsl{Alfred V.~Aho}, \textsl{John E.~Hopcraft}, \textsl{Jeffrey D.~Ullman}:
      \emph{The Design and Analysis of Computer Algorithms}. 

\item \textsl{Alfred V.~Aho}, \textsl{John E.~Hopcraft}, \textsl{Jeffrey D.~Ullman}:
      \emph{Data Structures and Algorithms}. 

      prim\"are Quelle f\"ur Vorlesung
\item \textsl{Frank M.~Carrano}: \\
      \emph{Data Abstraction and Problem Solving with \texttt{C++}}.

      gute Darstellungen der Algorithmen, kaum Theorie
\item \textsl{Frank M.~Carrano} and \emph{Janet Prichard}: \\
      \emph{Data Abstraction and Problem Solving with \textsl{Java}: Walls and Mirrors}.

      Neuauflage von 3., \textsl{Java} statt \texttt{C++}.

\item \textsl{Thomas H.~Cormen} and \textsl{Charles E.~Leiserson}: \\
      \emph{Introduction to Algorithms}. 

      sehr theroretisch, sehr viele Algorithmen, gut zum Nachschlagen.

\item \textsl{Robert Sedgewick}: 
      \emph{Algorithms in \texttt{C}}.

\item \textsl{Robert Sedgewick}: 
      \emph{Algorithms in \textsl{Java}, Part 1--4}.
    
      wie oben, \textsl{Java} statt \texttt{C}
\end{enumerate}


\vspace*{\fill}
\tiny \addtocounter{mypage}{1}
\rule{17cm}{1mm}
Einf\"uhrung \hspace*{\fill} Seite \arabic{mypage}
\end{slide}

%%%%%%%%%%%%%%%%%%%%%%%%%%%%%%%%%%%%%%%%%%%%%%%%%%%%%%%%%%%%%%%%%%%%%%%%

\begin{slide}{}
\normalsize

\begin{center}
Beispiele f\"ur Test-Funktionen
\end{center}
\vspace*{0.5cm}

\footnotesize

\begin{enumerate}
\item $s_1$ = ``{\tt int simple(char* x) \{ return 0; \}}''

      Test-Funktion mit dem Namen \texttt{simple}.
\item $s_2$ = ``{\tt int loop(char* x) \{ while (1) ++x; \}}''

      Test-Funktion mit dem Namen \texttt{loop}. 
\item $s_3$ = ``{\tt int loop(char* x);}''

      keine Test-Funktion: \\[0.3cm]
      \hspace*{1.3cm}  nur Deklaration, keine Definition.
\item $s_4$ = ``{\tt int hugo(char* x) begin i := 1; end;}''

      keine Test-Funktion: l\"asst sich nicht parsen.
\item $s_5$ = ``{\tt int hugo(int x) \{ return i*i; \}}''

      keine Test-Funktion: \\[0.3cm]
      Argument hat Typ \texttt{int} statt \texttt{char*}.
\end{enumerate}

Es gilt:
\begin{enumerate}
\item {\tt simple(\symbol{34}emil\symbol{34}) $\leadsto 0$}
\item {\tt simple(\symbol{34}emil\symbol{34}) $\downarrow$}
\item {\tt loop(\symbol{34}hugo\symbol{34}) $\uparrow$}
\end{enumerate}


\vspace*{\fill}
\tiny \addtocounter{mypage}{1}
\rule{17cm}{1mm}
Test-Funktionen  \hspace*{\fill} Seite \arabic{mypage}
\end{slide}

%%%%%%%%%%%%%%%%%%%%%%%%%%%%%%%%%%%%%%%%%%%%%%%%%%%%%%%%%%%%%%%%%%%%%%%%

\begin{slide}{}
\normalsize

\begin{center}
Halte-Problem
\end{center}
\vspace*{0.5cm}

\footnotesize
Gibt es \texttt{C}-Funktion \\[0.3cm]
\hspace*{1.3cm} 
$\texttt{int stops}(\texttt{char*}\;t,\;\texttt{char*}\;a)$ \\[0.3cm]
mit folgenden Eigenschaften:
\begin{enumerate}
\item $t \not\in T\!F \quad\Leftrightarrow\quad \mathtt{stops}(t, a) \leadsto 2$.
\item $t \in T\!F \,\wedge\, \mathtt{name}(t) = n \,\wedge\, n(a)\downarrow  \quad\Leftrightarrow\;$ 
      $\mathtt{stops}(t, a) \leadsto 1$.
\item $t \in T\!F \,\wedge\, \mathtt{name}(t) = n \,\wedge\, n(a)\uparrow \quad\Leftrightarrow\;$ 
      $\mathtt{stops}(t, a) \leadsto 0$.
\end{enumerate}


\footnotesize
Betrachte Test-Funktion \textsl{Turing}
\begin{verbatim}
    Turing := "int turing(char* x) {
                   int result = stops(x, x);
                   if (result == 1) {
                       while (1) {
                           ++result;
                       }
                   }
                   return result;
               }"
\end{verbatim}
Es gilt: $\textsl{Turing}\in T\!F$ und $\mathtt{name}(\textsl{Turing}) = \mathtt{turing}$.

\vspace*{\fill}
\tiny \addtocounter{mypage}{1}
\rule{17cm}{1mm}
Halte-Problem  \hspace*{\fill} Seite \arabic{mypage}
\end{slide}

%%%%%%%%%%%%%%%%%%%%%%%%%%%%%%%%%%%%%%%%%%%%%%%%%%%%%%%%%%%%%%%%%%%%%%%%

\begin{slide}{}
\normalsize

\begin{center}
Äquivalenz-Problem
\end{center}
\vspace*{0.5cm}

\footnotesize
\textbf{Geg.}: $n_1$, $n_2$: Namen zweier $k$-stelliger \texttt{C}-Funktionen \\[0.1cm]
\hspace*{1.8cm} $a_1$, $\cdots$, $a_k$: Argumente f\"ur $n_1$ und $n_2$

\textbf{Def.}: $n_1(a_1,\cdots,a_k) \simeq n_2(a_1,\cdots,a_k)$ \\[0.1cm]
g.d.w. einer der beiden folgen F\"alle auftritt:
\begin{enumerate}
\item $n_1(a_1,\cdots,a_k)\uparrow \quad\wedge\quad n_2(a_1,\cdots,a_k)\uparrow$
\item $\exists r: n_1(a_1,\cdots,a_k) \leadsto r \quad\wedge\quad n_2(a_1,\cdots,a_k) \leadsto r$
\end{enumerate}

\begin{verbatim}
       int equal(char* p1, char* p2, char* a)
\end{verbatim}
l\"ost \emph{Äquivalenz-Problem} falls gilt:
\begin{enumerate}
\item $p_1 \not\in T\!F \;\vee\; p_2 \not\in T\!F \quad\Leftrightarrow\quad \mathtt{equal}(p_1, p_2, a) \leadsto 2$.
\item Falls 
  \begin{enumerate}
  \item $p_1 \in T\!F \;\wedge\; \mathtt{name}(p_1) = n_1$,
  \item $p_2 \in T\!F \;\wedge\; \mathtt{name}(p_2) = n_2$ \quad und
  \item $n_1(a) \simeq n_2(a)$
  \end{enumerate}
    gilt, dann muss gelten: \\[0.1cm]
   \hspace*{1.3cm}  $\mathtt{equal}(p_1, p_2, a) \leadsto 1$.
\item Ansonsten gilt \\[0.1cm]
      \hspace*{1.3cm} $\mathtt{equal}(p_1, p_2, a) \leadsto 0$.
\end{enumerate}


\vspace*{\fill}
\tiny \addtocounter{mypage}{1}
\rule{17cm}{1mm}
Äquivalenz-Problem  \hspace*{\fill} Seite \arabic{mypage}
\end{slide}

%%%%%%%%%%%%%%%%%%%%%%%%%%%%%%%%%%%%%%%%%%%%%%%%%%%%%%%%%%%%%%%%%%%%%%%%

\begin{slide}{}
\normalsize

\begin{center}
Unl\"osbarkeit des Äquivalenz-Problems
\end{center}
\vspace*{0.5cm}

\footnotesize
\begin{Verbatim}[ frame         = lines, 
                  framesep      = 0.3cm, 
                  labelposition = bottomline,
                  numbers       = left,
                  numbersep     = -0.2cm,
                  xleftmargin   = 0.0cm,
                  xrightmargin  = 0.0cm
                ]
   int stops(char* p, char* a) {
       int e = 
       equal("int loop(char* x) { while (1) ++x; }", 
             p, a);
       if (e == 2) {
           return 2;
       } else {
           return 1 - e;
       }
   }
\end{Verbatim}

\textbf{Behauptung}: \texttt{stops} l\"ost Halte-Problem.

Setze: \textsl{Loop} := \texttt{\symbol{34}int loop(char* x) \{ while (1) ++x; \}\symbol{34}}

Betrachte $\texttt{stops}(p, a)$.
\begin{enumerate}
\item Fall: $p \not\in T\!F$

      Dann $\texttt{equal}(\textsl{Loop}, p, a) = 2$.

      Also $\texttt{stops}(p,a) = 2$
\item Fall: $p \in T\!F$ und $\mathtt{name}(p) = n$ und $n(a) \downarrow$

      Dann $e = \mathtt{equal}(\textsl{Loop}, p, a) = 0$.

      Also $\texttt{stops}(p,a) = 1 - 0 = 1$
\item Fall: $p \in T\!F$ und $\mathtt{name}(p) = n$ und $n(a)\uparrow$

      Dann $e = \mathtt{equal}(\textsl{Loop}, p, a) = 0$.

      Also $\texttt{stops}(p,a) = 1 - 1 = 0$
\end{enumerate}
Damit l\"ost \texttt{stops} das Halte-Problem!

\vspace*{\fill}
\tiny \addtocounter{mypage}{1}
\rule{17cm}{1mm}
Äquivalenz-Problem  \hspace*{\fill} Seite \arabic{mypage}
\end{slide}

%%%%%%%%%%%%%%%%%%%%%%%%%%%%%%%%%%%%%%%%%%%%%%%%%%%%%%%%%%%%%%%%%%%%%%%%

\begin{slide}{}
\normalsize

\begin{center}
Fibonacci--Zahlen
\end{center}
\vspace*{0.5cm}

\footnotesize
Kaninchen-Farm: Business-Plan f\"ur 
$$\textsl{Bunnies Incorporated}^{\,T\!M}$$
\begin{enumerate}
\item Jedes Kaninchen-Weibchen bringt jeden Monat ein
      neues Kaninchen-Paar  zur Welt.
\item Junge Kaninchen haben nach 2 Monaten Junge.
\item Vereinfachung: Kaninchen leben ewig. 
\end{enumerate}
Wie entwickelt sich nach $n$ Monaten die Zahl $k(n)$ der Kaninchen-Paare, wenn im ersten Monat
1 Paar junge Kaninchen eingekauft wird?

L\"osung: Rekursions-Gleichung
\begin{enumerate}
\item $k(0) = 1$
\item $k(1) = 1$
\item $k(2) = 2$ (Paar hat erstes Mal Junge)
\item Allgemein: \\[0.3cm]
      \hspace*{1.3cm} $k(n+2) = k(n + 1) + k(n)$ 
      \begin{enumerate}
      \item Alle Kaninchen-Paare, die es im Monat $n + 1$ gab, gibt es auch noch
            im Monat $n + 2$.
      \item Alle Kaninchen-Paare, die es schon im Monat $n$ gab, bringen im Monat
            $n + 2$ ein neues Paar zur Welt.
      \end{enumerate}
\end{enumerate}
\vspace*{\fill}
\tiny \addtocounter{mypage}{1}
\rule{17cm}{1mm}
Rekurrenzen  \hspace*{\fill} Seite \arabic{mypage}
\end{slide}

%%%%%%%%%%%%%%%%%%%%%%%%%%%%%%%%%%%%%%%%%%%%%%%%%%%%%%%%%%%%%%%%%%%%%%%%

\begin{slide}{}
\normalsize

\begin{center}
Berechnung der Rechenzeit eines Algorithmus
\end{center}
\vspace*{0.5cm}

\footnotesize
M\"ogliches Vorgehen:
\begin{enumerate}
\item Implementiere Algorithmus in Programmiersprache
\item Z\"ahle alle Operationen (RG aufstellen, l\"osen)
      \begin{enumerate}
      \item Zuweisungen,
      \item arithmetische Operationen,
      \item $\cdots$
      \end{enumerate}
\item Schlage Ausf\"uhrungs-Zeiten im Prozessorhandbuch nach    
\end{enumerate}

Probleme dieses Verfahrens
\begin{enumerate}
\item sehr aufwendig
\item praktisch bei modernen Prozessoren gar nicht durchf\"uhrbar,
      da Verhalten des Caches kaum vorhersehbar
\item Ergebnis vom Prozessor abh\"angig
\item Ergebnis beschreibt Programm statt Algorithmus
\end{enumerate}

\framebox{\framebox{
\begin{minipage}[c]{16cm}
    Wir brauchen abstrakten Begriff der Rechenzeit, um Komplexit\"at abstrakter Algorithmen zu
    untersuchen.
\end{minipage}
}}

\vspace*{\fill}
\tiny \addtocounter{mypage}{1}
\rule{17cm}{1mm}
Rekurrenzen  \hspace*{\fill} Seite \arabic{mypage}
\end{slide}

%%%%%%%%%%%%%%%%%%%%%%%%%%%%%%%%%%%%%%%%%%%%%%%%%%%%%%%%%%%%%%%%%%%%%%%%

\begin{slide}{}
\normalsize

\begin{center}
$\Oh$-Notation
\end{center}
\vspace*{0.5cm}

\footnotesize
Rechen-Zeiten  verschiedener Algorithmen zur Berechnung der Fibonacci-Zahlen:
\begin{enumerate}
\item Algorithmus zur Berechnung von $\mathtt{fib}(n)$: \\[0.3cm]
      \hspace*{1.3cm} 
      $\bruch{1}{\sqrt{5}} * (\lambda_1^n - \lambda_2^n) - 1$ Additionen

      mit $\lambda_{1/2} = \bruch{1}{2}(1 \pm \sqrt{5})$
\item Algorithmus zur Berechnung von $\mathtt{fib}(n)$: \\[0.3cm]
      \hspace*{1.3cm} 
      $4 * (n - 2) + 5$ Additionen, \\[0.3cm]
      mit $c$ unabh\"angig von $n$.
\end{enumerate}
Gesucht: Komplexit\"ats-Begriff, der
\begin{enumerate}
\item \underline{Wachstum} der Funktionen erfasst.
\item Terme niedriger Ordnung vernachl\"assigt.

      ($\lambda_2^n$ im 1. Fall, Konstante $5$ im 2. Fall)
\item Konstante Faktoren vernachl\"assigt.

      (Faktor $\bruch{1}{\sqrt{5}}$ im 1. Fall, Faktor 4 im zweiten Fall)
\end{enumerate}

\vspace*{\fill}
\tiny \addtocounter{mypage}{1}
\rule{17cm}{1mm}
Rekurrenzen  \hspace*{\fill} Seite \arabic{mypage}
\end{slide}

%%%%%%%%%%%%%%%%%%%%%%%%%%%%%%%%%%%%%%%%%%%%%%%%%%%%%%%%%%%%%%%%%%%%%%%%

\begin{slide}{}
\normalsize

\begin{center}
Eigenschaften der $\Oh$-Notation
\end{center}
\vspace*{0.5cm}

\footnotesize
\textbf{Satz}: Seien $f_1,f_2,g,h: \mathbb{N} \rightarrow \mathbb{N}$ und $c \in \mathbb{N}$. Dann gilt:
\begin{enumerate}
\item $f \in \Oh(f)$
\item $g \in \Oh(f) \Rightarrow c * g \in \Oh(f)$
\item $g \in \Oh(f) \Rightarrow g + c \in \Oh(f)$, falls $1 \in \Oh(f)$
\item $f \in \Oh(h) \;\&\; g \in \Oh(h) \Rightarrow f + g \in \Oh(h)$
\item $f \in \Oh(g) \;\&\; g \in \Oh(h) \Rightarrow f \in \Oh(h)$
\item $\lim\limits_{n \rightarrow \infty} \bruch{f(n)}{g(n)}\; \mbox{existiert}
       \Rightarrow f \in \Oh(g)$
  
\end{enumerate}

Seien $f:\mathbb{N} \rightarrow  \mathbb{N}$, \ $g:\mathbb{N} \rightarrow  \mathbb{N}$, \ und \ $h:\mathbb{N} \rightarrow  \mathbb{N}$.

\textbf{Schreibweise}: $f(n) = g(n) + O\bigg(h(n)\bigg)$ g.d.w \\[0.3cm]
\hspace*{1.3cm} $f(n) - g(n) \in O\bigg(h(n)\bigg)$.

\textbf{Beispiel}: $f(n) = n^2 + 5 * n*log(n) + 3*n$: \\[0.3cm]
\hspace*{1.3cm}   $f(n) = n^2 + \Oh(n*log(n))$ 


\begin{enumerate}
\item Kompexit\"at des 1. Alg. zur Ber. von $\mathtt{fib}(n)$: \\[0.3cm]
      \hspace*{1.3cm} 
      $\bruch{1}{\sqrt{5}} * (\lambda_1^n - \lambda_2^n) - 1 \;\in\;\Oh(\lambda_1^n)$
\item Kompexit\"at des 2. Alg. zur Ber. von $\mathtt{fib}(n)$: \\[0.3cm]
      \hspace*{1.3cm} $4 * (n - 2) + c \;\in\; \Oh(n)$
\item Kompexit\"at des 3. Alg. zur Ber. von $\mathtt{fib}(n)$: \\[0.3cm]
      \hspace*{1.3cm} $d \in \Oh(1)$
\end{enumerate}

\vspace*{\fill}
\tiny \addtocounter{mypage}{1}
\rule{17cm}{1mm}
Rekurrenzen  \hspace*{\fill} Seite \arabic{mypage}
\end{slide}

%%%%%%%%%%%%%%%%%%%%%%%%%%%%%%%%%%%%%%%%%%%%%%%%%%%%%%%%%%%%%%%%%%%%%%%%

\begin{slide}{}

\footnotesize
Sei $k \in \mathbb{N}$ fest.
\begin{enumerate}
\item $n^k \in \Oh(n^{k+1})$, \quad denn $\lim\limits_{n \rightarrow \infty} \bruch{n^k}{n^{k+1}} = 0$
\item $n^k \in \Oh(\lambda^n)$ falls $\lambda > 1$, denn  $\lim\limits_{n \rightarrow \infty} \bruch{n^k}{\lambda^n} = 0$ 
\item $\log_2(n) = \Oh(n)$, denn $\lim\limits_{n \rightarrow \infty} \bruch{\log_2(n)}{n} = 0$
\end{enumerate}

\textbf{Aufgabe}: Zeigen Sie \\[0.3cm]
\hspace*{1.3cm} $\log_{2}(n) \in \Oh(\log_{10}(n))$

Konkrete Beispiele:
\begin{enumerate}
\item $3 * n^2 + 5 * n + \sqrt{n} \in \Oh(n^2)$
\item $7 * n + \log_2(n)^2 \in \Oh(n)$
\item $\sqrt{n} + \log_2(n) \in \Oh\left(\sqrt{n}\right)$
\end{enumerate}

\textbf{Sprechweisen}: $f:\mathbb{N} \rightarrow  \mathbb{N}$ sei Rechenzeit eines Alg.
\begin{enumerate}
\item $f\in \Oh(n)$: linearer Algorithmus
\item $f\in \Oh(n^2)$: quadratischer Algorithmus
\item $f\in \Oh(\log_2(n))$: logarithmischer Algorithmus
\item $f\in \Oh(\lambda^n)$ mit $\lambda > 1$: exponentieller Algorithmus
\item $f\in \Oh(n*\log(n))$: n-log(n) Algorithmus
\end{enumerate}

\vspace*{\fill}
\tiny \addtocounter{mypage}{1}
\rule{17cm}{1mm}
Rekurrenzen  \hspace*{\fill} Seite \arabic{mypage}
\end{slide}

%%%%%%%%%%%%%%%%%%%%%%%%%%%%%%%%%%%%%%%%%%%%%%%%%%%%%%%%%%%%%%%%%%%%%%%%

\begin{slide}{}
\normalsize

\begin{center}
Teile und Herrsche
\end{center}
\vspace*{0.5cm}

\footnotesize
Programm zur Berechnung von $x^n$ f\"ur $n \in \mathbb{n}$
\begin{verbatim}
double power(double x, unsigned n) {
    if (n == 0)
        return 1;
    double y = power(x, n / 2);
    if (n % 2 == 0) {
        return y * y;
    } else {
        return x * y * y;
    }
}
\end{verbatim}
\textbf{Beh}.: \quad $\texttt{power}(x, n) = x^n$

\textbf{Beweis}: Induktion \"uber $n$
\begin{enumerate}
\item I.A.: $n = 0$ 

      $\mathtt{power}(x, 0) = 1 = x^0$
\item Fallunterscheidung nach $n \;\%\; 2$:
  \begin{enumerate}
  \item $n \;\%\; 2 = 0$, also $n = 2 * m$, $n/2 = m$.
        $$
        \begin{array}{lcl}
        \mathtt{power}(x,2*m) & =                & \mathtt{power}(x,m) * \mathtt{power}(x,m) \\
                              & \stackrel{IV}{=} & x^m * x^m = x^{2m} \\
        \end{array}
        $$
  \item $n \;\%\; 2 = 1$, also $n = 2 * m + 1$, $n/2 = m$.
        $$
        \begin{array}{lcl}
        \mathtt{power}(x,2*m+1) & =                & x * \mathtt{power}(x,m) * \mathtt{power}(x,m) \\
                                & \stackrel{IV}{=} & x* x^m * x^m = x^{2m+1} \\
        \end{array}
        $$\hspace*{\fill} $\Box$
  \end{enumerate}
\end{enumerate}


\vspace*{\fill}
\tiny \addtocounter{mypage}{1}
\rule{17cm}{1mm}
Rekurrenzen  \hspace*{\fill} Seite \arabic{mypage}
\end{slide}

%%%%%%%%%%%%%%%%%%%%%%%%%%%%%%%%%%%%%%%%%%%%%%%%%%%%%%%%%%%%%%%%%%%%%%%%

\begin{slide}{}
\normalsize

\begin{center}
Komplexit\"ats-Absch\"atzung
\end{center}
\vspace*{0.5cm}

\footnotesize
G\"unstigster Fall:  $n\;\%\;2 = 0$ f\"ur jeden rekursiven Aufruf.

$a_n$: Anzahl Multiplikationen bei Berechnung $\mathtt{power}(x,n)$ \\[0.3cm]
\hspace*{1.3cm} $a_n = a_{n/2} + 1$ \quad mit $a_0 = 0$
 
Ansatz: $n = 2^k$, $b_k := a_{2^k}$ liefert \\[0.3cm]
\hspace*{1.3cm} $b_{k+1} = b_k + 1$ \quad mit $b_0 = a_{2^0} = a_1 = 1$.

Setze $k \mapsto k + 1$: \\[0.3cm]
\hspace*{1.3cm} $b_{k+2} = b_{k+1} + 1$

Subtraktion: \\[0.3cm]
\hspace*{1.3cm} $b_{k+2} - b_{k+1} = b_{k+1} - b_k$ 

Also: $b_{k+2} = 2 * b_{k+1} - b_k$

Charakteristisches Polynom: \\[0.3cm]
\hspace*{1.3cm} $\chi(x) = x^2 - 2 * x + 1 = (x-1)^2$

Allgemeine L\"osung: \\[0.3cm]
\hspace*{1.3cm} $b_k = \alpha * 1^k + \beta * k * 1^k = a + \beta * k$

Rand-Bedingungen: \\[0.3cm]
\hspace*{1.3cm} $b_0 = 1 = \alpha$ \\[0.3cm]
\hspace*{1.3cm} $b_1 = b_0 + 1 = 2 = \alpha + \beta$

Also: $\alpha = 1$ und $\beta = 1$

L\"osung: $b_k = k + 1$, folglich $a_{2^k} = k + 1$

Setze $n = 2^k$, also $k = \log_2(n)$, dann: \\[0.3cm]
\hspace*{1.3cm} $a_n = \log_2(n) + 1 \in \Oh\left(\log_2(n)\right)$



\vspace*{\fill}
\tiny \addtocounter{mypage}{1}
\rule{17cm}{1mm}
Rekurrenzen  \hspace*{\fill} Seite \arabic{mypage}
\end{slide}

%%%%%%%%%%%%%%%%%%%%%%%%%%%%%%%%%%%%%%%%%%%%%%%%%%%%%%%%%%%%%%%%%%%%%%%%

\begin{slide}{}
\normalsize

\begin{center}
Komplexit\"ats-Absch\"atzung (Fortsetzung)
\end{center}
\vspace*{0.5cm}

\footnotesize
Ung\"unstigster Fall: $n = 2^k - 1$, denn $n/2 = 2^{k-1} - 1$ \\[0.3cm]
\hspace*{1.3cm} $a_n = a_{n/2} + 2$ mit $a_0 = 0$

Setze $b_k := a_{2^k - 1}$ \\[0.3cm]
\hspace*{1.3cm} $b_{k+1} = a_{(2^{k+1} - 1)/2} + 2 = a_{2^k - 1} + 2 = b_k + 2$

Rand-Bedingungen: $b_0 = a_0 = 0$, $b_1 = a_1 = 2$

Setze $k \mapsto k + 1$: \\[0.3cm]
\hspace*{1.3cm} $b_{k+2} = b_{k+1} + 2$

Subtraktion: \\[0.3cm]
\hspace*{1.3cm} $b_{k+2} = 2 * b_{k+1} - b_k$

Charakteristisches Polynom: \\[0.3cm]
\hspace*{1.3cm} $\chi(x) = x^2 - 2 * x + 1 = (x-1)^2$

Allgemeine L\"osung: \\[0.3cm]
\hspace*{1.3cm} $b_k = \alpha * 1^k + \beta * k * 1^k = a + \beta * k$

Rand-Bedingungen: \\[0.3cm]
\hspace*{1.3cm} $b_0 = 0 = \alpha$ \\[0.3cm]
\hspace*{1.3cm} $b_1 = 2 = \alpha + \beta$

Also: $\alpha = 0$ und $\beta = 2$

L\"osung: $b_k = 2 * k$, folglich $a_{2^k-1} = 2*k$

Setze $n = 2^k-1$, also $k = \log_2(n+1)$, dann: \\[0.3cm]
\hspace*{1.3cm} $a_n = 2*\log_2(n + 1)  \in \Oh\left(\log_2(n)\right)$

\vspace*{\fill}
\tiny \addtocounter{mypage}{1}
\rule{17cm}{1mm}
Rekurrenzen  \hspace*{\fill} Seite \arabic{mypage}
\end{slide}

\end{document}

%%% Local Variables: 
%%% mode: latex
%%% TeX-master: t
%%% End: 
