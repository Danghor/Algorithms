\chapter{Sortier-Algorithmen}
Im Folgenden gehen wir davon aus, dass wir eine Liste $L$ gegeben haben, deren Elemente
aus einer Menge $M$ entstammen.  
Die Elemente von M k\"onnen wir \emph{vergleichen}, das
hei{\ss}t, dass es auf der Menge $M$  eine Relation
$\leq$ gibt, die \emph{reflexiv}, \emph{anti-symmetrisch} und \emph{transitiv} ist, es gilt also  
\begin{enumerate}
\item $\forall x \el M \colon x \leq x$.
\item $\forall x, y \el M \colon x \leq y \wedge y \leq x \rightarrow x = y$.
\item $\forall x, y, z \el M \colon x \leq y \wedge y \leq z \rightarrow x \leq z$. 
\end{enumerate}
Ein Paar $\langle M, \leq \rangle$ bestehend aus einer Menge und einer bin\"aren Relation
$\leq \subseteq M \times M$ mit diesen Eigenschaften bezeichnen wir als eine
\emph{partielle Ordnung}.  Gilt zus\"atzlich \\[0.1cm]
\hspace*{1.3cm} $\forall x, y \el M \colon x \leq y \vee y \leq x$, \\[0.1cm]
so bezeichnen wir $\langle M, \leq \rangle$ als eine \emph{totale Ordnung}.
\vspace*{0.3cm}

\noindent
\textbf{Beispiele}:
\begin{enumerate}
\item $\langle\mathbb{N}, \leq \rangle$ ist eine totale Ordnung.
\item $\langle 2^{\mathbb{N}}, \subseteq \rangle$ ist eine partielle Ordnung aber keine totale Ordnung, denn beispielsweise sind die Mengen
      $\{1\}$ und $\{2\}$ nicht vergleichbar, es gilt \\[0.1cm]
      \hspace*{1.3cm} $\{1\} \not\subseteq \{2\}$ und $\{2\} \not\subseteq \{1\}$.
\item Ist $P$ die Menge der Mitarbeiter einer Firma und definieren wir f\"ur zwei
      Mitarbeiter $a,b \el M$\\[0.1cm]
      \hspace*{1.3cm} $a < b$ \quad g.d.w.~ \quad  $a$ verdient weniger als $b$, \\[0.1cm]
      so ist $\langle P, \leq \rangle$ eine partielle Ordnung.
\end{enumerate}
\textbf{Bemerkung}: In dem letzten Beispiel haben wir anstelle der Relation $\leq$ die
Relation $<$ definiert.  Ist eine Relation $<$ gegeben, so ist die dazugeh\"orige Relation
$\leq$ wie folgt definiert: \\[0.1cm]
\hspace*{1.3cm} $x \leq y \leftrightarrow x < y \vee x = y$. \\[0.1cm]
Betrachten wir die obigen Beispiele und \"uberlegen uns, in welchen F\"allen es m\"oglich ist,
eine Liste von Elemente zu sortieren, so stellen wir fest, dass dies im ersten und dritten Fall m\"oglich
ist, im zweiten Fall aber keinen Sinn macht.  Offensichtlich ist eine
totale Ordnung hinreichend zum Sortieren aber, wie das dritte Beispiel zeigt, nicht
unbedingt notwendig.  Eine partielle Ordnung reicht hingegen zum Sortieren nicht aus.
Wir f\"uhren daher einen weiteren Ordnungs-Begriff ein.

\begin{Definition}[Quasi-Ordnung]  \hspace*{\fill} \\
{\em
  Ein Paar $\langle M, \preceq\rangle$ ist eine \emph{Quasi-Ordnung}, falls $\preceq$ eine
  bin\"are Relation auf $M$ ist, f\"ur die gilt:
  \begin{enumerate}
  \item $\forall x \el M \colon x \preceq x$. \hspace*{\fill} (Reflexivit\"at)
  \item $\forall x, y, z \el M \colon x \preceq y \wedge y \preceq z \rightarrow x \preceq
    z$. \hspace*{\fill} (Transitivit\"at)
  \end{enumerate}
  Gilt zus\"atzlich \\[0.1cm]
  \hspace*{1.3cm} $\forall x, y \el M \colon x \preceq y \vee y \preceq x$, \\[0.1cm]
  so bezeichnen wir $\langle M, \preceq \rangle$ als eine \emph{totale Quasi-Ordnung}, was
  wir als \textsc{TQO} abk\"urzen.
}
\end{Definition}
Bei dem Begriff der Quasi-Ordnung wird im Unterschied zu dem Begriff der partiellen Ordnung auf die
Eigenschaft der Anti-Symmetrie verzichtet.  Trotzdem sind die Begriffe fast gleichwertig, denn wenn
$\langle M, \preceq \rangle$ eine Quasi-Ordnung ist, so kann auf $M$ eine Äquivalenz-Relation
$\approx$ durch 
\[ x \approx y \stackrel{def}{\longleftrightarrow} x \preceq y \wedge y \preceq x \]
definiert werden.
Setzen wir die Ordnung $\preceq$ auf die von der Relation $\approx$ erzeugten Äquivalenz-Klassen
fort,  so kann gezeigt werden, dass diese Fortsetzung eine partielle Ordnung ist.
\vspace*{0.3cm}

Es sei nun $\langle M, \preceq \rangle$ eine \textsc{TQO}.  Dann ist das \emph{Sortier-Problem} wie
folgt definiert:
\begin{enumerate}
\item Gegeben ist eine Liste $L$ von Elementen aus M.
\item Gesucht ist eine Liste $S$ mit folgenden Eigenschaften: 
  \begin{enumerate}
  \item $S$ ist aufsteigend sortiert: \\[0.1cm]
        \hspace*{1.3cm} 
        $\forall i \el \{ 1, \cdots, \#S-1 \} \colon S(i) \preceq S(i+1)$ \\[0.1cm]
        Hier bezeichnen wir die L\"ange der Liste $S$ mit $\#S$.
  \item $S$ und $L$ enthalten dieselben Elemente, es gilt also \\[0.1cm]
        \hspace*{1.3cm} 
        $\textsl{set}(L) = \textsl{set}(S)$. \\[0.1cm]
        Dabei ist f\"ur eine Liste $L$ der Ausdruck $\textsl{set}(L)$ als die Menge
        der Elemente der Liste $L$ definiert, es gilt \\[0.1cm]
        \hspace*{1.3cm} 
        $\textsl{set}(L) := \bigl\{ L(i) \mid i \el \{ 1, \cdots, \#L\} \bigr\}$.
  \item Die Elemente treten in $L$ und $S$ mit derselben H\"aufigkeit auf: \\[0.1cm]
        \hspace*{1.3cm} 
        $\forall x\el M \colon \textsl{count}(x,L) = \textsl{count}(x,S)$.
        \\[0.1cm]
        Dabei z\"ahlt die Funktion $\textsl{count}(x,L)$ wie oft das Element $x$ in der
        Liste $L$ auftritt: \\[0.1cm]
        \hspace*{1.3cm}
        $\textsl{count}(x,L) := \# \bigl\{ i \el \{1,\cdots,\#L\} \mid L(i) = x \bigr\}$.

        \textbf{Bemerkung}: Die letzte Forderung impliziert offensichtlich die zweite
        Forderung, denn wenn alle Elemente in den Listen $L$ und $S$ mit derselben
        H\"aufigkeit auftreten, dann enthalten $L$ und $S$ nat\"urlich auch dieselben Elemente.
  \end{enumerate}
\end{enumerate}

In diesem Kapitel pr\"asentieren wir verschiedene Algorithmen, die das Sortier-Problem
l\"osen, die also  zum Sortieren von Listen benutzt werden k\"onnen.
Wir stellen zun\"achst zwei Algorithmen vor, die sehr einfach zu implementieren sind, deren
Effizienz aber zu w\"unschen \"ubrig l\"asst.  Im Anschluss daran pr\"asentieren wir zwei
effizientere Algorithmen, deren Implementierung aber etwas aufwendiger ist.

\section{Sortieren durch Einf\"ugen}
Wir stellen zun\"achst einen sehr einfachen Algorithmus vor, der als 
``\emph{Sortieren durch Einf\"ugen}'' (engl. \emph{insertion sort}) bezeichnet wird.
Wir beschreiben den Algorithmus durch \emph{Gleichungen}.
Der Algorithmus arbeitet nach dem folgenden Schema:
\begin{enumerate}
\item Ist die zu sortierende Liste $L$ leer, so wird als Ergebnis
      die leere Liste zur\"uck gegeben: \\[0.1cm]
      \hspace*{1.3cm} $\mathtt{sort}([]) = []$
\item Andernfalls muss die Liste $L$ die Form $[x] + R$ haben.
      Dann sortieren wir den Rest $R$ und f\"ugen das Element $x$ in diese Liste so ein,
      dass die Liste sortiert bleibt. \\[0.1cm]
      \hspace*{1.3cm} $\mathtt{sort}\bigl([x] + R\bigr) = \mathtt{insert}\bigl(x, \mathtt{sort}(R)\bigr)$
\end{enumerate}
Das Einf\"ugen eines Elements $x$ in eine sortierte Liste $S$ erfolgt nach dem folgenden Schema:
\begin{enumerate}
\item Falls die Liste $S$ leer ist, ist das Ergebnis $[x]$: \\[0.1cm]
      \hspace*{1.3cm} $\mathtt{insert}(x,[]) = [x]$.
\item Sonst hat $S$ die Form $[y] + R$.  Wir vergleichen $x$ mit $y$.
      \begin{enumerate}
      \item Falls $x \preceq y$ ist, k\"onnen wir $x$ vorne an die Liste $S$ anf\"ugen: \\[0.1cm]
            \hspace*{1.3cm} $x \preceq y \rightarrow \mathtt{insert}\bigl(x, [y] + R\bigr) = [x,y] + R$. 
      \item Andernfalls f\"ugen wir $x$ rekursiv in die Liste $R$ ein: \\[0.1cm]
            \hspace*{1.3cm} $\neg x \preceq y \rightarrow \mathtt{insert}\bigl(x, [y] + R\bigr) = [y] + \mathtt{insert}(x,R)$. 
      \end{enumerate}
\end{enumerate}
Dieser Algorithmus l\"asst sich leicht in \textsl{Java} umsetzen. Abbildung
\ref{fig:insertion-sort} zeigt das resultierende Programm.
\begin{enumerate}
\item In Zeile 4 deklarieren wir die Member-Variable \texttt{mList}
      als Objekt vom Typ \texttt{LinkedList<Double>}.
      Die Klasse \texttt{LinkedList} rep\"asentiert 
      verkettete Listen und diese Datenstruktur ist zur Umsetzung der rekursiven
      Gleichungen am besten geeignet.
\item In Zeile 6 definieren wir den Konstruktor der Klasse \texttt{InsertionSort}.
      Dieser Konstruktor hat die Aufgabe, die Member-Variable \texttt{mList} zu
      initialisieren.  
\item In Zeile 12 bis 33 implementieren wir die Methoden \texttt{sort()} ind
      \texttt{insert()}.  Dabei benutzen wir die folgenden Methoden der Klasse
      \texttt{LinkedList}:
      \begin{enumerate}
      \item $L\texttt{.isEmpty}()$ testet, ob die Liste $L$ leer ist.
      \item $L\texttt{.removeFirst}()$ entfernt das erste Element aus der Liste $L$
            und liefert als Ergebnis das entfernte Element zur\"uck.
      \item $L\texttt{.addFirst}(x)$ f\"ugt das Element $x$ als erstes Element in die Liste
            $L$ ein.
      \item $L\texttt{.getFirst}()$ liefert als Ergebnis das erste Element der Liste $L$.
            Die Liste $L$ wird dabei nicht ver\"andert.
      \end{enumerate}
\item Zeile 34 definieren wir der Vollst\"andigkeit halber noch eine Methode \texttt{main}(),
      mit der wir die Klasse \texttt{InsertionSort} testen k\"onnen.
\end{enumerate}

\begin{figure}[!ht]
  \centering
\begin{Verbatim}[ frame         = lines, 
                  framesep      = 0.3cm, 
                  labelposition = bottomline,
                  numbers       = left,
                  numbersep     = -0.2cm,
                  xleftmargin   = 0.8cm,
                  xrightmargin  = 0.8cm
                ]
    import java.util.*;
    
    public class InsertionSort {
        LinkedList<Double> mList;
        
        InsertionSort(Double[] a) {
            mList = new LinkedList<Double>();
            for (Double x : a) {
                mList.add(x);
            }
        }    
        public void sort() {
            if (mList.isEmpty()) {
                return;
            }
            Double x = mList.removeFirst();
            sort();
            insert(x);
        }
        private void insert(Double x) {
            if (mList.isEmpty()) {
                mList.addFirst(x);
                return;
            }
            Double y = mList.getFirst();
            if (x <= y) {
                mList.addFirst(x);
            } else {
                mList.removeFirst();  // remove y
                insert(x);
                mList.addFirst(y);
            }
        }        
        public static void main(String[] args) {
            Double[] a = { 3.0, 7.0, 5.0, 2.0, 4.0, 11.0, 1.0 };
            InsertionSort IS = new InsertionSort(a);
            System.out.println(IS.mList);
            IS.sort();
            System.out.println(IS.mList);
        }
    }
\end{Verbatim}
\vspace*{-0.3cm}
  \caption{Der Algorithmus ``\emph{Sortieren durch Einf\"ugen}''}
  \label{fig:insertion-sort}
\end{figure} 

\subsection{Nachweis der Korrektheit}
Wir zeigen, dass der Algorithmus ``\emph{Sortieren durch Einf\"ugen}'' korrekt ist.
Dazu ben\"otigen wir zun\"achst einige Hilfs-Funktionen, die wir \"uber bedingte Gleichungen
definieren. 

\begin{Definition}[$\mathtt{le}(x,L)$]
{\em
  F\"ur ein Element $x\el M$ und eine Liste $L$ definieren wir die Funktion
  $\mathtt{le}(x, L)$ so, dass $\mathtt{le}(x, L)$ genau dann gilt, wenn f\"ur  alle
  Elemente $y$ aus $L$ die Ungleichung $x \preceq y$ gilt, es soll also gelten: \\[0.1cm]
  \hspace*{1.3cm} $\forall i \el \{ 1, \cdots, \#L \} \colon x \leq L(i)$. \\[0.1cm]
  Um mit der Funktion $\mathtt{le}$ arbeiten zu k\"onnen, geben wir jetzt eine 
  formale Definition durch Induktion \"uber $L$.
  \begin{enumerate}
  \item $\mathtt{le}(x, []) = \mathtt{true}$
  \item $x \preceq y \rightarrow \mathtt{le}(x, [y] + R) = \mathtt{le}(x, R)$
  \item $\neg x \preceq y \rightarrow \mathtt{le}(x, [y] + R) = \mathtt{false}$
        \hspace*{\fill} $\Box$
  \end{enumerate}
}
\end{Definition}
Die letzten beiden Gleichungen k\"onnen wir zu einer Gleichungen zusammen fassen: \\[0.1cm]
\hspace*{1.3cm} 
$\mathtt{le}(x, [y] + R) = \bigl(x \preceq y \wedge \mathtt{le}(x, R)\bigr)$. 
\\[0.1cm]
Die Syntax dieser Gleichung bedarf einer Erl\"auterung: Wir fassen hier $\preceq$ als eine
zweistellige Funktion mit der Signatur \\[0.1cm]
\hspace*{1.3cm} $\preceq: M \times M \rightarrow \mathbb{B}$ \\[0.1cm]
auf.  Die Funktion nimmt als Eingaben also zwei Elemente der Menge $M$ 
und liefert als Ergebnis einen Wahrheitswert aus der Menge 
$\mathbb{B} = \{ \mathtt{true}, \mathtt{false} \}$.  Genauso fassen wir den aussagenlogischen Junktor
$\wedge$ auf als eine zweistellige Funktion mit der Typ-Spezifikation \\[0.1cm]
\hspace*{1.3cm} $\wedge : \mathbb{B} \times \mathbb{B} \rightarrow \mathbb{B}$. 

\begin{Definition}[$\texttt{isSorted}(L)$]
{\em
  F\"ur eine Liste $L$ definieren wir die Funktion $\texttt{isSorted}(L)$ so,
  dass  $\texttt{isSorted}(L)$ genau dann den Wert \texttt{true} hat, wenn die Liste $L$ sortiert ist.
  Die formale Definition erfolgt durch Induktion \"uber $L$.
  \begin{enumerate}
  \item Die leere Liste ist sicher sortiert, also gilt \\[0.1cm]
        \hspace*{1.3cm} 
        $\mathtt{isSorted}([]) = \mathtt{true}$.
  \item Die Liste $[x] + R$ ist sicher dann sortiert, wenn einerseits $x$ kleiner-gleich
        den Elementen aus $L$ ist und wenn andererseits die Liste $R$ sortiert ist. \\[0.1cm]
        \hspace*{1.3cm} 
        $\mathtt{isSorted}([x]+R) = \bigl(\mathtt{le}(x,R) \wedge \mathtt{isSorted}(R)\bigr)$.
        \hspace*{\fill} $\Box$
  \end{enumerate}
}
\end{Definition}
Unser Ziel ist es zu zeigen, dass f\"ur eine
Liste $L$ immer $\texttt{isSorted}\bigl(\mathtt{sort}(L)\bigr)$ gilt.
Dazu ben\"otigen wir aber noch einige Hilfss\"atze.  

\begin{Lemma}[Distributivit\"at von \texttt{le} \"uber \texttt{insert}] \hspace*{\fill} \\
{\em
  Sind $x$ und $y$ Elemente aus $M$ und ist $L$ eine Liste, so gilt: \\[0.1cm]
  \hspace*{1.3cm} 
  $\mathtt{le}\bigl(x, \mathtt{insert}(y,L) \bigr) \leftrightarrow x \preceq y \wedge \mathtt{le}(x,L)$. 
}  
\end{Lemma}

\noindent
\textbf{Beweis}: Wir f\"uhren den Beweis durch Wertverlaufs-Induktion nach der Definition von \texttt{insert}.
\begin{enumerate}
\item Fall:  $L = []$.   Dann gilt 
      $$
      \begin{array}{lll}
                       & \mathtt{le}\bigl(x, \mathtt{insert}(y,[]) \bigr)                                        \\
       \leftrightarrow & \mathtt{le}\bigl(x, [y] \bigr)                   & \mbox{nach Definition von \texttt{insert}} \\
       \leftrightarrow & x \preceq y \wedge \mathtt{le}\bigl(x, [] \bigr) & \mbox{nach Definition von \texttt{le}}     \\
       \leftrightarrow & x \preceq y \wedge \mathtt{le}\bigl(x, L \bigr) & \mbox{wegen}\; L = []. 
      \end{array}
      $$
\item Fall: $L = [z] + R$ und $y \preceq z$.  Dann gilt 
            $$ 
            \begin{array}{cll}
                            & \mathtt{le}\bigl( x, \mathtt{insert}(y, [z] + R)\bigr) \\
            \leftrightarrow & \mathtt{le}\bigl( x, [y,z] + R\bigr)                   & \mbox{nach Definition von \texttt{insert}}\\
            \leftrightarrow & x \preceq y \wedge \mathtt{le}\bigl( x, [z] + R\bigr)  & \mbox{nach Definition von \texttt{le}}\\
            \leftrightarrow & x \preceq y \wedge \mathtt{le}\bigl( x, L\bigr)        & \mbox{wegen $L = [z] + R$}.
            \end{array}
            $$
\item Fall: $L = [z] + R$ und $\neg y \preceq z$.  
            In diesem Fall wird der Wert von $\texttt{insert}(x,L)$ durch die folgende
            bedingte Gleichung berechnet: \\[0.1cm]
            \hspace*{1.3cm} 
            $\neg y \preceq z \rightarrow \mathtt{insert}(y, [z] + R) = [z] + \mathtt{insert}(y,R)$. \\[0.1cm]
            Daher lautet die Induktions-Voraussetzung f\"ur diesen Fall \\[0.1cm]
            \hspace*{1.3cm} 
            $\mathtt{le}\bigl(x, \mathtt{insert}(y,R)\bigr) \leftrightarrow x \preceq y \wedge \mathtt{le}(x, R)$. 
            \\[0.1cm]
            Damit haben wir
            $$ 
            \begin{array}[b]{cll}
                            & \mathtt{le}\bigl(x, \mathtt{insert}(y, [z] + R)\bigr) \\
            \leftrightarrow & \mathtt{le}\bigl(x, [z] + \mathtt{insert}(y,R) \bigr)                & \mbox{nach Definition von \texttt{insert}}\\
            \leftrightarrow & x \preceq z \wedge \mathtt{le}\bigl( x, \mathtt{insert}(y,R) \bigr)  & \mbox{nach Definition von \texttt{le}}\\
            \leftrightarrow & x \preceq z \wedge x \preceq y \wedge \mathtt{le}\bigl( x, R\bigr)   & \mbox{nach I.V.}\\
            \leftrightarrow & x \preceq y \wedge x \preceq z \wedge \mathtt{le}\bigl( x, R\bigr)   & \mbox{wegen Kommutativit\"at von $\wedge$}\\
            \leftrightarrow & x \preceq y \wedge \mathtt{le}\bigl( x, [z] + R\bigr)                & \mbox{nach Definition von \texttt{le}}\\
            \leftrightarrow & x \preceq y \wedge \mathtt{le}\bigl( x, L\bigr)                      & \mbox{wegen $L = [z] + R$}.\\
            \end{array}\hspace*{1.6cm} \Box
            $$
\end{enumerate}

\begin{Lemma}[Transitivit\"at von \texttt{le}] \label{lemma:le-trans}
{\em
  Es gilt: \\[0.1cm]
  \hspace*{1.3cm} $x \preceq y \wedge \mathtt{le}(y,L) \rightarrow \mathtt{le}(x,L)$.
}  
\end{Lemma}

\noindent
\textbf{Beweis}: Wir f\"uhren den Beweis durch Induktion nach $L$.
\begin{enumerate}
\item[I.A.:] $L = []$. Nach Definition von \texttt{le} gilt \\[0.1cm]
      \hspace*{1.3cm} $\mathtt{le}(x,[]) \leftrightarrow \mathtt{true}$.
\item[I.S.:] $L = [z] + R$.

     Wir formen zun\"achst die Voraussetzung $\mathtt{le}(y,L)$ um: \\[0.1cm]
     \hspace*{1.3cm} 
     $\mathtt{le}(y,L) \leftrightarrow \mathtt{le}(y,[z] + R) \leftrightarrow y \preceq z \wedge \mathtt{le}(y, R)$. 
     \\[0.1cm]
     Aus der Voraussetzung $\mathtt{le}(y,L)$ folgt also \\[0.1cm]
     \hspace*{1.3cm} $(y \preceq z) \leftrightarrow \mathtt{true}$, \hspace*{\fill} (1) \\[0.1cm]
     \hspace*{1.3cm} $\mathtt{le}(y,R) \leftrightarrow \mathtt{true}$.  \hspace*{\fill} (2) \\[0.1cm]
     Aus der Voraussetzung $x \preceq y$ und (1) folgt mit der Transitivit\"at der Relation $\preceq$ \\[0.1cm]
     \hspace*{1.3cm} $x \preceq z \leftrightarrow \mathtt{true}$. \hspace*{\fill} (3) \\[0.1cm]
     Aus der Voraussetzung $x \preceq y$ und (2) folgt mit der Induktions-Voraussetzung \\[0.1cm]
     \hspace*{1.3cm} $\mathtt{le}(x, R) \leftrightarrow \mathtt{true}$. \hspace*{\fill} (4) \\[0.1cm]
     Jetzt k\"onnen wir die Behauptung zeigen: 
     $$
     \begin{array}[b]{cll}
        & \mathtt{le}(x, L) \\
      \leftrightarrow & \mathtt{le}(x, [z] + R) \\
      \leftrightarrow & x \preceq z \wedge \mathtt{le}(x, R)   &          \mbox{nach Definition von \texttt{le}} \\
      \leftrightarrow & \mathtt{true} \wedge \mathtt{le}(x, R) &          \mbox{wegen (3)} \\
      \leftrightarrow & \mathtt{true} \wedge \mathtt{true}     &          \mbox{wegen (4)} \\
      \leftrightarrow & \mathtt{true}.  &
     \end{array}\hspace*{\fill} \Box
     $$
\end{enumerate}


\begin{Korollar}
{\em
  Es sei $L = [y] + R$. Dann gilt \\[0.1cm]
  \hspace*{1.3cm} 
  $x \preceq y \wedge \mathtt{isSorted}(L) \rightarrow \mathtt{le}(x,L)$.
}
\end{Korollar}

\noindent
\textbf{Beweis}: Wir nehmen an, dass \\[0.1cm]
\hspace*{1.3cm} $x \preceq y$ \quad und \quad $\textsl{isSorted}(L)$ \\[0.1cm]
gilt und zeigen, dass daraus \\[0.1cm]
\hspace*{1.3cm} $\mathtt{le}(x,L)$ \\[0.1cm]
folgt.  Wir setzen in die Annahme $\textsl{isSorted}(L)$ f\"ur $L$ den Wert $[y] + R$ ein und erhalten 
$$
\begin{array}[b]{lll}
                & \mathtt{isSorted}([y] + R) \\
\leftrightarrow & \mathtt{le}(y, R) \wedge \mathtt{isSorted}(R) & \mbox{nach Definition von \texttt{isSorted}} \\
\rightarrow     & \mathtt{le}(y, R)  \\
\rightarrow     & \mathtt{le}(x, R) & \mbox{wegen der Transitivit\"at von \texttt{le}}. 
\end{array}\hspace*{2.6cm} \Box
$$

\begin{Lemma}[Distributivit\"at von \texttt{isSorted} \"uber \texttt{insert}] \hspace*{\fill}\\
{\em
  Es sei $x\el M$ und $S$ eine Liste. Dann gilt \\[0.1cm]
  \hspace*{1.3cm} $ \mathtt{isSorted}\bigl(\mathtt{insert}(x,S) \bigr) \leftrightarrow \mathtt{isSorted}(S)$. 
}  
\end{Lemma}
  
\noindent
\textbf{Beweis}: Wir f\"uhren den Beweis durch Wertverlaufs-Induktion nach der Definition von \texttt{insert}.
\begin{enumerate}
\item Fall: $S = []$.  Es gilt 
      $$
      \begin{array}{lll}
                      & \mathtt{isSorted}\bigl(\mathtt{insert}(x,[]) \bigr) \\
      \leftrightarrow & \mathtt{isSorted}\bigl([x] \bigr)                        & \mbox{nach Definition von \texttt{insert}} \\
      \leftrightarrow & \mathtt{le}(x,[]) \wedge \mathtt{isSorted}\bigl([]\bigr) & \mbox{nach Definition von \texttt{isSorted}} \\
      \leftrightarrow & \mathtt{true}     \wedge \mathtt{isSorted}\bigl([]\bigr) & \mbox{nach Definition von \texttt{le} } \\
      \leftrightarrow & \mathtt{isSorted}\bigl([]\bigr)                          &                                         \\
      \leftrightarrow & \mathtt{isSorted}\bigl(S\bigr)                           & \mbox{wegen $S=[]$}.                     \\
      \end{array}
      $$
\item Fall: $S = [y] + R$ und $x \preceq y$.  Dann gilt 
            $$
            \begin{array}{cll}
               & \mathtt{isSorted}\bigl( \mathtt{insert}(x, [y] + R\bigr) \\
             \leftrightarrow & \mathtt{isSorted}\bigl( [x,y] + R\bigr)                    & \mbox{nach Definition von $\mathtt{insert}$}\\
             \leftrightarrow & \mathtt{isSorted}\bigl( [x]+ S\bigr)                       & \mbox{wegen $S = [y] + R$ }\\
             \leftrightarrow & \mathtt{le}(x, S) \wedge \mathtt{isSorted}\bigl( S\bigr)   & \mbox{nach Definition von $\mathtt{isSorted}$}\\
             \leftrightarrow & \mathtt{true} \wedge \mathtt{isSorted}\bigl( S\bigr)       & \mbox{nach dem Korollar zur Transitivit\"at von \texttt{le}}\\
             \leftrightarrow & \mathtt{isSorted}\bigl( S\bigr)                              \\
            \end{array}
            $$
\item Fall:  $S = [y] + R$ und $\neg x \preceq y$.  Da wir voraussetzen, dass die Relation $\preceq$ eine \textsc{TQO} ist,
            muss $y \preceq x$ gelten.  Damit haben wir:
            $$
            \begin{array}[b]{cll}
               & \mathtt{isSorted}\bigl( \mathtt{insert}(x, [y] + R\bigr) \\
             \leftrightarrow & \mathtt{isSorted}\bigl( [y] + \mathtt{insert}(x, R)\bigr)   & \mbox{nach Definition von $\mathtt{insert}$}\\
             \leftrightarrow & \mathtt{le}\bigl(y, \mathtt{insert}(x,R) \bigr) \wedge \mathtt{isSorted}\bigl( \mathtt{insert}(x,R)\bigr)   
                     & \mbox{nach Definition von $\mathtt{isSorted}$}\\
             \leftrightarrow & y \preceq x \wedge \mathtt{le}\bigl(y, R \bigr) \wedge \mathtt{isSorted}\bigl( \mathtt{insert}(x,R)\bigr)   
                     & \mbox{Distributivit\"at von \texttt{le} \"uber \texttt{insert}}\\
             \leftrightarrow & y \preceq x \wedge \mathtt{le}\bigl(y, R \bigr) \wedge \mathtt{isSorted}\bigl( R \bigr)   
                     & \mbox{nach IV}\\
             \leftrightarrow & \mathtt{true} \wedge \mathtt{le}\bigl(y, R \bigr) \wedge \mathtt{isSorted}\bigl( R \bigr)   
                     & \mbox{wegen $y \preceq x$}\\
             \leftrightarrow & \mathtt{le}\bigl(y, R \bigr) \wedge \mathtt{isSorted}\bigl( R \bigr)   \\
             \leftrightarrow & \mathtt{isSorted}\bigl( [y] + R \bigr)  &
                       \mbox{nach Definition von \texttt{isSorted}} \\
             \leftrightarrow & \mathtt{isSorted}\bigl( L \bigr)  &
                       \mbox{wegen $L = [y] + R$}. \\
            \end{array} \hspace*{-0.5cm} \Box
            $$
\end{enumerate}

\begin{Satz}[Korrektheit von \texttt{sort}, Teil 1] \hspace*{\fill}\\
{\em
  Es sei $L$ eine Liste. Dann gilt \\[0.1cm]
  \hspace*{1.3cm} $\mathtt{isSorted}\bigl(\mathtt{sort}(L) \bigr)$. 
}  
\end{Satz}

\noindent
\textbf{Beweis}: Wir f\"uhren den Beweis durch Wertverlaufs-Induktion nach der Definition von \texttt{sort}.
\begin{enumerate}
\item[I.A.:] $L = []$. Dann gilt 
      $$
      \begin{array}[b]{lll}
                      & \mathtt{isSorted}\bigl(\mathtt{sort}([]) \bigr) \\
      \leftrightarrow & \mathtt{isSorted}\bigl([] \bigr) & \mbox{nach Definition von \texttt{sort}} \\
      \leftrightarrow & \mathtt{true}                    & \mbox{nach Definition von \texttt{isSorted}}.
      \end{array}
      $$
\item[I.S.:] $L = [x] + R$.  Es gilt \\[0.1cm]
      \hspace*{1.3cm} 
      $\mathtt{sort}([x] + R) = \mathtt{insert}\bigl(x, \mathtt{sort}(R)\bigr)$. \\[0.1cm]
      Daher gilt nach Induktions-Voraussetzung \\[0.1cm]
      \hspace*{1.3cm} 
      $\mathtt{isSorted}\bigl(\mathtt{sort}(R) \bigr)$.  \\[0.1cm]
      Also gilt 
      $$
      \begin{array}[b]{lll}
                      & \mathtt{isSorted}\bigl(\mathtt{sort}([x] + R)\bigr) \\
      \leftrightarrow & \mathtt{isSorted}\Bigl(\mathtt{insert}\bigl(x, \mathtt{sort}(R)\bigr)\Bigr) &
      \mbox{nach Definition \texttt{sort}}  \\
      \leftrightarrow & \mathtt{isSorted}\bigl(\mathtt{sort}(R)\bigr) &
      \mbox{Distributivit\"at von \texttt{isSorted} \"uber \texttt{insert}}  \\
      \leftrightarrow & \mathtt{true} &
      \mbox{nach I.V.}  \\
      \end{array}
      \hspace*{-0.5cm} \Box
      $$
\end{enumerate}

\begin{Definition}[$\texttt{eq}(x,y)$]
{\em
  F\"ur zwei Elemente $x$ und $y$ aus einer Menge $M$  definieren wir die Funktion
  $\mathtt{eq}: M \times M \rightarrow \mathbb{N}$  wie folgt
  \[
  \mathtt{eq}(x,y) = \left\{
  \begin{array}{ll}
    1 & \quad \mbox{falls}\quad x = y; \\
    0 & \quad \mbox{falls}\quad x \not= y. \\
  \end{array}
  \right.
  \]
}
\end{Definition}

\noindent
Damit k\"onnen wir die Funktion \\[0.1cm]
\hspace*{1.3cm} $\texttt{count}\colon M \times \textsl{List}(M) \rightarrow \mathbb{N}$ \\[0.1cm]
die z\"ahlt, wie oft ein Element $x$ in einer Liste $L$ vorkommt, durch Gleichungen definieren:
\begin{enumerate}
\item Fall: $L = []$.

      $\mathtt{count}(x,[]) = 0$.
\item Fall $L = [y] + R$.  

      $\mathtt{count}(x,[y] + R) = \mathtt{eq}(x,y) + \mathtt{count}(x, R)$.
\end{enumerate}

\begin{Lemma}[Distributivit\"at von \texttt{count} \"uber \texttt{insert}]\hspace*{\fill}\\
{\em
  Es seien $x$ und $y$ Elemente der Menge $M$ und $S$ sei eine Liste von Elementen aus $M$. Dann gilt \\[0.1cm]
  \hspace*{1.3cm} 
  $\mathtt{count}\bigl(x, \mathtt{insert}(y, S) \bigr) = \mathtt{eq}(x,y) + \mathtt{count}(x,S)$.
}  
\end{Lemma}

\noindent
\textbf{Beweis}: Wir f\"uhren den Beweis durch Wertverlaufs-Induktion.
\begin{enumerate}
\item Fall: $S = []$. Es gilt
      $$
      \begin{array}{cll}
        & \mathtt{count}\bigl(x, \mathtt{insert}(y, S) \bigr) \\
      = & \mathtt{count}\bigl(x, \mathtt{insert}(y, []) \bigr) \\
      = & \mathtt{count}\bigl(x, [y]) \bigr) \\
      = & \mathtt{eq}(x,y) + \mathtt{count}\bigl(x, []) \bigr) \\
      = & \mathtt{eq}(x,y) + \mathtt{count}(x,S).
      \end{array}
      $$
\item Fall: $S = [z] + R$ und $y \preceq z$. Dann gilt:
        $$
      \begin{array}{cll}
        & \mathtt{count}\bigl(x, \mathtt{insert}(y, S) \bigr) \\
      = & \mathtt{count}\bigl(x, \mathtt{insert}(y, [z] + R) \bigr) \\
      = & \mathtt{count}\bigl(x, [y,z] + R) \bigr) & \mbox{nach Definition von \texttt{insert}}\\
      = & \mathtt{eq}(x,y) + \mathtt{count}\bigl(x, [z] + R) \bigr) &
          \mbox{nach Definition von \texttt{count}} \\
      = & \mathtt{eq}(x,y) + \mathtt{count}(x,S).
      \end{array}
      $$
\item Fall: $S = [z] + R$ und $\neg y \preceq z$.  Dann gilt:
      $$
      \begin{array}[b]{cll}
        & \mathtt{count}\bigl(x, \mathtt{insert}(y, S) \bigr) \\
      = & \mathtt{count}\bigl(x, \mathtt{insert}(y, [z] + R) \bigr) \\
      = & \mathtt{count}\bigl(x, [z] + \mathtt{insert}(y,R) \bigr) & \mbox{nach Definition von \texttt{insert}}\\
      = & \mathtt{eq}(x,z) + \mathtt{count}\bigl(x, \mathtt{insert}(y, R) \bigr) &
          \mbox{nach Definition von \texttt{count}} \\
      = & \mathtt{eq}(x,z) + \mathtt{eq}(x,y) + \mathtt{count}(x, R) &
          \mbox{nach Induktions-Voraussetzung} \\
      = & \mathtt{eq}(x,y) + \mathtt{eq}(x,z) + \mathtt{count}(x, R)
          \\
      = & \mathtt{eq}(x,y) + \mathtt{count}(x, [z] + R) &
          \mbox{nach Definition von \texttt{count}} \\

      = & \mathtt{eq}(x,y) + \mathtt{count}(x,S).
      \end{array}\hspace*{\fill} \Box
      $$
\end{enumerate}

\begin{Satz}[Distributivit\"at von \texttt{count} \"uber \texttt{sort}] \hspace*{\fill}\\
{\em
  Ist $x$ ein Element und $L$ eine Liste, so gilt \\[0.1cm]
  \hspace*{1.3cm} $\texttt{count}\bigl(x, \texttt{sort}(L) \bigr) = \texttt{count}(x,L)$.
}
\end{Satz}

\noindent
\textbf{Beweis}: Wir f\"uhren den Beweis durch Wertverlaufs-Induktion nach der Definition von \texttt{sort}.
\begin{enumerate}
\item Fall: $L = []$. Es gilt
      $$
      \begin{array}{cll}
        & \mathtt{count}\bigl(x, \mathtt{sort}( L) \bigr) \\
      = & \mathtt{count}\bigl(x, \mathtt{sort}([]) \bigr) \\
      = & \mathtt{count}\bigl(x, []) \bigr) \\
      = & \mathtt{count}\bigl(x, L) \bigr). 
      \end{array}
      $$
\item Fall: $L = [y] + R$. 
        $$
      \begin{array}[b]{cll}
        & \mathtt{count}\bigl(x, \mathtt{sort}(L) \bigr) \\
      = & \mathtt{count}\bigl(x, \mathtt{sort}([y] + R) \bigr) \\
      = & \mathtt{count}\bigl(x, \mathtt{insert}(y, \mathtt{sort}(R)) \bigr) \\
      = & \mathtt{eq}(x,y) + \mathtt{count}(x,\mathtt{sort}(R)) &
         \mbox{wegen der Distributivit\"at von \texttt{count} \"uber \texttt{insert}} \\
      = & \mathtt{eq}(x,y) + \mathtt{count}(x,R) &
         \mbox{nach Induktions-Voraussetzung} \\
      = & \mathtt{count}(x,[y] + R) &
         \mbox{nach Definition von \texttt{count}} \\
      = & \mathtt{count}(x,L) &
         \mbox{wegen $L = [y] + R$.}
      \end{array}\hspace*{\fill} \Box
      $$
\end{enumerate}
Damit haben wir nun auch den zweiten Teil der Korrektheit von  
``\emph{Sortieren durch Einf\"ugen}'' nachgewiesen.

\subsection{Komplexit\"at}
Wir berechnen nun die Anzahl der Vergleichs-Operationen, die bei einem Aufruf von
``\texttt{Sortieren durch Einf\"ugen}'' in Zeile 26 von Abbildung
\ref{fig:insertion-sort} auf Seite \pageref{fig:insertion-sort} durchgef\"uhrt werden. Dazu
berechnen wir zun\"achst die Anzahl der Aufrufe von ``\texttt{<=}'', die bei einem Aufruf
von $\texttt{insert}(x,L)$ im schlimmsten Fall bei einer Liste der L\"ange $n$
durchgef\"uhrt werden.  Wir bezeichnen diese Anzahl mit $a_n$. Dann haben wir \\[0.1cm]
\hspace*{1.3cm} $a_0 = 0$ \quad und \quad $a_{n+1} = a_n + 1$. \\[0.1cm]
Dies ist eine lineare inhomogene Rekurrenz-Gleichungen erster Ordnung mit der Inhomogenit\"at $c_{-1} = 1$.  Das charakteristische
Polynom lautet \\[0.1cm]
\hspace*{1.3cm} $\chi(x) = x - 1$ \\[0.1cm]
und offenbar ist $\mathtt{sp}(\chi) = 0$.  Also hat die spezielle L\"osung die Form
 \\[0.1cm]
\hspace*{1.3cm}
$a_n = \varepsilon * n$ \quad mit $\varepsilon = \bruch{c_{-1}}{\chi'(1)} = \bruch{\;1\;}{1} = 1$,
\\[0.1cm]
 die spezielle L\"osung ist also \\[0.1cm]
\hspace*{1.3cm} $a_n = n$.
\\[0.1cm]
Damit lautet die allgemeine L\"osung: \\[0.1cm]
\hspace*{1.3cm} $a_n = \alpha * 1^n + n$. \\[0.1cm]
Durch Einsetzen der Anfangs-Bedingungen erhalten wir die Gleichung \\[0.1cm]
\hspace*{1.3cm} $0 = \alpha * 1^n + 1 * 0$, \quad also $\alpha = 0$. \\[0.1cm]
Damit lautet die L\"osung \\[0.1cm]
\hspace*{1.3cm} $a_n = n$, \\[0.1cm]
im schlimmsten Falle f\"uhrt der Aufruf von $\mathtt{insert}(x,L)$ bei einer Liste $L$ mit
$n$ Elementen also $n$ Vergleichs-Operationen durch, denn wir m\"ussen dann $x$ mit jedem
Element aus $L$ vergleichen.  Wir berechnen nun die Anzahl der
Vergleichs-Operationen, die im schlimmsten Fall beim Aufruf von $\texttt{sort}(L)$ f\"ur
eine Liste der L\"ange $L$ durchgef\"uhrt werden.  Wir bezeichnen dieses Anzahl mit $b_n$.
Offenbar gilt \\[0.1cm]
\hspace*{1.3cm} $b_0 = 0$ \quad und \quad $b_{n+1} = b_n + n$, \hspace*{\fill} (1)\\[0.1cm]
denn f\"ur eine Liste $L = [x] + R$ der L\"ange $n+1$ wird zun\"achst f\"ur die Liste $R$ rekursiv
die Funktion $\mathtt{sort}(R)$ aufgerufen. Das liefert den Summanden $b_n$. Anschlie{\ss}end wird
mit $\mathtt{insert}(x, \mathtt{Sorted})$ 
das erste Element in diese Liste eingef\"ugt.  Wir hatten oben gefunden, dass dazu
schlimmstenfalls $n$
Vergleichs-Operationen notwendig sind, was den Summanden $n$ erkl\"art.

Die Rekurrenz-Gleichungen (1) ist eine lineare inhomogene Rekurrenz-Gleichungen erster Ordnung mit einer
nicht-konstanten Inhomogenit\"at.
Um diese Rekurrenz-Gleichungen zu l\"osen, verwenden wir die Methode des diskreten Differenzierens.
Dazu f\"uhrt wir in der Gleichung (1) die Substitution
$n \mapsto n + 1$ durch: \\[0.1cm]
\hspace*{1.3cm} $b_{n+2} = b_{n+1} + n+1$, \hspace*{\fill} (2)\\[0.1cm]
Wir subtrahieren nun die Gleichung (1) von der Gleichung (2) ab und erhalten nach Vereinfachung \\[0.1cm]
\hspace*{1.3cm} $b_{n+2} = 2 * b_{n+1} - b_n + 1$. \\[0.1cm]
Dies ist eine lineare inhomogene Rekurrenz-Gleichungen zweiter Ordnung mit konstanter
Inhomogenit\"at.  F\"ur das charakteristische Polynom gilt $\chi(x) = (x-1)^2$.  Offenbar hat
dieses Polynom eine doppelte Null-Stelle an der Stelle $x=1$.  Um eine spezielle L\"osung zu
erhalten, machen wir also den Ansatz \\[0.1cm]
\hspace*{1.3cm} $\displaystyle b_n = \varepsilon * n^2$. \\[0.1cm]
Einsetzen in die Gleichung (2) ergibt: \\[0.1cm]
\hspace*{1.3cm} 
$\varepsilon * (n+2)^2 = 2 * \varepsilon * (n+1)^2 - \varepsilon * n^2 + 1$. \\[0.1cm]
Diese Gleichung vereinfacht sich zu \\[0.1cm]
\hspace*{1.3cm} $4* \varepsilon = 2 * \varepsilon + 1$ \\[0.1cm]
und daraus folgt sofort $\varepsilon = \frac{1}{2}$. Also ist \\[0.1cm]
\hspace*{1.3cm} $a_n = \bruch{\;1\;}{2} n^2$ \\[0.1cm]
eine spezielle L\"osung der Rekurrenz-Gleichungen (2).   Daher ergibt sich die allgemeine
L\"osung als \\[0.1cm]
\hspace*{1.3cm} $\displaystyle b_n = \alpha * 1^n + \beta * n * 1^n + \frac{1}{2} * n^2$. \\[0.1cm]
Wir finden $\alpha$ und $\beta$ indem wir die Anfangs-Bedingungen $a_0 = 0$ und 
$a_1 = a_0 + 0 = 0$ hier einsetzen.  Das liefert die beiden Gleichungen \\[0.1cm]
\hspace*{1.3cm} $0 = \alpha$ \quad und \quad $\displaystyle 0 = \alpha + \beta + \frac{1}{2}$, \quad also \quad  $\alpha = 0$ \quad und \quad $\beta = - \frac{1}{2}$.  
\\[0.1cm]
Damit lautet die L\"osung unserer Rekurrenz-Gleichungen (1) \\[0.1cm]
\hspace*{1.3cm} 
$\displaystyle b_n = -\frac{1}{2} * n + \frac{1}{2} * n^2 = \frac{1}{2} n*(n-1)$. \\[0.1cm]
Im schlimmsten Fall werden also $\Oh(n^2)$ Vergleiche durchgef\"uhrt, der Algorithmus 
``\emph{Sortieren durch Einf\"ugen}'' erfordert einen quadratischen Aufwand.
Sie k\"onnen sich \"uberlegen, dass der schlimmste Fall genau dann eintritt, wenn die zu
sortierende Liste $L$ absteigend sortiert ist, so dass die gr\"o{\ss}ten Elemente gerade am
Anfang der Liste stehen.

Der g\"unstigste Fall f\"ur den Algorithmus ``\emph{Sortieren durch Einf\"ugen}'' liegt dann
vor, wenn die zu sortierende Liste bereits aufsteigend sortiert ist.  Dann wird beim
Aufruf von $\mathtt{insert}(x,\mathtt{Sorted})$ nur ein einziger Vergleich durchgef\"uhrt.
Die Rekurrenz-Gleichungen f\"ur die Anzahl der Vergleiche in $\mathtt{sort}(L)$ lautet dann \\[0.1cm]
\hspace*{1.3cm} $b_0 = 0$ \quad und \quad $b_{n+1} = b_n + 1$. \hspace*{\fill} (1)\\[0.1cm]
Die L\"osung dieser Rekurrenz-Gleichung haben wir oben berechnet, sie lautet $b_n = n$.
Im g\"unstigsten Falle ist der Algorithmus ``\emph{Sortieren durch Einf\"ugen}'' also linear.
\pagebreak

\section{Sortieren durch Auswahl}
Wir stellen als n\"achstes den Algorithmus ``\emph{Sortieren durch Auswahl}''
(engl. \emph{selection sort}) vor.  Der Algorithmus kann wie folgt beschrieben werden:
\begin{enumerate}
\item Ist die zu sortierende Liste $L$ leer, so wird als Ergebnis
      die leere Liste zur\"uck gegeben: \\[0.1cm]
      \hspace*{1.3cm} $\mathtt{sort}([]) = []$
\item Andernfalls suchen wir in der Liste $L$ das kleinste Element und entfernen dieses
      Element aus $L$.  Wir sortieren rekursiv die resultierende Liste, die ja ein Element
      weniger enth\"alt.  Zum Schluss f\"ugen wir das kleinste Element vorne an die sortierte
      Liste an: \\[0.1cm]
      \hspace*{1.3cm} 
      $L \not= [] \rightarrow \mathtt{sort}\bigl(L\bigr) = \bigl[\texttt{min}(L)\bigr]
      + \mathtt{sort}\bigl(\mathtt{delete}(\texttt{min}(L), L)\bigr)$.
\end{enumerate}
Der Algorithmus um ein Auftreten eines Elements $x$ aus einer Liste $L$ zu entfernen, kann ebenfalls leicht rekursiv
formuliert werden. Wir unterscheiden drei F\"alle:
\begin{enumerate}
\item Falls $L$ leer ist, gilt \\[0.1cm]
      \hspace*{1.3cm} $\mathtt{delete}(x, []) = []$.
\item Falls $x$ gleich dem ersten Element der Liste $L$ ist, gibt die Funktion den Rest
      $R$ zur\"uck: \\[0.1cm]
      \hspace*{1.3cm} 
      $\mathtt{delete}(x, [x] + R) = R$.
\item Andernfalls wird das Element $x$ rekursiv aus $R$ entfernt: \\[0.1cm]
      \hspace*{1.3cm}   
      $x \not = y \rightarrow \mathtt{delete}(x, [y] + R) = [y] + \mathtt{delete}(x,R)$.
\end{enumerate}
Schlie{\ss}lich geben wir noch rekursive Gleichungen an um das Minimum einer Liste zu berechnen:
\begin{enumerate}
\item Das Minimum der leeren Liste ist gr\"o{\ss}er als alles andere \\[0.1cm]
      \hspace*{1.3cm} $\mathtt{min}([]) = \infty$.
\item Um das Minimum der Liste $[x] + R$ zu berechnen, berechnen wir rekursiv das Minimum
      von $R$ und benutzen die zweistellige Minimums-Funktion: \\[0.1cm]
      \hspace*{1.3cm} 
      $\mathtt{min}([x] + R) = \mathtt{min}\bigl(x, \mathtt{min}(R) \bigr)$. 

      Dabei ist die zweistellige Minimums-Funktion wie folgt definiert: \\[0.1cm]
      \hspace*{1.3cm} 
      $\mathtt{min}(x,y) = \left\{
      \begin{array}{ll}
        x  & \mbox{falls $x \preceq y\,$;} \\
        y  & \mbox{sonst.} \\
      \end{array}\right.
      $
\end{enumerate}
Die Implementierung dieses Algorithmus in \textsl{Java} sehen Sie in Abbildung
\ref{fig:selection-sort} auf Seite \pageref{fig:selection-sort}.  Es war nicht notwendig,
die Funktion $\textsl{delete}()$ zu implementieren, denn die Methode $\textsl{remove}()$
der Klasse $\texttt{LinkedList<T>}$ leistet das Gleiche wie die oben beschriebene Funktion
$\textsl{delete}()$.

\begin{figure}[!ht]
  \centering
\begin{Verbatim}[ frame         = lines, 
                  framesep      = 0.3cm, 
                  labelposition = bottomline,
                  numbers       = left,
                  numbersep     = -0.2cm,
                  xleftmargin   = 0.8cm,
                  xrightmargin  = 0.8cm
                ]
    import java.util.*;
    
    public class SelectionSort {
        LinkedList<Double> mList;
        
        SelectionSort(Double[] a) {
            mList = new LinkedList<Double>();
            for (Double x : a) {
                mList.add(x);
            }
        }
        public void sort() {
            if (mList.isEmpty()) {
                return;
            }
            Double x = min();
            mList.remove(x);
            sort();
            mList.addFirst(x);
        }    
        private Double min() {
            Double min = mList.getFirst();
            for (Double x : mList) {
                min = Math.min(min, x);
            }
            return min;
        }
    }
\end{Verbatim}
\vspace*{-0.3cm}
  \caption{Der Algorithmus ``\emph{Sortieren durch Auswahl}''}
  \label{fig:selection-sort}
\end{figure}

\subsection{Nachweis der Korrektheit}
Um den Nachweis der Korrektheit erbringen zu k\"onnen, brauchen wir eine Reihe von
Hilfss\"atzen.

\begin{Lemma}[Vertr\"aglichkeit von \texttt{min} und \texttt{le}] \label{l2} 
{\em 
F\"ur  $x\el M$ und eine Liste $L$ gilt: \\[0.1cm]
\hspace*{1.3cm} $\mathtt{le}(x, L) \leftrightarrow x \preceq \texttt{min}(L)$.
}
\end{Lemma}
\textbf{Beweis}:  Wir f\"uhren den Beweis durch Induktion \"uber die Liste $L$.
\begin{enumerate}
\item Fall: $L = []$. Es gilt:
      \[
      \begin{array}[b]{cl}
                       & \mathtt{le}(x, L) \\
       \leftrightarrow & \mathtt{le}(x, []) \\
       \leftrightarrow & \mathtt{true} \\
       \leftrightarrow & x \preceq \infty \\
       \leftrightarrow & x \preceq \mathtt{min}([]) \\
       \leftrightarrow & x \preceq \mathtt{min}(L).
      \end{array}     
      \]
\item Fall: $L = [y] + R$.  Es gilt:
      \[
      \begin{array}[b]{cll}
                       & \mathtt{le}(x, L)  & \\
       \leftrightarrow & \mathtt{le}(x, [y] + R)  & \\
       \leftrightarrow & x \preceq y \wedge \mathtt{le}(x, R)  & \mbox{nach Definition von \texttt{le}} \\
       \leftrightarrow & x \preceq y \wedge x \preceq \mathtt{min}(R)  & \mbox{nach Induktions-Voraussetzung} \\
       \leftrightarrow & x \preceq \mathtt{min}\bigl(y, \mathtt{min}(R) \bigr) &
          \mbox{wegen $c \preceq \mathtt{min}(a,b) \leftrightarrow c \preceq a \wedge c \preceq b$} \\
       \leftrightarrow & x \preceq \mathtt{min}\bigl( [y] + R \bigr) &
          \mbox{nach Definition von \texttt{min}} \\
       \leftrightarrow & x \preceq \mathtt{min}\bigl(L \bigr). &
      \end{array} \hspace*{2cm} \Box
      \]
\end{enumerate}

\begin{Korollar} \label{l3}
{\em
  F\"ur jede Liste $L$ gilt: \ $\mathtt{le}\bigl(\mathtt{min}(L), L\bigr)$.
}
\end{Korollar}
\textbf{Beweis}: Setzen wir in dem letzten Lemma f\"ur $x$ den Wert $\mathtt{min}(L)$ ein,
so erhalten wir: \\[0.1cm]
\hspace*{1.3cm}
 $\mathtt{le}\bigl(\mathtt{min}(L), L) \leftrightarrow \mathtt{min}(L) \preceq \texttt{min}(L) \leftrightarrow \mathtt{true}$. 
\hspace*{\fill} $\Box$

\begin{Lemma}[Distributivit\"at von \texttt{le} \"uber \texttt{delete}] 
\label{l4} \hspace*{\fill} \\
{\em
  Sind  $x,y \el M$  und es gelte $x \preceq y$.  Dann gilt f\"ur jede Liste $L$: \\[0.1cm]
  \hspace*{1.3cm} $\mathtt{le}\bigl(x,\mathtt{delete}(y,L)\bigr) \leftrightarrow \mathtt{le}(x, L)$.
}
\end{Lemma}
\textbf{Beweis} durch Wertverlaufs-Induktion nach der Definition von $\mathtt{delete}(x,L)$.
\begin{enumerate}
\item Fall: $L = []$. Es gilt 

      \hspace*{1.3cm} 
      $\mathtt{le}\bigl(x, \mathtt{delete}(y,L)\bigr) \leftrightarrow \mathtt{le}\bigl(x, \mathtt{delete}(y,[])\bigr) 
       \leftrightarrow \mathtt{le}(x, []) \leftrightarrow \mathtt{le}(x,L)$.
\item Fall: $L = [y] + R$. Es gilt 
      \[
      \begin{array}[b]{cll}
                      & \mathtt{le}\bigl(x,\mathtt{delete}(y,L)\bigr) \\
      \leftrightarrow & \mathtt{le}\bigl(x,\mathtt{delete}(y,[y] + R)\bigr) \\
      \leftrightarrow & \mathtt{le}\bigl(x,R\bigr) \\
      \leftrightarrow & \mathtt{true} \wedge \mathtt{le}\bigl(x,R\bigr) \\
      \leftrightarrow & x \preceq y \wedge \mathtt{le}\bigl(x,R\bigr) & \mbox{wegen $x \preceq y$}\\
      \leftrightarrow & \mathtt{le}\bigl(x,[y] + R\bigr) & \mbox{nach Definition von \texttt{le}}\\
      \leftrightarrow & \mathtt{le}\bigl(x,L\bigr).
      \end{array}
      \]
\item Fall: $L = [z] + R$ mit $y \not= z$.  Es gilt:
       \[ 
       \begin{array}[b]{cll}
                         & \mathtt{le}\bigl(x, \mathtt{delete}(y, L)\bigr)      \\
         \leftrightarrow & \mathtt{le}\bigl(x, \mathtt{delete}(y,[z] + R)\bigr) \\
         \leftrightarrow & \mathtt{le}\bigl(x, [z] + \mathtt{delete}(y,R)\bigr) & 
              \mbox{nach Definition von \texttt{delete}} \\
         \leftrightarrow & x \preceq z \wedge \mathtt{le}\bigl(x, \mathtt{delete}(y,R)\bigr) & 
              \mbox{nach Definition von \texttt{le}} \\
         \leftrightarrow & x \preceq z \wedge \mathtt{le}\bigl(x, R\bigr) & 
              \mbox{nach Induktions-Voraussetzung} \\
         \leftrightarrow & \mathtt{le}\bigl(x, [z] + R\bigr) \\
         \leftrightarrow & \mathtt{le}\bigl(x, L\bigr). 
       \end{array} \hspace*{\fill} \Box
       \]
\end{enumerate}

\begin{Lemma}[Distributivit\"at von \texttt{le} \"uber \texttt{sort}] \label{l6} 
{\em
Ist $x \el M$ und $L$ eine Liste, so gilt \\[0.1cm]
\hspace*{1.3cm}  $\mathtt{le}\bigl(x, \mathtt{sort}(L)\bigr) \leftrightarrow \mathtt{le}(x, L)$.
}
\end{Lemma}
\textbf{Beweis} durch Wertverlaufs-Induktion nach der Definition von $\texttt{sort}(L)$:
\begin{enumerate}
\item Fall: $L = []$. Dann gilt:

      \hspace*{1.3cm} 
      $\mathtt{le}\bigl(x, \mathtt{sort}(L)\bigr) \leftrightarrow \mathtt{le}\bigl(x, \mathtt{sort}([])\bigr) 
         \leftrightarrow \mathtt{le}\bigl(x, []\bigr) \leftrightarrow \mathtt{le}\bigl(x, L\bigr)$.
\item Fall: $L \not= []$.  Zur Abk\"urzung setzen wir $m = \mathtt{min}(L)$.
     \[
     \begin{array}[b]{cll}
                        & \mathtt{le}\bigl(x, \mathtt{sort}(L)\bigr) \\
        \leftrightarrow & \mathtt{le}\Bigl(x, [m] + \mathtt{sort}\bigl(\mathtt{delete}(m, L)\bigr)\Bigr) \\
        \leftrightarrow & x \preceq m \wedge \mathtt{le}\Bigl(x, \mathtt{sort}\bigl(\mathtt{delete}(m,L)\bigr)\Bigr) &
                          \mbox{nach Definition von \texttt{le}} \\
        \leftrightarrow & x \preceq m \wedge \mathtt{le}\Bigl(x, \mathtt{delete}(m,L) \Bigr) &
                          \mbox{nach Induktions-Voraussetzung} \\
        \leftrightarrow & x \preceq m \wedge \mathtt{le}\bigl(x, L \bigr) &
                          \mbox{denn unter der Voraussetzung $x \preceq m$ gilt nach } \\
                        & & \mbox{Lemma \ref{l4}: $\mathtt{le}\Bigl(x, \mathtt{delete}(m,L) \Bigr) \leftrightarrow \mathtt{le}\bigl(x, L \bigr)$} \\
        \leftrightarrow & x \preceq \mathtt{min}(L) \wedge \mathtt{le}\bigl(x, L \bigr) &
                          \mbox{wegen $m = \mathtt{min}(L)$} \\
        \leftrightarrow & \mathtt{le}\bigl(x, L \bigr) \wedge \mathtt{le}\bigl(x, L \bigr) &
                          \mbox{denn nach Lemma \ref{l2} gilt: $x \preceq \mathtt{min}(L) \leftrightarrow \mathtt{le}(x,L)$ } \\
        \leftrightarrow & \mathtt{le}\bigl(x, L \bigr).
     \end{array}\hspace*{\fill} \Box
     \]
\end{enumerate}


\begin{Satz}[Korrektheit von \emph{Sortieren durch Auswahl}, 1. Teil] 
{\em 
  F\"ur beliebige Listen $L$ gilt: \\[0.1cm]
  \hspace*{1.3cm} $\mathtt{isSorted}\bigl(\mathtt{sort}(L)\bigr)$.
}
\end{Satz}
\textbf{Beweis} durch Wertverlaufs-Induktion nach der Definition der Funktion \texttt{sort}:
\begin{enumerate}
\item Fall: $L=[]$.  Es gilt:

      \hspace*{1.3cm} 
       $\mathtt{isSorted}\bigl(\mathtt{sort}(L)\bigr) \leftrightarrow \mathtt{isSorted}\bigl(\mathtt{sort}([])\bigr) \leftrightarrow
          \mathtt{isSorted}\bigl([]\bigr) \leftrightarrow \mathtt{true}$.
\item Fall: $L \not= []$. Zur Abk\"urzung setzen wir $m = \mathtt{min}(L)$.  Es gilt:
      \[
      \begin{array}[b]{cll}
          & \mathtt{isSorted}\bigl(\mathtt{sort}(L)\bigr) \\ 
        \leftrightarrow & \mathtt{isSorted}\Bigl([m] + \mathtt{sort}\bigl(\mathtt{delete}(m,L)\bigr)\Bigr) &
                          \mbox{nach Definition von \texttt{sort}}     \\
        \leftrightarrow & \mathtt{le}\Bigl(m,\, \mathtt{sort}\bigl(\mathtt{delete}(m,\,L)\bigr)\Bigr) \;\wedge\; \\
                        & \mathtt{isSorted}\Bigl(\mathtt{sort}\bigl(\mathtt{delete}(m,\,L)\bigr)\Bigr) & 
                          \mbox{nach Definition von \texttt{isSorted}}     \\
        \leftrightarrow & \mathtt{le}\Bigl(m,\, \mathtt{sort}\bigl(\mathtt{delete}(m,\,L)\bigr)\Bigr) \;\wedge\;
                          \mathtt{true} &
                          \mbox{nach Induktions-Voraussetzung}     \\
        \leftrightarrow & \mathtt{le}\Bigl(m,\, \mathtt{sort}\bigl(\mathtt{delete}(m,\,L)\bigr)\Bigr) \\
        \leftrightarrow & \mathtt{le}\bigl(m,\, \mathtt{delete}(m,L)\bigr) &
                          \mbox{nach Lemma \ref{l6}}     \\
        \leftrightarrow & \mathtt{le}\bigl(m, L\bigr) &
                          \mbox{wegen $m \preceq m$ nach Lemma \ref{l4}}     \\
        \leftrightarrow & \mathtt{le}\bigl(\mathtt{min}(L), L\bigr) &
                          \mbox{wegen $m = \mathtt{min}(L)$}     \\
        \leftrightarrow & \mathtt{true}. &
                          \mbox{nach Korollar \ref{l3}}     
      \end{array}\hspace*{\fill} \Box
      \]
\end{enumerate}
Um das n\"achste Lemma elegant formulieren zu k\"onnen, ben\"otigen wir noch die Hilfsfunktion \\[0.1cm]
\hspace*{1.3cm} $\texttt{member}: M \times \mathtt{List}(M) \rightarrow \{0,1\}$, \\[0.1cm]
die \"uberpr\"uft, ob ein gegebenes Element in einer Liste auftritt.  Der Aufruf $\mathtt{member}(x,L)$
liefert genau dann 1, wenn $x$ in der Liste $L$ auftritt.  Rekursiv k\"onnen wir die Funktion 
\texttt{member} wie folgt definieren: 
\begin{enumerate}
\item $\mathtt{member}(x,[]) = 0$.
\item $\mathtt{member}(x,[x] + R) = 1$.
\item Falls $x \not= y$ ist, setzen wir $\mathtt{member}(x,[y] + R) = \mathtt{member}(x,R)$.
\end{enumerate}

\noindent
\textbf{Aufgabe}: Beweisen Sie die beiden folgenden Lemmata.
\begin{Lemma}[Distributivit\"at von \texttt{count} \"uber \texttt{delete}] \hspace*{\fill} \\[0.1cm]
{\em
  F\"ur beliebige Elemente $x$ und $y$ und f\"ur beliebige  Listen $L$ gilt: \\[0.1cm]
  \hspace*{1.3cm} 
  $\mathtt{count}\bigl(x, \mathtt{delete}(y,L)\bigr) = \mathtt{count}(x,L) - \mathtt{eq}(x,y)*\mathtt{member}(y,L)$.
}
\end{Lemma}

\noindent
Wir f\"uhren den \textbf{Beweis} durch Wertverlaufs-Induktion nach der Definition der Funktion \texttt{delete}:
\begin{enumerate}
\item Fall: $L = []$
      \[
      \begin{array}[b]{cll}
         & \mathtt{count}\bigl(x, \mathtt{delete}(y,L) \bigr) \\
       = & \mathtt{count}\bigl(x, \mathtt{delete}(y,[]) \bigr) \\
       = & \mathtt{count}\bigl(x, [] \bigr) \\
       = & \mathtt{count}\bigl(x, [] \bigr) - \mathtt{eq}(x,y) * 0\\
       = & \mathtt{count}\bigl(x, [] \bigr) - \mathtt{eq}(x,y) * \mathtt{member}(y,[])\\
       = & \mathtt{count}\bigl(x,  L \bigr) - \mathtt{eq}(x,y) * \mathtt{member}(y,L).
      \end{array}
      \]
\item Fall: $L = [y] + R$
      \[
      \begin{array}[b]{cll}
         & \mathtt{count}\bigl(x, \mathtt{delete}(y,L) \bigr) \\
       = & \mathtt{count}\bigl(x, \mathtt{delete}(y,[y] + R) \bigr) \\
       = & \mathtt{count}\bigl(x, R \bigr) \\
       = & \mathtt{eq}(x,y) + \mathtt{count}\bigl(x, R \bigr) - \mathtt{eq}(x,y) * 1 \\
       = & \mathtt{eq}(x,y) + \mathtt{count}\bigl(x, R \bigr) - \mathtt{eq}(x,y) * \mathtt{member}(y, [y] + R) \\
       = & \mathtt{count}\bigl(x, [y] + R \bigr) - \mathtt{eq}(x,y) * \mathtt{member}(y, [y] + R) \\
       = & \mathtt{count}\bigl(x,  L \bigr) - \mathtt{eq}(x,y) * \mathtt{member}(y,L).
      \end{array}
      \]
\item Fall: $L = [z] + R$ mit $y \not= z$
      \[
      \begin{array}[b]{cll}
         & \mathtt{count}\bigl(x, \mathtt{delete}(y,L) \bigr) \\
       = & \mathtt{count}\bigl(x, \mathtt{delete}(y,[z] + R) \bigr) \\
       = & \mathtt{count}\bigl(x, [z] + \mathtt{delete}(y, R) \bigr) \\
       = & \mathtt{eq}(x,z) + \mathtt{count}\bigl(x, \mathtt{delete}(y, R) \bigr) \\
       = & \mathtt{eq}(x,z) + \mathtt{count}\bigl(x, R \bigr) - \mathtt{eq}(x,y) * \mathtt{member}(y,R)&
           \mbox{nach Induktions-Voraussetzung}\\
       = & \mathtt{eq}(x,z) + \mathtt{count}\bigl(x, R \bigr) - \mathtt{eq}(x,y) * \mathtt{member}(y,[z]+R) &
           \mbox{nach Definition von \texttt{member}}\\
       = & \mathtt{count}\bigl(x, [z] + R \bigr) - \mathtt{eq}(x,y) * \mathtt{member}(y,[z]+R) &
           \mbox{nach Definition von \texttt{count}}\\
       = & \mathtt{count}\bigl(x, L \bigr) - \mathtt{eq}(x,y) * \mathtt{member}(y,L). 
      \end{array}\hspace*{\fill} \Box
      \]
\end{enumerate}

\begin{Lemma} 
{\em
  F\"ur jede nicht-leere Liste $L$ gilt: \\[0.1cm]
  \hspace*{1.3cm} 
  $\mathtt{member}\bigl(\mathtt{min}(L), L\bigr) = 1$.
}
\end{Lemma}
Beweis durch Wertverlaufs-Induktion nach der Definition der Funktion \texttt{min}.
\begin{enumerate}
\item $L = [x]$.  Es gilt
      \[
        \mathtt{member}\bigl(\mathtt{min}(L), L\bigr)  = 
        \mathtt{member}\bigl(\mathtt{min}([x]), [x]\bigr) 
         =  \mathtt{member}\bigl(x, [x]\bigr) = 1
      \]
\item Sei $L = [x] + R$.  Wir f\"uhren eine Fall-Unterscheidung nach der Gr\"o{\ss}e von $x$
      durch:
      \begin{enumerate}
      \item $x \preceq \mathtt{min}(R)$. \\[0.1cm]
        \hspace*{1.3cm} 
      $
      \begin{array}[b]{cll}
        & \mathtt{member}\bigl( \mathtt{min}(L), L \bigr) \\
      = & \mathtt{member}\bigl( \mathtt{min}([x]+R), [x] + R \bigr) \\
      = & \mathtt{member}\bigl( \mathtt{min}(x,\mathtt{min}(R)), [x] + R \bigr) \\
      = & \mathtt{member}\bigl( x, [x] + R \bigr) & 
          \mbox{wegen $x \preceq \mathtt{min}(R)$} \\
      = & 1 & 
          \mbox{nach Definition von \texttt{member}} \\
      \end{array}
      $
      \item $\neg x \preceq \mathtt{min}(R)$. \\[0.1cm]
        \hspace*{1.3cm} 
      $
      \begin{array}[b]{cll}
        & \mathtt{member}\bigl( \mathtt{min}(L), L \bigr) \\
      = & \mathtt{member}\bigl( \mathtt{min}([x]+R), [x] + R \bigr) \\
      = & \mathtt{member}\bigl( \mathtt{min}(x,\mathtt{min}(R)), [x] + R \bigr) \\
      = & \mathtt{member}\bigl( \mathtt{min}(R), [x] + R \bigr) & 
          \mbox{nach Definition von \texttt{min}} \\
      = & \mathtt{member}\bigl( \mathtt{min}(R), R \bigr) & 
          \mbox{nach Definition von \texttt{member}} \\
        & & \mbox{denn es muss $\mathtt{min}(R) \not= x$ gelten} \\
      = & 1. & 
          \mbox{nach Induktions-Voraussetzung} 
      \end{array}\hspace*{\fill} \Box
      $
      \end{enumerate}
\end{enumerate}

\begin{Satz}[Distributivit\"at von \texttt{count} \"uber \texttt{sort}] \hspace*{\fill} \\[0.1cm]
{\em
  Es sei $x$ ein Element und $L$ sei eine Liste.  Dann gilt \\[0.1cm]
  \hspace*{1.3cm} $\mathtt{count}\bigl(x, \mathtt{sort}(L)\bigr) = \mathtt{count}(x, L)$.
}
\end{Satz}

\noindent
\textbf{Beweis} durch Wertverlaufs-Induktion nach der Definition von \texttt{sort}.
\begin{enumerate}
\item Fall: $L=[]$.  Nach Definition von $\mathtt{sort}$ gilt 

     \hspace*{1.3cm} 
     $\mathtt{count}\bigl(x, \mathtt{sort}(L)\bigr) = \mathtt{count}\bigl(x, \mathtt{sort}([])\bigr) = \mathtt{count}\bigl(x,[]\bigr) = \mathtt{count}\bigl(x,L\bigr)$.
\item Fall: $L \not= []$.  Es sei $m = \mathtt{min}(L)$. Dann gilt:
     \[ 
     \begin{array}[b]{cll}
        & \mathtt{count}\bigl(x, \mathtt{sort}(L)\bigr) \\
      = & \mathtt{count}\Bigl(x, [m] + \mathtt{sort}\bigl(\mathtt{delete}(m,L)\bigr)\Bigr) &
         \mbox{nach Definition von \texttt{sort}} \\
      = & \mathtt{eq}(x,m) + \mathtt{count}\Bigl(x, \mathtt{sort}\bigl(\mathtt{delete}(m,L)\bigr)\Bigr) &
         \mbox{nach Definition von \texttt{count}} \\
      = & \mathtt{eq}(x,m) + \mathtt{count}\bigl(x, \mathtt{delete}(m,L)\bigr) &
         \mbox{nach Induktions-Voraussetzung} \\
      = & \mathtt{eq}(x,m) + \mathtt{count}\bigl(x, L) - \mathtt{eq}(x,m)*\mathtt{member}(m,L) &
         \mbox{Distributivit\"at von \texttt{count} \"uber \texttt{delete}} \\
      = & \mathtt{eq}(x,m) + \mathtt{count}\bigl(x, L) - \mathtt{eq}(x,m) &
         \mbox{denn $\mathtt{member}(m,L) = 1$} \\
      = & \mathtt{count}\bigl(x, L).
     \end{array}\hspace*{\fill} \Box
     \]
\end{enumerate}

\subsection{Komplexit\"at}
Um die Komplexit\"at von ``\emph{Sortieren durch Auswahl}'' analysieren zu k\"onnen, m\"ussen
wir zun\"achst die Anzahl der Vergleiche, die bei der Berechnung von $\mathtt{min}(L)$
durchgef\"uhrt werden, bestimmen.  Es gilt \\[0.1cm]
\hspace*{1.3cm} 
$\mathtt{min}([x_1,x_2,x_3,\cdots,x_n]) = \mathtt{min}(x_1, \mathtt{min}(x_2, \mathtt{min}(x_3, \cdots \mathtt{min}(x_{n-1},x_n) \cdots )))$. 
\\[0.1cm]
Also wird bei der Berechnung von $\texttt{min}(L)$ f\"ur eine Liste $L$ der L\"ange $n$ der Operator
\texttt{min} insgesamt $(n-1)$-mal aufgerufen.  Jeder Aufruf von \texttt{min} bedingt dann
einen Aufruf des Vergleichs-Operators ``$\preceq$''.
Bezeichnen wir die Anzahl der Vergleiche bein Aufruf von $\texttt{sort}(L)$ f\"ur eine
Liste der L\"ange $L$ mit $b_n$, so finden wir also \\[0.1cm]
\hspace*{1.3cm} $b_0 = 0$ \quad und \quad $b_{n+1} = b_n + n$, \hspace*{\fill}\\[0.1cm]
denn um eine Liste mit $n+1$ Elementen zu sortieren, muss zun\"achst das Minimum dieser
Liste berechnet werden.  Dazu sind $n$ Vergleiche notwendig.  Dann wird das Minimum aus
der Liste entfernt und die Rest-Liste, die ja nur noch $n$ Elemente enth\"alt, wird rekursiv
sortiert.  Das liefert den Beitrag $b_n$ in der obigen Summe.

Bei der Berechnung der Komplexit\"at von ``\emph{Sortieren durch Einf\"ugen}'' hatten wir die
selbe Rekurrenz-Gleichung erhalten.  Die L\"osung dieser Rekurrenz-Gleichung lautet also \\[0.1cm]
\hspace*{1.3cm} $b_n = \frac{1}{2} * n^2 - \frac{1}{2}* n = \frac{1}{2} * n^2 + \Oh(n)$. \\[0.1cm]
Das sieht so aus, als ob die Anzahl der Vergleiche beim ``\emph{Sortieren durch
  Einf\"ugen}'' genau so w\"are wie beim ``\emph{Sortieren durch Auswahl}''.  Das stimmt aber
nicht.  Bei ``\emph{Sortieren durch Einf\"ugen}'' ist die Anzahl der durchgef\"uhrten
Vergleiche im schlimmsten Fall $\frac{1}{2}n*(n-1)$, beim ``\emph{Sortieren durch
  Auswahl}'' ist Anzahl der Vergleiche \underline{immer} $\frac{1}{2}n*(n-1)$.  Der Grund
ist, dass zur Berechnung des Minimums einer Liste mit $n$ Elementen immer $n-1$ Vergleiche
erforderlich sind.  Um aber ein Element in eine Liste mit $n$ Elementen einzuf\"ugen, sind
im \underline{Durchschnitt} nur etwa $\frac{1}{2}n$ Vergleiche erforderlich, denn im Schnitt sind etwa
die H\"alfte der Elemente kleiner als das einzuf\"ugende Element und daher m\"ussen beim Einf\"ugen
in eine sortierte Liste der L\"ange $n$ im Durchschnitt nur die ersten $\frac{n}{2}$
Elemente betrachtet werden.  Daher ist die durchschnittliche Anzahl von Vergleichen beim
``\emph{Sortieren durch Einf\"ugen}'' $\frac{1}{4}n^2 + \Oh(n)$, also halb so gro{\ss} wie beim
``\emph{Sortieren durch Auswahl}''.


\subsection{Eine feldbasierte Implementierung}
In der Anwendung besteht die Aufgabe h\"aufig darin, ein Feld von Daten zu sortieren.
Bisher haben wir nur Listen sortiert.  Daher pr\"asentieren wir
zum Abschluss unserer Diskussion des Algorithmus ``\emph{Sortieren durch Auswahl}''
noch eine Implementierung dieses Algorithmus' in der Sprache \textsl{Java},
in der die zu sortierenden Daten in einem Feld an Stelle einer Liste vorliegen.
Abbildung \ref{fig:MinSortAlgorithm.java} auf Seite \pageref{fig:MinSortAlgorithm.java} zeigt
diese Implementierung.  Die Klasse \texttt{MinSortAlgorithm} implementiert die
Schnittstelle \\[0.1cm]
\hspace*{1.3cm} \texttt{SortingAlgorithm}, \\[0.1cm]
die in Abbildung
\ref{fig:SortingAlgorithm.java} gezeigt wird.  Diese Schnittstelle schreibt die
Implementierung einer einzigen Methode vor, der Methode \\[0.1cm]
\hspace*{1.3cm} \texttt{void sort();} \\[0.1cm]
Die Einhaltung dieser Schnittstelle ist notwendig, um das Programm sp\"ater in eine Umgebung
zur Algorithmen-Visualisierung einbinden zu k\"onnen.

\begin{figure}[!ht]
  \centering
\begin{Verbatim}[ frame         = lines, 
                  framesep      = 0.3cm, 
                  labelposition = bottomline,
                  numbers       = left,
                  numbersep     = -0.2cm,
                  xleftmargin   = 0.3cm,
                  xrightmargin  = 0.3cm
                ]
    public interface SortingAlgorithm {
        public void sort();
    }
\end{Verbatim}
\vspace*{-0.3cm}
  \caption{Das Interface ``\texttt{SortingAlgorithm}''}
  \label{fig:SortingAlgorithm.java}
\end{figure}


In Zeile 3 der Implementierung der Klass \texttt{MinSortAlgorithm} (Abbildung
\ref{fig:MinSortAlgorithm.java})
 definieren wir das zu sortierende Feld und in Zeile 4 definieren wir ein Objekt
vom Typ \texttt{Comparator}.  Dieses Objekt hat eine Methode \\[0.1cm]
\hspace*{1.3cm} \texttt{int compare(Double x, Double y);}\\[0.1cm]
mit deren Hilfe wir zwei Zahlen vergleichen k\"onnen.  Es gilt \\[0.1cm]
\hspace*{1.3cm} 
$\mathtt{compare}(x,y) = \left\{
 \begin{array}[c]{rl}
  -1 & \mbox{falls $x < y$;} \\ 
   0 & \mbox{falls $x = y$;} \\ 
   1 & \mbox{falls $x > y$.} \\ 
 \end{array}\right.
$ %\$
\\[0.1cm]
Sowohl das zu sortierende Feld als auch der Komparator werden dem Konstruktor in Zeile 
6 als Argument mitgegeben.  Der Konstruktor initialisiert mit diesen Argumenten die
entsprechenden Member-Variablen.   

\begin{figure}[!ht]
  \centering
\begin{Verbatim}[ frame         = lines, 
                  framesep      = 0.3cm, 
                  labelposition = bottomline,
                  numbers       = left,
                  numbersep     = -0.2cm,
                  xleftmargin   = 0.3cm,
                  xrightmargin  = 0.3cm
                ]
    public class MinSortAlgorithm implements SortingAlgorithm
    {
        private Double[]           mArray;
        private Comparator<Double> mComparator;
    
        MinSortAlgorithm(Double[] array, Comparator<Double> comparator) {
            mArray      = array;
            mComparator = comparator;
        }
        public void sort() { sort(0); }
    
        private void sort(int i) {
            if (i == mArray.length)
                return;
            int minIndex = minIndex(i);
            swap(i, minIndex);
            sort(i + 1);
        }
        private int minIndex(int first) {
            int minIndex = first;
            for (int i = first + 1; i < mArray.length; ++i) {
                if (mComparator.compare(mArray[minIndex], mArray[i]) > 0) {
                    minIndex = i;
                }
            }
            return minIndex;
        }
        private void swap(int i, int j) {
            if (i == j) return;
            Double temp = mArray[i];
            mArray[i]   = mArray[j];
            mArray[j]   = temp;
        }
    }
\end{Verbatim}
\vspace*{-0.3cm}
  \caption{Der Algorithmus ``\emph{Sortieren durch Auswahl}''}
  \label{fig:MinSortAlgorithm.java}
\end{figure}

Die Implementierung der Methode \texttt{sort()} in Zeile 10 ist trivial, denn die ganze
Arbeit wird in der Hilfs-Methode \texttt{sort(int)} abgewickelt.  F\"ur eine 
nat\"urliche Zahl $i$ mit $i < \mathtt{mArray.length}$ sortiert der Aufruf 
 $\texttt{sort}(i)$ 
den Teil des Feldes, dessen Indizes gr\"o{\ss}er oder gleich dem Argument $i$ sind, nach dem
Aufruf $\mathtt{sort}(i)$ ist also die Liste \\[0.1cm]
\hspace*{1.3cm} 
$\bigl[\;\mathtt{mArray}[i],\;\mathtt{mArray}[i+1],\;\cdots,\;\mathtt{mArray}[\mathtt{mArray.length}-1] \;\bigr]$
\\[0.1cm]
sortiert.  Die Implementierung der Methode \texttt{sort(int)} ist rekursiv:
\begin{enumerate}
\item Falls $i = \mathtt{mArray.length}$ ist, dann ist der zu sortierende Teil des 
      Feldes leer und folglich ist nichts zu tun.
\item Andernfalls berechnet der Aufruf $\texttt{minIndex}(i)$ in Zeile 15 einen Index
      \texttt{minIndex} so, dass das Element \texttt{mArray[minIndex]} in dem zu
      sortierenden Teil der Liste minimal ist, es gilt also \\[0.1cm]
      \hspace*{1.3cm} 
      $\forall j \in \bigl\{\, i,\, i+1,\, \cdots,\, \texttt{mArray.length} -1\, \} \colon\;
       \texttt{mArray}[\texttt{minIndex}] \leq \texttt{mArray}[i]$. \\[0.1cm]
      Anschlie{\ss}end wird das Element, das an der Stelle \texttt{minIndex} steht mit dem
      Element an der Stelle $i$ vertauscht.  Damit steht an der Stelle $i$ jetzt ein
      kleinstes Element der Liste \\[0.1cm]
      \hspace*{1.3cm} $\bigl[\;\mathtt{mArray}[i],\;\mathtt{mArray}[i+1],\;\cdots,\;\mathtt{mArray}[\mathtt{mArray.length}-1] \;\bigr]$.
      \\[0.1cm]
      Sortieren wir danach rekursiv die Liste \\[0.1cm]
      \hspace*{1.3cm} $\bigl[\;\mathtt{mArray}[i+1],\;\cdots,\;\mathtt{mArray}[\mathtt{mArray.length}-1] \;\bigr]$,
      \\[0.1cm]
      so ist anschlie{\ss}end auch die Liste \\[0.1cm]
      \hspace*{1.3cm}
      $\bigl[\;\mathtt{mArray}[i],\;\mathtt{mArray}[i+1],\;\cdots,\;\mathtt{mArray}[\mathtt{mArray.length}-1] \;\bigr]$
      \\[0.1cm]
      sortiert, denn $\texttt{mArray}[i]$ ist ja ein minimales Element.
\end{enumerate}
Die Implementierung der Methode $\texttt{minIndex}(\textsl{first})$ berechnet den Index eines kleinsten
Elements iterativ.  Zun\"achst wird \texttt{minIndex} mit dem Index \textsl{first}
initialisiert.  Anschlie{\ss}end wird eine Schleife durchlaufen.  Falls ein
Index $i$ gefunden wird, so dass das Element $\mathtt{mArray}[i]$ kleiner als das Element
$\texttt{mArray}[\textsl{minIndex}]$ ist, dann wird \texttt{minIndex} auf $i$ gesetzt.
Dadurch ist gew\"ahrleistet dass \texttt{minIndex} am Ende der Schleife tats\"achlich auf das
kleinste Element zeigt.
\pagebreak

\section{Sortieren durch Mischen}
Wir stellen nun einen Algorithmus zum Sortieren vor, der f\"ur gro{\ss}e Listen erheblich
effizienter ist als die beiden bisher betrachteten Algorithmen.  Der Algorithmus wird als
``\emph{Sortieren durch Mischen}'' (engl. \emph{merge sort}) bezeichnet und verl\"auft nach dem folgenden Schema:
\begin{enumerate}
\item Hat die zu sortierende Liste $L$ weniger als zwei Elemente, so wird $L$ zur\"uck
      gegeben: \\[0.1cm]
      \hspace*{1.3cm} $\#L < 2 \rightarrow \mathtt{sort}(L) = L$.
\item Ansonsten wird die Liste in zwei etwa gleich gro{\ss}e Listen zerlegt.
      Diese Listen werden rekursiv sortiert und anschlie{\ss}end so gemischt, dass
      das Ergebnis sortiert ist: \\[0.1cm]
      \hspace*{1.3cm} 
 $\#L \geq 2 \rightarrow \mathtt{sort}(L) = \mathtt{merge}\Bigl(\mathtt{sort}\bigl(\mathtt{split}_1(L)\bigr), \mathtt{sort}\bigl(\mathtt{split}_2(L)\bigr)\Bigr)$
     \\[0.1cm]
     Hier bezeichnen $\texttt{split}_1$ und $\mathtt{split}_2$ die Funktionen, die eine Liste in zwei Teil-Listen zerlegen
     und \texttt{merge} ist eine Funktion, die zwei sortierte Listen so mischt, dass das Ergebnis wieder sortiert ist.
\end{enumerate}
Abbildung \ref{fig:merge-sort} zeigt die Umsetzung dieser sortierten Gleichungen 
in ein \textsl{Java}-Programm, das eine verkettete Liste sortiert. 
Die beiden Funktionen $\mathtt{split}_1$ und $\mathtt{split}_2$ haben wir dabei
zu einer Funktion \texttt{split} zusammengefasst, die zwei Listen zur\"uck liefert. Ein
Aufruf der Form
\\[0.2cm]
\hspace*{1.3cm}
$\mathtt{split}(L, L_1, L_2)$
\\[0.2cm]
verteilt die Elemente der Liste $L$ auf die beiden Listen $L_1$ und $L_2$.  
Damit das funktioniert, m\"ussen die Variablen $L_1$ und $L_2$ zu Beginn des Aufrufs auf
leere Listen verweisen.  Damit k\"onnen wir die Details des \textsl{Java}-Programms in
Abbildung \ref{fig:merge-sort} verstehen:


\begin{figure}[!ht]
  \centering
\begin{Verbatim}[ frame         = lines, 
                  framesep      = 0.3cm, 
                  labelposition = bottomline,
                  numbers       = left,
                  numbersep     = -0.2cm,
                  xleftmargin   = 0.8cm,
                  xrightmargin  = 0.8cm
                ]
    public LinkedList<Double> sort(LinkedList<Double> list) {
        if (list.size() < 2) {
            return list;
        }
        LinkedList<Double> first  = new LinkedList<Double>();
        LinkedList<Double> second = new LinkedList<Double>();
        split(list, first, second);
        LinkedList<Double> firstSorted  = sort(first);
        LinkedList<Double> secondSorted = sort(second);
        return merge(firstSorted, secondSorted);
    }
\end{Verbatim}
\vspace*{-0.3cm}
  \caption{Die Methode \texttt{sort}.}
  \label{fig:merge-sort}
\end{figure}
\begin{enumerate}
\item Wenn die zu sortierende Liste aus weniger als zwei Elementen besteht,
      dann ist diese Liste bereits sortiert und wir k\"onnen diese Liste in Zeile 3
      unver\"andert zur\"uck geben.
\item In den Zeilen 5 und 6 legen wir zwei leere Listen \texttt{first} und \texttt{second}
      an.
\item Der Aufruf von \texttt{split} verteilt die Elemente der zu sortierenden Liste
      auf die beiden Listen \texttt{first} und \texttt{second}.
\item In den Zeilen 8 und 9 werden diese Listen durch einen rekursiven Aufruf der Methode
      \texttt{sort} sortiert.
\item Die sortierten Teillisten werden dann in Zeile 10 durch den Aufruf der Methode
      \texttt{merge} zu einer sortierten Liste zusammengefasst.
\end{enumerate}

\noindent
Als n\"achstes \"uberlegen wir uns, wie wir die Funktion $\texttt{split}$ durch Gleichungen spezifizieren k\"onnen.
\begin{enumerate}
\item Falls $L$ leer ist, produziert $\mathtt{split}(L)$ zwei leere Listen:\\[0.1cm]
      \hspace*{1.3cm} 
      $\mathtt{split}(\texttt{[]}) = \mathtt{[} \texttt{[]}, \texttt{[]} \mathtt{]}$.
\item Falls $L$ genau ein Element besitzt, stecken wir dieses in die erste Liste: \\[0.1cm]
      \hspace*{1.3cm} 
      $\mathtt{split}(\mathtt{[}x\mathtt{]}) = \mathtt{[} \texttt{[}x\texttt{]}, \texttt{[]} \mathtt{]}$.
\item Sonst hat $L$ die Form $\mathtt{[}x, y\mathtt{]} + R$.
      Dann spalten wir rekursiv $R$ in zwei Listen auf. Vor die erste Liste f\"ugen wir $x$ an,
      vor die zweite Liste f\"ugen wir $y$ an:
      \\[0.1cm]
      \hspace*{1.3cm} 
      $\mathtt{split}(R) = \mathtt{[}R_1, R_2\mathtt{]} \rightarrow
      \mathtt{split}\bigl(\mathtt{[}x, y\mathtt{]} + R\bigr) = \bigl[ [x] + R_1, [y] + R_2 \bigr]$.
\end{enumerate}
Abbildung \ref{fig:split} auf Seite \pageref{fig:split} 
zeigt die Umsetzung dieser bedingten Gleichungen in \textsl{Java}.


\begin{figure}[!ht]
  \centering
\begin{Verbatim}[ frame         = lines, 
                  framesep      = 0.3cm, 
                  labelposition = bottomline,
                  numbers       = left,
                  numbersep     = -0.2cm,
                  xleftmargin   = 0.3cm,
                  xrightmargin  = 0.3cm
                ]
    public void split(LinkedList<Double> list, 
                      LinkedList<Double> first, LinkedList<Double> second) 
    {
        if (list.size() == 0) {
            return;
        }
        Double x = list.removeFirst();
        if (list.size() == 0) {
            first.addFirst(x);
            return;
        }
        Double y = list.removeFirst();
        split(list, first, second);
        first .addFirst(x);
        second.addFirst(y);
    }
\end{Verbatim}
\vspace*{-0.3cm}
  \caption{Die Methode \texttt{split}.}
  \label{fig:split}
\end{figure}
\begin{enumerate}
\item Falls die Liste, deren Elemente auf zwei Listen verteilt werden sollen, leer ist,
      so ist nichts weiter zu tun, denn wir setzen voraus, dass die Listen \texttt{first}
      und \texttt{second} beim Aufruf der Methode \texttt{split} leer sind.
\item Andernfalls entfernen wir zun\"achst des erste Element aus der Liste \texttt{list} und
      speichern es in der Variablen $x$.
      Falls die Liste danach leer ist, f\"ugen wir das Element $x$
      in die Liste \texttt{first} ein und beenden den Methoden-Aufruf.
\item Wenn der Kontrollfluss in Zeile 12 ankommt, dann muss die Liste \texttt{list} beim Aufruf
      wenigstens zwei Elemente gehabt haben.  Wir entfernen nun das zweite dieser beiden
      Elemente und speichern es in der Variablen $y$.
\item Anschlie{\ss}end teilen wir das, was jetzt noch von der Liste \texttt{list} \"ubrig ist,
      durch den rekursiven Aufruf von \texttt{split} auf die Listen \texttt{first} und
      \texttt{second} auf.
\item Zum Abschluss f\"ugen wir das Element $x$ in die Liste \texttt{first} ein
      und das Element $y$ schieben wir in die Liste \texttt{second}.
\end{enumerate}


Als letztes spezifizieren wir, wie zwei sortierte Listen $L_1$ 
und $L_2$ so gemischt werden k\"onnen, dass das Ergebnis anschlie{\ss}end wieder sortiert ist.
\begin{enumerate}
\item Falls die Liste $L_1$ leer ist, ist das Ergebnis $L_2$: \\[0.1cm]
      \hspace*{1.3cm} 
      $\mathtt{merge}([], L_2) = L_2$.
\item Falls die Liste $L_2$ leer ist, ist das Ergebnis $L_1$: \\[0.1cm]
      \hspace*{1.3cm} 
      $\mathtt{merge}(L_1, []) = L_1$.
\item Andernfalls hat $L_1$ die Form $[x] + R_1$ und $L_2$ hat die Gestalt $[y] + R_2$.
      Dann f\"uhren wir eine Fallunterscheidung nach der relativen Gr\"o{\ss}e von $x$ und $y$ durch:
      \begin{enumerate}
      \item $x \preceq y$.

            In diesem Fall mischen wir $R_1$ und $L_2$ und setzen $x$ an den Anfang dieser Liste:\\[0.1cm]
            \hspace*{1.3cm} 
            $x \preceq y \rightarrow \mathtt{merge}\bigl([x]+R_1, [y]+R_2\bigr) = [x] + \mathtt{merge}\bigl(R_1,[y]+R_2\bigr)$.
      \item $\neg x \preceq y$.

            In diesem Fall mischen wir $L_1$ und $R_2$ und setzen $y$ an den Anfang dieser Liste:\\[0.1cm]
            \hspace*{1.3cm} 
            $\neg x \preceq y \rightarrow \mathtt{merge}\bigl([x]+R_1, [y]+R_2\bigr) = [y] + \mathtt{merge}\bigl([x] + R_1,R_2\bigr)$.
      \end{enumerate}
\end{enumerate}
Abbildung \ref{fig:merge} auf Seite \pageref{fig:merge} 
zeigt die Umsetzung dieses Algorithmus in \textsl{Java}.

\begin{figure}[!ht]
  \centering
\begin{Verbatim}[ frame         = lines, 
                  framesep      = 0.3cm, 
                  labelposition = bottomline,
                  numbers       = left,
                  numbersep     = -0.2cm,
                  xleftmargin   = 0.8cm,
                  xrightmargin  = 0.8cm
                ]
    public LinkedList<Double> merge(LinkedList<Double> first, 
                                    LinkedList<Double> second) 
    {
        if (first .size() == 0) { return second; }
        if (second.size() == 0) { return first;  }
        LinkedList<Double> result;
        Double x = first .getFirst();
        Double y = second.getFirst();
        if (x < y) {
            first.removeFirst();
            result = merge(first, second);
            result.addFirst(x);
        } else {
            second.removeFirst();
            result = merge(first, second);
            result.addFirst(y);
        }
        return result;
    }
\end{Verbatim}
\vspace*{-0.3cm}
  \caption{Die Methode \texttt{merge}.}
  \label{fig:merge}
\end{figure}

\begin{enumerate}
\item Falls eine der beiden Listen leer ist, so geben wir als Ergebnis die
      andere Liste zur\"uck.
\item Wenn der Kontrollfluss in Zeile 7 ankommt, dann wissen wir, dass beide Listen
      nicht leer sind.  Wir holen uns jeweils das erste Element der beiden Listen
      und speichern diese in den Variablen $x$ und $y$ ab.  Da wir diese Elemente
      aber mit Hilfe der Methode \texttt{getFirst} bekommen, bleiben diese Elemente
      zun\"achst Bestandteil der beiden Listen.
\item Anschlie{\ss}end pr\"ufen wir, welche der beiden Variablen die kleinere ist.
      \begin{enumerate}
      \item Falls $x$ kleiner als $y$ ist, so entfernen wir $x$ aus der ersten Liste
            und mischen rekursiv die verk\"urzte erste Liste mit der zweiten Liste.
            Anschlie{\ss}en f\"ugen wir $x$ an den Anfang der Ergebnis-Liste ein.
      \item Andernfalls entfernen wir $y$ aus der zweiten Liste
            und mischen rekursiv die erste Liste mit der verk\"urzten zweiten Liste und
            f\"ugen dann $y$ am Anfang der beim rekursiven Aufruf erhaltenen Liste ein.
      \end{enumerate}
\end{enumerate}
Zum Abschluss zeigen wir in Abbildung \ref{fig:MergeSort}, wie die eben diskutierten
Methoden zusammen spielen.

\begin{figure}[!ht]
\centering
\begin{Verbatim}[ frame         = lines, 
                  framesep      = 0.3cm, 
                  labelposition = bottomline,
                  numbers       = left,
                  numbersep     = -0.2cm,
                  xleftmargin   = 0.0cm,
                  xrightmargin  = 0.0cm,
                  commandchars  = \\\{\}
                ]
    import java.util.*;
    
    public class MergeSort 
    \{  
        public LinkedList<Double> sort(LinkedList<Double> list) \{
            \(\cdots\)
        \}
        public void split(LinkedList<Double> list,
                          LinkedList<Double> first, LinkedList<Double> second) 
        \{
            \(\cdots\)
        \}
        public LinkedList<Double> merge(LinkedList<Double> first, 
                                        LinkedList<Double> second) 
        \{
            \(\cdots\)
        \}    
        public static void main(String[] args) \{
            Double[] a = \{ 3.0, 7.0, 5.0, 2.0, 4.0, 2.0, 11.0, 1.0 \};
            LinkedList<Double> list = new LinkedList<Double>();
            for (Double x : a) \{
                list.addFirst(x);
            \}
            MergeSort MS = new MergeSort();
            System.out.println(list);
            list = MS.sort(list);
            System.out.println(list);
        \}
    \}
\end{Verbatim}
\vspace*{-0.3cm}
\caption{Die Klasse \texttt{MergeSort}}
\label{fig:MergeSort}
\end{figure}


\subsection{Korrektheit}
Um die Korrektheit des Algorithmus ``\emph{Sortieren durch Mischen}'' zu zeigen, 
ben\"otigen wir zun\"achst einige Hilfs-S\"atze, die das Verhalten der Funktionen \texttt{split} und
\texttt{merge} beschreiben.

\begin{Lemma}[Distributivit\"at von \texttt{le} \"uber \texttt{merge}] \hspace*{\fill} \\
{\em
  F\"ur beliebige Listen $L_1$ und $L_2$ gilt: \\[0.1cm]
  \hspace*{1.3cm} 
  $\texttt{le}\bigl(x,\texttt{merge}(L_1,L_2)\bigr) \leftrightarrow \texttt{le}(x,L_1) \wedge \mathtt{le}(x,L_2)$.
}
\end{Lemma}

\noindent
\textbf{Beweis}: Wir f\"uhren den Beweis durch Wertverlaufs-Induktion entsprechend der Definition
der Funktion \texttt{merge}.
\begin{enumerate}
\item Fall: $L_1 = []$.  Wegen $\texttt{merge}([],L_2) = L_2$ gilt einerseits \\[0.1cm]
      \hspace*{1.3cm} 
       $\texttt{le}\bigl(x,\texttt{merge}(L_1,L_2)\bigr) \leftrightarrow \texttt{le}\bigl(x,\texttt{merge}([],L_2)\bigr) \leftrightarrow \texttt{le}(x,L_2)$ \\[0.1cm]
       und andererseits haben wir wegen $\texttt{le}(x,[]) \leftrightarrow \mathtt{true}$ \\[0.1cm]
       \hspace*{1.3cm} 
       $\texttt{le}(x,L_1) \wedge \mathtt{le}(x,L_2) \leftrightarrow \texttt{le}(x,[])
       \wedge \mathtt{le}(x,L_2) \leftrightarrow \mathtt{true} \wedge \mathtt{le}(x,L_2) \leftrightarrow \mathtt{le}(x,L_2)$.
\item Fall: $L_2 = []$.  Der Beweis ist analog zum 1.~Fall.
\item Fall: $L_1 = [y] + R_1$, $L_2 = [z] + R_2$, $y \preceq z$.  Es gilt:
      \[
      \begin{array}{cll}
         & \mathtt{le}\bigl(x, \mathtt{merge}(L_1, L_2) \bigr)\\
       \leftrightarrow & \mathtt{le}\bigl(x, \mathtt{merge}([y]+R_1, [z]+R_2) \bigr)\\
       \leftrightarrow & \mathtt{le}\bigl(x, [y] + \mathtt{merge}(R_1, [z]+R_2) \bigr) &
                         \mbox{wegen $y \preceq z$} \\
       \leftrightarrow & x \preceq y \wedge \mathtt{le}\bigl(x, \mathtt{merge}(R_1, [z]+R_2) \bigr) &
                         \mbox{nach Definition von \texttt{le}} \\
       \leftrightarrow & x \preceq y \wedge \mathtt{le}(x, R_1) \wedge \mathtt{le}(x, [z]+R_2) \bigr) &
                         \mbox{nach Induktions-Voraussetzung} \\
       \leftrightarrow &  \mathtt{le}(x, [y] + R_1) \wedge \mathtt{le}(x, [z]+R_2) \bigr) &
                         \mbox{nach Definition von \texttt{le}} \\
       \leftrightarrow &  \mathtt{le}(x, L_1) \wedge \mathtt{le}(x, L_2) \bigr) 
      \end{array}
      \]
\item Fall:  $L_1 = [y] + R_1$, $L_2 = [z] + R_2$, $\neg y \preceq z$.  

      Der Beweis verl\"auft analog zum 3.~Fall. \hspace*{\fill} $\Box$
\end{enumerate}

\begin{Lemma}[Vertr\"aglichkeit von \texttt{isSorted} und \texttt{merge}] \hspace*{\fill} \\
{\em
  F\"ur beliebige Listen $L_1$ und $L_2$ gilt: \\[0.1cm]
  \hspace*{1.3cm} 
  $\texttt{isSorted}\bigl(\texttt{merge}(L_1,L_2)\bigr) \leftrightarrow \mathtt{isSorted}(L_1) \wedge \mathtt{isSorted}(L_2)$.
}
\end{Lemma}

\noindent
\textbf{Beweis} durch Wertverlaufs-Induktion nach der Definition der Funktion \texttt{merge}.
\begin{enumerate}
\item Fall: $L_1 = []$.  Es gilt 
      \[
      \begin{array}{cll}
                        & \mathtt{isSorted}\bigl(\mathtt{merge}(L_1, L_2)\bigr) \\
        \leftrightarrow & \mathtt{isSorted}\bigl(\mathtt{merge}([], L_2)\bigr) \\
        \leftrightarrow & \mathtt{isSorted}(L_2) & \mbox{nach Definition von \texttt{merge}} \\
        \leftrightarrow & \mathtt{true} \wedge \mathtt{isSorted}(L_2)  \\
        \leftrightarrow & \mathtt{isSorted}([]) \wedge \mathtt{isSorted}(L_2)  &
                          \mbox{nach Definition von \texttt{isSorted}} \\
        \leftrightarrow & \mathtt{isSorted}(L_1) \wedge \mathtt{isSorted}(L_2)  &
      \end{array}
      \]
\item Fall: $L_2 = []$. 

      Dieser Fall ist analog zum 1.~Fall.
\item Fall: $L_1 = [x_1] + R_1$,   $L_2 = [x_2] + R_2$, $x_1 \preceq x_2$.
      \[
      \begin{array}{cl}
        & \texttt{isSorted}\bigl(\texttt{merge}(L_1, L_2)\bigr) \\
      \leftrightarrow  & \texttt{isSorted}\bigl(\texttt{merge}([x_1] + R_1, [x_2] + R_2)\bigr) \\
      \leftrightarrow  & \texttt{isSorted}\bigl([x_1] + \texttt{merge}(R_1, [x_2] + R_2)\bigr) \\
                       &          \mbox{nach Definition von \texttt{merge}}  \\
      \leftrightarrow  & \texttt{le}\bigl(x_1,\texttt{merge}(R_1, [x_2] + R_2)\bigr) \wedge 
                         \texttt{isSorted}\bigl(\texttt{merge}(R_1, L_2)\bigr)   \\
                       & \mbox{nach Definition von \texttt{isSorted}}  \\
      \leftrightarrow  & \texttt{le}(x_1,R_1) \wedge \mathtt{le}(x_1, [x_2] + R_2) \wedge 
                          \texttt{isSorted}\bigl(\texttt{merge}(R_1, L_2)\bigr)   \\
                       & \mbox{wegen der Distributivit\"at von \texttt{le} \"uber \texttt{merge}}  \\
      \leftrightarrow  & \texttt{le}(x_1,R_1) \wedge \mathtt{le}(x_1, [x_2] + R_2) \wedge 
                         \texttt{isSorted}(R_1) \wedge \mathtt{isSorted}(L_2) \\
                       & \mbox{nach Induktions-Voraussetzung}  \\
      \leftrightarrow  & \texttt{le}(x_1,R_1) \wedge \texttt{isSorted}(R_1) \wedge \mathtt{le}(x_1, [x_2] + R_2) \wedge \mathtt{isSorted}(L_2) \\
      \leftrightarrow  & \texttt{isSorted}(L_1) \wedge \mathtt{le}(x_1, [x_2] + R_2) \wedge \mathtt{isSorted}(L_2) \\
                       & \mbox{nach Definition von \texttt{isSorted}} \\
      \leftrightarrow  & \texttt{isSorted}(L_1) \wedge x_1 \preceq x_2 \wedge \mathtt{le}(x_1, R_2) \wedge \mathtt{isSorted}(L_2) \\
                       & \mbox{nach Definition von \texttt{le}} \\
      \leftrightarrow  & \texttt{isSorted}(L_1) \wedge \mathtt{le}(x_1, R_2) \wedge \mathtt{isSorted}(L_2) \\
                       & \mbox{wegen $x_1\preceq x_2$} \\
      \end{array}
      \]
      Der Beweis ist abgeschlossen, wenn wir \\[0.1cm]
      \hspace*{1.3cm} $\mathtt{isSorted}(L_2) \leftrightarrow \mathtt{le}(x_1, R_2) \wedge \mathtt{isSorted}(L_2)$
      \\[0.1cm]
      zeigen k\"onnen. Dazu reicht es offenbar, wenn wir \\[0.1cm]
      \hspace*{1.3cm} $\mathtt{isSorted}([x_2] + R_2) \rightarrow \mathtt{le}(x_1, R_2) \wedge \mathtt{isSorted}([x_2] + R_2)$
      \hspace*{\fill} $(\star)$ \\[0.1cm]
      zeigen, denn die umgekehrte Richtung ist trivial.  Um $(\star)$ zu zeigen, reicht es
      wenn wir \\[0.1cm]
      \hspace*{1.3cm}
      $\mathtt{isSorted}([x_2] + R_2) \rightarrow \mathtt{le}(x_1, R_2)$ \\[0.1cm]
      nachweisen.  Nach Definition von \texttt{isSorted} gilt \\[0.1cm]
      \hspace*{1.3cm} $\mathtt{isSorted}([x_2] + R_2) \leftrightarrow \mathtt{le}(x_2,R_2) \wedge \mathtt{isSorted}(R_2)$.
      \\[0.1cm]
      Aus $\mathtt{isSorted}([x_2] + R_2)$ folgt also insbesondere \\[0.1cm]
      \hspace*{1.3cm} $\mathtt{le}(x_2,R_2)$. \\[0.1cm]
      Da $x_1 \preceq x_2$ ist, folgt mit dem Lemma \"uber die Transitivit\"at von \texttt{le} (Lemma \ref{lemma:le-trans}): \\[0.1cm]
      \hspace*{1.3cm} $\mathtt{le}(x_1,R_2)$ \\[0.1cm]
      und das war zu zeigen.
\item Fall:  $L_1 = [x_1] + R_1$,   $L_2 = [x_2] + R_2$, $x_1 > x_2$.
      
      Der Beweis verl\"auft analog zum 3.~Fall. \hspace*{\fill} $\Box$
\end{enumerate}

\begin{Satz}[Vertr\"aglichkeit von \texttt{isSorted} und \texttt{sort}] \hspace*{\fill} \\
{\em
  F\"ur beliebige Listen $L$ von Elementen aus $M$ gilt \\[0.1cm]
  \hspace*{1.3cm} $\mathtt{isSorted}\bigl(\mathtt{sort}(L)\bigr)$.
}
\end{Satz}

\noindent
\textbf{Beweis}: Wir f\"uhren den Beweis durch Wertverlaufs-Induktion entsprechend der Definition
der Funktion \texttt{sort}.
\begin{enumerate}
\item Fall: $\#L < 2$.  Es gilt \\[0.1cm]
      \hspace*{1.3cm} 
      $\mathtt{isSorted}\bigl(\mathtt{sort}(L)\bigr) \leftrightarrow \mathtt{isSorted}(L) \leftrightarrow \mathtt{true}$, \\[0.1cm]
      denn alle Listen mit weniger als zwei Elementen sind sortiert.
\item Fall: $\#L \geq 2$. Es sei $L = [x_1,x_2] + R$.  Zun\"achst definieren wir \\[0.1cm]
      \hspace*{1.3cm} $[R_1,R_2] = \mathtt{split}(R)$. \\[0.1cm]
      Dann gilt 
      \[
      \begin{array}[b]{cl}
                      & \mathtt{isSorted}\bigl(\mathtt{sort}(L)\bigr) \\
      \leftrightarrow & \mathtt{isSorted}\bigl(\mathtt{sort}([x_1,x_2] + R)\bigr) \\
      \leftrightarrow & \mathtt{isSorted}\Bigl(\mathtt{merge}\bigl(\mathtt{sort}([x_1] + R_1), \mathtt{sort}([x_2] + R_2)\bigr)\Bigr) \\
                      & \mbox{nach Definition von \texttt{sort}} \\
      \leftrightarrow & \mathtt{isSorted}\bigl(\mathtt{sort}([x_1] + R_1)\bigr) \wedge \mathtt{isSorted}\bigl(\mathtt{sort}([x_2] + R_2)\bigr) \\
                      & \mbox{wegen der Distributivit\"at von \texttt{isSorted} \"uber \texttt{merge}} \\
      \leftrightarrow & \mathtt{true} \wedge \mathtt{true} \\
                      & \mbox{nach Induktions-Voraussetzung} \\
      \leftrightarrow & \mathtt{true} 
      \end{array}\hspace*{\fill} \Box
      \]
\end{enumerate}

\begin{Satz}[Vertr\"aglichkeit von \texttt{count} und \texttt{sort}] \hspace*{\fill} \\
{\em
  F\"ur beliebige $x\el M$ und beliebige Listen $L$ von Elementen aus $M$ gilt \\[0.1cm]
  \hspace*{1.3cm} $\mathtt{count}\bigl(x,\mathtt{sort}(L)\bigr) = \mathtt{count}(x,L)$.
}
\end{Satz}

\noindent
\textbf{Aufgabe}:  Formulieren und beweisen Sie die Lemmata, die Sie zum Beweis des letzten Satzes ben\"otigen
und beweisen Sie dann den Satz.

\subsection{Komplexit\"at}
Wir wollen wieder berechnen, wieviele Vergleiche beim Sortieren einer Liste mit $n$
Elementen durchgef\"uhrt werden.  Dazu analysieren wir zun\"achst, wieviele Vergleiche zum
Mischen zweier Listen $L_1$ und $L_2$ ben\"otigt werden.  Wir definieren eine Funktion \\[0.1cm]
\hspace*{1.3cm} 
$\mathtt{cmpCount}: \textsl{List}(M) \times \textsl{List}(M) \rightarrow \mathbb{N}$ \\[0.1cm]
so dass f\"ur zwei Listen $L_1$ und $L_2$ der Term
$\mathtt{cmpCount}(L_1, L_2)$ die Anzahl Vergleiche angibt, die bei Berechnung von $\texttt{merge}(L_1,L_2)$ erforderlich sind. 
Wir behaupten, dass f\"ur beliebige Listen $L_1$ und $L_2$  \\[0.1cm]
\hspace*{1.3cm} $\mathtt{cmpCount}(L_1, L_2) \leq \# L_1 + \# L_2$ \\[0.1cm]
gilt.  F\"ur eine Liste $L$ bezeichnet dabei $\#L$ die Anzahl der Elemente der Liste.
 Wir f\"uhren den Beweis durch Induktion nach der Summe $\#L_1 + \#L_2$.
\begin{enumerate}
\item[I.A.:] $\#L_1 + \#L_2=0$.

             Dann m\"ussen $L_1$ und $L_2$ leer sein und somit ist beim Aufruf
             von $\mathtt{merge}(L_1, L_2)$ kein Vergleich erforderlich.  Also gilt \\[0.1cm]
             \hspace*{1.3cm} $\mathtt{cmpCount}(L_1, L_2) = 0 \leq 0 = \#L_1 + \#L_2$.
\item[I.S.:] $\#L_1 + \#L_2 = n+1$.

             Falls entweder $L_1$ oder $L_2$ leer ist, so ist kein Vergleich erforderlich
             und wir haben 
             \\[0.1cm]
             \hspace*{1.3cm}
             $\mathtt{cmpCount}(L_1,L_2) = 0 \leq \#L_1 + \#L_2$.
             \\[0.1cm]
             Wir nehmen also an, dass gilt: \\[0.1cm]
             \hspace*{1.3cm} $L_1 = [x] + R_1$ \quad und \quad $L_2 = [y] + R_2$.
             \\[0.1cm]
             Wir f\"uhren eine Fallunterscheidung bez\"uglich der relativen Gr\"o{\ss}e von $x$ und $y$ 
             durch.
             \begin{enumerate}
             \item $x \preceq y$.  Dann gilt \\[0.1cm]
                   \hspace*{1.3cm} 
                   $\mathtt{merge}\bigl([x] + R_1, [y] + R_2\bigr) = [x] +
                   \mathtt{merge}\bigl(R_1, [y] + R_2\bigr)$. \\[0.1cm]
                   Also haben wir: \\[0.1cm]
                   \hspace*{1.3cm} 
                   $\mathtt{cmpCount}(L_1, L_2) = 1 + \mathtt{cmpCount}(R_1, L_2) \stackrel{IV}{\leq} 1 + \#R_1 + \#L_2 = \#L_1 + \#L_2$.
             \item $\neg x \preceq y$.  Dieser Fall ist v\"ollig analog zum 1.~Fall. \hspace*{\fill} $\Box$
             \end{enumerate}
\end{enumerate}
Wir bezeichnen nun die Anzahl der Vergleiche, die beim Aufruf von
$\mathtt{sort}(L)$ f\"ur eine Liste $L$ der L\"ange $n$
schlimmstenfalls durchgef\"uhrt werden m\"ussen mit $a_n$.  Zur Vereinfachung nehmen wir an, dass $n$ die Form \\[0.1cm]
\hspace*{1.3cm} $\displaystyle n = 2^k$ \qquad f\"ur ein $k \el \mathbb{N}$ \\[0.1cm]
hat und definieren $b_k = a_n = a_{2^k}$.  Zun\"achst berechnen wir den Anfangs-Wert
$\displaystyle b_0 = a_{2^0} = a_1 = 0$.  Im rekursiven Fall wird zur Berechnung
von $\mathtt{sort}(L)$ die Liste $L$ zun\"achst durch $\texttt{split}$ in zwei
gleich gro{\ss}e Listen geteilt, die dann rekursiv sortiert werden. Anschlie{\ss}end
werden die sortierten Listen
gemischt.  Das liefert f\"ur das Sortieren einer Liste der L\"ange $2^{k+1}$ die Rekurrenz-Gleichung \\[0.1cm]
\hspace*{1.3cm} $b_{k+1} = 2 * b_k + 2^{k+1}$, \hspace*{\fill} (1) \\[0.1cm]
denn das Mischen der beiden halb so gro{\ss}en Listen kostet schlimmstenfalls $2^k +
2^k = 2^{k+1}$ Vergleiche und das rekursive Sortieren 
der beiden Teil-Listen
kostet insgesamt $2*b_k$ Vergleiche.

Um diese Rekurrenz-Gleichung zu l\"osen, f\"uhren wir in (1) die Substitution $k \mapsto k+1$ durch und erhalten \\[0.1cm]
\hspace*{1.3cm}  $b_{k+2} = 2 * b_{k+1} + 2^{k+2}$. \hspace*{\fill} (2) \\[0.1cm]
Wir multiplizieren Gleichung (1) mit dem Faktor 2 und subtrahieren die erhaltene Gleichung von Gleichung (2). 
Dann erhalten wir \\[0.1cm]
\hspace*{1.3cm}  $b_{k+2} - 2 * b_{k+1} = 2 * b_{k+1} - 4 * b_k$. \hspace*{\fill} (3) \\[0.1cm]
Diese Gleichung vereinfachen wir zu \\[0.1cm]
\hspace*{1.3cm}  $b_{k+2} = 4 * b_{k+1} - 4 * b_k$. \hspace*{\fill} (4) \\[0.1cm]
Diese Gleichung ist eine homogene lineare Rekurrenz-Gleichung 2. Ordnung.
Das charakteristische Polynom der zugeh\"origen homogenen Rekurrenz-Gleichung lautet \\[0.1cm]
\hspace*{1.3cm} $\displaystyle \chi(x) = x^2 - 4 *x + 4 = (x-2)^2$. \\[0.1cm]
Weil das charakteristische Polynom an der Stelle $x=2$ eine doppelte Null-Stelle hat, 
lautet die allgemeine L\"osung \\[0.1cm]
\hspace*{1.3cm} $\displaystyle b_k = \alpha * 2^k + \beta * k * 2^k$. \hspace*{\fill} (5) \\[0.1cm]
Setzen wir hier den Wert f\"ur $k=0$ ein, so erhalten wir \\[0.1cm]
\hspace*{1.3cm} $0 = \alpha$. \\[0.1cm]
Aus (1) erhalten wir den Wert $b_1 = 2 * b_0 + 2^1 = 2$.  Setzen wir also in Gleichung (5) f\"ur $k$ den Wert 1 ein, 
so finden wir \\[0.1cm]
\hspace*{1.3cm} $2 = 0 * 2^1 + \beta * 1 * 2^1$,  \\[0.1cm]
also muss $\beta = 1$ gelten.  Damit lautet die L\"osung \\[0.1cm]
\hspace*{1.3cm} $b_k = k * 2^k$. \\[0.1cm]
Aus $n = 2^k$ folgt $k = \log_2(n)$ und daher gilt f\"ur $a_n$ \\[0.1cm]
\hspace*{1.3cm} $a_n = n * \log_2(n)$. \\[0.1cm]
Wir sehen also, dass beim ``\emph{Sortieren durch Mischen}'' f\"ur gro{\ss}e Listen wesentlich weniger Vergleiche
durchgef\"uhrt werden m\"ussen, als dies bei den beiden anderen Algorithmen der Fall ist.

Zur Vereinfachung haben wir bei der obigen Rechnung nur eine obere Absch\"atzung
der Anzahl der Vergleiche durchgef\"uhrt.  Eine exakte Rechnung zeigt, dass im
schlimmsten Fall \\[0.1cm]
\hspace*{1.3cm}  $n * \log_2(n) - n + 1$ \\[0.1cm]
Vergleiche beim ``\emph{Sortieren durch Mischen}'' einer nicht-leeren Liste der L\"ange $n$
durchgef\"uhrt werden m\"ussen.

\subsection{Eine feldbasierte Implementierung}
Abbildung \ref{fig:MergeSortAlgorithm.java} auf Seite \pageref{fig:MergeSortAlgorithm.java}
zeigt eine feldbasierte  Implementierung des Algorithmus ``\emph{Sortieren durch
  Mischen}''.  Die Klasse \texttt{MergeSortAlgorithm} ist von der in der Übung
diskutierten Klasse \texttt{SortingAlgorithm} abgeleitet.
Wir diskutieren zun\"achst die Member-Variablen.
\begin{enumerate}
\item Das zu sortierende Feld wird in der Member-Variablen \texttt{mArray} 
      abgelegt.  Diese Member-Variable ist in der Oberklasse \texttt{SortingAlgorithm}
      definiert und wird von dieser Klasse geerbt.
\item \texttt{mAux} ist ein Hilfsfeld, das wir w\"ahrend der Durchf\"uhrung des Algorithmus
      ben\"otigen um Werte zwischenzuspeichern.
\item Der Komparator, den wir zum Vergleich zweier Elemente ben\"otigen,
      wird in der Variablen \texttt{mComparator} abgespeichert.  Diese Variable ist
      ebenfalls in der Oberklasse \texttt{SortingAlgorithm} definiert.
      
      Der Komparator stellt die Methode \\[0.1cm]
      \hspace*{1.3cm} 
      $\texttt{mComparator.compare}(\mathtt{Element}\ x, \mathtt{Element}\ y)$  \\[0.1cm]
      zur Verf\"ugung, mit deren Hilfe wir zwei Elements $x$ und $y$ vergleichen k\"onnen.
\end{enumerate}

\begin{figure}[!ht]
  \centering
\begin{Verbatim}[ frame         = lines, 
                  framesep      = 0.3cm, 
                  labelposition = bottomline,
                  numbers       = left,
                  numbersep     = -0.2cm,
                  xleftmargin   = 0.2cm,
                  xrightmargin  = 0.0cm
                ]
    import java.util.*;
    
    public class MergeSortAlgorithm<Element> extends SortingAlgorithm<Element>
    {
        private Element[] mAux;        
        
        MergeSortAlgorithm(Element[] array, Comparator<Element> comparator) {
            super(array, comparator);
            mAux = (Element[]) new Object[mArray.length];
        }
        public void sort() {
            mergeSort(0, mArray.length);        
        }
        private void mergeSort(int start, int end) {
            if (end - start < 2)
                return;
            int middle = (start + end) / 2;                         
            mergeSort( start,  middle );  
            mergeSort( middle, end    );    
            merge( start, middle, end ); 
        }
        private void merge(int start, int middle, int end) {    
            for (int i = start; i < end; ++i) {
                mAux[i] = mArray[i]; 
            }
            int idx1 = start;
            int idx2 = middle;
            int i    = start;
            while (idx1 < middle && idx2 < end) {
                if (mComparator.compare(mAux[idx1], mAux[idx2]) <= 0) {
                    mArray[i++]  = mAux[idx1++]; 
                } else {
                    mArray[i++]  = mAux[idx2++]; 
                }
            }
            while (idx1 < middle) {
                mArray[i++]  = mAux[idx1++];
            }
            while (idx2 < end) {
                mArray[i++]  = mAux[idx2++];  
            }
        }
    }
\end{Verbatim}
\vspace*{-0.3cm}
  \caption{Die Klasse \texttt{MergeSortAlgorithm}.}
  \label{fig:MergeSortAlgorithm.java}
\end{figure}
Der Konstruktor in Zeile 7 erwartet als Argumente das zu sortierende Feld und den
Komparator zum Vergleichen zweier Elemente.  Er speichert diese Argumente 
durch den Aufruf des Konstruktors der Oberklasse in den
Member-Variablen der Oberklasse ab und legt gleichzeitig das Hilfsfeld \texttt{mAux} an.

Der Algorithmus ``\emph{Sortieren durch Mischen}'' \texttt{sort}() wird in der Methode
\texttt{mergeSort} implementiert.  Diese Methode erh\"alt die Argumente  \texttt{start} und
\texttt{end}, die den  Bereich des Feldes eingrenzen, der zu sortieren ist: 
Der Aufruf \texttt{mergeSort}(\textsl{start}, \textsl{end}) sortiert nur der Bereich \\[0.1cm]
\hspace*{1.3cm} \texttt{mArray[\textsl{start}]}, $\cdots$, \texttt{mArray[\textsl{end}-1]}. \\[0.1cm]
Die feldbasierte Implementierung  weicht von der listenbasierten Implementierung ab, da
wir keine Funktion \texttt{split} mehr ben\"otigen, um die Liste aufzuspalten. 
Statt dessen berechnen wir die Mitte des Feldes
\texttt{mArray} mit der Formel \\[0.1cm]
\hspace*{1.3cm} \texttt{middle = (start + end) / 2;} \\[0.1cm]
und spalten das Feld dann an dem Index \texttt{middle} in zwei etwa gleich gro{\ss}e
Teile, die wir in den Zeilen 19 und 20 rekursiv sortieren.

Anschlie{\ss}end rufen wir in Zeile 21 die Methode \texttt{merge} aus, die die beiden
sortierten Felder zu einem sortierten Feld zusammenfasst.  Diese Methode ist in den Zeilen
22 --- 42 implementiert.  Die Methode erh\"alt 3 Argumente: Die Parameter \texttt{start},
\texttt{middle} und \texttt{end} spezifizieren die beiden Teilfelder, die zu mischen sind.
Das erste Teilfeld besteht aus den Elementen \\[0.1cm]
\hspace*{1.3cm} \texttt{mArray[start]}, $\cdots$, \texttt{mArray[middle-1]}, \\[0.1cm]
das zweite Teilfeld besteht aus den Elementen \\[0.1cm]
\hspace*{1.3cm} \texttt{mArray[middle]}, $\cdots$, \texttt{mArray[end-1]}. \\[0.1cm]
Der Aufruf setzt voraus, dass die beiden Teilfelder bereits sortiert sind.
Das Mischen funktioniert dann wie folgt.
\begin{enumerate}
\item Zun\"achst werden die Daten aus den beiden zu sortierenden Teilfelder
      in Zeile 24 in das Hilfs-Feld \texttt{mAux} kopiert.
\item Anschlie{\ss}end definieren wir drei Indizes:
      \begin{enumerate}
      \item \texttt{idx1} zeigt auf das n\"achste zu untersuchende Element im ersten
            Teilfeld des Arrays \texttt{mAux}.
      \item \texttt{idx2} zeigt auf das n\"achste zu untersuchende Element im zweiten
            Teilfeld des Arrays \texttt{mAux}.
      \item \texttt{i} gibt die Position im Ergebnis-Feld \texttt{mArray} an, in die das
            n\"achste Element geschrieben wird.
      \end{enumerate}
\item Solange weder das erste noch das zweite Teilfeld des Arrays \texttt{mAux}
      vollst\"andig abgearbeitet ist, vergleichen wir in Zeile 30 die Elemente aus den beiden Teilfeldern
      und schreiben das kleinere von beiden an die Stelle, auf die der Index \texttt{i} zeigt.
\item Falls bei diesem Vergleich eines der beiden Felder vor dem anderen ersch\"opft ist,
      m\"ussen wir anschlie{\ss}end die restlichen Elemente des verbleibenden Teilfeldes
      in das Ergebnis-Feld kopieren.  Die Schleife in Zeile 36 --- 38 wird aktiv, wenn das
      zweite Teilfeld zuerst ersch\"opft wird.  Dann werden dort die verbleibenden Elemente
      des ersten Teilfeldes in das Feld \texttt{mArray} kopiert.  Ist umgekehrt das erste
      Teilfeld zuerst ersch\"opft, dann werden in Zeile 39 --- 41 die verbleibenden Elemente
      des zweiten Teilfeldes in das Feld \texttt{mArray} kopiert. 
\end{enumerate}
\pagebreak

\subsubsection{Eine nicht-rekursive Implementierung von \emph{Sortieren durch Mischen}}

\begin{figure}[!ht]
  \centering
\begin{Verbatim}[ frame         = lines, 
                  framesep      = 0.3cm, 
                  labelposition = bottomline,
                  numbers       = left,
                  numbersep     = -0.2cm,
                  xleftmargin   = 0.0cm,
                  xrightmargin  = 0.0cm
                ]
    public void sort() {
        mergeSort();
    }
    private void mergeSort() {
        for (int l = 1; l < mArray.length; l *= 2) {
            int k;
            for (k = 0; l * (k + 1) <= mArray.length; k += 2) {
                merge(l * k, l * (k + 1), Math.min(l * (k + 2), mArray.length));
            }
        }
    }
\end{Verbatim}
\vspace*{-0.3cm}
  \caption{Eine sequentielle Implementierung des Merge-Sort-Algorithmus}
  \label{fig:MergeSortNRAlgorithm.java}
\end{figure}

\noindent
Die in Abbildung \ref{fig:MergeSortAlgorithm.java} gezeigte Implementierung des
Merge-Sort-Algorithmus ist rekursiv.  Die Effizienz einer rekursiven Implementierung ist
in der Regel schlechter als die Effizienz einer sequentiellen Implementierung.
Der Grund ist, dass der Aufruf einer rekursiven Funktion relativ viel Zeit kostet, denn
beim Aufruf einer rekursiven Funktion m\"ussen einerseits die lokalen Variablen der Funktion auf dem
Stack gesichert werden und andererseits m\"ussen die Argumente, mit denen die Funktion
aufgerufen wird, auf den Stack gelegt werden.  Wir zeigen daher, wie sich der
Merge-Sort-Algorithmus sequentiell implementieren l\"asst.  Abbildung
\ref{fig:MergeSortNRAlgorithm.java} auf Seite \pageref{fig:MergeSortNRAlgorithm.java}
zeigt eine solche Implementierung.
Statt der rekursiven Aufrufe haben wir hier zwei ineinander geschachtelte
\texttt{for}-Schleifen.  Die Arbeitsweise des Algorithmus wird deutlich, wenn wir die
\emph{Invarianten} dieser Schleifen formulieren.  Eine solche \emph{Invariante} ist eine
logische Formel, die vor jedem Durchlauf der Schleife wahr ist.  Dieser Begriff
wird am besten durch ein Beispiel klar.  Die Invariante der \"au{\ss}eren Schleife besagt,
dass alle Teil-Felder der L\"ange $l$, die bei einem Index beginnen, der ein Vielfaches von
$l$ ist, sortiert sind.  Bezeichnen wir das zu sortierende Feld \textsl{mArray} jetzt der
K\"urze halber mit $x$, so hat ein solches Teilfeld die Form 
\\[0.1cm]
\hspace*{1.3cm}
$\Bigl[x[k\cdot l],\; x[k\cdot l+1],\; x[k\cdot l+2],\; \cdots,\; x[k\cdot l+(l-1)]\Bigr]$
\\[0.1cm]
und wenn $n$ die L\"ange des Feldes $x$ ist, dann l\"asst sich die Aussage, dass alle
diese Teilfelder sortiert sind, durch die Formel 
\\[0.1cm]
\hspace*{0.8cm}
$\forall k \in \{0, \cdots, n/l\}: \forall j\in\{0,\cdots,l-2\}: k\cdot l+j+1 < n \rightarrow x[k\cdot l+j] \preceq x[k\cdot l+j+1]$
\\[0.1cm]
beschreiben. Die Bedingung $k\cdot l+j+1 < n$  ist notwendig um sicherzustellen, dass der
Array-Zugriff $x[k\cdot l+j+1]$ definiert ist.  Wenn diese Bedingung am Anfang eines Schleifen-Durchlaufs
erf\"ullt sein soll, dann ist es die Aufgabe des Schleifen-Rumpfs diese Bedingung f\"ur den
n\"achsten Wert, den die Schleifen-Variable $l$ annimmt, sicherzustellen.  Der erste Wert
der Schleifen-Variable $l$ ist 1.   F\"ur diesen Wert ist die Schleifen-Invariante trivial
denn dann sagt die Invariante nur aus, dass Teilfelder der L\"ange 1 sortiert sind.  In der
Schleife wird der Wert von $l$ nach jedem Schleifen-Durchlauf verdoppelt.   
Es werden jeweils zwei Teilfelder der L\"ange $l$ genommen und so gemischt, dass das
resultierende Teilfeld, dass die L\"ange $2\cdot l$ hat, danach sortiert ist.  Die innere Schleife in
den Zeilen 7 bis 9 mischt f\"ur gerade Zahlen $k$ das $k$-te Teilfeld mit dem $k+1$-ten
Teilfeld, es werden also die Teilfelder
\\[0.1cm]
\hspace*{1.3cm}
$x[k\cdot l],\;x[k\cdot l+1],\;\cdots,\;x[k\cdot l+(l-1)]$ \quad und
\\[0.1cm]
\hspace*{1.3cm}
 $x[(k+1)\cdot l],\;x[(k+1)\cdot l+1],\;\cdots,\;x[(k+1)\cdot l+(l-1)]$ 
\\[0.1cm]
gemischt.  M\"oglicherweise hat das letzte Teilfeld eine L\"ange, die kleiner als $l$ ist.
Daher nehmen wir f\"ur das dritte Argument der Methode \textsl{merge} das Minimum der beiden
Zahlen $l\cdot(k+2)$ und $\textsl{mArray}.\textsl{length}$.

Der Algorithmus l\"asst sich noch dadurch verbessern, dass die Ergebnisse abwechselnd in dem 
Feld \texttt{mArray} und \texttt{mAux} zur\"uck gegeben werden, denn dann entf\"allt in der
Methode \textsl{merge} das Kopieren
des Feldes \texttt{mArray} in das Hilfs-Feld \texttt{mAux}.  
Es ist sogar m\"oglich, ganz auf das Teilfeld \texttt{mAux} zu verzichten.  
Die resultierende Implementierung ist allerdings wesentlich komplexer.  Daher gehen wir
auf diesen Punkt nicht weiter ein sondern verweisen die interessierten Leser auf den Artikel von
Katajainen et.~al.~\cite{katajainen:96}.

\section{Der \emph{Quick-Sort}-Algorithmus}
Der ``\emph{Quick-Sort}-Algorithmus'' funktioniert nach folgendem Schema:
\begin{enumerate}
\item Ist die zu sortierende Liste $L$ leer, so wird $L$
      zur\"uck gegeben: \\[0.1cm]
      \hspace*{1.3cm} $\mathtt{sort}([]) = []$.
\item Sonst hat $L$ die Form $L = [x] + R$.
      Dann verteilen wir die Elemente von $R$ so auf zwei Listen $S$ und $B$, 
      dass $S$ alle Elemente von $R$ enth\"alt, die kleiner-gleich $x$ sind, w\"ahrend $B$ die
      restlichen Elemente von $R$ enth\"alt. 
      Wir implementieren die Berechnung der beiden Listen \"uber eine Funktion
      $\mathtt{partition}$, die das Paar von Listen $S$ und $B$ erzeugt: 
      \\[0.2cm]
      \hspace*{1.3cm}
      $\textsl{partition}(x,R) = [S,B]$.
      \\[0.2cm]
      Hierbei gilt dann 
      \begin{enumerate}
      \item Die Listen $S$ und $B$ enthalten zusammen genau die Elemente der Liste $R$
            \[ \forall y \in R: \textsl{count}(y,S) + \textsl{count}(y,B) = \textsl{count}(y,R) \]
      \item Alle Elemente aus $S$ sind kleiner-gleich $x$, die Elemente aus $B$ sind
            gr\"o{\ss}er als $x$:
            \[ \forall y \in S: y \preceq x \quad \mbox{und} \quad \forall y \in B: x \prec y. \]
      \end{enumerate}
      Formal k\"onnen wir die Funktion $\textsl{partition}()$ durch die folgenden
      Gleichungen beschreiben:
      \begin{enumerate}
      \item $\textsl{partition}(x, []) = \bigl[[], []\bigr]$,
      \item \quad $\; x \preceq y\; \wedge \textsl{partition}(x,R) = \bigl[ S, B \bigr] \rightarrow 
             \textsl{partition}(x, [y] + R) = \bigl[ [x] + S,\, B \bigr]$
      \item $\neg\, (x \preceq y) \wedge \textsl{partition}(x,R) = \bigl[ S, B \bigr] \rightarrow 
             \textsl{partition}(x, [y] + R) = \bigl[ S,\, [x] + B \bigr]$,
      \end{enumerate}
      Anschlie{\ss}end sortieren wir die Listen $S$ und $B$ rekursiv.
      An die sortierte Liste $S$ h\"angen wir dann das Element $x$ und darauf folgt die
      sortierte Liste $B$.  Insgesamt wird dieser Algorithmus durch die folgende Gleichung
      beschrieben:
      \\[0.2cm]
      \hspace*{1.3cm}
      $\textsl{partition}(x,R) = [S,B] \rightarrow 
       \textsl{sort}([x] + R) = \textsl{sort}(S) + [x] + \textsl{sort}(B)$.
\end{enumerate}
Abbildung \ref{fig:quick-sort} zeigt die Umsetzung dieser Überlegung 
in einem \textsl{Java}-Programm.

\begin{figure}[!ht]
  \centering
\begin{Verbatim}[ frame         = lines, 
                  framesep      = 0.3cm, 
                  labelposition = bottomline,
                  numbers       = left,
                  numbersep     = -0.2cm,
                  xleftmargin   = 0.0cm,
                  xrightmargin  = 0.0cm
                ]
    public class QuickSort
    {  
        private static void partition(Double pivot, 
                                      LinkedList<Double> list,
                                      LinkedList<Double> small, 
                                      LinkedList<Double> big) 
        {
            if (list.isEmpty()) { return; }
            Double x = list.removeFirst();
            if (x <= pivot) { small.addFirst(x); } 
            else            { big  .addFirst(x); }
            partition(pivot, list, small, big);
        }
        public static LinkedList<Double> sort(LinkedList<Double> list) {
            if (list.isEmpty()) {
                return list;
            }
            Double             pivot = list.removeFirst();
            LinkedList<Double> small = new LinkedList<Double>();
            LinkedList<Double> big   = new LinkedList<Double>();
            partition(pivot, list, small, big);
            LinkedList<Double> smallSorted = sort(small);
            LinkedList<Double> bigSorted   = sort(big);
            smallSorted.add(pivot);
            smallSorted.addAll(bigSorted);
            return smallSorted;
        }    
    }
\end{Verbatim}
\vspace*{-0.3cm}
  \caption{Der \emph{Quick-Sort}-Algorithmus.}
  \label{fig:quick-sort}
\end{figure}

\begin{enumerate}
\item Der \emph{Quick-Sort}-Algorithmus ben\"otigt keine globalen Variablen,
      daher enth\"alt die Klasse \texttt{QuickSort} keine Member-Variablen.
      Folglich k\"onnen die Methoden dieser Klasse als \texttt{static}
      deklariert werden.
\item Die Methode \texttt{partition} hat die Aufgabe, eine Liste \texttt{list} anhand eines
      gegebenen \emph{Pivot-Elements} \texttt{pivot} so in zwei Listen aufzuteilen, dass nach einem
      Aufruf der Form 
      \[ \texttt{partition(pivot, list, small, big)} \]
      die Liste \texttt{small} alle die Elemente aus 
      \texttt{list} enth\"alt, die kleiner-gleich \texttt{pivot} sind, w\"ahrend \texttt{big}
      die Elemente enth\"alt, die gr\"o{\ss}er als \texttt{pivot} sind.
      Voraussetzung ist dabei, dass die Listen \texttt{small} und \texttt{big} vor dem
      Aufruf leer sind.  

      Falls die Listen \texttt{small} und \texttt{big} bei dem Aufruf nicht leer sind,
      so ist die Idee, dass die Elemente aus \texttt{list} entsprechend ihrer Gr\"o{\ss}e
      den beiden Listen \texttt{small} und \texttt{big} hinzugef\"ugt werden.
      \begin{enumerate}
      \item Falls \texttt{list} leer ist, so gibt es keine Elemente, die verteilt werden
            m\"ussen.
      \item Andernfalls wird das erste Element von \texttt{list} aus \texttt{list}
            entfernt und, je nach dem wie gro{\ss} es im Vergleich zu dem Pivot-Element
            \texttt{pivot} ist, in die Liste \texttt{small} oder \texttt{big} eingef\"ugt.
      \item Schlie{\ss}lich werden die restlichen Elemente von \texttt{list} durch einen
            rekursiven Aufruf der Funktion \texttt{partition} auf die Listen \texttt{small}
            und \texttt{big} verteilt.
      \end{enumerate}
\item Die Methode \texttt{sort} sortiert die als Argument \"ubergebene Liste. 
      \begin{enumerate}
      \item Falls diese Liste leer ist, kann sie unver\"andert zur\"uck gegeben werden.
      \item Andernfalls wird das erste Element der Liste \texttt{list} aus dieser Liste
            entfernt und in der Variablen \texttt{pivot}
            abgespeichert.
      \item Dann werden zwei leere Listen \texttt{small} und \texttt{big} angelegt.
      \item Der anschlie{\ss}ende Aufruf der Methode \texttt{partition} verteilt die restlichen
            Elemente der \"ubergebenen Liste auf diese beiden Listen.
      \item Danach werden diese Listen durch rekursive Aufrufe der Methode
            \texttt{sort} sortiert.
      \item Zum Schluss wird an die sortierte Liste der Elemente, die kleiner als das
            Pivot-Element sind, erst das Pivot-Element und dann die sortierte Liste der Elemente,
            die gr\"o{\ss}er als das Pivot-Element sind, angeh\"angt.
      \item Die so erhaltene Liste wird als Ergebnis zur\"uck gegeben.
      \end{enumerate}
\end{enumerate}

\subsection{Komplexit\"at}
Als n\"achstes analysieren wir die Komplexit\"at von \emph{Quick-Sort}.
Dazu untersuchen wir wieder die Zahl der Vergleiche, die beim Aufruf von
$\textsl{sort}(L)$ f\"ur eine Liste $L$ mit $n$ 
Elementen durchgef\"uhrt werden. Dazu betrachten wir zun\"achst die Anzahl der Vergleiche,
die wir bei einem Aufruf der Form
\\[0.2cm]
\hspace*{1.3cm}
$\textsl{partition}(x,L,S,B)$ \\[0.2cm]
f\"ur eine Liste $L$ mit $n$ Elementen durchf\"uhren m\"ussen. 
Da wir jedes der $n$ Elemente der Liste $L$ mit $x$ vergleichen m\"ussen, ist klar, dass 
daf\"ur insgesamt $n$ Vergleiche erforderlich sind.

\subsubsection{Komplexit\"at im schlechtesten Fall}
Wir berechnen als n\"achstes die Anzahl $a_n$ von Vergleichen, die wir im schlimmsten Fall
beim Aufruf von $\textsl{sort}(L)$ f\"ur eine Liste $L$ der L\"ange $n$ durchf\"uhren m\"ussen.
Der schlimmste Fall tritt beispielsweise dann ein, wenn die Liste \texttt{small} leer ist und 
die Liste \texttt{big} folglich  die L\"ange $n-1$ hat.   F\"ur die Anzahl
$a_n$ der Vergleiche gilt in diesem Fall \\[0.1cm]
\hspace*{1.3cm} $a_n = a_{n-1} + n - 1$. \\[0.1cm]
Der Term $n-1$ r\"uhrt von den $n-1$ Vergleichen, die beim Aufruf von
$\textsl{partition}(x,R)$ in Zeile 6 von Abbildung \ref{fig:quick-sort} durchgef\"uhrt
werden und der Term $a_{n-1}$ erfasst die Anzahl der Vergleiche, die beim rekursiven Aufruf
von $\textsl{sort}(L_2)$ ben\"otigt werden.

Die Anfangs-Bedingung f\"ur die Rekurrenz-Gleichung lautet $a_0 = 0$, denn beim Sortieren einer leeren Liste sind keine
Vergleiche notwendig.  Damit l\"asst sich die obige Rekurrenz-Gleichung mit dem \emph{Teleskop-Verfahren} l\"osen:
\[
\begin{array}{lcl}
  a_n & = & a_{n-1} + (n-1) \\
      & = & a_{n-2} + (n-2) + (n-1) \\
      & = & a_{n-3} + (n-3) + (n-2) + (n-1) \\
      & = & \vdots \\
      & = & a_{0} + 0 + 1 + 2 + \cdots  + (n-2) + (n-1) \\
      & = & 0 + 0 + 1 + 2 + \cdots  + (n-2) + (n-1) \\[0.1cm]
      & = & \sum\limits_{i=0}^{n-1} i  =  \frac{1}{2} n \cdot(n - 1) =\frac{1}{2} n^2 - \frac{1}{2} n.
\end{array}
\]
Damit ist $a_n$ in diesem Fall genauso gro{\ss} wie im schlimmsten Fall des Algorithmus' \emph{Sortieren durch Einf\"ugen}.
Es ist leicht zu sehen, dass der schlechteste Fall dann eintritt, wenn die zu sortierende Liste $L$
bereits sortiert ist.  Es existieren Verbesserungen des \emph{Quick-Sort}-Algorithmus, f\"ur
die der schlechteste Fall nicht bei sortierten Listen eintritt.  Wir gehen sp\"ater n\"aher darauf ein.

\subsubsection{Durchschnittliche Komplexit\"at}
Der Algorithmus \emph{Quick-Sort} w\"urde seinen Namen zu Unrecht tragen, wenn er im
\emph{Durchschnitt} ein Komplexit\"at der Form $\Oh(n^2)$ h\"atte.  Wir analysieren nun die
\emph{durchschnittliche} Anzahl von Vergleichen $d_{n}$, die wir beim Sortieren einer
Liste $L$ mit $n$ Elementen erwarten m\"ussen.  Im Allgemeinen gilt: Ist $L$ eine Liste
mit $n+1$ Elementen, so ist die Zahl der Elemente der Liste \texttt{small}, die in Zeile 19
von Abbildung \ref{fig:quick-sort} berechnet wird, ein Element der Menge
$\{0,1,2,\cdots,n\}$.  Hat die Liste \texttt{small} insgesamt $i$ Elemente, so enth\"alt die
Liste \texttt{big} $n-i$ Elemente.  Gilt $\#\mathtt{small} = i$, so werden
zum rekursiven Sortieren von $\mathtt{small}$ und $\mathtt{big}$ durchschnittlich
\\[0.1cm]
\hspace*{1.3cm} $d_i + d_{n-i}$ \\[0.1cm]
Vergleiche durchgef\"uhrt.  Bilden wir den Durchschnitt dieses Wertes f\"ur alle $i \el
\{0,1,\cdots,n\}$, so erhalten wir
f\"ur $d_{n+1}$ die Rekurrenz-Gleichung 
\begin{equation}
  \label{eq:qs1}
  d_{n+1} = n + \frac{1}{n+1} \sum_{i=0}^n (d_i + d_{n-i})  
\end{equation}
Der Term $n$ stellt die Vergleiche in Rechnung, die beim Aufruf von 
\\[0.2cm]
\hspace*{1.3cm}
\texttt{partition(pivot, list, small, big)}
\\[0.2cm]
durchgef\"uhrt werden.  Um die Rekurrenz-Gleichung (1) zu vereinfachen, bemerken wir
zun\"achst, dass f\"ur beliebige Funktionen $f:\mathbb{N} \rightarrow \mathbb{N}$ folgendes gilt:
\begin{eqnarray}
\sum_{i=0}^n f(n-i) & = & f(n) + f(n-1) + \cdots + f(1) + f(0) \\
                    & = & f(0) + f(1) + \cdots + f(n-1) + f(n) \\
 \label{eq:qssum}
                    & = & \sum_{i=0}^n f(i) 
\end{eqnarray}
Damit vereinfacht sich die Rekurrenz-Gleichung (\ref{eq:qs1}) zu 
\begin{equation}
  \label{eq:qs2}
  d_{n+1} = n + \frac{2}{n+1} \cdot \sum_{i=0}^n d_i.   
\end{equation}
Um diese Rekurrenz-Gleichung l\"osen zu k\"onnen, substituieren wir $n \mapsto n+1$ und erhalten
\begin{equation}
  \label{eq:qs3}
   d_{n+2} = n+1 + \frac{2}{n+2} \cdot \sum_{i=0}^{n+1} d_i.  
\end{equation}
Wir multiplizieren nun Gleichung (\ref{eq:qs3}) mit $n+2$ und Gleichung (\ref{eq:qs2}) mit $n+1$ und
erhalten 
\begin{eqnarray}
  \label{eq:qs4}
 (n+2)\cdot d_{n+2} & = & (n+2)\cdot(n+1) + 2 \cdot \sum_{i=0}^{n+1} d_i \qquad \mbox{und} \\
  \label{eq:qs5}
 (n+1)\cdot d_{n+1} & = & (n+1)\cdot n + 2 \cdot \sum_{i=0}^n d_i.  
\end{eqnarray}
Wir bilden die Differenz der Gleichungen (\ref{eq:qs4}) und (\ref{eq:qs5}) und beachten,
dass sich die Summationen bis auf den Term $2\cdot d_{n+1}$ gerade gegenseitig aufheben.
Dann erhalten wir
\begin{equation}
  \label{eq:qs6}
 (n+2)\cdot d_{n+2} - (n+1)\cdot \displaystyle d_{n+1} = (n+2)\cdot(n+1) - (n+1)\cdot n+2 \cdot d_{n+1}
\end{equation}
Diese Gleichung vereinfachen wir zu
\begin{equation}
  \label{eq:qs7}
(n+2)\cdot d_{n+2} = (n+3)\cdot \displaystyle d_{n+1} + 2\cdot(n+1).  
\end{equation}
Einer genialen Eingebung folgend teilen wir diese Gleichung durch $(n+2) \cdot(n+3)$ und
erhalten 
\begin{equation}
  \label{eq:qs8}
 \frac{1}{n+3} \cdot d_{n+2} = \frac{1}{n+2}\cdot d_{n+1} + \frac{2\cdot(n+1)}{(n+2)\cdot(n+3)}.
\end{equation}
Als n\"achstes bilden wir die Partialbruch-Zerlegung von dem Bruch
\[ \frac{2\cdot(n+1)}{(n+2)\cdot(n+3)}. \] 
Dazu machen wir den Ansatz
\[ \frac{2\cdot(n+1)}{(n+2)\cdot(n+3)} = \frac{\alpha}{n+2} + \frac{\beta}{n+3}.\]
Wir multiplizieren diese Gleichung mit dem Hauptnenner und erhalten
\[ 2\cdot n + 2 = \alpha \cdot (n+3) + \beta \cdot (n+2), \]
was sich zu 
\[ 2\cdot n + 2 = (\alpha + \beta) \cdot n + 3 \cdot \alpha  + 2 \cdot \beta \]
vereinfacht.  Ein Koeffizientenvergleich liefert dann das lineare Gleichungs-System:
\begin{eqnarray*}
  2 & = & \alpha + \beta \\
  2 & = & 3 \cdot \alpha + 2 \cdot \beta 
\end{eqnarray*}
Ziehen wir die erste Gleichung zweimal von der zweiten Gleichung ab, so erhalten wir
 $\alpha = -2$ und Einsetzen in die erste Gleichung liefert $\beta = 4$.
Damit k\"onnen wir die Gleichung (\ref{eq:qs8}) als 
\begin{equation}
  \label{eq:qs9}
 \frac{1}{n+3} \cdot d_{n+2} = \frac{1}{n+2}\cdot d_{n+1} - \frac{2}{n+2} + \frac{4}{n+3}  
\end{equation}
schreiben.  Wir definieren $\displaystyle a_n = \frac{d_n}{n+1}$ und erhalten dann aus der
letzten Gleichung 
\[ a_{n+2} = a_{n+1} - \frac{2}{n+2} + \frac{4}{n+3} \]
Die Substitution $n \mapsto n-2$ vereinfacht diese Gleichung zu 
\begin{equation}
  \label{eq:qs10}
 a_{n} = a_{n-1} - \frac{2}{n} + \frac{4}{n+1}  
\end{equation}
Diese Gleichung k\"onnen wir mit dem Teleskop-Verfahren l\"osen.  Es gilt 
\begin{equation}
  \label{eq:qs11}
 a_{n} = 4 \cdot \sum_{i=1}^n \frac{1}{i+1} - 2 \cdot \sum_{i=1}^n \frac{1}{i}.  
\end{equation}
Wir vereinfachen diese Summe:
\[
\begin{array}{lcl}
 a_{n} & = & \displaystyle 4 \cdot \sum_{i=1}^n \frac{1}{i+1} - 2 \cdot \sum_{i=1}^n \frac{1}{i} \\[0.5cm]
       & = & \displaystyle 4 \cdot \sum_{i=2}^{n+1} \frac{1}{i} - 2 \cdot \sum_{i=1}^n \frac{1}{i} \\[0.5cm]
       & = & \displaystyle 4 \cdot \frac{1}{n+1} - 4 \cdot \frac{1}{1} + 4 \cdot \sum_{i=1}^{n} \frac{1}{i} - 2 \cdot \sum_{i=1}^n \frac{1}{i} \\[0.5cm]
       & = & \displaystyle 4 \cdot \frac{1}{n+1} - 4 \cdot \frac{1}{1} + 2 \cdot \sum_{i=1}^{n} \frac{1}{i}  \\[0.5cm]
       & = & \displaystyle - \frac{4 \cdot n}{n+1}  + 2 \cdot \sum_{i=1}^{n} \frac{1}{i}  
\end{array}
\]
Um unsere Rechnung abzuschlie{\ss}en, berechnen wir eine N\"aherung f\"ur die Summe 
\[ H_n = \sum\limits_{i=1}^{n}\frac{1}{i}.\]
Der Wert $H_n$ wird in der Mathematik als die $n$-te \emph{harmonische Zahl} bezeichnet.
Dieser Wert h\"angt mit dem Wert $\ln(n)$ zusammen, Leonhard Euler hat gezeigt, dass f\"ur
gro{\ss}e $n$ die Approximation
\[ \sum\limits_{i=1}^n \frac{1}{i} \approx \ln(n)  \]
benutzt werden kann.  Genauer hat er folgendes gezeigt:
\[ H_n = \ln(n) + \Oh(1). \]
Wir haben bei dieser Gleichung eine Schreibweise benutzt, die wir bisher noch nicht
eingef\"uhrt haben.  Sind $f$, $g$, $h$ Funktionen aus $\mathbb{R}^\mathbb{N}$, so schreiben
wir 
\[ f(n) = g(n) + \Oh\bigl(h(n)\bigr) \quad \mathrm{g.d.w.} \quad
   f(n) - g(n) \el \Oh\bigl(h(n)\bigr). 
\]
Wegen $d_n = (n+1) \cdot a_{n}$ gilt jetzt: 
\begin{eqnarray*}  
 d_n & = & -4 \cdot n + 2 \cdot(n+1) \cdot H_n \\
     & = & -4 \cdot n + 2 \cdot(n+1) \cdot \bigl(\ln(n) + \Oh(1)\bigr) \\
     & = & 2 \cdot n \cdot \ln(n) + \Oh(n).
\end{eqnarray*}
Wir vergleichen dieses Ergebnis mit dem Ergebnis, das wir bei der Analyse von
``\emph{Sortieren durch Mischen}'' erhalten haben.  Dort hatte sich die Anzahl
der Vergleiche, die zum Sortieren eine Liste mit $n$ Elementen durchgef\"uhrt
werden musste,
als \\[0.1cm]
\hspace*{1.3cm} $n \cdot \log_2(n) + \Oh(n)$ \\[0.1cm]
ergeben.  Wegen $\ln(n) = \ln(2) \cdot \log_2(n)$ ben\"otigen wir bei Quick-Sort im Durchschnitt \\[0.1cm]
\hspace*{1.3cm} $2 \cdot \ln(2) \cdot n \cdot \log_2(n)$ \\[0.1cm]
Vergleiche, also etwa $2 \cdot \ln(2) \approx 1.39$ mehr Vergleiche als beim
``\emph{Sortieren durch Mischen}''.

\subsection{Eine feldbasierte Implementierung von \emph{Quick-Sort}}
Zum Abschluss geben wir eine feldbasierte Implementierung des \emph{Quick-Sort}-Algorithmus
an.  Abbildung \ref{fig:QuickSort.java} zeigt diese Implementierung.

\begin{figure}[!ht]
  \centering
\begin{Verbatim}[ frame         = lines, 
                  framesep      = 0.3cm, 
                  labelposition = bottomline,
                  numbers       = left,
                  numbersep     = -0.2cm,
                  xleftmargin   = 0.0cm,
                  xrightmargin  = 0.0cm
                ]
    import java.util.*;
    
    public class QuickSortAlgorithm<Element> extends SortingAlgorithm<Element>
    {    
        QuickSortAlgorithm(Element[] array, Comparator<Element> comparator) {
            super(array, comparator);
        }
        public void sort() {
            quickSort(0, mArray.length - 1); 
        }
        private void quickSort(int start, int end) {
            if (end <= start) { return; }
            int splitIdx = partition(start, end);
            quickSort(start, splitIdx - 1);  
            quickSort(splitIdx + 1, end );    
        }
        private int partition(int start, int end) {
            Element x     = mArray[start];
            int     left  = start + 1;
            int     right = end;
            while (true) {
                while (left <= end && mComparator.compare(mArray[left],x) <= 0) {
                    ++left;
                }
                while (mComparator.compare(mArray[right], x) > 0) {
                    --right;
                }
                if (left > right) { break; }
                swap(left, right);
            }
            swap(start, right);
            return right;    
        }
        private void swap(int i, int j) {
            Element e = mArray[i];
            mArray[i] = mArray[j];
            mArray[j] = e;
        }
    }
\end{Verbatim}
\vspace*{-0.3cm}
  \caption{Implementierung des \emph{Quick-Sort}-Algorithmus in \textsl{Java}.}
  \label{fig:QuickSort.java}
\end{figure}

\begin{enumerate}
\item Im Gegensatz zu der feldbasierten Implementierung des \emph{Merge-Sort-Algorithmus}
      ben\"otigen wir diesmal kein zus\"atzliches Hilfsfeld.  Die Klasse
      \texttt{QuickSortAlgorithm} hat daher nur die beiden Member-Variablen
      \texttt{mArray} und \texttt{mComparator}, die sie von der Oberklasse
      \texttt{SortingAlgorithm} erbt.  Diese Member-Variablen werden durch den 
      Konstruktor initialisiert.
\item Die Methode \texttt{sort} wird auf die Implementierung der Methode
      \texttt{quickSort} zur\"uck gef\"uhrt.  Diese Methode bekommt die beiden Parameter
      \texttt{start} und \texttt{end}.
      \begin{enumerate}
      \item \texttt{start} gibt den Index des ersten Elementes des Teilfeldes an,
            das zu sortieren ist.
      \item \texttt{end} gibt den Index des letzten Elementes des Teilfeldes an,
            das zu sortieren ist.  
      \end{enumerate}
      Der Aufruf \texttt{quickSort(start, end)} sortiert die Elemente \\[0.1cm]
      \hspace*{1.3cm} 
      \texttt{mArray[start]}, \texttt{mArray[start+1]}, $\cdots$, \texttt{mArray[end]}
      \\[0.1cm]
      des Feldes \texttt{mArray}, d.~h.~nach dem Aufruf gilt:\\[0.1cm]
      \hspace*{1.3cm}
      $\texttt{mArray[start]}\preceq\texttt{mArray[start+1]}\preceq\cdots\preceq\texttt{mArray[end]}$.
      \\[0.1cm]
      Die Implementierung der Methode \texttt{quickSort}
      entspricht weitgehend der listenbasierten Implementierung.  Der
      wesentliche Unterschied besteht darin, dass die Funktion \texttt{partition},
      die in Zeile 13 aufgerufen wird,  die
      Elemente des Feldes \texttt{array} so umverteilt, dass hinterher alle Elemente,
      die kleiner oder gleich dem \emph{Pivot-Element} sind, links vor dem
      Index \texttt{splitIdx} stehen, w\"ahrend die restlichen Elemente rechts von dem Index stehen.
      Das Pivot-Element selbst steht hinterher an der durch \texttt{splitIdx}
      bezeichneten Stelle.
\item Die Schwierigkeit bei der Implementierung von \emph{Quick-Sort} liegt
      in der Codierung der Methode \texttt{partition}, die in Zeile 17 beginnt.
      Die Funktion \texttt{partition} erh\"alt zwei Argumente:
      \begin{enumerate}
      \item \texttt{start} ist der Index des ersten  Elementes in dem aufzuspaltenden Teilbereich.
      \item \texttt{end}   ist der Index des letzten Elementes in dem aufzuspaltenden Teilbereich.
      \end{enumerate}
      Die Funktion \texttt{partition} 
      liefert als Resultat einen Index $\mathtt{splitIdx}$ aus der Menge \\[0.1cm]
      \hspace*{1.3cm} 
      $\mathtt{splitIdx} \el \{ \mathtt{start},\; \mathtt{start}+1,\; \cdots,\; \mathtt{end} \}$.\\[0.1cm]
      Au{\ss}erdem wird der Teil des Feldes zwischen \texttt{start} und \texttt{end} so
      umsortiert, dass nach dem Aufruf der Methode gilt:
      \begin{enumerate}
      \item Alle Elemente mit Index aus der Menge $\{\mathtt{start}, \cdots,
        \mathtt{splitIdx}-1\}$ kommen in der Ordnung
            ``$\preceq$'' vor dem Element an der Stelle \texttt{splitIdx}: \\[0.1cm]
            \hspace*{1.3cm} 
            $\forall i \el \{\mathtt{start}, \cdots, \mathtt{splitIdx}-1\} \colon\, \mathtt{mArray}[i] \preceq \mathtt{mArray}[\mathtt{splitIdx}]$.
      \item Alle Elemente mit Index aus der Menge $\{\mathtt{splitIdx}+1, \cdots, \mathtt{end}\}$ kommen
            in der Ordnung ``$\preceq$'' hinter dem Element an der Stelle \texttt{splitIdx}: \\[0.1cm]
            \hspace*{1.3cm} 
            $\forall i \el \{\mathtt{splitIdx}+1, \cdots, \mathtt{end}\} \colon\, \mathtt{mArray}[\mathtt{splitIdx}]\prec \mathtt{mArray}[i]$.
      \end{enumerate}
      Der Algorithmus, um diese Bedingungen zu erreichen,  w\"ahlt zun\"achst das Element
      \texttt{mArray[start]} als sogenanntes
      \emph{Pivot-Element} aus.  Anschlie{\ss}end l\"auft der Index \texttt{leftIdx} ausgehend von
      dem Index $\texttt{start} + 1$ von links nach rechts bis ein Element gefunden wird, das gr\"o{\ss}er als 
      das Pivot-Element ist. Analog l\"auft der Index $\mathtt{rightIdx}$ ausgehend von dem
      Index \texttt{end} von rechts nach links, bis ein Element gefunden wird, dass kleiner
      oder gleich dem Pivot-Element ist. Falls nun \texttt{leftIdx} kleiner als
      \texttt{rightIdx} ist, werden die entsprechenden Elemente des Feldes ausgetauscht.  In
      dem Moment, wo \texttt{leftIdx} gr\"o{\ss}er oder gleich \texttt{rightIdx} wird, wird dieser
      Prozess abgebrochen. Jetzt wird noch das Pivot-Element in die Mitte gestellt,
      anschlie{\ss}end wird \texttt{rightIdx} zur\"uck gegeben.
\item Der Aufruf $\texttt{swap}(i, j)$ vertauscht die Elemente des Arrays \texttt{mArray},
      die die Indizes $i$ und $j$ haben.  
\end{enumerate}
Der einfachste Weg um zu verstehen, wie die Methode \texttt{partition} funktioniert,
besteht darin, diese Methode anhand eines Beispiels auszuprobieren.  Wir betrachten dazu 
einen Ausschnitt aus einem Feld, der die Form 
\[ \cdots,\; 7,\; 2,\; 9,\; 1,\; 8,\; 5,\; 11,\;\cdots \]
hat.  Wir nehmen an, dass der Index \texttt{start} die Position der Zahl 7 angibt und das
der Index \texttt{end} auf die 11 zeigt.  
\begin{enumerate}
\item Dann zeigt der Index \texttt{left} zun\"achst auf
      die Zahl 2 und der Index \texttt{right} zeigt auf die die Zahl 11.
\item Die erste \texttt{while}-Schleife vergleicht zun\"achst die Zahl 2 mit der Zahl 7.
      Da $2 \leq 7$ ist, wird der Index \texttt{left} inkrementiert, so dass er jetzt auf 
      die Zahl 9 zeigt.
\item Anschlie{\ss}end vergleicht die  erste \texttt{while}-Schleife die Zahlen 9 und 7.
      Da $\neg(9 \leq 7)$ ist, wird die erste \texttt{while}-Schleife abgebrochen.
\item Nun startet die zweite \texttt{while}-Schleife und vergleicht die Zahlen 7 und 11.
      Da $7 < 11$ ist, wird der Index \texttt{right} dekrementiert und zeigt nun auf die
      Zahl 5.
\item Da $\neg (7 < 5)$ ist, wird auch die zweite \texttt{while}-Schleife abgebrochen.
\item Anschlie{\ss}end wir gepr\"uft, ob der Index \texttt{right} bereits \"uber den Index
      \texttt{left} hin\"uber gelaufen ist und somit $\mathtt{left} \geq \mathtt{right}$
      gilt.   Dies ist aber nicht der Fall, denn \texttt{left} zeigt auf die 9,
      w\"ahrend \texttt{right} auf die 5 zeigt, die rechts von der 9 liegt.
      Daher wird die \"au{\ss}ere \texttt{while}-Schleife noch nicht abgebrochen.
\item Jetzt werden die Elemente, auf die die Indizes \texttt{left} und \texttt{right}
      zeigen, vertauscht.  In diesem Fall werden also die Zahlen 9 und 5 vertauscht.
      Damit hat der Ausschnitt aus dem Feld die Form
      \[ \cdots,\; 7,\; 2,\; 5,\; 1,\; 8,\; 9,\; 11,\;\cdots \]
\item Jetzt geht es in die zweite Runde der \"au{\ss}eren \texttt{while}-Schleife.
      Zun\"achst vergleichen wir in der inneren \texttt{while}-Schleife die Elemente
      5 und 7.  Da $5 \leq 7$ ist, wird der Index \texttt{left} inkrementiert.
\item Dann vergleichen wir die Zahlen 1 und 7.  Da $1 \leq 7$ ist, wird der Index
      \texttt{left} ein weiteres Mal inkrementiert und zeigt nun auf die 8.
\item Der Vergleich $8 \leq 7$ f\"allt negativ aus, daher wird die erste
      \texttt{while}-Schleife jetzt abgebrochen.
\item Die zweite \texttt{while}-Schleife vergleicht nun 7 und 9.  Da $7 < 9$ ist,
      wird der Index \texttt{right} dekrementiert und zeigt jetzt auf die 8.
\item Anschlie{\ss}end werden die Zahlen 7 und 8 verglichen.  Da auch $7 < 8$ gilt, wird der
      Index \texttt{right} ein weiteres Mal dekrementiert und zeigt nun auf die 1.
\item Jetzt werden die Zahlen 7 und 1 verglichen.  Wegen $\neg (7 < 1)$ bricht nun die
      zweite \texttt{while}-Schleife ab.
\item Nun wird gepr\"uft, ob der Index \texttt{right} \"uber den Index
      \texttt{left} hin\"uber gelaufen ist und somit $\mathtt{left} \geq \mathtt{right}$
      gilt.  Diesmal ist der Test positiv, denn \texttt{left} zeigt auf die 8,
      w\"ahrend \texttt{right} auf die 1 zeigt, die links von der 8 steht.
      Also wird die \"au{\ss}ere Schleife durch den \texttt{break}-Befehl in Zeile 28
      abgebrochen.
\item Zum Abschluss wird das Pivot-Element, das durch den Index \texttt{start}
      identifiziert wird, mit dem Element vertauscht, auf das der Index \texttt{right}
      zeigt,  wir vertauschen also die Elemente 7 und 1.  Damit hat das Feld die Form
      \[ \cdots,\; 1,\; 2,\; 5,\; 7,\; 8,\; 9,\; 11,\;\cdots \]
      Als Ergebnis wird nun der Index \texttt{right}, der jetzt auf das Pivot-Element
      zeigt, zur\"uck gegeben.
\end{enumerate}
 
\subsection{Korrektheit}
Die Implementierung der Methode \texttt{partition} ist trickreich.  Daher untersuchen wir
die Korrektheit der Methode jetzt im Detail.  Zun\"achst formulieren wir Invarianten, die
f\"ur die \"au{\ss}ere \texttt{while}-Schleife, die sich von Zeile 21 bis Zeile 30 erstreckt,
gelten.  Insgesamt gibt es f\"unf Invarianten:
\begin{enumerate}
\item[(I1)] $\forall i \in \{ \mathtt{start}+1, \cdots, \mathtt{left} - 1 \} \colon\; \mathtt{mArray}[i] \preceq x$
\item[(I2)] $\forall j \in \{ \mathtt{right}+1, \cdots, \mathtt{end} \} \colon\; x \prec \mathtt{mArray}[j]$
\item[(I3)] $\mathtt{start}+1 \leq \mathtt{left}$ 
\item[(I4)] $\mathtt{right} \leq \mathtt{end}$ 
\item[(I5)] $\mathtt{left} \leq \mathtt{right} + 1$
\end{enumerate}
Wir weisen die G\"ultigkeit dieser Invarianten nach.  Dazu ist zun\"achst zu zeigen,
dass diese Invarianten dann erf\"ullt sind, wenn die Schleife zum ersten Mal durchlaufen wird.
Zu Beginn gilt \\[0.1cm]
\hspace*{1.3cm} $\texttt{left} = \mathtt{start} + 1$. \\[0.1cm]
Daraus folgt sofort, dass die dritte Invariante anfangs erf\"ullt ist.  Au{\ss}erdem gilt dann \\[0.1cm]
\hspace*{1.3cm} 
$\{ \mathtt{start}+1, \cdots, \mathtt{left}-1 \} = \{ \mathtt{start}+1, \cdots, \mathtt{start} \} = \{\}$
\\[0.1cm]
und damit ist auch klar, dass die erste Invariante gilt, denn f\"ur $\mathtt{left} = \mathtt{start}+1$
ist die erste Invariante eine leere Aussage.  Weiter gilt zu Beginn \\[0.1cm]
\hspace*{1.3cm} $\mathtt{right} = \mathtt{end}$,
\\[0.1cm]
woraus unmittelbar die G\"ultigkeit der vierten Invariante folgt.  Au{\ss}erdem gilt dann \\[0.1cm]
\hspace*{1.3cm} 
$\{ \mathtt{right}+1, \cdots, \mathtt{end} \} = \{ \mathtt{end}+1, \cdots, \mathtt{end} \} = \{\}$,
\\[0.1cm]
so dass auch die zweite Invariante trivialerweise erf\"ullt ist.  F\"ur die f\"unfte Invariante gilt anfangs
 \\[0.1cm]
\hspace*{1.3cm} 
$\mathtt{left} \leq \mathtt{right} + 1 \;\leftrightarrow\; \mathtt{start} + 1 \leq \mathtt{end} + 1 \;\leftrightarrow\;
  \mathtt{start} \leq \mathtt{end} \;\leftrightarrow\; \mathtt{true}$,
\\[0.1cm]
denn die Methode $\texttt{partition}(\texttt{start}, \texttt{end})$ wird nur aufgerufen,
falls $\mathtt{start} < \mathtt{end}$ ist.
\vspace*{0.3cm}

\noindent
Als n\"achstes zeigen wir, dass die Invarianten bei einem Schleifen-Durchlauf erhalten
bleiben.  
\begin{enumerate}
\item Die erste Invariante gilt, weil \texttt{left} nur dann inkrementiert wird,
      wenn vorher \\[0.1cm]
      \hspace*{1.3cm} $\texttt{mArray}[\texttt{left}] \preceq x$ \\[0.1cm]
      gilt.  Wenn die Menge $\{\mathtt{start}+1, \cdots, \mathtt{left}-1\}$ also um 
      $i = \mathtt{left}$ vergr\"o{\ss}ert wird, so ist sichergestellt, dass f\"ur dieses $i$ gilt:
      \\[0.1cm]
      \hspace*{1.3cm} $\texttt{mArray}[i] \preceq x$.
\item Die zweite Invariante gilt, weil \texttt{right} nur dann dekrementiert wird,
      wenn vorher \\[0.1cm]
      \hspace*{1.3cm} $x \prec \texttt{mArray}[\texttt{right}]$ \\[0.1cm]
      gilt.  Wenn die Menge $\{\mathtt{right}+1, \cdots, \mathtt{end}\}$ also um 
      $i = \mathtt{right}$ vergr\"o{\ss}ert wird, so ist sichergestellt, dass f\"ur dieses $i$ gilt
      \\[0.1cm]
      \hspace*{1.3cm} $x \prec \texttt{mArray}[i]$.
\item Die G\"ultigkeit der dritten Invariante folgt aus der Tatsache, dass \texttt{left}
      in der ganzen Schleife h\"ochstens inkrementiert wird.  Wenn also zu Beginn 
      $\mathtt{start} + 1 \leq \mathtt{left}$ gilt, so wird dies immer gelten.  
\item Analog ist die vierten Invariante g\"ultig, weil zu Beginn $\mathtt{right} \leq \mathtt{end}$ gilt und
      \texttt{right} immer nur dekrementiert wird.
\item Aus den ersten beiden Invarianten (I1) und (I2) folgt: \\[0.1cm]
      \hspace*{1.3cm} 
      $\{ \mathtt{start}+1, \cdots, \mathtt{left} - 1 \} \cap \{ \mathtt{right}+1, \cdots, \mathtt{end} \} = \{\}$,
      \\[0.1cm]
      denn ein Element des Arrays kann nicht gleichzeigt kleiner-gleich $x$ und 
      gr\"o{\ss}er $x$ sein.
      Wenn $\texttt{right} + 1 \leq \texttt{end}$ ist, dann ist die zweite Menge
      nicht-leer und es folgt \\[0.1cm]
      \hspace*{1.3cm} $\mathtt{left} - 1 < \mathtt{right} + 1$ \quad und das impliziert \quad $\mathtt{left} \leq \mathtt{right} + 1$. 
      \\[0.1cm]
      Andernfalls gilt $\texttt{right}=\mathtt{end}$.  Dann haben wir 
      \\[0.1cm]
      \hspace*{1.3cm}
      $\mathtt{left} \leq \mathtt{right} + 1 \;\leftrightarrow\;
       \mathtt{left} \leq \mathtt{end} + 1 \;\leftrightarrow\; \mathtt{true}$,
      \\[0.1cm]
      denn wenn $\texttt{left} > \mathtt{end}$ ist, wird \texttt{left} in der ersten
      Schleife nicht mehr erh\"oht.  \texttt{left} wird nur dann und auch nur um 1
      inkrementiert, solange $\texttt{left} \leq \mathtt{end}$ gilt.
      Also kann \texttt{left} maximal den Wert $\texttt{end} + 1$ annehmen.
\end{enumerate}
Um den Beweis der Korrektheit abzuschlie{\ss}en, muss noch gezeigt werden, dass 
alle \texttt{while}-Schleifen terminieren.  F\"ur die erste innere
\texttt{while}-Schleife folgt das daraus, dass bei jedem Schleifen-Durchlauf die Variable
\texttt{left} inkrementiert wird.  Da die Schleife andererseits der Bedingung \\[0.1cm]
\hspace*{1.3cm} $\texttt{left} \leq \mathtt{end}$ \\[0.1cm]
enth\"alt, kann \texttt{left} nicht beliebig oft inkrementiert werden und die Schleife muss
irgendwann abbrechen. 

Die zweite innere
\texttt{while}-Schleife terminiert, weil einerseits \texttt{right} in jedem Schleifen-Durchlauf
dekrementiert wird und andererseits aus der dritten und der f\"unften Invariante folgt: \\[0.1cm]
\hspace*{1.3cm} $\texttt{right} + 1\geq \texttt{left} \geq \mathtt{start} + 1$. \\[0.1cm]
Die \"au{\ss}ere \texttt{while}-Schleife terminiert, weil die Menge \\[0.1cm]
\hspace*{1.3cm} $M = \{ \mathtt{left}, \cdots, \mathtt{right} \}$ \\[0.1cm]
st\"andig verkleinert wird.  Um das zu sehen, f\"uhren wir eine Fall-Unterscheidung durch:
\begin{enumerate}
\item Fall: Nach dem Ende der Schleife in Zeile 22 -- 24 gilt \\[0.1cm]
      \hspace*{1.3cm} $\mathtt{left} > \mathtt{end}$. \\[0.1cm]
      Diese Schleife bricht also ab, weil die Bedingung $\mathtt{left} \leq \mathtt{end}$
      verletzt ist.  Wir haben oben schon gesehen, dass dann \\[0.1cm]
      \hspace*{1.3cm} 
      $\texttt{left} = \mathtt{end} + 1$ \quad und \quad $\mathtt{right} = \mathtt{end}$
      \\[0.1cm]
      gelten muss.  Daraus folgt aber sofort \\[0.1cm]
      \hspace*{1.3cm} $\mathtt{left} > \mathtt{right}$ \\[0.1cm]
      und folglich wird die Schleife durch den Befehl \texttt{break} in Zeile 28 abgebrochen.
\item Fall: Nach dem Ende der Schleife in Zeile 6 --- 8 gilt \\[0.1cm]
      \hspace*{1.3cm} $\mathtt{left} \leq \mathtt{end}$. \\[0.1cm]
      Die Schleife bricht also ab, weil die Bedingung 
      $\mathtt{mArray}[\mathtt{left}] \preceq x$ verletzt ist, es gilt also \\[0.1cm]
      \hspace*{1.3cm} $x \prec \mathtt{mArray}[\mathtt{left}]$. \\[0.1cm]
      Analog gilt nach dem Abbruch der zweiten inneren \texttt{while}-Schleife \\[0.1cm]
      \hspace*{1.3cm} $\mathtt{mArray}[\mathtt{right}] \preceq x$. \\[0.1cm]
      Wenn die \"au{\ss}ere Schleife nun nicht abbricht weil $\texttt{left} < \mathtt{right}$ ist,
      dann werden die Elemente $\mathtt{mArray}[\mathtt{left}]$ und
      $\mathtt{mArray}[\mathtt{right}]$ vertauscht.  Nach dieser Vertauschung gilt 
      offenbar \\[0.1cm]
      \hspace*{1.3cm} 
      $x \prec \mathtt{mArray}[\mathtt{right}]$ \quad und \quad $\mathtt{mArray}[\mathtt{left}] \preceq x$. 
      \\[0.1cm]
      Wenn nun also die \"au{\ss}ere Schleife erneut durchlaufen wird, dann wird die zweite
      innere Schleife mindestens einmal durchlaufen, so dass also \texttt{right}
      dekrementiert wird und folglich die Menge $M = \{ \mathtt{left}, \cdots, \mathtt{right} \}$ 
      um ein Element verkleinert wird.  Das geht aber nur endlich oft, denn sp\"atestens
      wenn die Menge leer ist, gilt $\mathtt{left} = \mathtt{right} + 1$
      und die Schleife wird durch den Befehl \texttt{break} in Zeile 28 abgebrochen. 
\end{enumerate}
Jetzt haben wir alles Material zusammen, um die Korrektheit unserer Implementierung zu zeigen.
Da die Schleife abbricht, gilt $\mathtt{left} > \mathtt{right}$.  Wegen der f\"unften
Invariante gilt $\texttt{left} \leq \mathtt{right} + 1$.  Also gibt es nur noch die
M\"oglichkeit
\\[0.2cm]
\hspace*{1.3cm}
$\texttt{left} = \mathtt{right} +
1$.
\\[0.2cm]  
Wegen den ersten beiden Invarianten
wissen wir also \\[0.1cm]
\hspace*{1.3cm} 
 $\forall i \in \{ \mathtt{start}+1, \cdots, \mathtt{right} \} \colon\; \mathtt{mArray}[i] \preceq x$ \\[0.1cm]
\hspace*{1.3cm} 
 $\forall j \in \{ \mathtt{right}+1, \cdots, \mathtt{end} \} \colon\; x \prec \mathtt{mArray}[j]$
\\[0.1cm]
Durch das \texttt{swap} in Zeile 17 wird nun $x$ mit dem Element an der Position
\texttt{right} vertauscht.  Dann sind anschlie{\ss}end alle Elemente links von $x$
kleiner-gleich $x$ und alle Elemente rechts von $x$ sind gr\"o{\ss}er.
Damit ist die Korrektheit von \texttt{partition}() nachgewiesen.

\subsection{M\"ogliche Verbesserungen}
In der Praxis gibt es noch eine Reihe Tricks, um die Implementierung von \emph{Quick-Sort}
effizienter zu machen:
\begin{enumerate}
\item Anstatt immer das erste Element als Pivot-Element zu w\"ahlen,
      werden drei Elemente aus der zu sortierenden Liste ausgew\"ahlt,
      z.~B.~das erste, das letzte und ein Element aus der Mitte des Feldes.
      Als Pivot-Element wird dann das Element gew\"ahlt, was der Gr\"o{\ss}e nach zwischen den
      anderen Elementen liegt.

      Der Vorteil dieser Strategie liegt darin, dass der schlechteste Fall, bei dem die Laufzeit
      von \emph{Quick-Sort} quadratisch ist, wesentlich unwahrscheinlicher wird.
      Insbesondere kann der schlechteste Fall nicht mehr bei Listen auftreten, die bereits
      sortiert sind.
\item Falls weniger als 10 Elemente zu sortieren sind, wird auf ``\emph{Sortieren durch Einf\"ugen}''
      zur\"uck gegriffen.
\end{enumerate}
Der Artikel von Bentley and M. Douglas McIlroy ``\emph{Engineering a Sort Function}''
\cite{bentley:93} beschreibt diese und weitere Verbesserungen des Quick-Sort Algorithmus. 
%\textbf{Aufgabe}: Implementieren Sie diese Verbesserungen.

\section{Bewertung der Algorithmen}
F\"ur die Praxis empfehle ich den Einsatz von ``\emph{Sortieren durch Mischen}'', denn dieser Algorithmus hat auch im schlechtesten Fall 
eine Komplexit\"at der Form $\Oh\bigl(n\cdot \ln(n)\bigr)$, der Algorithmus ist folglich robust.
Es ist zwar in der Praxis so, dass \emph{Quick-Sort} f\"ur viele Anwendungen fast doppelt   
so schnell ist wie ``\emph{Sortieren durch Mischen}'', trotzdem gibt es zwei Gr\"unde, die gegen den Einsatz von
\emph{Quick-Sort} sprechen:
\begin{enumerate}
\item Die Implementierung von \emph{Quick-Sort} in einer Sprache wie \textsl{Java} ist erheblich komplizierter als 
      die Implementierung von ``\emph{Sortieren durch Mischen}''.
      Bei \emph{Quick-Sort} ist es insbesondere leicht, subtile Fehler zu machen, die durch Tests nur schwer 
      zu entdecken sind.  Es gibt eine Reihe Lehrb\"ucher die diesen Punkt dadurch belegen, dass sie
      \underline{fehlerhafte} Versionen von \emph{Quick-Sort} angeben.
\item Im schlechtesten Fall ist die Komplexit\"at von \emph{Quick-Sort} eben doch quadratisch und dann dauert das
      Sortieren wesentlich l\"anger als beim ``\emph{Sortieren durch Mischen}''.
\end{enumerate}
Es gibt allerdings eine Situation, in der die Verwendung von \emph{Quick-Sort} dem ``\emph{Sortieren durch Mischen}'' 
vorzuziehen ist.  Eine Implementierung von \emph{Quick-Sort} in \textsl{Java} kommt mit vergleichsweise wenig Speicherplatz aus,
denn au{\ss}er dem zu sortierenden Feld wird eigentlich nur noch etwas Platz auf dem Stack ben\"otigt.  Hingegen
ist bei der Implementierung von ``\emph{Sortieren durch Mischen}'' ein Hilfs-Feld erforderlich, das dieselbe Gr\"o{\ss}e hat
wie das zu sortierende Feld.  Es wird also in etwa doppelt soviel Speicherplatz ben\"otigt wie bei 
\emph{Quick-Sort}.

\paragraph{Historisches}  
Der Quick-Sort-Algorithmus wurde von Charles Antony Richard Hoare \cite{hoare:61}
entwickelt, der Merge-Sort-Algorithmus geht auf John von Neumann zur\"uck.

%%% Local Variables: 
%%% mode: latex
%%% TeX-master: "algorithmen"
%%% End: 