%%%%%%%%%%%%%%%%%%%%%%%%%%%%%%%%%%%%%%%%%%%%%%%%%%%%%%%%%%%%%%%%%%%%%%%%

\begin{slide}{}
\normalsize

\begin{center}
Darstellung des Schach--Bretts in \texttt{C}
\end{center}
\vspace*{0.5cm}

\footnotesize
Vor\"uberlegung:
\begin{itemize}
\item  alle Damen m\"ussen in verschiedenen Reihen stehen
\item  es m\"ussen 8 Damen gesetzt werden 
\item  also muss in jeder Reihe eine Dame stehen.
\item  Reihen werden sukkzessive besetzt
\item  Z\"ahlung f\"angt bei 0 an
\end{itemize}

Darstellung des Bretts als \texttt{C} \texttt{struct} mit 2 Komponenten
\begin{enumerate}
\item Anzahl der bereits gesetzten Damen \\[0.3cm]
      \hspace*{1.3cm} \texttt{numberQueens}
\item Spezifikation der Spalten, in denen Damen gesetzt sind durch
      Feld \\[0.3cm]
      \hspace*{1.3cm} \texttt{column[8]} \\[0.3cm]
      Dame in $i$-ter Zeile steht in Spalte $\mathtt{column}[i]$
\end{enumerate}

\begin{verbatim}
    #define BOARD_SIZE 8

    typedef struct {
        unsigned numberQueens;  
        unsigned column[BOARD_SIZE];
    } Board;
\end{verbatim}

\vspace*{\fill}
\tiny \addtocounter{mypage}{1}
\rule{17cm}{1mm}
Backtracking  \hspace*{\fill} Seite \arabic{mypage}
\end{slide}

%%%%%%%%%%%%%%%%%%%%%%%%%%%%%%%%%%%%%%%%%%%%%%%%%%%%%%%%%%%%%%%%%%%%%%%%

\begin{slide}{}
\normalsize

\begin{center}
Graphische Darstellung des Schach--Bretts
\end{center}
\vspace*{0.5cm}

\footnotesize

\begin{verbatim}
             Board = {
                       4,
                       { 7, 1, 3, 6 }
                     }
\end{verbatim}
\begin{itemize}
\item 4 Damen sind gesetzt
\item Dame in der 0--ten Reihe steht in Spalte 7
\item Dame in der 1--ten Reihe steht in Spalte 1
\item Dame in der 2--ten Reihe steht in Spalte 3
\item Dame in der 3--ten Reihe steht in Spalte 6
\end{itemize}
Z\"ahlung beginnt bei 0, wegen \texttt{C} Indizierung der Felder
\vspace*{0.5cm}

\hspace*{4cm}
\vbox{\offinterlineskip
   \hrule height1pt
   \hbox{\vrule width1pt\chess
         \vbox{\hbox{0Z0Z0Z0L}
               \hbox{ZQZ0Z0Z0}
               \hbox{0Z0L0Z0Z}
               \hbox{Z0Z0Z0L0}
               \hbox{0Z0Z0Z0Z}
               \hbox{Z0Z0Z0Z0}
               \hbox{0Z0Z0Z0Z}
               \hbox{Z0Z0Z0Z0}}%
         \vrule width1pt}
   \hrule height1pt}





\vspace*{\fill}
\tiny \addtocounter{mypage}{1}
\rule{17cm}{1mm}
Backtracking  \hspace*{\fill} Seite \arabic{mypage}
\end{slide}

%%%%%%%%%%%%%%%%%%%%%%%%%%%%%%%%%%%%%%%%%%%%%%%%%%%%%%%%%%%%%%%%%%%%%%%%

\begin{slide}{}
\normalsize

\footnotesize
\setlength{\unitlength}{1.5cm}

\begin{picture}(10,10)

\thicklines
\put(1,1){\line(1,0){8}}
\put(1,1){\line(0,1){8}}
\put(1,9){\line(1,0){8}}
\put(9,1){\line(0,1){8}}

\thinlines
\multiput(1,2)(0,1){7}{\line(1,0){8}}
\multiput(2,1)(1,0){7}{\line(0,1){8}}
\put(2.15,7.15){{\chess Q}}
\put(1.15,2.15){{\chess Q}}
\put(8.15,1.15){{\chess Q}}
\multiput(3.3,6.6)(0.15,-0.15){3}{$.$}
\multiput(2.35,3.35)(0.15,+0.15){3}{$.$}
\multiput(3.35,4.35)(0.15,+0.15){3}{$.$}
\put(1.05,8.35){{\tiny $\langle 0, 0 \rangle$}}
\put(4.02,5.35){{\scriptsize $\langle x, y \rangle$}}
\put(5.02,4.55){{\tiny $\langle x\!+\!1, $}}
\put(5.12,4.15){{\tiny $y\!+\!1\rangle$}}
\multiput(6.3,3.6)(0.15,-0.15){3}{$.$}
\put(7.05,2.55){{\tiny $\langle x\!+\!d,$}}
\put(7.15,2.15){{\tiny $y\!+\!d\rangle$}}

\put(5.02,6.55){{\tiny $\langle x\!+\!1,$}}
\put(5.15,6.15){{\tiny $y\!-\!1 \rangle$}}
\put(6.35,7.35){$.^{\mbox{$.^{\mbox{$.$}}$}}$}
\put(7.02,8.55){{\tiny $\langle x\!+\!d,$}}
\put(7.15,8.15){{\tiny $ y\!-\!d \rangle$}}
\end{picture}
\vspace*{-1.0cm}

\begin{enumerate}
\item  Steigende  Diagonale: \\[0.3cm]
      \hspace*{0.3cm} 
      $x + y = (x + 1) + (y - 1) = \cdots = (x + d) + (y - d)$ \\[0.3cm]
      Damen in Zeile $i$ und Zeile $j$ in selber steigender Diag. \\[0.3cm]
      \hspace*{1.3cm}      $i + \mathtt{column}[i] = j + \mathtt{column}[j]$
\item Fallende Diagonale: \\[0.3cm]
      \hspace*{0.3cm} 
      $x - y = (x + 1) - (y + 1) = \cdots = (x + d) - (y + d)$ \\[0.3cm]
      Damen in Zeile $i$ und Zeile $j$ in selber fallender Diag. \\[0.3cm]
      \hspace*{1.3cm}  $i - \mathtt{column}[i] = j - \mathtt{column}[j]$
\end{enumerate}


\vspace*{\fill}
\tiny \addtocounter{mypage}{1}
\rule{17cm}{1mm}
Backtracking  \hspace*{\fill} Seite \arabic{mypage}
\end{slide}

%%%%%%%%%%%%%%%%%%%%%%%%%%%%%%%%%%%%%%%%%%%%%%%%%%%%%%%%%%%%%%%%%%%%%%%%

\begin{slide}{}
\normalsize

\begin{center}
Test, ob Dame gesetzt werden kann
\end{center}
\vspace*{0.5cm}

\footnotesize
\begin{tabular}{ll}
\textbf{Gegeben}: & \begin{tabular}[t]{ll}
                   Brett & \texttt{board} \\
                   Zeile & $x=\texttt{board->numberQueens}$ \\
                   Spalte & $y=\texttt{nextColumn}$ \\[0.3cm]
                    \end{tabular} \\
\textbf{Frage}:   & Kann Dame  in $\langle x, y \rangle$ gesetzt werden? \\[0.3cm]
\textbf{Vorgehen:} & 1. Überpr\"ufe Spalte \texttt{nextColumn} \\[0.3cm]
                  & 2. Überpr\"ufe fallende Diagonale \\[0.3cm]
                  & 3. Überpr\"ufe steigende Diagonale \\[0.3cm]
\end{tabular}

\begin{verbatim}
bool check(Board* board, unsigned nextColumn) 
{
    unsigned row = board->numberQueens;
    for (unsigned i = 0; i < board->numberQueens; ++i)
    {
        if (board->column[i] == nextColumn) 
        {
            return false;
        }
        if (i - board->column[i] == row - nextColumn) 
        {
            return false;
        }
        if (i + board->column[i] == row + nextColumn)  
        {
            return false;
        }
    }
    return true;
}
\end{verbatim}


\vspace*{\fill}
\tiny \addtocounter{mypage}{1}
\rule{17cm}{1mm}
Backtracking  \hspace*{\fill} Seite \arabic{mypage}
\end{slide}

%%%%%%%%%%%%%%%%%%%%%%%%%%%%%%%%%%%%%%%%%%%%%%%%%%%%%%%%%%%%%%%%%%%%%%%%

\begin{slide}{}
\normalsize

\begin{center}
Hilfs--Funktionen
\end{center}
\vspace*{0.5cm}

\footnotesize

Erzeugung eines leeren Bretts
\begin{enumerate}
\item Speicherplatz allozieren
\item \texttt{numberQueens} mit 0 initialisieren
\end{enumerate}
\begin{verbatim}
Board* createEmpty()
{
    Board* newBoard = malloc( sizeof(Board) );
    newBoard->numberQueens = 0;
    return newBoard;
}
\end{verbatim}

Hinzuf\"ugen einer Dame (erzeugt neues Brett)
\begin{enumerate}
\item Speicherplatz f\"ur neues Brett allozieren
\item \texttt{numberQueens} initialisieren
\item vorhandene Damen kopieren
\item neue Dame hinzuf\"ugen
\end{enumerate}

\begin{verbatim}
Board* addQueen(Board* board, unsigned nextColumn)
{
    Board* newBoard = malloc( sizeof(Board) );
    newBoard->numberQueens = board->numberQueens + 1;
    for (unsigned i = 0; i < board->numberQueens; ++i) {
        newBoard->column[i] = board->column[i];
    }
    newBoard->column[board->numberQueens] = nextColumn;
    return newBoard;
}
\end{verbatim}

\vspace*{\fill}
\tiny \addtocounter{mypage}{1}
\rule{17cm}{1mm}
Backtracking  \hspace*{\fill} Seite \arabic{mypage}
\end{slide}

%%%%%%%%%%%%%%%%%%%%%%%%%%%%%%%%%%%%%%%%%%%%%%%%%%%%%%%%%%%%%%%%%%%%%%%%

\begin{slide}{}
\normalsize

\begin{center}
Backtracking
\end{center}
\vspace*{0.5cm}

\footnotesize
Algorithmus
\begin{enumerate}
\item Wenn alle Damen auf dem Brett sind: fertig!
\item Finde n\"achste Spalte, in die Dame gesetzt werden kann
      und setze Dame: neues Brett.
\item Rekursiv: Versuche, neues Brett zu vervollst\"andigen.
\item Funktioniert: fertig.
\item Sonst: versuche n\"achste Spalte
\end{enumerate}

\begin{verbatim}
Board* complete(Board* board)
{
    if (board->numberQueens == BOARD_SIZE) {
        return board;
    }
    for (unsigned i = 0; i < BOARD_SIZE; ++i) {
        if (check(board, i)) {
            Board* nextBoard = addQueen(board, i);
            printBoard(nextBoard);
            Board* result = complete(nextBoard);
            if (result != 0) {
                return result;
            } else {
                // !!! backtrack !!!
                printBoard(board);
                free(nextBoard);
                continue;
            }
        }
    }
    return 0;
}
\end{verbatim}

\vspace*{\fill}
\tiny \addtocounter{mypage}{1}
\rule{17cm}{1mm}
Backtracking  \hspace*{\fill} Seite \arabic{mypage}
\end{slide}

%%%%%%%%%%%%%%%%%%%%%%%%%%%%%%%%%%%%%%%%%%%%%%%%%%%%%%%%%%%%%%%%%%%%%%%%

\begin{slide}{}
\normalsize

\begin{center}
Backtracking: Anwendung
\end{center}
\vspace*{0.5cm}

\footnotesize
Grammatik f\"ur arithmetische Ausdr\"ucke: \\[0.3cm]
\hspace*{1.3cm} $G_{\mathtt{calc}} = \langle T, N, R, S \rangle$ 
\begin{enumerate}
\item $T = \{ \mathbf{number} \}$
\item $N = \{ \textsl{Sum}, \textsl{Prod}, \textsl{Factor} \}$
\item Die Menge der Regeln ist wie folgt gegeben: \\[0.3cm]
      \hspace*{1.3cm} $
      \begin{array}{lcl}
         \textsl{Sum}  & \rightarrow & \textsl{Sum}\; ``\mathtt{+}\cq \; \textsl{Prod}    \\
                       & |           & \textsl{Sum}\; ``\mathtt{-}\cq \; \textsl{Prod}    \\
                       & |           & \textsl{Prod}                                      \\[0.3cm]
         \textsl{Prod} & \rightarrow & \textsl{Prod}\; ``\mathtt{*}\cq \; \textsl{Factor} \\
                       & |           & \textsl{Prod}\; ``\mathtt{/}\cq \; \textsl{Factor} \\
                       & |           & \textsl{Factor}                                    \\[0.3cm]
       \textsl{Factor} & \rightarrow & ``\mathtt{(}\cq \;\textsl{Sum}\; ``\mathtt{)}\cq   \\
                       & |           & \mathbf{number}                                    \\
      \end{array}
      $
\item $S = \textsl{Sum}$ 
\end{enumerate}
Vorteile dieser Grammatik
\begin{enumerate}
\item Grammatik ist \emph{eindeutig}: Jeder Ausdruck aus $\Ll(G_{\mathtt{calc}})$
      hat genau einen Parse--Baum.
\item Grammatik ber\"ucksichtigt 
      \begin{enumerate}
      \item \emph{Punkt vor Strich}: \\[0.3cm]
             \hspace*{1.3cm} $x+y*z = x + (y*z)$
      \item Links--Assoziativit\"at von ``\texttt{-}'' und ``\texttt{/}'' \\[0.3cm]
            \hspace*{1.3cm} $x - y - z = (x - y) - z$
      \end{enumerate}
\end{enumerate}



\vspace*{\fill}
\tiny \addtocounter{mypage}{1}
\rule{17cm}{1mm}
Backtracking  \hspace*{\fill} Seite \arabic{mypage}
\end{slide}

%%%%%%%%%%%%%%%%%%%%%%%%%%%%%%%%%%%%%%%%%%%%%%%%%%%%%%%%%%%%%%%%%%%%%%%%

\begin{slide}{}
\normalsize

\begin{center}
Projekt: Taschenrechner
\end{center}
\vspace*{0.5cm}

\footnotesize
\begin{tabular}{ll}
\textbf{Ziel}:     & Sprache $\Ll(G_{\mathtt{calc}})$ der arithmetische Ausdr\"ucke    \\
                  & mit \emph{recursive descent} Parser auswerten  \\[0.3cm]
\textbf{Problem}:  & Grammatik $G_{\mathtt{calc}}$ ist links--rekursiv                 \\[0.3cm]
\textbf{L\"osung}:   & Recursive Descent Parser muss backtracken!          \\[0.3cm]
\textbf{Nachteil}: & resultierender Parser nicht sehr effizient          \\[0.3cm]
\textbf{Vorteil}:  & einfach zu implementieren                         
\end{tabular}

\textbf{Vorgehen}:
\begin{enumerate}
\item Festlegung der Daten--Strukturen

      Jedem Nicht--Terminal $X \in N$ entspricht eine \texttt{struct} \\[0.3cm]
      \hspace*{1.3cm} $\mathtt{typedef\ struct} \;X\mathtt{*}\;\;  X\mathtt{Ptr;}$
      \begin{enumerate}
      \item \textsl{Sum} entspricht  \texttt{struct Sum}
      \item \textsl{Prod} entspricht \texttt{struct Prod}
      \item \textsl{Factor} entspricht \texttt{struct Factor}
      \end{enumerate}
\item Festlegung der Funktionen

      Jedem Nicht--Terminal $X \in N$ entspricht  Funktion \\[0.3cm]
      \hspace*{1.3cm} $X\mathtt{Ptr\ }\mathtt{parse}X(\mathtt{char*\ begin,\ char*\ end})$
      \\[0.3cm]
      die die Sprache $\Ll(X)$  erkennt. \\[0.3cm]
        \hspace*{1.3cm} 
      \begin{tabular}{ll}
       \texttt{begin}: & Pointer auf erstes Zeichen   \\
       \texttt{end}:   & Pointer \textbf{hinter} letztes Zeichen  \\
      \end{tabular} \\[0.3cm]
      Funktion erfolgreich, wenn \\[0.3cm]
      \hspace*{1.3cm} $\texttt{*begin}\;\texttt{*(begin+1)}\;\cdots\;\texttt{*(end-1)} \in \Ll(X)$
\end{enumerate}

\vspace*{\fill}
\tiny \addtocounter{mypage}{1}
\rule{17cm}{1mm}
Backtracking  \hspace*{\fill} Seite \arabic{mypage}
\end{slide}

%%%%%%%%%%%%%%%%%%%%%%%%%%%%%%%%%%%%%%%%%%%%%%%%%%%%%%%%%%%%%%%%%%%%%%%%

\begin{slide}{}
\normalsize

\begin{center}
Festlegung der Daten--Strukturen
\end{center}
\vspace*{0.5cm}

\footnotesize
Genaue Analogie zwischen Grammatik und Daten--Struktur: \\[0.3cm]
\hspace*{1.3cm} 
$\begin{array}{lcl}
         \textsl{Sum}  & \rightarrow & \textsl{Sum}\; ``\mathtt{+}\cq \; \textsl{Prod}    \\
                       & |           & \textsl{Sum}\; ``\mathtt{-}\cq \; \textsl{Prod}    \\
                       & |           & \textsl{Prod}                                      \\[0.3cm]
      \end{array}
      $    

\begin{verbatim}
    struct Sum {
        char     operation; 
        SumPtr   arg1;
        ProdPtr  arg2;
        unsigned ctr;  
    };
\end{verbatim}

Bedeutung der Komponenten \\[0.3cm]
\hspace*{0.3cm} 
\begin{tabular}{ll}
 \texttt{operation}: & entweder ``\texttt{+}'' oder ``\texttt{-}'' \\[0.3cm]
 \texttt{arg1}:      & erstes Argument oder 0                      \\[0.3cm]
 \texttt{arg2}:      & zweites Argument oder einziges Argument     \\[0.3cm]
 \texttt{ctr}:       & nur zur graphischen Ausgabe 
\end{tabular}

Semantik: 
\begin{enumerate}
\item Fall: \texttt{arg1 != 0}: \\[0.3cm]
      Fall liegt vor, wenn eine der ersten beiden Grammatik Regeln verwendet wurde.
      Semantik: \\[0.3cm]
      \hspace*{1.3cm}  $\mathtt{arg1} \; \mathtt{operation} \; \mathtt{arg2}$
\item Fall: \texttt{arg1 == 0}: \\[0.3cm]
      Fall liegt vor, wenn die letzte Grammatik Regeln verwendet wurde.
      Semantik: \\[0.3cm]
      \hspace*{1.3cm}  $\mathtt{arg2}$
\end{enumerate}

\vspace*{\fill}
\tiny \addtocounter{mypage}{1}
\rule{17cm}{1mm}
Backtracking  \hspace*{\fill} Seite \arabic{mypage}
\end{slide}

%%%%%%%%%%%%%%%%%%%%%%%%%%%%%%%%%%%%%%%%%%%%%%%%%%%%%%%%%%%%%%%%%%%%%%%%

\begin{slide}{}
\normalsize

\begin{center}
Festlegung der Daten--Strukturen
\end{center}
\vspace*{0.5cm}

\footnotesize
Grammatik--Regeln f\"ur \textsl{Prod} \\[0.3cm]
\hspace*{1.3cm} 
$\begin{array}{lcl}
         \textsl{Prod} & \rightarrow & \textsl{Prod}\; ``\mathtt{*}\cq \; \textsl{Factor} \\
                       & |           & \textsl{Prod}\; ``\mathtt{/}\cq \; \textsl{Factor} \\
                       & |           & \textsl{Factor}                                    \\
 \end{array}$    

\begin{verbatim}
    struct Prod {
        char      operation; 
        ProdPtr   arg1;
        FactorPtr arg2;
        unsigned ctr; 
    };
\end{verbatim}

Bedeutung der Komponenten \\[0.3cm]
\hspace*{0.3cm} 
\begin{tabular}{ll}
 \texttt{operation}: & entweder ``\texttt{*}'' oder ``\texttt{/}'' \\[0.3cm]
 \texttt{arg1}:      & erstes Argument oder 0                      \\[0.3cm]
 \texttt{arg2}:      & zweites Argument oder einziges Argument \\[0.3cm]
 \texttt{ctr}:       & nur zur graphischen Ausgabe 
\end{tabular}

Semantik: 
\begin{enumerate}
\item Fall: \texttt{arg1 != 0}: \\[0.3cm]
      Fall liegt vor, wenn eine der ersten beiden Grammatik Regeln verwendet wurde.
      Semantik: \\[0.3cm]
      \hspace*{1.3cm}  $\mathtt{arg1} \; \mathtt{operation} \; \mathtt{arg2}$
\item Fall: \texttt{arg1 == 0}: \\[0.3cm]
      Fall liegt vor, wenn die letzte Grammatik Regeln verwendet wurde.
      Semantik: \\[0.3cm]
      \hspace*{1.3cm}  $\mathtt{arg2}$
\end{enumerate}

\vspace*{\fill}
\tiny \addtocounter{mypage}{1}
\rule{17cm}{1mm}
Backtracking  \hspace*{\fill} Seite \arabic{mypage}
\end{slide}
%%%%%%%%%%%%%%%%%%%%%%%%%%%%%%%%%%%%%%%%%%%%%%%%%%%%%%%%%%%%%%%%%%%%%%%%

\begin{slide}{}
\normalsize

\begin{center}
Festlegung der Daten--Strukturen
\end{center}
\vspace*{0.5cm}

\footnotesize
Grammatik--Regeln f\"ur \textsl{Factor} \\[0.3cm]
\hspace*{1.3cm} $\begin{array}{lcl}
       \textsl{Factor} & \rightarrow & ``\mathtt{(}\cq \;\textsl{Sum}\; ``\mathtt{)}\cq   \\
                       & |           & \mathbf{number}                                    \\
 \end{array}$    

\begin{verbatim}
    struct Factor {
        int      number;
        SumPtr   sumPtr;
        unsigned ctr; 
    };
\end{verbatim}

Bedeutung der Komponenten \\[0.3cm]
\hspace*{1.3cm} 
\begin{tabular}{ll}
 \texttt{number}:    & Zahl, falls zweite Regel verwendet wurde                    \\[0.3cm]
 \texttt{sumPtr}:    & geklammerter Ausdruck, falls erste Regel \\
                     & verwendet wurde                    \\[0.3cm]
 \texttt{ctr}:       & nur zur graphischen Ausgabe 
\end{tabular}

Semantik: 
\begin{enumerate}
\item Fall: \texttt{sumPtr != 0}: \\[0.3cm]
      Fall liegt vor, wenn die erste Grammatik Regeln verwendet wurde.
      Semantik: \\[0.3cm]
      \hspace*{1.3cm}  $\mathtt{sumPtr}$
\item Fall: \texttt{sumPtr == 0}: \\[0.3cm]
      Fall liegt vor, wenn die letzte Grammatik Regeln verwendet wurde.
      Semantik: \\[0.3cm]
      \hspace*{1.3cm}  $\mathbf{number}$
\end{enumerate}

\vspace*{\fill}
\tiny \addtocounter{mypage}{1}
\rule{17cm}{1mm}
Backtracking  \hspace*{\fill} Seite \arabic{mypage}
\end{slide}

%%%%%%%%%%%%%%%%%%%%%%%%%%%%%%%%%%%%%%%%%%%%%%%%%%%%%%%%%%%%%%%%%%%%%%%%

\begin{slide}{}
\normalsize

\begin{center}
Parsen von \textsl{Sum}
\end{center}
\vspace*{0.5cm}

\footnotesize
\textbf{Grammatik--Regeln} \\[0.3cm]
\hspace*{1.3cm} 
$\begin{array}{lcl}
         \textsl{Sum}  & \rightarrow & \textsl{Sum}\; ``\mathtt{+}\cq \; \textsl{Prod}    \\
                       & |           & \textsl{Sum}\; ``\mathtt{-}\cq \; \textsl{Prod}    \\
                       & |           & \textsl{Prod}                                      \\[0.3cm]
      \end{array}
      $    

\textbf{Algorithmus}
\begin{enumerate}
\item Versuche, letzte Regel anzuwenden

      Falls erfolgreich: fertig!
\item Suche  Zeichen ``\texttt{+}'' oder ``\texttt{-}'' in Intervall  \\[0.3cm]
      \hspace*{1.3cm} \texttt{[begin, end-1]} \\[0.3cm]
      Sei \texttt{ptr} gefunden mit 
      \begin{enumerate}
      \item \texttt{begin < ptr < end}
      \item $\texttt{*ptr} \in \{ ``\mathtt{+}\cq, ``\mathtt{-}\cq \}$
      \end{enumerate}
      Falls erfogreich:
      \begin{enumerate}
      \item Parse rekursiv \texttt{[begin,ptr]} als \textsl{Sum}
      \item Parse rekursiv \texttt{[ptr+1,end]} als \textsl{Prod}
      \end{enumerate}
      Falls erfogreich: fertig

      Sonst backtracken: Gehe zu 2. und suche n\"achstes Zeichen ``\texttt{+}'' oder ``\texttt{-}''.
\item Falls Suche nach ``\texttt{+}'' oder ``\texttt{-}'' scheitert: 

      Parsen von \textsl{Sum} nicht m\"oglich \\[0.3cm]
      \hspace*{1.3cm} \texttt{return 0;}
\end{enumerate}

\vspace*{\fill}
\tiny \addtocounter{mypage}{1}
\rule{17cm}{1mm}
Backtracking  \hspace*{\fill} Seite \arabic{mypage}
\end{slide}

%%%%%%%%%%%%%%%%%%%%%%%%%%%%%%%%%%%%%%%%%%%%%%%%%%%%%%%%%%%%%%%%%%%%%%%%

\begin{slide}{}
\normalsize

\begin{center}
Parsen von \textsl{Sum}\end{center}
\vspace*{0.5cm}

\footnotesize
\begin{verbatim}
SumPtr parseSum(char* begin, char* end)
{
    ProdPtr prod = parseProd(begin, end);
    if (prod != 0) {
        SumPtr sum = malloc( sizeof(struct Sum) );
        sum->arg1 = 0;
        sum->arg2 = prod;
        sum->ctr  = nodeCounter++;
        return sum;
    }
    for (char* ptr = begin + 1; ptr < end; ++ptr) 
    {
        if (*ptr == '+' || *ptr == '-') {
            SumPtr firstSum = parseSum(begin, ptr);
            if (firstSum != 0) {
                ProdPtr prod = parseProd(ptr + 1, end);
                if (prod != 0) {
                    SumPtr sum = 
                        malloc( sizeof(struct Sum) );
                    sum->operation = *ptr;
                    sum->arg1 = firstSum;
                    sum->arg2 = prod;
                    sum->ctr  = nodeCounter++;
                    return sum;
                } else {
                    free(firstSum);
                }
            }
        }
    }
    return 0;
}
\end{verbatim}

\vspace*{\fill}
\tiny \addtocounter{mypage}{1}
\rule{17cm}{1mm}
Backtracking  \hspace*{\fill} Seite \arabic{mypage}
\end{slide}

%%%%%%%%%%%%%%%%%%%%%%%%%%%%%%%%%%%%%%%%%%%%%%%%%%%%%%%%%%%%%%%%%%%%%%%%

\begin{slide}{}
\normalsize

\begin{center}
Parsen von \textsl{Fact}
\end{center}
\vspace*{0.5cm}

\footnotesize
\textbf{Grammatik--Regeln} \\[0.3cm]
\hspace*{1.3cm} 
$\begin{array}{lcl}
         \textsl{Factor} & \rightarrow & ``\mathtt{(}\cq \;\textsl{Sum}\; ``\mathtt{)}\cq   \\
                         & |           & \mathbf{number}                                    \\
 \end{array}$

\textbf{Algorithmus}:
\begin{enumerate}
\item Falls erstes Zeichen ``\texttt{(}'' ist, wende erste Regel an.
      \begin{enumerate}
      \item Parse rekursiv \textsl{Sum}: \\[0.3cm]
            \hspace*{1.3cm} \texttt{parseSum(begin + 1, end - 1);}
      \item Überpr\"ufe, ob letztes Zeichen ``\texttt{)}'' ist.
      \end{enumerate}
\item Falls erstes Zeichen Ziffer ist, wende zweite Regel an.
    
      Werte der Ziffern im {\sc Ascii}--Alphabet \\[0.8cm]
\hspace*{-1.3cm} 
      \begin{tabular}{|l|l|l|l|l|l|l|l|l|l|}
      \hline
          \texttt{'0'} & \texttt{'1'} &\texttt{'2'} &\texttt{'3'} &\texttt{'4'} &\texttt{'5'} &\texttt{'6'} &\texttt{'7'} &\texttt{'8'} &\texttt{'9'} \\
      \hline
          48 & 49 & 50 & 51 & 52 & 53 & 54 & 55 & 56 & 57 \\
      \hline
      \end{tabular}
      \vspace*{0.8cm}

      \textbf{Beobachtung}: Werte sind kontinuierlich verteilt. 
      \begin{verbatim}
      unsigned char2unsigned(char c) {
           return c - '0';
      }
      \end{verbatim}
\end{enumerate}

\vspace*{\fill}
\tiny \addtocounter{mypage}{1}
\rule{17cm}{1mm}
Backtracking  \hspace*{\fill} Seite \arabic{mypage}
\end{slide}

%%%%%%%%%%%%%%%%%%%%%%%%%%%%%%%%%%%%%%%%%%%%%%%%%%%%%%%%%%%%%%%%%%%%%%%%

\begin{slide}{}
\normalsize

\begin{center}
Parsen von \textsl{Fact}
\end{center}
\vspace*{0.5cm}

\footnotesize
\begin{verbatim}
FactorPtr parseFactor(char* begin, char* end) {
    if (*begin == '(') {
        SumPtr sum = parseSum(begin + 1, end - 1);
        if (sum != 0 && *(end-1) == ')') {
            FactorPtr factor = 
                malloc( sizeof(struct Factor) );
            factor->sumPtr = sum;
            factor->ctr    = nodeCounter++;
            return factor;
        } else {
            free(sum);
            return 0;
        }
    } 
    if (isdigit(*begin)) {
        int num = 0;
        while (isdigit(*begin) && begin < end) {
            num = 10 * num + *begin - '0';
            ++begin;
        }
        if (begin == end) {
            FactorPtr factor = 
                malloc( sizeof(struct Factor) );
            factor->number = num;
            factor->sumPtr = 0;
            factor->ctr    = nodeCounter++;
            return factor;
        } 
    }
    return 0;
}
\end{verbatim}

\vspace*{\fill}
\tiny \addtocounter{mypage}{1}
\rule{17cm}{1mm}
Backtracking  \hspace*{\fill} Seite \arabic{mypage}
\end{slide}



%%% Local Variables: 
%%% mode: latex
%%% TeX-master: "backtracking.tex"
%%% End: 
