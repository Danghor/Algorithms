\documentclass{article}
\usepackage{german}
\usepackage[latin1]{inputenc}

\usepackage{a4wide}
\usepackage{amssymb}
\usepackage{epsfig}

\setlength{\textwidth}{15cm}

\newcommand{\cq}{\symbol{34}}
\newcommand{\Ll}{{\cal L}}
\newcommand{\Rl}{{\cal R}}
\newcommand{\NS}{{\cal N\!S}}
\newcommand{\cl}[1]{{\cal #1}}
\renewcommand{\labelenumi}{(\alph{enumi})}

\newcounter{aufgabe}

\newcommand{\exercise}{\vspace*{0.1cm}
\stepcounter{aufgabe}

\noindent
\textbf{Aufgabe \arabic{aufgabe}}: }


\begin{document}

\noindent
{\large Aufgaben zur Vorlesung  ``{\sl Algorithmen und Datenstrukturen}''}
\vspace{0.5cm}


\exercise
\begin{enumerate}
\item L\"osen Sie die Rekursions-Gleichung \\[0.2cm]
      \hspace*{1.3cm} $a_{n+2} = a_n + 2$ \\[0.2cm]
      f\"ur die Anfangs-Bedingungen $a_0 = 2$ und $a_1 = 1$.
      \hspace*{\fill} (10 Punkte)
\item L\"osen Sie die Rekursions-Gleichung \\[0.2cm]
      \hspace*{1.3cm} $a_{n+2} = 2 \cdot a_n - a_{n+1}$ \\[0.2cm]
      f\"ur die Anfangs-Bedingungen $a_0 = 0$ und $a_1 = 3$.
      \hspace*{\fill} (10 Punkte)
\end{enumerate}
\vspace{0.3cm}

\textbf{Hinweis}: 
\begin{enumerate}
\item Bei der L\"osung der folgenden Aufgabe sind selbstverst\"andlich
      die in der Vorlesung vorgestellten Algorithmen zu verwenden.
\item Sie k\"onnen bei der folgenden Aufgabe das Ergebnis 
      graphisch als Baum angeben.
\end{enumerate}
\vspace{0.2cm}

\exercise
Der AVL-Baum $t$ sei durch den folgenden Term gegeben,
wobei zur Vereinfachung auf die Angabe der Werte, die mit den Schl\"usseln
assoziiert sind, verzichtet wurde.
\\[0.2cm]
\hspace*{1.3cm}
$t = \textsl{node}(17, \textsl{node}(8, \textsl{node}(2, \textsl{nil}, \textsl{nil}),
 \textsl{node}(10, \textsl{nil}, \textsl{nil})), \textsl{node}(23,\textsl{nil},\textsl{nil}))$
%\epsfig{file=avl1,scale=0.5}
\begin{enumerate}
\item F\"ugen Sie  in diesem Baum den Schl\"ussel \texttt{13} ein und geben Sie den
      resultierenden Baum an.   \\[0.2cm]
      \hspace*{\fill} (4 Punkte)
\item F\"ugen Sie in dem Baum aus Teil den Schl\"ussel \texttt{15} ein und geben Sie den
      resultierenden Baum an.
      \hspace*{\fill} (3 Punkte)
\item Entfernen Sie den Schl\"ussel 2 aus dem unter Teil (b) berechneten Baum und geben Sie
      den resultierenden Baum an.
      \hspace*{\fill} (4 Punkte)
\end{enumerate}

%\exercise
%\begin{enumerate}
%\item Zeigen Sie, dass f\"ur die Funktion \textsl{merge}, die wir im Skript definiert haben,
%      folgende Gleichung gilt:
%      \\
%      \hspace*{1.3cm}
%      $\textsl{count}(x, \textsl{merge}(L_1, L_2)) = \textsl{count}(x, L_1) + \textsl{count}(x, L_2)$
%      \hspace*{\fill} (8 Punkte)
%\item Zeigen Sie, dass f\"ur die Funktion \textsl{split}, die wir im Skript definiert haben,
%      folgende Gleichung gilt:
%      \\
%      \hspace*{1.3cm}
%      $\textsl{split}(L) = [L_1, L_2] \rightarrow \textsl{count}(x, L) = \textsl{count}(x, L_1) + \textsl{count}(x, L_2)$
%      \hspace*{\fill} (7 Punkte)
%\end{enumerate}
\pagebreak

\exercise
Betrachten Sie das folgende Programm:
\begin{verbatim}
    sum := procedure(n) {
        i := 0;
        s := 0;
        while (i <= n) {
            s := i + s;
            i := i + 1;
        }
        return s;
    }
\end{verbatim}
Die Funktion $\textsl{sum}$  soll die folgende Spezifikation erf\"ullen:
\\[0.2cm]
\hspace*{1.3cm}
$\textsl{sum}(n) = \frac{1}{2} \cdot n \cdot (n + 1)$
\begin{enumerate}
\item Weisen Sie mit Hilfe des Hoare-Kalk\"uls nach, dass das Programm korrekt ist.
\item Beweisen Sie mit Hilfe der Methode der symbolischen Programm-Ausf\"uhrung,
      dass das Programm korrekt ist.
\end{enumerate}
\vspace{0.3cm}

\exercise
Im Abschnitt 8.2 des Skriptes
werden Gleichungen angegeben, die das Einf\"ugen und L\"oschen in einem Heap beschreiben.
In diesem Zusammenhang sollen Sie in dieser Aufgabe  einige zus\"atzliche Methoden auf
bin\"aren B\"aumen durch bedingte Gleichungen spezifizieren.
\begin{enumerate}
\item Spezifizieren Sie eine Methode \textsl{isHeap}, so
      dass f\"ur einen bin\"aren Baum $b \in \mathcal{B}$ der Ausdruck 
      $b.\mathtt{isHeap}()$ genau dann den Wert $\mathtt{true}$ hat, wenn $b$ die
      \emph{Heap-Bedingung} erf\"ullt.      
\item Spezifizieren Sie eine Methode \textsl{isBalanced}, so
      dass f\"ur einen bin\"aren Baum $b \in \mathcal{B}$ der Ausdruck 
      $b.\mathtt{isBalanced}()$ genau dann den Wert $\mathtt{true}$ hat, wenn $b$ die
      \emph{Balancierungs-Bedingung} f\"ur \emph{Heaps} erf\"ullt.  
      \hspace*{\fill} (10 Punkte)
\end{enumerate}
\vspace{0.3cm}

\exercise
Es sei $f(n) := \biggl(\sum\limits_{i=1}^n \frac{1}{i}\biggr) - \ln(n)$.
Zeigen Sie $f(n)\in \mathcal{O}\bigl(1\bigr)$. \hspace*{\fill} (12 Punkte)
      
\noindent
\textbf{Hinweis}: Zeigen Sie die Ungleichung
\\[0.2cm]
\hspace*{1.3cm}
$0 \leq \biggl(\sum\limits_{i=1}^n \frac{1}{i}\biggr) - \ln(n) \leq 1$
\\[0.2cm]
indem Sie die Summe $\sum\limits_{i=1}^n \frac{1}{i}$ durch geeignete Integrale absch\"atzen.
\vspace{0.3cm}
\pagebreak


\exercise
Es sei $\mathcal{B}$ die Menge der bin\"aren B\"aume, die im Skript definiert wird.  
\begin{enumerate}
\item Spezifizieren Sie eine Methode \\[0.2cm]
      \hspace*{1.3cm}
      $\textsl{isOrdered}: \mathcal{B} \rightarrow \mathbb{B}$
      \\[0.2cm]
      durch bedingte Gleichungen.  F\"ur einen bin\"aren Baum $b$ soll der Aufruf
      $b.\textsl{isOrdered}()$ genau dann \texttt{true} zur\"uck liefern, wenn $b\in \mathcal{B}_<$
      gilt.
      \hspace*{\fill} (8 Punkte)

      \textbf{Hinweis}: Definieren Sie sich geeignete Hilfsfunktionen.
\item Es sei $\textsl{insert}()$ die in Abschnitt 6.1 definierte Methode.
      Nehmen Sie an, dass Sie f\"ur alle $b\in \mathcal{B}_<$, alle Schl\"ussel $k$ und alle Werte
      $v$ die Gleichung
      \\[0.2cm]
      \hspace*{1.3cm} $b.\textsl{insert}(k,v).\textsl{isOrdered}() = \texttt{true}$
      \\[0.2cm]
      beweisen sollen.  Geben Sie an, welche Lemmata \"uber die in Teil (a) definierten
      Hilfsfunktionen zu einem solchen Beweis ben\"otigt werden.
      \hspace*{\fill} (4 Punkte)
\item Zeigen Sie nun f\"ur geordnete bin\"are B\"aume die Gleichung
      \\[0.2cm]
      \hspace*{1.3cm} $b.\textsl{insert}(k,v).\textsl{isOrdered}() = \texttt{true}$ 
      \hspace*{\fill} (12 Punkte)
      \\[0.2cm]
      Sie d\"urfen dabei die Lemmata, die Sie in Teil (b) angeben sollen, benutzen.
\end{enumerate}
\vspace{0.3cm}

\exercise
Es gelte $\Sigma = \{ \mathtt{a},\,\mathtt{b},\,\mathtt{c},\,\mathtt{d},\,\mathtt{e},\,\mathtt{f} \}$.
Die H\"aufigkeit, mit der diese Buchstaben in dem zu kodierenden String $s$ auftreten, sei durch die
folgende Tabelle gegeben:

\begin{center}
\begin{tabular}[t]{|l|r|r|r|r|r|r|}
\hline
Buchstabe  & \texttt{a} & \texttt{b} & \texttt{c} & \texttt{d} & \texttt{e} & \texttt{f} \\
\hline
H\"aufigkeit &          8 &          9 &         10 &         11 &         12 &         13 \\
\hline
\end{tabular}
\end{center}
\begin{enumerate}
\item Berechnen sie einen optimalen Kodierungs-Baum f\"ur die angegebenen H\"aufigkeiten.
\item Geben Sie die Kodierung der einzelnen Buchstaben an, die sich aus diesem Baum ergibt.
\end{enumerate}

\exercise
\begin{enumerate}
\item Use the LZW algorithm to encode the string ``\texttt{aabbaaabbb}''.  Compute the
      compression factor for this string.
\item Decode the list 
      \\[0.2cm]
      \hspace*{1.3cm}
      $[97, 98, 98, 128, 131]$
      \\[0.2cm]
      using the LZW algorithm.
\end{enumerate}


\end{document}

%%% Local Variables: 
%%% mode: latex
%%% TeX-master: t
%%% End: 
