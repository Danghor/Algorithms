\chapter{Graphentheorie}
Wir wollen zum Abschluss der Vorlesung wenigstens ein graphentheoretisches Problem
vorstellen: Das Problem der Berechnung  k\"urzester Wege. 


\section{Die Berechnung k\"urzester Wege}
Um das Problem der Berechnung k\"urzester Wege formulieren zu k\"onnen, f\"uhren wir zun\"achst 
den Begriff des \emph{gewichteten Graphen} ein.  

\begin{Definition}[Gewichteter Graph]
{\em
  Ein  {\em gewichteter Graph} ist ein Tripel
   $\langle \nodes, \edges, \weight{\cdot} \rangle$ so dass gilt:
  \begin{enumerate}
  \item $\nodes$ ist eine Menge von \emph{Knoten}.
  \item $\edges \subseteq \nodes \times \nodes$ ist eine Menge von \emph{Kanten}.
  \item $\weight{\cdot}: \edges \rightarrow \N \backslash\{0\}$ ist eine Funktion,
        die jeder Kante eine positive \emph{L\"ange} zuordnet.
        \conclude
  \end{enumerate}
}
\end{Definition}

\noindent
Ein \emph{Pfad} $P$ ist eine Liste der Form \\[0.2cm]
\hspace*{1.3cm} $P = [ x_1, x_2, x_3, \cdots, x_n ]$ \\[0.2cm]
so dass f\"ur alle $i = 1, \cdots, n-1$ gilt: \\[0.2cm]
\hspace*{1.3cm} $\pair(x_i,x_{i+1}) \in \edges$. \\[0.2cm]
Die Menge aller Pfade bezeichnen wir mit $\paths$.
Die L\"ange eines Pfads definieren wir als die Summe der L\"ange aller Kanten:
\\[0.2cm]
\hspace*{1.3cm} $\Weight{[x_1,x_2, \cdots, x_n]} \df \sum\limits_{i=1}^{n-1} \Weight{\pair(x_i,x_{i+1})}$. \\[0.2cm]
Ist $p = [x_1, x_2, \cdots, x_n]$ ein Pfad, so sagen wir, dass $p$ den Knoten $x_1$ mit dem
Knoten $x_n$ verbindet.   Die Menge alle Pfade, die den Knoten $v$ mit dem Knoten $w$
verbinden, bezeichnen wir als \\[0.2cm]
\hspace*{1.3cm} 
 $\paths(v,w) \df \bigl\{ [x_1, x_2, \cdots, x_n] \in \paths \mid x_1 = v \,\wedge\, x_n = w \}$.
\\[0.2cm]
Damit k\"onnen wir nun das Problem der Berechnung k\"urzester Wege formulieren.
\begin{Definition}[K\"urzeste-Wege-Problem]
{\em
  Gegeben sei ein gewichteter Graph $G = \langle \nodes, \edges, \weight{\cdot} \rangle$ 
  und ein  Knoten $\source \in \nodes$.  Dann besteht das 
  {\em k\"urzeste-Wege-Problem}  darin, die folgende Funktion zu berechnen: \\[0.2cm]
  \hspace*{1.3cm} $\spath: \nodes \rightarrow \N$ \\[0.1cm]
  \hspace*{1.3cm} $\spath(v) \df \mathtt{min}\bigl\{ \weight{p} \mid p \in \paths(\source,v) \bigr\}$.
  \conclude  
}
\end{Definition}

\subsection{Ein naiver Algorithmus zur L\"osung des k\"urzeste-Wege-Problems}
Als erstes betrachten wir einen ganz naiven Algorithmus zur L\"osung des k\"urzeste-Wege-Problems.
Die Idee ist, dass wir eine Funktion \\[0.2cm]
\hspace*{1.3cm} $\texttt{dist}: \nodes \rightarrow \N \cup \{\Omega\}$ \\[0.2cm]
definieren, die f\"ur einen Punkt $u \in \nodes$ eine obere Absch\"atzung des Abstandes zum Knoten
\textsl{source} angibt, es soll also immer gelten: \\[0.2cm]
\hspace*{1.3cm} $\textsl{dist}(u) \not= \Omega \;\rightarrow\; \spath(u) \leq
\textsl{dist}(u)$. \\[0.2cm]
Die Funktion \textsl{dist} liefert zu einem Knoten $x$ die k\"urzeste bisher
bekannte Entfernung zum Knoten \textsl{source}.  Solange noch kein Pfad von
\textsl{source} zu dem Knoten $u$ gefunden worden ist, gilt $\textsl{dist}(u) = \Omega$.
Anfangs ist die Funktion $\textsl{dist}()$ also nur f\"ur den Knoten \textsl{source}
definiert, es gilt \\[0.2cm]
\hspace*{1.3cm} $\textsl{dist}(\textsl{source}) = 0$. \\[0.2cm]
Sp\"ater, wenn wir f\"ur einen Knoten $u$ einen Pfad gefunden haben, der den Knoten
\textsl{source} mit dem Knoten $u$ verbindet und wenn es zus\"atzlich eine Kante
$\pair(u,v)$ gibt, die den Knoten $u$ mit einem anderen Knoten $v$ verbindet, dann wissen
wir, dass auch der Knoten $v$ von \textsl{source} erreichbar ist.  Zus\"atzlich wissen wir,
dass dieser Weg die L\"ange $\textsl{dist}(u) + \weight{\pair(u,v)}$ hat.  Falls bisher also
$\textsl{dist}(v)$ undefiniert war, weil wir noch keinen Weg gefunden hatten, der
\textsl{source} mit $v$ verbindet, k\"onnen wir \\
\hspace*{1.3cm} $\textsl{dist}(v) := \textsl{dist}(u) + \weight{\pair(u,v)}$ \\[0.2cm]
setzen.  Diese Zuweisung ist ebenfalls g\"ultig wenn $\textsl{dist}(v)$ bereits definiert
ist aber einen Wert hat, der gr\"o{\ss}er als $\textsl{dist}(u) + \weight{\pair(u,v)}$ ist.
Wir fassen diese \"Uberlegungen in den beiden
\textsl{ASM}-Regeln zusammen, die in Abbildung \ref{fig:rule-naive} dargestellt sind.  
Die Abk\"urzung \textsc{ASM} steht f\"ur \emph{abstract state machine}.
Dieser Begriff wurde von 
Yuri Gurevich \cite{gurevich:91} eingef\"uhrt und von Egon B\"orger \cite{boerger:2003} zur
Spezifikation und Verifikation von 
Algorithmen  propagiert und weiterentwickelt.  ASMs sind eine Art Pseudo-Code.  Die 
wesentlichen Eigenschaften von ASMs sind wie folgt:
\begin{enumerate}
\item ASMs bestehen aus Regeln.  Dabei besteht jede Regel
      aus einem Namen, einer Bedingung und einer Menge von Zuweisungen.
\item Bei der Abarbeitung der Regeln wird willk\"urlich eine Regel ausgew\"ahlt, deren
      Bedingung wahr ist und die Zuweisungen dieser Regel werden ausgef\"uhrt.
\item Bei Zuweisungen k\"onnen nicht nur Variablen ge\"andert werden, sondern es k\"onnen auch die Werte,
      die eine Funktion an einer Stelle annimmt, ver\"andert werden k\"onnen.  Eine Zuweisung
      kann also die Form
      \\[0.2cm]
      \hspace*{1.3cm}
      $f(x) := y$
      \\[0.2cm]
      haben.  Diese Zuweisung \"andert die Funktion $f$ so ab, dass die Funktion anschlie{\ss}end an der Stelle $x$
      den Wert $y$ annimmt.
\item Wenn es keine Regel mehr gibt, deren Bedingung wahr ist, dann h\"alt die ASM an.
\item Die Zuweisungen einer Regel werden alle gleichzeitig ausgef\"uhrt.  In der Regel 
      \\[0.2cm]
      \hspace*{0.8cm} \texttt{\underline{Rule}} \textsl{swap}            \\
      \hspace*{1.3cm} \texttt{\underline{if} x < y \underline{then}}     \\
      \hspace*{1.8cm} \texttt{x := y;}                                   \\
      \hspace*{1.8cm} \texttt{y := x;}                                   \\
      \hspace*{1.3cm} \texttt{\underline{endif}}                         \\[0.2cm]
      werden die beiden Zuweisungen also gleichzeitig ausgef\"uhrt, so dass im Endeffekt die
      Werte von \texttt{x} und \texttt{y} vertauscht werden.
\end{enumerate}
Wie ASMs im Detail funktionieren, erkl\"aren wir bei der Diskussion des in Abbildung
\ref{fig:rule-naive} gegebenen Beispiels.

\begin{figure}[!ht]
  \centering
\begin{Verbatim}[ frame         = lines, 
                  framesep      = 0.3cm, 
                  labelposition = bottomline,
                  numbers       = left,
                  numbersep     = -0.2cm,
                  commandchars  = \\\{\},
                  xleftmargin   = 0.0cm,
                  xrightmargin  = 0.0cm
                ]
    \underline{Rule} \textsl{Init}
        \underline{if}  \(\textsl{dist}(\textsl{source}) = \Omega\)
        \underline{then}
            \(\textsl{dist}(\textsl{source})\) := \(0\);
        \underline{endif}
        
    \underline{Rule} \textsl{Run}
        \underline{if} \underline{choose} \(\langle{}u,v\rangle\in\edges\) \underline{satisf}y\underline{in}g
            \(\textsl{dist}(u) \not= \Omega\) \underline{and}  \((\textsl{dist}(v) = \Omega\) \underline{or} \(\textsl{dist}(u)+\weight{\langle{}u,v\rangle}<\textsl{dist}(v))\)
        \underline{then}
            \(\textsl{dist}(v)\) := \(\textsl{dist}(u)+\weight{\langle{}u,v\rangle}\);
        \underline{endif}
\end{Verbatim}
\vspace*{-0.3cm}
  \caption{ASM-Regeln zur L\"osung des k\"urzeste-Wege-Problems.}
  \label{fig:rule-naive}
\end{figure} 
\begin{enumerate}
\item Die erste Regel hat den Namen \textsl{Init}.
      In dieser Regel wird  $\textsl{dist}(\textsl{source})$ auf den Wert 0
      gesetzt, wenn die Funktion \textsl{dist} an der Stelle \textsl{source}
      noch undefiniert ist.  Diese Regel kann nur einmal ausgef\"uhrt werden,
      denn nach Ausf\"uhrung der Regel ist $\textsl{dist}(source)$ von $\Omega$ verschieden.
\item Die zweite Regel benutzt das Konstrukt \texttt{choose}.
      Dieses Konstrukt hat allgemein die Form 
      \\[0.2cm]
      \hspace*{1.3cm}
      \texttt{choose} $\langle x_1, \cdots, x_n\rangle \in M: F(x_1,\cdots,x_n)$
      \\[0.2cm]
      Hierbei sind $x_1, \cdots, x_n$ verschiedene Variablen, $M$ ist eine Menge von
      $n$-Tupeln und \linebreak 
      $F(x_1,\cdots,x_n)$ ist eine
      logische Formel, in der diese Variablen auftreten.
      Das \texttt{choose}-\linebreak
      Konstrukt liefert genau dann als Ergebnis \texttt{true}
      zur\"uck, wenn es in der Menge $M$ ein Tupel $\langle t_1,\cdots,t_n\rangle$ gibt, so dass die Formel 
      $F(t_1,\cdots,t_n)$ wahr wird.  In diesem Fall werden gleichzeitig die Variablen
      $x_1,\cdots,x_n$ mit den entsprechenden Elementen belegt.
      
      Bei der zweiten Regel suchen wir \"uber das \texttt{choose}-Konstrukt
      eine Kante $\pair(u,v)$, f\"ur die gilt: 
      \begin{enumerate}
      \item $\textsl{dist}(u)$ ist definiert, es gibt also einen Pfad von
            dem Knoten $\textsl{source}$  zu dem Knoten $u$.
      \item $\textsl{dist}(v)$ ist undefiniert oder gr\"o{\ss}er als 
            $\textsl{dist}(u)+\weight{\langle{}u,v\rangle}$. 
      \end{enumerate}
      Dann k\"onnen wir die Absch\"atzung
      f\"ur den Abstand $\textsl{dist}(v)$ von dem Knoten $\textsl{source}$ zu dem Knoten
      $v$ zu dem Wert $\textsl{dist}(u)+\weight{\langle{}u,v\rangle}$ verbessern.
\end{enumerate}
Der Algorithmus um das k\"urzeste-Wege-Problem zu l\"osen besteht nun darin,
dass wir die ASM-Regeln solange ausf\"uhren, wie dies m\"oglich ist. 
Der Algorithmus terminiert, denn die Regel \textsl{Init} kann nur einmal ausgef\"uhrt
werden und die Regel \textsl{Run} vermindert bei jeder Ausf\"uhrung den Wert
der Funktion $\textsl{dist}()$ an einem Punkt.  Da der Werte-Bereich dieser Funktion aus
nat\"urlichen Zahlen besteht, geht das nur endlich oft.


\subsection{Der Algorithmus von Moore}
\begin{figure}[!ht]
  \centering
\begin{Verbatim}[ frame         = lines, 
                  framesep      = 0.3cm, 
                  labelposition = bottomline,
                  numbers       = left,
                  numbersep     = -0.2cm,
                  commandchars  = \\\{\},
                  xleftmargin   = 0.8cm,
                  xrightmargin  = 0.8cm
                ]
    \underline{Rule} \textsl{Init}
        \underline{if}   \(\textsl{dist}(\textsl{source}) = \Omega\)
        \underline{then} \(\textsl{dist}(\textsl{source})\) := \(0\);
             \textsl{mode}\,        := \textsl{scan};
             \textsl{Fringe}\,      := \(\{ \textsl{source} \}\);
        \underline{endif}
        
    \underline{Rule} \textsl{Scan}
        \underline{if}   \(\textsl{mode} = \mathtt{scan}\) \underline{and} \underline{choose} \(u\in\textsl{Fringe}\) 
        \underline{then}
           \({\cal E}\!\)      := \(\textsl{edges}(u)\);
            \(\textsl{Fringe}\) := \(\texttt{Fringe} \backslash \{u\}\);
            \(\textsl{mode}\)   := \texttt{relabel};
        \underline{endif}

    \underline{Rule} \textsl{Relabel}
        \underline{if}        \(\textsl{mode} = \textsl{relabel}\)
             \underline{and}  \underline{choose} \(\langle{}u,v\rangle\in{\cal{}E}\) \underline{satisfying}  
                      \(\textsl{dist}(v)=\Omega\) \underline{or} \(\textsl{dist}(u)+\weight{\langle{}u,v\rangle}<\textsl{dist}(v)\);
        \underline{then}
            \(\textsl{dist}(v)\) := \(\textsl{dist}(u)+\weight{\langle{}u,v\rangle}\);
            \(\textsl{Fringe}\;\) := \(\texttt{Fringe}\cup\{v\}\);
        \underline{else} 
            \(\textsl{mode}\;\)   := \texttt{scan};
        \underline{endif}
\end{Verbatim}
\vspace*{-0.3cm}
  \caption{Algorithmus von Moore zur L\"osung des k\"urzeste-Wege-Problems.}
  \label{fig:rules-moore}
\end{figure} 

\noindent
Der oben gezeigte Algorithmus l\"asst sich zwar prinzipiell implementieren, er ist aber 
viel zu ineffizient um praktisch n\"utzlich zu sein.
Beim naiven Algorithmus ist die Frage, in welcher Reihenfolge Knoten ausgew\"ahlt werden,
nicht weiter spezifiert.   Edward F. Moore \cite{moore:59} hat den Algorithmus in naheliegender Weise
verbessert,  indem er \"uber die Auswahl der Knoten Buch f\"uhrte.  
Dazu benutzen wir die Variable \textsl{Fringe}, die die Menge aller Knoten enth\"alt, von denen aus
noch k\"urzere Pfade gefunden werden k\"onnen.  Am Anfang enth\"alt diese Menge nur den Knoten
\texttt{source}.  Jedes Mal, wenn f\"ur einem Knoten $v$ die Funktion
$\mathtt{dist}(v)$ ge\"andert wird, wird $v$ der Menge \textsl{Fringe} hinzugef\"ugt.
Umgekehrt wird $v$ aus der Menge \textsl{Fringe} entfernt, wenn alle Kanten, die von $v$
ausgehen, betrachtet worden sind.  Um leichter \"uber diese Kanten iterieren zu k\"onnen,
nehmen wir an, dass eine Funktion \\[0.2cm]
\hspace*{1.3cm} $\textsl{edges}: \nodes \rightarrow 2^{\edges}$ \\[0.2cm]
gegeben ist, die f\"ur einen gegebenen Knoten $u$ die Menge aller Kanten $\pair(u,v)$
berechnet, die von dem Knoten $u$ ausgehen.  Es gilt also \\[0.2cm]
\hspace*{1.3cm} $\textsl{edges}(u) = \{ \pair(u,v) \mid \pair(u,v) \in \edges \}$.
\\[0.2cm]
Der Algorithmus l\"auft nun in drei Phasen ab.
\begin{enumerate}
\item In der \emph{Initialisierungs}-Phase setzen wir $\textsl{dist}(\textsl{source}) := 0$.
\item In der \emph{Scanning}-Phase w\"ahlen wir einen Knoten $u$ aus der Menge \textsl{Fringe} aus,
      entfernen ihn aus dieser Menge und setzen \\[0.2cm]
      \hspace*{1.3cm} ${\cal E} := \textsl{edges}(u)$. \\[0.2cm]
      Ansschlie{\ss}end wechseln wir in die \emph{Relabeling}-Phase.
\item In der \emph{Relabeling}-Phase w\"ahlen wir eine Kante 
      $\pair(u,v) \in {\cal E}$ aus, f\"ur die \\[0.2cm]
      \hspace*{1.3cm} $\textsl{dist}(v)=\Omega$ \quad oder \quad
      $\textsl{dist}(u) + \weight{\langle{}u,v\rangle} < \textsl{dist}(v)$ \\[0.2cm]
      gilt.  Dann \"andern wir die Abstands-Funktion \textsl{dist} f\"ur den Knoten $v$ ab
      und f\"ugen gleichzeitig den Knoten $v$ der Menge \textsl{Fringe} hinzu.

      Falls wir keinen Knoten finden k\"onnen, f\"ur den wir die 
      Funktion $\textsl{dist}(u)$ verkleinern k\"onnen, wechseln wir wieder in die \emph{Scanning}-Phase zur\"uck.
\end{enumerate}
Der Algorithmus bricht ab, wenn die Menge $\textsl{Fringe}$ leer wird.
Abbildung \ref{fig:rules-moore} auf Seite \pageref{fig:rules-moore} zeigt die
Spezifikation dieses Algorithmus durch eine ASM.

\subsection{Der Algorithmus von Dijkstra}
\begin{figure}[!htt]
  \centering
\begin{Verbatim}[ frame         = lines, 
                  framesep      = 0.3cm, 
                  labelposition = bottomline,
                  numbers       = left,
                  numbersep     = -0.2cm,
                  commandchars  = \\\{\},
                  xleftmargin   = 0.8cm,
                  xrightmargin  = 0.8cm
                ]
    \underline{Rule} \textsl{Init}
        \underline{if}   \(\textsl{dist}(\textsl{source}) = \Omega\)
        \underline{then} 
             \(\textsl{Fringe}.\textsl{insert}(0,\textsl{source})\);
             \textsl{dist}(\textsl{source}) := \(0\);
             \textsl{Visited}      := \(\{ \textsl{source} \}\);
             \textsl{mode}         := \textsl{scan};
        \underline{endif}
        
    \underline{Rule} \textsl{Scan}
        \underline{if}     \(\textsl{mode} = \textsl{scan}\)
           \underline{and} \underline{not} \texttt{Fringe}.\texttt{isEmpty}()  
        \underline{then}
            \(\langle{d,u}\rangle\) := \textsl{Fringe}.\textsl{top}();
            \textsl{Fringe}.remove();
            \textsl{Visited} := \(\textsl{Visited} \cup \{ u \}\);
           \({\cal E}\,\)    := \(\textsl{edges}(u)\);            
            \(\textsl{mode}\;\) := \texttt{relabel};
        \underline{endif}

    \underline{Rule} \textsl{Relabel}
        \underline{if}        \(\textsl{mode} = \textsl{relabel}\)
             \underline{and}  \underline{choose} \(\langle{}u,v\rangle\in{\cal{}E}\) \underline{satisfying}  
                      \(\textsl{dist}(v)=\Omega\) \underline{or} \(\textsl{dist}(u)+\weight{\langle{}u,v\rangle}<\textsl{dist}(v)\);
        \underline{then}
            \(\textsl{dist}(v)\) := \(\textsl{dist}(u)+\weight{\langle{}u,v\rangle}\);
            \underline{if} \(\textsl{dist}(v) = \Omega\) \underline{then}
                \(\textsl{Fringe}\) := \(\texttt{Fringe}.\textsl{insert}(\textsl{dist}(v),v)\);
            \underline{else}
                \(\textsl{Fringe}\) := \(\texttt{Fringe}.\textsl{change}(\textsl{dist}(v),v)\);
            \underline{endif} 
        \underline{else} 
            \(\textsl{mode}\;\) := \texttt{scan};
        \underline{endif}
\end{Verbatim}
\vspace*{-0.3cm}
  \caption{ASM-Regeln f\"ur den Algorithmus von Dijkstra.}
  \label{fig:rules-dijkstra}
\end{figure} 

\noindent
Im Algorithmus von Moore ist die Frage, in welcher Weise die Knoten aus der Menge
\textsl{Fringe} ausgew\"ahlt werden, nicht weiter spezifiziert.  
Die Idee bei dem  von Edsger W.~Dijkstra (1930 -- 2002) im Jahre 1959 ver\"offentlichten
Algorithmus \cite{dijkstra:59}
besteht darin, in der Regel \textsl{Scan} immer den Knoten auszuw\"ahlen, der den geringsten Abstand zu
dem Knoten \textsl{source} hat.
Dazu wird die Menge \textsl{Fringe} durch eine Priorit\"ats-Warteschlange
implementiert.  Als Priorit\"aten w\"ahlen wir die Entfernungen zu dem Knoten \texttt{source}.
Abbildung \ref{fig:rules-dijkstra} auf Seite \pageref{fig:rules-dijkstra} zeigt die 
Spezifikation des Algorithmus von Dijkstra zur Berechnung der k\"urzesten Wege.
Gegen\"uber dem Algorithmus von Moore hat sich vor allem die Regel \textsl{Scan} ge\"andert,
denn dort w\"ahlen wir jetzt immer den Knoten aus der Menge \textsl{Fringe}, der den
kleinsten Abstand zum Knoten \textsl{source} hat.

In den ASM-Regeln taucht noch eine Variable mit dem Namen \textsl{Visited} auf.
Diese Variable bezeichnet die Menge der Knoten, die der Algorithmus schon \textsl{besucht}
hat.  Genauer sind das die Knoten, die aus der Priorit\"ats-Warteschlange \texttt{Fringe}
entfernt wurden und f\"ur die dann anschlie{\ss}end in der Regel \textsl{Relabel} alle
benachbarten Knoten untersucht wurden.  Die Menge \textsl{Visited} hat keine Bedeutung f\"ur
die eigentliche Implementierung des Algorithmus.  Die Variable wird eingef\"uhrt um eine Invariante formulieren
zu k\"onnen, die f\"ur den Beweis der Korrektheit des Algorithmus zentral ist.  Die Invariante lautet
\\[0.2cm]
\hspace*{1.3cm}
$\forall u\in\mathtt{Visited}: \textsl{dist}(u) = \textsl{sp}(u)$.
\\[0.2cm]
F\"ur alle Knoten aus \textsl{Visited} liefert die Funktion \textsl{dist}() also bereits den
k\"urzesten Abstand zum Knoten \textsl{source}.  
\vspace*{0.1cm}

\noindent
\textbf{Beweis}: Wir zeigen durch Induktion, dass jedes Mal wenn wir einen Knoten $u$ in die Menge
\textsl{Visited} einf\"ugen, die Gleichung $\textsl{dist}(u) = \textsl{sp}(u)$ gilt.
In den ASM-Regeln gibt es zwei Stellen, bei denen wir der Menge \textsl{Visited} neue
Elemente hinzuf\"ugen.
\begin{enumerate}
\item[I.A.:]
      In Zeile 6 f\"ugen wir den Start-Knoten \textsl{source} in die Menge \textsl{Visited}
      ein.  Wegen $\textsl{sp}(\textsl{source}) = 0$ ist die Behauptung in diesem Fall
      offensichtlich.
\item[I.S.:]
      In Zeile 16 f\"ugen wir den Knoten $u$ in die Menge \textsl{Visited} ein.
      Wir betrachten nun die Situation unmittelbar vor dem Einf\"ugen von $u$.
      Wir k\"onnen annehmen, dass $u$ noch nicht in der Menge \textsl{Visited}
      enthalten ist, denn sonst wird $u$  ja nicht wirklich in \textsl{Visited} eingef\"ugt.
      Wir f\"uhren den Beweis nun indirekt und nehmen an, dass 
      \\[0.2cm]
      \hspace*{1.3cm} $\textsl{dist}(u) > \textsl{sp}(u)$
      \\[0.2cm]
      gilt.  Dann gibt es einen k\"urzesten Pfad 
      \\[0.2cm]
      \hspace*{1.3cm} $p = [ x_0 = \textsl{source}, x_1, \cdots, x_n = u ]$
      \\[0.2cm]
      von \textsl{source} nach $u$, der insgesamt die L\"ange $\textsl{sp}(u)$ hat.
      Es sei  $i\in\{0,\cdots,n-1\}$ der Index f\"ur den 
      \\[0.2cm]
      \hspace*{1.3cm}
      $x_0\in \textsl{Visited}$, $\cdots$, $x_i\in \textsl{Visited}$ \quad aber \quad $x_{i+1} \not\in \mathtt{Visited}$,
      \\[0.2cm]
      gilt, $x_i$ ist also der erste Knoten aus dem Pfad $p$, f\"ur den $x_{i+1}$ nicht mehr
      in der Menge
      \textsl{Visited} liegt.  Nachdem $x_i$ in die Menge Visited eingef\"ugt wurde,
      wurde f\"ur alle Knoten, die mit $x_i$ \"uber eine Kante verbunden sind,
      die Funktion \textsl{dist}() neu ausgerechnet.  Insbesondere
      wurde auch $\textsl{dist}(x_{i+1})$ neu berechnet und der Knoten $x_{i+1}$ wurde 
      sp\"atestens zu diesem Zeitpunkt in die Menge \textsl{Fringe} eingef\"ugt.
      Au{\ss}erdem wissen wir, dass $\textsl{dist}(x_{i+1}) = \textsl{sp}(x_{i+1})$ gilt,
      denn nach Induktions-Voraussetzung gilt $\textsl{dist}(x_i) = \textsl{sp}(x_i)$
      und die Kante $\pair(x_i,x_{i+1})$ ist Teil eines k\"urzesten Pfades von $x_i$ nach $x_{i+1}$.
      
      Da wir nun angenommen haben, dass $x_{i+1} \not\in \textsl{Visited}$ ist,
      muss $x_{i+1}$ immer noch in der Pri\-ori\-t\"ats-Warteschlange \textsl{Fringe} liegen.
      Also muss $\textsl{dist}(x_{i+1}) \geq \textsl{dist}(u)$ gelten,
      denn sonst w\"are $x_{i+1}$ vor $u$ aus der Priorit\"ats-Warteschlange entfernt worden.
      Wegen $\textsl{sp}(x_{i+1}) = \textsl{dist}(x_{i+1})$ haben wir dann aber
      den Widerspruch 
      \\[0.2cm]
      \hspace*{1.3cm} 
      $\textsl{sp}(u) \geq \textsl{sp}(x_{i+1}) = \textsl{dist}(x_{i+1}) \geq
      \textsl{dist}(u) > \textsl{sp}(u)$.
      \hspace*{\fill} $\Box$
\end{enumerate}

\subsection{Implementierung in \textsl{Java}}
\begin{figure}[!ht]
\centering
\begin{Verbatim}[ frame         = lines, 
                  framesep      = 0.3cm, 
                  labelposition = bottomline,
                  numbers       = left,
                  numbersep     = -0.2cm,
                  xleftmargin   = 0.8cm,
                  xrightmargin  = 0.8cm,
                ]
    import java.util.*;
    
    public class Node implements Comparable<Node>
    {
        private String     mName;
        private List<Edge> mEdges;
    
        public Node(String name) {
            mName  = name;
            mEdges = new LinkedList<Edge>();
        }   
        public String     toString() { return mName;  }
        public String     getName () { return mName;  }
        public List<Edge> getEdges() { return mEdges; }
        public void       setEdges(List<Edge> edges) { mEdges = edges; }
            
        public int compareTo(Node node) {
            return mName.compareTo(node.mName);
        }
    }
\end{Verbatim}
\vspace*{-0.3cm}
\caption{Die Klasse Node.}
\label{fig:Graph/Node.java}
\end{figure}

Zun\"achst m\"ussen wir \"uberlegen, wie wir einen Graphen repr\"asentieren wollen.
Abbildung \ref{fig:Graph/Node.java} zeigt die Klasse \texttt{Node}, mit der wir die Knoten
des Graphen repr\"asentieren.
\begin{enumerate}
\item Die Klasse \texttt{Node} implementiert die Schnittstelle \texttt{Comparable},
      damit wir sp\"ater Knoten als Schl\"ussel einer \texttt{TreeMap} verwenden k\"onnen.
      Dies ist bei der Funktion $\textsl{dist}()$ erforderlich.
\item Ein Knoten hat einen eindeutigen Namen, der in der Member-Variablen \texttt{mName}
      abgespeichert wird.  Dieser Name ist beim Einlesen eines Graphen n\"utzlich.
\item Weiterhin verwaltet ein Knoten eine Liste von Kanten in der Member-Variablen
      \texttt{mEdges}.  Diese Liste repr\"asentiert den Funktionswert $\textsl{edges}(\mathtt{this})$.
\item Die Methode $\textsl{compareTo}()$ vergleicht Knoten anhand ihres Namens.
\end{enumerate}

\begin{figure}[!ht]
\centering
\begin{Verbatim}[ frame         = lines, 
                  framesep      = 0.3cm, 
                  labelposition = bottomline,
                  numbers       = left,
                  numbersep     = -0.2cm,
                  xleftmargin   = 0.8cm,
                  xrightmargin  = 0.8cm,
                ]
    class Edge {
        private Node    mSource;
        private Node    mTarget;
        private Integer mLength;
        
        public Edge(Node source, Node target, Integer length) {
            mSource = source;
            mTarget = target;
            mLength = length;
        }   
        public Node    getSource() { return mSource; }
        public Node    getTarget() { return mTarget; }
        public Integer getLength() { return mLength; }
        public String  toString () {
            return "<" + mSource + ", " + mTarget + ">: " + mLength;
        }
    }
\end{Verbatim}
\vspace*{-0.3cm}
\caption{Die Klasse \texttt{Edge}.}
\label{fig:Graph/Edge.java}
\end{figure}

\noindent
Die Klasse \texttt{Edge} repr\"asentiert eine Kante $\pair(x,y)$ in unserem Graphen.
\begin{enumerate}
\item Die Variable \texttt{mSource} entspricht dem Start-Knoten $x$ der Kante $\pair(x,y)$.
\item Die Variable \texttt{mTarget} entspricht dem Ziel-Knoten $y$ der Kante $\pair(x,y)$.
\item Die Variable \texttt{mLength} gibt die L\"ange der Kante $\pair(x,y)$ an.
\end{enumerate}

\begin{figure}[!ht]
  \centering
\begin{Verbatim}[ frame         = lines, 
                  framesep      = 0.3cm, 
                  labelposition = bottomline,
                  numbers       = left,
                  numbersep     = -0.2cm,
                  commandchars  = \\\{\}, 
                  xleftmargin   = 0.0cm,
                  xrightmargin  = 0.0cm
                ]
    public class Dijkstra \{
        \(\vdots\)
        public Map<Node, Integer> shortestPath(Node source) 
        \{
            Map<Node, Integer> dist = new TreeMap<Node, Integer>();
            dist.put(source, 0);
            HeapTree<Integer, Node> fringe = new HeapTree<Integer, Node>();
            fringe.insert(0, source);
            while (!fringe.isEmpty()) \{
                Pair<Integer, Node> p     = fringe.top();
                Integer             distU = p.getFirst();
                Node                u     = p.getSecond();
                fringe.remove();
                for (Edge edge: u.getEdges()) \{
                    Node v = edge.getTarget();
                    if (dist.get(v) == null) \{
                        Integer d = distU + edge.getLength();
                        dist.put(v, d);
                        fringe.insert(d, v);
                    \} else \{
                        Integer oldDist = dist.get(v);
                        Integer newDist = dist.get(u) + edge.getLength();
                        if (newDist < oldDist) \{
                            dist.put(v, newDist);
                            fringe.change(newDist, v);
                        \}
                    \}
                \}
            \}
            return dist;
        \}
    \}
\end{Verbatim}
\vspace*{-0.3cm}
  \caption{Dijkstra's Algorithmus zur L\"osung des k\"urzeste-Wege-Problems.}
  \label{fig:dijkstra}
\end{figure}

\noindent 
Abbildung \ref{fig:dijkstra} auf Seite \pageref{fig:dijkstra} zeigt eine
Implementierung des von Dijkstra vorgeschlagenen Algorithmus in \textsl{Java}.
Die Methode $\textsl{shortestPath}()$ bekommt als Argument einen Knoten \texttt{source}.
Sie berechnet den Abstand aller anderen Knoten zu diesem Knoten.
\begin{enumerate}
\item In Zeile 5 und 6 initialsieren wir  die Funktion \textsl{dist} und
      implementieren die Zuweisung \\[0.2cm]
      \hspace*{1.3cm} $\textsl{dist}(\textsl{source}) \df 0$.
\item In Zeile 7 und 8 wird die Menge \texttt{fringe} initialisiert. 
      Diese Menge repr\"asentieren wir durch eine Priorit\"ats-Warteschlange,
      wobei wir nicht die von \textsl{Java} zur Verf\"ugung gestellte Klasse benutzen
      sondern die Klasse, die wir im Kapitel \ref{chap:prioqueue}
      entwickelt haben.  Dies ist erforderlich, weil die von \textsl{Java} zur Verf\"ugung
      gestellte Klasse \texttt{PriorityQueue} keine Methode $\textsl{change}()$ anbietet,
      mit der die Priorit\"at eines Elementes ge\"andert werden kann.

      Am Anfang enth\"alt die Priorit\"ats-Warteschlange \textsl{fringe} nur den Knoten \texttt{source}.
\item Die \texttt{while}-Schleife in Zeile 9 -- 29 implementiert die Scanning-Phase.
      Solange die Priorit\"ats-Warteschlange \textsl{fringe} nicht leer ist,
      holen wir den Knoten $u$ mit dem k\"urzesten Abstand zum Knoten \textsl{source}
      aus der Warteschlange heraus.
\item Die Relabeling-Phase wird durch die \texttt{for}-Schleife in Zeile 18 -- 27
      implementiert.  Hierbei iterieren wir \"uber alle Kanten $\pair(u,v)$, die
      von dem Knoten $u$ ausgehen.  Dann gibt es zwei F\"alle:
      \begin{enumerate}
      \item Falls die Funktion \textsl{dist} f\"ur den Knoten $v$ noch undefiniert
            ist, dann realisieren wir in Zeile 17 die Zuweisung \\[0.2cm]
            \hspace*{1.3cm} $\textsl{dist}(v) \df \textsl{dist}(u) + \weight{\pair(u,v)}$.
            \\[0.2cm]
            Gleichzeitig f\"ugen wir den Knoten $v$ in die Menge $\textsl{Fringe}$ ein.
      \item Andernfalls ist $\textsl{dist}(v)$ schon definiert.  Dann kommt es
            darauf an, ob der neu entdeckte Weg von \textsl{source} nach $v$
            \"uber $u$ k\"urzer ist als die L\"ange des bisher gefundenen Pfades.
            Falls dies so ist, \"andern wir die Funktion \textsl{dist}
            entsprechend ab.  Gleichzeitig m\"ussen wir die Priorit\"at des Knotens
            $v$ in der Warteschlange erh\"ohen.
      \end{enumerate}
\end{enumerate}

\subsection{Komplexit\"at}
Wenn ein Knoten $u$ aus der Warteschlange \textsl{Fringe} entfernt wird, ist er anschlie{\ss}end ein Element der
Menge \textsl{Visited} und aus der oben gezeigten Invariante folgt, dass dann 
\\[0.2cm]
\hspace*{1.3cm}
$\textsl{sp}(u) = \textsl{dist}(u)$
\\[0.2cm]
gilt.  Daraus folgt aber notwendigerweise, dass der Knoten $u$ nie wieder in die Menge \textsl{Fringe}
eingef\"ugt werden kann, denn ein Knoten $v$ wird nur dann in \textsl{Fringe} neu eingef\"ugt, wenn die Funktion
$\textsl{dist}(v)$ noch undefiniert ist.  Das Einf\"ugen eines Knoten in eine Priorit\"ats-Warteschlange mit $n$
Elementen kostet eine Rechenzeit, die durch $\Oh\bigl(\log_2(n)\bigr)$ abgesch\"atzt werden kann.  Da die
Warteschlange sicher nie mehr als $\#V$ knoten enthalten kann und da jeder Knoten maximal einmal eingef\"ugt
werden kann, liefert das einen Term der Form 
\\[0.2cm]
\hspace*{1.3cm}
$\Oh\bigl(\#V \cdot \log_2(\#V)\bigr)$ 
\\[0.2cm]
f\"ur das Einf\"ugen der Knoten.  Neben dem Aufruf von $\textsl{fringe}.\textsl{insert}(d,v)$ m\"ussen
wir auch die Komplexit\"at des Aufrufs $\textsl{fringe}.\textsl{change}(\mathtt{newDist}, v)$ analysieren.
Die Anzahl dieser Aufrufe ist durch die Anzahl der Kanten begrenzt, die zu dem Knoten $v$ hinf\"ugen.
Da ein Aufruf von $q.\textsl{change}()$ f\"ur eine Priorit\"ats-Warteschlange $q$ mit $n$ Elementen Rechenzeit
in der H\"ohe von $\Oh\bigl(\log_2(n)\bigr)$ erfordert, haben wir also insgesamt f\"ur den Aufruf von
$\textsl{change}()$ die Absch\"atzung
\\[0.2cm]
\hspace*{1.3cm}
$\Oh\bigl(\#E \cdot \log_2(\#V)\bigr)$
\\[0.2cm]
Dabei bezeichnet $\#E$ die Anzahl der Kanten. Damit erhalten wir f\"ur die
Komplexit\"at von Dijkstra's Algorithmus den Ausdruck \\[0.2cm]
\hspace*{1.3cm} $\Oh\bigl((\#\edges + \#\nodes) * \ln(\#\nodes)\bigr)$. \\[0.2cm]
Ist die Zahl der Kanten, die von den Knoten ausgehen k\"onnen, durch eine feste Zahl begrenzt
(z.B. wenn von jedem Knoten nur maximal 4 Kanten ausgehen), so
kann  die Gesamt-Zahl der Kanten durch ein festes Vielfaches der Knoten-Zahl abgesch\"atzt
werden.  Dann ist  die Komplexit\"at f\"ur Dijkstra's Algorithmus zur  Bestimmung der k\"urzesten Wege
durch den Ausdruck  
\\[0.2cm]
\hspace*{1.3cm}
$\Oh\bigl(\#\nodes * \log_2(\#\nodes)\bigr)$ 
\\[0.2cm]
gegeben.




%%% Local Variables: 
%%% mode: latex
%%% TeX-master: "algorithmen"
%%% End: 
