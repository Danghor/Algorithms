\chapter{Grenzen der Berechenbarkeit}
In jeder Disziplin der Wissenschaft wird die Frage gestellt, welche Grenzen die
verwendeten Methoden haben.   Wir wollen daher in diesem Kapitel beispielhaft ein Problem
untersuchen, bei dem die Informatik an ihre Grenzen st\"o{\ss}t.  Es handelt sich um das Halte-Problem. 

\section{Das Halte-Problem}
Das Halte-Problem ist die Frage, ob eine gegebene Funktion f\"ur eine bestimmte Eingabe
terminiert.  Wir werden zeigen, dass dieses Problem nicht durch ein Programm gel\"ost werden
kann.  Dazu f\"uhren wir folgende Definition ein.

\begin{Definition}[Test-Funktion] 
{\em Ein String $t$ ist eine \emph{Test-Funktion} mit Namen $n$ wenn $t$ 
 die Form \\[0.3cm]
\hspace*{1.3cm} {\tt $n$ := procedure(x) \{ $\cdots$ \}} \\[0.3cm]
hat, und sich als Definition einer \textsc{SetlX}-Funktion parsen l\"asst.  Die Menge der
Test-Funktionen bezeichnen wir mit $T\!F$.  Ist $t \in T\!F$ und hat den Namen $n$, so
schreiben wir $\mathtt{name}(t) = n$.} \hspace*{\fill} $\Box$
\end{Definition}

\noindent
\textbf{Beispiele}:  
\begin{enumerate}
\item $s_1$ = ``{\tt simple := procedure(x) \{ return 0; \}}''

      $s_1$ ist eine (sehr einfache) Test-Funktion mit dem Namen \texttt{simple}.
\item $s_2$ = ``{\tt loop := procedure(x) \{ while (true) \{ x := x + 1; \} \}}''

      $s_2$ ist eine Test-Funktion mit dem Namen \texttt{loop}. 
\item $s_3$ = ``{\tt hugo := procedure(x) \{ return ++x; \} }''

      $s_3$ ist keine Test-Funktion, denn da \textsc{SetlX} den Pr\"afix-Operator
      ``\texttt{++}'' nicht unterst\"utzt, l\"asst sich der String $s_3$  nicht fehlerfrei parsen.
\end{enumerate}
Um das Halte-Problem \"ubersichtlicher formulieren zu k\"onnen, f\"uhren wir noch drei
zus\"atzliche Notationen ein.
\begin{Notation}[$\leadsto$, $\downarrow$, $\uparrow$]
{\em
Ist $n$ der Name einer \texttt{C}-Funktion und sind $a_1$, $\cdots$, $a_k$ Argumente, die
vom Typ her der Deklaration von $n$ entsprechen, so schreiben wir \\[0.3cm]
\hspace*{1.3cm} $n(a_1, \cdots, a_k) \leadsto r$ \\[0.3cm]
wenn der Aufruf $n(a_1, \cdots, a_k)$ das Ergebnis $r$ liefert.  Sind wir an dem Ergebnis
selbst nicht interessiert, sondern wollen nur angeben, dass ein Ergebnis existiert, so
schreiben wir \\[0.3cm]
\hspace*{1.3cm} $n(a_1, \cdots, a_k) \,\downarrow$ \\[0.3cm]
und sagen, dass der Aufruf $n(a_1, \cdots, a_k)$ \emph{terminiert}.
Terminiert der Aufruf $n(a_1, \cdots, a_k)$ nicht, so schreiben wir \\[0.3cm]
\hspace*{1.3cm} $n(a_1, \cdots, a_k) \,\uparrow$ \\[0.3cm]
und sagen, dass der Aufruf $n(a_1, \cdots, a_k)$ \emph{divergiert}.
\hspace*{\fill} $\Box$
}
\end{Notation}

\noindent
\textbf{Beispiele}: Legen wir die Funktions-Definitionen zugrunde, die wir im Anschluss an
die Definition des Begriffs der Test-Funktion gegeben haben, so gilt:
\begin{enumerate}
\item {\tt simple(\symbol{34}emil\symbol{34}) $\leadsto 0$}
\item {\tt simple(\symbol{34}emil\symbol{34}) $\downarrow$}
\item {\tt loop(\symbol{34}hugo\symbol{34}) $\uparrow$}
\end{enumerate}

\noindent
Das \emph{Halte-Problem} f\"ur \textsc{SetlX}-Funktionen ist die Frage, ob es eine
\texttt{SetlX}-Funktion \\[0.1cm] 
\hspace*{1.3cm} 
$\texttt{stops := procedure}(t,\;a)\; \{\;\cdots\;\}$ \\[0.1cm]
gibt, die als Eingabe eine Testfunktion $t$ und einen String $a$ erh\"alt und die folgende
Eigenschaft hat:
\begin{enumerate}
\item $t \not\in T\!F \quad\Leftrightarrow\quad \mathtt{stops}(t, a) \leadsto 2$.

      Der Aufruf \texttt{stops($t$, $a$)} liefert genau dann den Wert 2 zur\"uck, 
      wenn $t$ keine Test-Funktion ist.

\item $t \in T\!F \,\wedge\, \mathtt{name}(t) = n \,\wedge\, n(a)\downarrow \quad\Leftrightarrow\quad
       \mathtt{stops}(t, a) \leadsto 1$.

      Der Aufruf \texttt{stops($t$, $a$)} liefert genau dann den Wert 1 zur\"uck,
      wenn $t$ eine Test-Funktion mit Namen $n$ ist und der Aufruf $n(a)$ terminiert.

\item $t \in T\!F \,\wedge\, \mathtt{name}(t) = n \,\wedge\, n(a)\uparrow \quad\Leftrightarrow\quad
       \mathtt{stops}(t, a) \leadsto 0$.

      Der Aufruf \texttt{stops($t$, $a$)} liefert genau dann den Wert 0 zur\"uck,
      wenn $t$ eine Test-Funktion mit Namen $n$ ist und der Aufruf $n(a)$ \underline{nicht} terminiert.
\end{enumerate}
Falls eine \textsc{SetlX}-Funktion \texttt{stops} mit den obigen Eigenschaften existiert, dann
sagen wir, dass das Halte-Problem f\"ur \textsc{SetlX} entscheidbar ist.

\begin{Theorem}[Turing, 1936]
{\em
  Das Halte-Problem ist unentscheidbar.
} 
\end{Theorem}

\noindent
\textbf{Beweis}:  Zun\"achst eine Vorbemerkung.  Um die Unentscheidbarkeit des
Halte-Problems nachzuweisen m\"ussen wir zeigen, dass etwas, n\"amlich eine Funktion mit
gewissen Eigenschaften nicht existiert.  Wie kann so ein Beweis \"uberhaupt funktionieren?
Wie k\"onnen wir \"uberhaupt zeigen, dass irgendetwas nicht existiert?
Die einzige M\"oglichkeit zu zeigen, dass etwas nicht existiert ist indirekt:
Wir nehmen also an, dass eine Funktion \texttt{stops} existiert, die das Halte-Problem l\"ost.
Aus dieser Annahme werden wir einen Widerspruch ableiten.  Dieser Widerspruch zeigt
uns dann, dass eine Funktion \texttt{stops} mit den gew\"unschten Eigenschaften nicht
existieren kann.
Um zu einem Widerspruch zu kommen, definieren wir den String \textsl{turing} wie in Abbildung
\ref{fig:turing-string} gezeigt.

\begin{figure}[!h]
  \centering
\begin{Verbatim}[ frame         = lines, 
                  framesep      = 0.3cm, 
                  labelposition = bottomline,
                  numbers       = left,
                  numbersep     = -0.2cm,
                  xleftmargin   = 0.8cm,
                  xrightmargin  = 0.8cm,
                  commandchars  = \\\{\},
                ]
    \textsl{turing} := "alan := procedure(x) \{
                   result := stops(x, x);
                   if (result == 1) \{
                       while (true) \{
                           print("... looping ...");
                       \}
                   \}
                   return result;
               \};"
\end{Verbatim}
  \vspace*{-0.3cm}
  \caption{Die Definition des Strings \textsl{turing}.}
  \label{fig:turing-string}
\end{figure}

Mit dieser Definition ist klar, dass \textsl{turing} eine Test-Funktion mit dem Namen
``\texttt{alan}'' ist: \\[0.3cm]
\hspace*{1.3cm} $\textsl{turing} \in T\!F \;\wedge\; \mathtt{name}(\textsl{turing}) = \mathtt{alan}$. \\[0.3cm]
Damit sind wir in der Lage, den String \textsl{Turing} als Eingabe der Funktion \texttt{stops}
zu verwenden.  Wir betrachten nun den folgenden Aufruf: \\[0.3cm]
\hspace*{1.3cm} \texttt{stops(\textsl{turing}, \textsl{turing});} \\[0.3cm]
Offenbar ist \textsl{turing} eine Test-Funktion.  Daher k\"onnen nur zwei F\"alle auftreten:
\\[0.1cm]
\hspace*{1.3cm} 
$\mathtt{stops}(\textsl{turing}, \textsl{turing}) \leadsto 0 \quad \vee\quad
 \mathtt{stops}(\textsl{turing}, \textsl{turing}) \leadsto 1$. \\[0.1cm]
Diese beiden F\"alle analysieren wir nun im Detail:
\begin{enumerate}
\item $\mathtt{stops}(\textsl{turing}, \textsl{turing}) \leadsto 0$. 

      Nach der Spezifikation von \texttt{stops} bedeutet dies \\[0.1cm]
      \hspace*{1.3cm} $\mathtt{alan}(\textsl{turing}) \uparrow$ \\[0.1cm]
      Schauen wir nun, was wirklich beim Aufruf \texttt{alan(\textsl{turing})} passiert:
      In Zeile 2 erh\"alt die Variable \texttt{result} den Wert 0 zugewiesen.  In Zeile 3
      wird dann getestet, ob \texttt{result} den Wert 1 hat.  Dieser Test schl\"agt fehl.
      Daher wird der Block der \texttt{if}-Anweisung nicht ausgef\"uhrt und die Funktion liefert als
      n\"achstes in Zeile 8 den Wert 0 zur\"uck.  Insbesondere terminiert der Aufruf also, im
      Widerspruch zu dem, was die Funktion \texttt{stops} behauptet hat.

      Damit ist der erste Fall ausgeschlossen.
\item  $\mathtt{stops}(\textsl{turing}, \textsl{turing}) \leadsto 1$. 

      Aus der Spezifikation der Funktion \texttt{stops} folgt, dass der Aufruf
      $\mathtt{alan}(\textsl{turing})$ terminiert: \\[0.1cm]
      \hspace*{1.3cm} $\mathtt{alan}(\textsl{turing}) \downarrow$ \\[0.1cm]
      Schauen wir nun, was wirklich beim Aufruf \texttt{alan(\textsl{turing})} passiert:
      In Zeile 2 erh\"alt die Variable \texttt{result} den Wert 1 zugewiesen.  In Zeile 3
      wird dann getestet, ob \texttt{result} den Wert 1 hat.  Diesmal gelingt der Test.
      Daher wird der Block der \texttt{if}-Anweisung ausgef\"uhrt.  Dieser Block
      besteht aber nur aus einer Endlos-Schleife, aus der wir nie wieder zur\"uck kommen.
      Das steht im Widerspruch zu dem, was die Funktion \texttt{stops} behauptet hat.

      Damit ist der zweite Fall ausgeschlossen.
\end{enumerate}
Insgesamt haben wir also in jedem Fall einen Widerspruch erhalten.  
Also ist die Annahme, dass die \textsc{SetlX}-Funktion \texttt{stops}
das Halte-Problem l\"ost, falsch.  Insgesamt haben wir gezeigt, dass es keine \textsc{SetlX}-Funktion
geben kann, die das Halte-Problem l\"ost. \hspace*{\fill} $\Box$
\vspace*{0.3cm}

\noindent
\textbf{Bemerkung}:
Der Nachweis, dass das Halte-Problem unl\"osbar ist, wurde 1936 von Alan Turing (1912 -- 1954)
\cite{turing:36} erbracht.  Turing hat das Problem damals nat\"urlich nicht f\"ur die Sprache
\textsc{SetlX} gel\"ost, sondern f\"ur die heute nach ihm benannten \emph{Turing-Maschinen}.  
Eine Turing-Maschine ist abstrakt gesehen nichts anderes als eine Beschreibung eines
Algorithmus.  Turing hat also gezeigt, dass es keinen Algorithmus gibt, der entscheiden
kann, ob ein gegebener anderer Algorithmus terminiert.
\vspace*{0.3cm}

\noindent
\textbf{Bemerkung}:
An dieser Stelle k\"onnen wir uns fragen, ob es vielleicht eine andere Programmier-Sprache
gibt, in der wir das Halte-Problem dann vielleicht doch l\"osen k\"onnten.  
Wenn es in dieses Programmier-Sprache Unterprogramme gibt, und wenn wir dort
Programm-Texte als Argumente von Funktionen \"ubergeben k\"onnen, dann ist leicht zu sehen,
dass der obige Beweis der 
Unl\"osbarkeit des Halte-Problems sich durch geeignete syntaktische Modifikationen auch auf
die andere Programmier-Sprache \"ubertragen l\"asst.
\pagebreak

\exercise
Wir nennen eine Menge $X$ \emph{abz\"ahlbar}, wenn  es eine Funktion \\[0.1cm]
\hspace*{1.3cm} $f: \mathbb{N} \rightarrow X$ \\[0.1cm]
gibt, so dass es f\"ur alle $x\in X$ ein $n \in \mathbb{N}$ gibt, so dass $x$ das Bild von
$n$ unter $f$ ist: \\[0.1cm]
\hspace*{1.3cm} $\forall x \in X: \exists n \in \mathbb{N}: x = f(n)$.
\\[0.1cm]
Zeigen Sie, dass die Potenz-Menge $2^\mathbb{N}$ der nat\"urlichen Zahlen $\mathbb{N}$ 
nicht abz\"ahlbar ist.  
\vspace*{0.2cm}

\noindent
\textbf{Hinweis}: Gehen Sie \"ahnlich vor wie beim Beweis der Unl\"osbarkeit des
Halte-Problems.  Nehmen Sie an, es g\"abe eine Funktion $f$, die die Teilmengen von
$\mathbb{N}$ aufz\"ahlt: \\[0.1cm]
\hspace*{1.3cm}  $\forall x \in 2^\mathbb{N}: \exists n \in \mathbb{N}: x = f(n)$.
\\[0.1cm]
Definieren Sie eine Menge \texttt{Cantor} wie folgt:
\\[0.1cm]
\hspace*{1.3cm} $\mathtt{Cantor} := \bigl\{ n \in \mathbb{N} \mid n \notin f(n) \bigr\}$.
\\[0.1cm]
Versuchen Sie nun, einen Widerspruch herzuleiten.


\section{Unl\"osbarkeit des \"Aquivalenz-Problems}
Es gibt noch eine ganze Reihe anderer Funktionen, die nicht berechenbar sind.  In der
Regel werden wir den Nachweis, dass eine bestimmt Funktion nicht berechenbar ist, dadurch f\"uhren, dass
wir zun\"achst annehmen, dass die gesuchte Funktion doch implementierbar ist.  Unter dieser Annahme
konstruieren wir dann eine Funktion, die das Halte-Problem l\"ost, was im Widerspruch zu dem am Anfang dieses Abschnitts
bewiesen Sachverhalts steht.
Dieser Widerspruch zwingt uns zu der Folgerung, dass die gesuchte Funktion nicht implementierbar ist.
Wir werden dieses Verfahren an einem Beispiel demonstrieren. Vorweg ben\"otigen wir aber
noch eine Definition.

\begin{Definition}[$\simeq$] 
{\em 
Es seien $n_1$ und $n_2$ Namen zweier \textsc{SetlX}-Funktionen und
  $a_1$, $\cdots$, $a_k$ seien Argumente, mit denen wir diese Funktionen f\"uttern k\"onnen. Wir definieren \\[0.1cm]
\hspace*{1.3cm} $n_1(a_1,\cdots,a_k) \simeq n_2(a_1,\cdots,a_k)$ \\[0.1cm]
g.d.w. einer der beiden folgen F\"alle auftritt:
\begin{enumerate}
\item $n_1(a_1,\cdots,a_k)\uparrow \quad\wedge\quad n_2(a_1,\cdots,a_k)\uparrow$
\item $\exists r: \Bigl(n_1(a_1,\cdots,a_k) \leadsto r \quad\wedge\quad n_2(a_1,\cdots,a_k) \leadsto r\Bigr)$

      In diesem Fall sagen wir, dass die beiden Funktions-Aufrufe 
      $n_1(a_1,\cdots,a_k) \simeq n_2(a_1,\cdots,a_k)$ \emph{partiell \"aquivalent} sind.
      \hspace*{\fill} $\Box$
\end{enumerate}}
\end{Definition}

\noindent
Wir kommen jetzt zum \emph{\"Aquivalenz-Problem}.  Die Funktion $\textsl{equal}$, die die Form
\\[0.2cm]
\hspace*{1.3cm}
$\texttt{equal := procedure(p1, p2, a) \{ ... \}}$
\\[0.2cm]
hat, m\"oge folgender Spezifikation gen\"ugen:
\begin{enumerate}
\item $p_1 \not\in T\!F \;\vee\; p_2 \not\in T\!F \quad\Leftrightarrow\quad \mathtt{equal}(p_1, p_2, a) \leadsto 2$.
\item Falls 
  \begin{enumerate}
  \item $p_1 \in T\!F \;\wedge\; \mathtt{name}(p_1) = n_1$,
  \item $p_2 \in T\!F \;\wedge\; \mathtt{name}(p_2) = n_2$ \quad und
  \item $n_1(a) \simeq n_2(a)$
  \end{enumerate}
    gilt, dann muss gelten: \\[0.1cm]
   \hspace*{1.3cm}  $\mathtt{equal}(p_1, p_2, a) \leadsto 1$.
\item Ansonsten gilt \\[0.1cm]
      \hspace*{1.3cm} $\mathtt{equal}(p_1, p_2, a) \leadsto 0$.
\end{enumerate}
Wir sagen, dass eine Funktion, die der eben angegebenen Spezifikation gen\"ugt, das
\emph{\"Aquivalenz-Problem} l\"ost.

\begin{Theorem}[Rice, 1953]
Das \"Aquivalenz-Problem ist unl\"osbar.  
\end{Theorem}

\noindent
\textbf{Beweis}:
Wir f\"uhren den Beweis indirekt und nehmen
an, dass es doch eine Implementierung der Funktion \texttt{equal} gibt, die das
\"Aquivalenz-Problem l\"ost.  Wir betrachten die in Abbildung 
\ref{fig:stops} angegeben Implementierung der Funktion \texttt{stops}.


\begin{figure}[!h]
  \centering
\begin{Verbatim}[ frame         = lines, 
                  framesep      = 0.3cm, 
                  labelposition = bottomline,
                  numbers       = left,
                  numbersep     = -0.2cm,
                  xleftmargin   = 0.3cm,
                  xrightmargin  = 0.3cm
                ]
     stops := procedure(p, a) {
         f := "loop := procedure(x) {           \n"
            + "    while (true) { x := x + x; } \n"
            + "    return 0;                    \n"
            + "};                               \n";                 
         e := equal(f, p, a);
         if (e == 2) {
             return 2;
         } else {
             return 1 - e;
         }
     }
\end{Verbatim}
  \vspace*{-0.3cm}
  \caption{Eine Implementierung der Funktion \texttt{stops}.}
  \label{fig:stops}
\end{figure}

Zu beachten ist, dass in Zeile 2 die Funktion \texttt{equal} mit einem String aufgerufen
wird, der eine Test-Funktion ist, und zwar mit dem Namen \texttt{loop}.  Diese 
Test-Funktion hat die
folgende Form:
\begin{verbatim}
      loop := procedure(x) { while (1) { x := x + x; } };
\end{verbatim}
Es ist offensichtlich, dass die Funktion \texttt{loop} f\"ur kein Ergebnis terminiert.
Ist also das Argument $p$ eine Test-Funktion mit Namen $n$, so liefert die Funktion
\texttt{equal} immer dann den Wert 1, 
wenn $n(a)$ nicht terminiert, andernfalls muss sie den Wert 0 zur\"uck geben.
Damit liefert die Funktion \texttt{stops} aber f\"ur eine Test-Funktion $p$ mit Namen $n$
und ein Argument 
$a$ genau dann 1, wenn der Aufruf $n(a)$ terminiert und w\"urde folglich das Halte-Problem
l\"osen.  Das kann nicht sein, also kann es keine Funktion  \texttt{equal}
geben, die das \"Aquivalenz-Problem l\"ost. 
\qed
\vspace*{0.3cm}

\noindent
Die Unl\"osbarkeit des \"Aquivalenz-Problems und vieler weiterer praktisch interessanter
Problem folgen aus einem 1953 von Henry G.~Rice \cite{rice:53} bewiesenen Satz.


%%% Local Variables: 
%%% mode: latex
%%% TeX-master: "algorithmen"
%%% End: 
