\chapter{Die $\mathcal{O}$-Notation} 
In diesem Kapitel  stellen wir
die $O$-Notation vor.  Diese beiden Begriffe ben\"otigen wir, um die Laufzeit von
Algorithmen analysieren zu k\"onnen.  Die Algorithmen selber stehen in diesem Kapitel noch
im Hintergrund.

\section{Motivation}
Wollen wir die Komplexit\"at eines Algorithmus absch\"atzen, so w\"are ein m\"ogliches Vorgehen
wie folgt: Wir kodieren den Algorithmus in einer Programmiersprache und berechnen,
wieviele Additionen, Multiplikationen, Zuweisungungen, und andere elementare Operationen
bei einer gegebenen Eingabe von dem Programm ausgef\"uhrt werden. Anschlie{\ss}end schlagen wir
im Prozessor-Handbuch nach, wieviel Zeit die einzelnen Operationen in Anspruch nehmen und
errechnen daraus die Gesamtlaufzeit des Programms.\footnote{
Da die heute verf\"ugbaren Prozessoren fast alle mit \emph{Pipelining} arbeiten, werden oft
mehrere Befehle gleichzeitig abgearbeitet. Da gleichzeitig auch das Verhalten des Caches
eine wichtige Rolle spielt, ist die genaue Berechnung der Rechenzeit faktisch unm\"oglich.}
Dieses Vorgehen ist aber in zweifacher Hinsicht problematisch:
\begin{enumerate}
\item Das Verfahren ist sehr kompliziert.
\item W\"urden wir den selben Algorithmus anschlie{\ss}end in einer anderen Programmier-Sprache
      kodieren, oder aber das Programm auf einem anderen Rechner laufen lassen, so w\"are
      unsere Rechnung wertlos und wir m\"ussten sie wiederholen.
\end{enumerate}
Der letzte Punkt zeigt, dass das Verfahren dem Begriff des Algorithmus, der ja eine
Abstraktion des Programm-Begriffs ist, nicht gerecht wird.  Ähnlich wie der Begriff des
Algorithmus von bestimmten Details einer Implementierung abstrahiert brauchen wir zur
Erfassung der rechenzeitlichen Komplexit\"at eines Algorithmus einen Begriff, der von
bestimmten Details der Funktion, die die Rechenzeit f\"ur ein gegebenes Programm berechnet,
abstrahiert.  Wir haben drei Forderungen an den zu findenden  Begriff.
\begin{itemize}
\item Der Begriff soll von konstanten Faktoren abstrahieren.
\item Der Begriff soll von \emph{unwesentlichen Termen} abstrahieren.

      Nehmen wir an, wir h\"atten ein Programm, dass zwei $n \times n$ Matrizen
      multipliziert und wir h\"atten f\"ur die Rechenzeit $T(n)$ dieses Programms in Abh\"angigkeit von
      $n$ die Funktion \\[0.1cm]
      \hspace*{1.3cm} $T(n) = 3 \cdot n^3 + 2 \cdot n^2 + 7$ \\[0.1cm]
      gefunden.  Dann nimmt der \emph{proportionale Anteil} des Terms $2\cdot n^2 + 7$ an der
      gesamten Rechenzeit mit wachsendem $n$ immer mehr ab.  Zur Verdeutlichung haben wir
      in einer Tabelle die Werte des proportionalen Anteils f\"ur 
      $n = 1,\; 10,\; 100,\; 1000,\, 10\,000$ aufgelistet: \\[0.3cm]
      \hspace*{1.3cm} 
      \begin{tabular}{|r|r|}
        \hline
        $n$  & \rule{0pt}{16pt} $\bruch{2 \cdot n^2 + 7}{3 \cdot n^3 + 2 \cdot n^2 + 7}$ \\[0.3cm]
        \hline
        \hline
        1       &  0.75000000000000  \\
        10      &  0.06454630495800  \\
        100     &  0.00662481908150  \\
        1000    &  0.00066622484855  \\
        10\,000 &  6.6662224852\,e\,-05  \\
       \hline
      \end{tabular}
\item Der Begriff soll das \emph{Wachstum} der Rechenzeit abh\"angig von \emph{Wachstum} der
      Eingaben erfassen. Welchen genauen Wert die Rechenzeit f\"ur kleine Werte der Eingaben
      hat, spielt nur eine untergeordnete Rolle, denn f\"ur kleine Werte der Eingaben 
      wird auch die Rechenzeit nur klein sein.
\end{itemize}
Wir bezeichnen die Menge der positiven reellen Zahlen mit $\R_+$ \\[0.1cm]
\hspace*{1.3cm} $\R_+ := \{ x \in \R \mid x > 0 \}$. \\[0.1cm]
Wir bezeichnen die Menge aller Funktionen von $\N$ nach
 $\R_+$ mit $\R_+^{\;\N}$, es gilt also: \\[0.1cm]
\hspace*{1.3cm} 
$\R_+^{\;\N} = \bigl\{ f \mid \mbox{$f$ ist Funktion der Form $f: \N \rightarrow \R_+$} \}$.

\begin{Definition}[$\Oh(f)$] {\em
  Es sei eine Funktion $f\in \R_+^\N$ 
  gegeben.   Dann definieren wir die Menge der Funktionen, die asymptotisch
  das gleiche Wachstumsverhalten haben wie die Funktion $f$, wie folgt:
  \\[0.1cm]
  \hspace*{1.3cm} 
  $ \Oh(f) \;:=\; \left\{ g \in \R_+^{\;\N} \mid \exists k \in \N \colon 
    \bigl(\exists c \in \R\colon \forall n \in \N \colon n \geq k \rightarrow g(n) \leq c \cdot f(n)\bigr) \right\}$.
  \hspace*{\fill} $\Box$
}
\end{Definition}
Was sagt die obige Definition aus? Zun\"achst kommt es auf kleine Werte des Arguments $n$
nicht an, denn die obige Formel sagt ja, dass $g(n) \leq c \cdot f(n)$ nur f\"ur die $n$ gelten
muss, f\"ur die  $n \geq k$ ist.  Au{\ss}erdem kommt es auf Proportionalit\"ats-Konstanten nicht
an, denn $g(n)$ muss ja nur kleinergleich $c \cdot f(n)$ sein und die Konstante $c$ k\"onnen wir
beliebig w\"ahlen.  Um den Begriff zu verdeutlichen, geben wir einige Beispiele.
\vspace*{0.3cm}

\noindent
\textbf{Beispiel}: Es gilt \\[0.1cm]
\hspace*{1.3cm} $3 \cdot n^3 + 2 \cdot n^2 + 7 \in \Oh(n^3)$. \\[0.1cm]
\textbf{Beweis}: Wir m\"ussen eine Konstante $c$ und eine Konstante $k$ angeben, so dass f\"ur
alle $n\in \N$ mit $n \geq k$ die Ungleichung
\\[0.1cm]
\hspace*{1.3cm} 
$3 \cdot n^3 + 2 \cdot n^2 + 7 \leq c \cdot n^3$
\\[0.1cm]
gilt.  Wir setzen  $k := 1$ und $c := 12$. Dann k\"onnen wir die Ungleichung 
\begin{equation}
  \label{eq:u1}
  1\leq n  
\end{equation}
voraussetzen und m\"ussen zeigen, dass daraus 
\begin{equation}
  \label{eq:u2}
  3 \cdot n^3 + 2 \cdot n^2 + 7 \leq 12 \cdot n^3  
\end{equation}
folgt. Erheben wir beide Seiten der  Ungleichung (\ref{eq:u1}) in die dritte Potenz, so sehen wir,
dass 
\begin{equation}
  \label{eq:u3pre}
  1 \leq n^3  
\end{equation}
gilt.  Diese Ungleichung multiplizieren wir auf beiden Seiten mit $7$ und erhalten: 
\begin{equation}
  \label{eq:u3}
  7 \leq 7 \cdot n^3
\end{equation}
Multiplizieren wir die Ungleichung (\ref{eq:u1}) mit $2\cdot n^2$, so erhalten wir 
\begin{equation}
  \label{eq:u4}
  2 \cdot n^2 \leq 2 \cdot n^3  
\end{equation}
Schlie{\ss}lich gilt trivialerweise 
\begin{equation}
  \label{eq:u5}
  3 \cdot n^3 \leq 3 \cdot n^3
\end{equation}
Die Addition der Ungleichungen (\ref{eq:u3}), (\ref{eq:u4}) und (\ref{eq:u5}) liefert nun \\[0.1cm]
\hspace*{1.3cm} $3 \cdot n^3 + 2 \cdot n^2 + 7 \leq 12 \cdot n^3$ \\[0.1cm]
und das war zu zeigen. \hspace*{\fill} $\Box$
\vspace*{0.3cm}

\noindent
\textbf{Beispiel}: Es gilt  $n \in \Oh(2^n)$. 
\vspace*{0.3cm}

\noindent
\textbf{Beweis}: Wir m\"ussen eine Konstante $c$ und eine Konstante $k$ angeben, so dass f\"ur
alle $n \geq k$
\\[0.1cm]
\hspace*{1.3cm} $ n \leq c \cdot 2^n$ \\[0.1cm]
gilt.  Wir setzen $k := 0$ und $c := 1$.  Wir zeigen \\[0.1cm]
\hspace*{1.3cm} $n \leq 2^n$ \quad f\"ur alle $n \in \N$ \\[0.1cm]
durch vollst\"andige Induktion \"uber $n$.
\begin{enumerate}
\item \textbf{I.A.}: $n = 0$

      Es gilt $0 \leq 1 = 2^0$.
\item \textbf{I.S.}: $n \mapsto n + 1$

      Einerseits gilt nach Induktions-Voraussetzung \\[0.1cm]
      \hspace*{1.3cm} $n \leq 2^n$, \hspace*{\fill} $(1)$ \\[0.1cm]
      andererseits haben wir \\[0.1cm]
      \hspace*{1.3cm} $1 \leq 2^n$. \hspace*{\fill} $(2)$ \\[0.1cm]
      Addieren wir $(1)$ und $(2)$, so erhalten wir \\[0.1cm]
      \hspace*{1.3cm} $n+1 \leq 2^n + 2^n = 2^{n+1}$. \hspace*{\fill} $\Box$

      \textbf{Bemerkung}: Die Ungleichung $1 \leq 2^n$ h\"atten wir eigentlich ebenfalls
      durch Induktion  nachweisen m\"ussen.
\end{enumerate}

\exercise
Zeigen Sie \\[0.1cm]
\hspace*{1.3cm} $n^2 \in \Oh(2^n)$.
\vspace*{0.3cm}

\noindent
Wir zeigen nun einige Eigenschaften der $\Oh$-Notation.

\begin{Satz}[Reflexivit\"at]
{\em
  F\"ur alle Funktionen $f\colon \N \rightarrow {\R_+}$ gilt \\[0.1cm]
  \hspace*{1.3cm} $f \in \Oh(f)$. 
}
\end{Satz}
\textbf{Beweis}: W\"ahlen wir $k:=0$ und $c:=1$, so folgt die Behauptung sofort aus der
Ungleichung \\[0.1cm]
\hspace*{1.3cm} $\forall n \in \N\colon f(n) \leq f(n)$. \hspace*{\fill} $\Box$

\begin{Satz}[Abgeschlossenheit unter Multiplikation mit Konstanten] \hspace*{\fill} \\
{\em
  Es seien  $f,g\colon \N \rightarrow {\R_+}$
   und $d \in \R_+$.  Dann gilt \\[0.1cm]
  \hspace*{1.3cm} $g \in \Oh(f) \Rightarrow d \cdot g \in \Oh(f)$.
}
\end{Satz}
\textbf{Beweis}: Aus $g \in \Oh(f)$ folgt, dass es Konstanten $c'\in \R_+$, $k'\in \N$ gibt,
so dass \\[0.1cm]
\hspace*{1.3cm} 
$\forall n \in \N \colon \bigl(n \geq k'  \rightarrow g(n) \leq c' \cdot f(n)\bigr)$ \\[0.1cm]
gilt.  Multiplizieren wir die Ungleichung mit $d$, so haben wir \\[0.1cm]
\hspace*{1.3cm} 
$\forall n \in \N \colon\bigl( n \geq k'  \rightarrow d \cdot g(n) \leq d \cdot c' \cdot f(n)\bigr)$ \\[0.1cm]
Setzen wir nun $k:=k'$ und $c := d \cdot c'$, so folgt \\[0.1cm]
\hspace*{1.3cm} $\forall n \in \N \colon \bigl(n \geq k  \rightarrow d \cdot g(n) \leq c \cdot f(n)\bigr)$ 
\\[0.1cm]
und daraus folgt $d \cdot g \in \Oh(f)$. \hspace*{\fill} $\Box$.

\begin{Satz}[Abgeschlossenheit unter Addition]
{\em
  Es seien $f,g,h \colon \N \rightarrow {\R_+}$.  Dann gilt \\[0.1cm]
  \hspace*{1.3cm} $f \in \Oh(h) \wedge g \in \Oh(h) \,\rightarrow\, f + g \in \Oh(h)$.
}
\end{Satz}
\textbf{Beweis}: Aus den Voraussetzungen $f \in \Oh(h)$ und $g \in \Oh(h)$ folgt, dass es
Konstanten $k_1,k_2\in \N$ und $c_1,c_2\in \R$ gibt, so dass \\[0.1cm]
\hspace*{1.3cm} 
$\forall n \in \N \colon \bigl( n \geq k_1 \rightarrow f(n) \leq c_1 \cdot h(n)\bigr)$ 
\quad und\\[0.1cm]
\hspace*{1.3cm} 
$\forall n \in \N \colon\bigl( n \geq k_2 \rightarrow g(n) \leq c_2 \cdot h(n)\bigr)$
\\[0.1cm]
gilt.  Wir setzen $k := \max(k_1,k_2)$ und $c:= c_1 + c_2$.  F\"ur $n \geq k$ gilt dann \\[0.1cm]
\hspace*{1.3cm} $f(n) \leq c_1 \cdot h(n)$ und $g(n) \leq c_2 \cdot h(n)$. \\[0.1cm]
Addieren wir diese beiden Gleichungen, dann haben wir f\"ur alle $n \geq k$ \\[0.1cm]
\hspace*{1.3cm} $f(n) + g(n) \leq (c_1 + c_2) \cdot h(n) = c \cdot h(n)$. \hspace*{\fill} $\Box$

\begin{Satz}[Transitivit\"at]
{\em
  Es seien $f,g,h \colon \N \rightarrow {\R_+}$.  Dann gilt \\[0.1cm]
  \hspace*{1.3cm} $f \in \Oh(g) \wedge g \in \Oh(h) \,\rightarrow\, f \in \Oh(h)$.
}
\end{Satz}
\textbf{Beweis}: Aus $f \in \Oh(g)$ folgt, dass es $k_1 \in \N$ und $c_1 \in \R$ gibt, so dass\\[0.1cm]
\hspace*{1.3cm} 
$\forall n \in \N \colon\bigl( n \geq k_1 \rightarrow f(n) \leq c_1 \cdot g(n)\bigr)$ \\[0.1cm]
gilt und aus $g \in \Oh(h)$ folgt, dass es $k_2 \in \N$ und $c_2 \in \R$ gibt, so dass \\[0.1cm]
\hspace*{1.3cm} 
$\forall n \in \N \colon\bigl( n \geq k_2 \rightarrow g(n) \leq c_2 \cdot h(n)\bigr)$ \\[0.1cm]
gilt.  Wir definieren $k:= \max(k_1,k_2)$ und $c := c_1 \cdot c_2$.  Dann haben wir f\"ur alle
$n \geq k$:\\[0.1cm]
\hspace*{1.3cm} $f(n) \leq c_1\cdot g(n)$ und $g(n) \leq c_2 \cdot h(n)$. \\[0.1cm]
Die zweite dieser Ungleichungen multiplizieren wir mit $c_1$ und erhalten \\[0.1cm]
\hspace*{1.3cm} $f(n) \leq c_1\cdot g(n)$ und $c_1\cdot g(n) \leq c_1\cdot c_2 \cdot h(n)$. \\[0.1cm]
Daraus folgt aber sofort $f(n) \leq c \cdot h(n)$. \hspace*{\fill} $\Box$

\begin{Satz}[Grenzwert-Satz] \label{limit}
{\em
  Es seien $f,g \colon \N \rightarrow {\R_+}$.  Au{\ss}erdem existiere der Grenzwert \\[0.1cm]
  \hspace*{1.3cm} $\lim\limits_{n \rightarrow \infty} \bruch{\,f(n)\,}{g(n)}$.  \\[0.1cm]
  Dann gilt $f \in \Oh(g)$. 
}
\end{Satz}
\textbf{Beweis}: Es sei \\[0.1cm]
\hspace*{1.3cm} $\lambda := \lim\limits_{n \rightarrow \infty} \bruch{f(n)}{g(n)}$.  \\[0.1cm]
Nach Definition des Grenzwertes gibt es dann eine Zahl $k \in \N$, so dass 
f\"ur alle $n\in \N$ mit $n \geq k$ die Ungleichung \\[0.1cm]
\hspace*{1.3cm} $\left| \bruch{f(n)}{g(n)} - \lambda \right| \leq 1$ \\[0.1cm]
gilt.  Multiplizieren wir diese Ungleichung mit $g(n)$, so erhalten wir \\[0.1cm]
\hspace*{1.3cm} $|f(n) - \lambda \cdot g(n)| \leq g(n)$. \\[0.1cm]
Daraus folgt wegen \\[0.1cm]
\hspace*{1.3cm} $f(n) \leq \bigl|f(n) - \lambda \cdot g(n)\bigr| + \lambda \cdot g(n)$ \\[0.1cm]
die Ungleichung \\[0.1cm]
\hspace*{1.3cm} $f(n) \leq g(n) + \lambda \cdot g(n) = (1 + \lambda) \cdot g(n)$. \\[0.1cm]
Definieren wir  $c := 1 +  \lambda$, 
so folgt f\"ur alle $n \geq k$ die Ungleichung $f(n) \leq c \cdot g(n)$. \hspace*{\fill} $\Box$
\vspace*{0.3cm}

\noindent
Wir zeigen die N\"utzlichkeit der obigen S\"atze anhand einiger Beispiele.
\vspace*{0.3cm}

\noindent
\textbf{Beispiel}: Es sei $k \in \N$.  Dann gilt\\[0.1cm]
\hspace*{1.3cm} $n^k \in \Oh(n^{k+1})$.
\vspace*{0.3cm}

\noindent
\textbf{Beweis}: Es gilt \\[0.1cm]
\hspace*{1.3cm} 
$\lim\limits_{n \rightarrow \infty} \bruch{n^{k}}{n^{k+1}} = \lim\limits_{n \rightarrow   \infty} \bruch{1}{n} = 0$.
\\[0.1cm]
Die Behauptung folgt nun aus dem Grenzwert-Satz. \hspace*{\fill} $\Box$
\vspace*{0.3cm}

\noindent
\textbf{Beispiel}: Es sei $k \in \N$ und $\lambda \in \R$ mit $\lambda > 1$.  Dann gilt\\[0.1cm]
\hspace*{1.3cm} $n^k \in \Oh(\lambda^n)$.
\vspace*{0.3cm}

\noindent
\textbf{Beweis}: Wir zeigen, dass 
\hspace*{1.3cm} 
\begin{equation}
  \label{eq:star}
  \lim\limits_{n \rightarrow \infty} \bruch{n^{k}}{\lambda^n} = 0  
\end{equation}
ist, denn dann folgt die Behauptung aus dem Grenzwert-Satz. 
Nach dem Satz von L'Hospital k\"onnen wir den Grenzwert wie folgt berechnen \\[0.1cm]
\hspace*{1.3cm} 
$\displaystyle \lim\limits_{n \rightarrow \infty} \bruch{n^{k}}{\lambda^n} =
\lim\limits_{x \rightarrow \infty} \bruch{x^{k}}{\lambda^x} =
\lim\limits_{x \rightarrow \infty} \bruch{\;\frac{d\,x^{k}}{dx}\;}{\frac{d\,\lambda^x}{dx}}$
\\[0.1cm]
Die Ableitungen  k\"onnen wir berechnen, es gilt: \\[0.1cm]
\hspace*{1.3cm}
 $\displaystyle \frac{d\,x^{k}}{dx} = k \cdot x^{k-1}$ \quad und \quad 
 $\displaystyle \frac{d\,\lambda^{x}}{dx} = \ln(\lambda) \cdot \lambda^x$. \\[0.1cm]
Berechnen wir die zweite Ableitung so sehen wir \\[0.1cm]
\hspace*{1.3cm}  
$\displaystyle \frac{d^{2}\,x^{k}}{dx^2} = k \cdot (k-1) \cdot x^{k-2}$ \quad und \quad 
 $\displaystyle \frac{d^2\,\lambda^{x}}{dx^2} = \ln(\lambda)^2 \cdot \lambda^x$. \\[0.1cm]
F\"ur die $k$-te Ableitung gilt analog \\[0.1cm]
\hspace*{1.3cm} 
$\displaystyle \frac{d^{k}\,x^{k}}{dx^k} = k \cdot (k-1) \cdot \cdots \cdot 1 \cdot x^{0} = k!$ \quad und \quad 
 $\displaystyle \frac{d^k\,\lambda^{x}}{dx^k} = \ln(\lambda)^k \cdot \lambda^x$. \\[0.1cm]
Wenden wir also den Satz von L'Hospital zur Berechnung des Grenzwertes  $k$ mal an, so
sehen wir \\[0.1cm]
\hspace*{1.3cm} 
$\displaystyle 
\lim\limits_{x \rightarrow \infty} \bruch{x^{k}}{\lambda^x} =
\lim\limits_{x \rightarrow \infty} \bruch{\rule[-10pt]{0pt}{10pt}\;\frac{d\,x^{k}}{dx}\;}{\rule{0pt}{12pt}\frac{d\,\lambda^x}{dx}} =
\lim\limits_{x \rightarrow \infty} \bruch{\rule[-10pt]{0pt}{10pt}\;\frac{d^2\,x^{k}}{dx^2}\;}{\rule{0pt}{12pt}\frac{d^2\,\lambda^x}{dx^2}} =
\cdots = 
\lim\limits_{x \rightarrow \infty} \bruch{\rule[-10pt]{0pt}{10pt}\;\frac{d^k\,x^{k}}{dx^k}\;}{\rule{0pt}{12pt}\frac{d^k\,\lambda^x}{dx^k}} =
\lim\limits_{x \rightarrow \infty} \bruch{k!}{\ln(\lambda)^k \lambda^x} = 0$.

\hspace*{\fill} $\Box$
\vspace*{0.3cm}

\noindent
\textbf{Beispiel}: Es gilt $\ln(n) \in \Oh(n)$.
\vspace*{0.3cm}

\noindent
\textbf{Beweis}: Wir benutzen Satz \ref{limit} und zeigen mit der Regel von L'Hospital,
dass \\[0.1cm]
\hspace*{1.3cm} 
$\lim\limits_{n \rightarrow \infty} \bruch{\;\ln(n)\;}{n} = 0$
\\[0.1cm]
ist.  Es gilt \\[0.1cm]
\hspace*{1.3cm} $\displaystyle \frac{d\, \ln(x)}{dx} = \frac{1}{x}$ 
\quad und \quad
 $\displaystyle \frac{d\, x}{dx} = 1$. \\[0.1cm]
Also haben wir \\[0.1cm]
\hspace*{1.3cm} 
$\displaystyle \lim\limits_{n \rightarrow \infty} \bruch{\;\ln(n)\;}{n} = 
\lim\limits_{x \rightarrow \infty} \bruch{\rule[-10pt]{0pt}{10pt}\;\frac{1}{x}\;}{1} = 
\lim\limits_{x \rightarrow \infty} \bruch{1}{x} = 0$. \hspace*{\fill} $\Box$
\vspace*{0.3cm}

\exercise
Zeigen Sie $\sqrt{n} \in \Oh(n)$.
\vspace*{0.3cm}

\noindent
\textbf{Beispiel}:  Es gilt $2^n \in \Oh(3^n)$, aber $3^n \notin \Oh(2^n)$.
\vspace*{0.3cm}

\noindent
\textbf{Beweis}:  Zun\"achst haben wir \\[0.1cm]
\hspace*{1.3cm} 
$\displaystyle\lim\limits_{n \rightarrow \infty} \bruch{2^n}{3^n} = 
 \lim\limits_{n \rightarrow \infty} \left(\bruch{2}{3}\right)^n = 0$.
\\[0.1cm]
Den Beweis, dass $3^n \notin \Oh(2^n)$ ist, f\"uhren wir indirekt und nehmen an, dass 
$3^n \in \Oh(2^n)$ ist.  Dann muss es Konstanten $c$ und $k$ geben, so dass f\"ur alle $n
\geq k$ gilt \\[0.1cm]
\hspace*{1.3cm} $3^n \leq c \cdot 2^n$. \\[0.1cm]
Wir logarithmieren beide Seiten dieser Ungleichung und finden \\[0.1cm]
\[
\begin{array}{llcl}
                & \ln(3^n) & \leq & \ln(c \cdot 2^n) \\[0.1cm]
\leftrightarrow\quad &  n \cdot \ln(3) & \leq & \ln(c) + n \cdot \ln(2) \\[0.1cm]
\leftrightarrow &  n \cdot \bigl(\ln(3) - \ln(2)\bigr) & \leq & \ln(c)  \\[0.1cm]
\leftrightarrow &  n  & \leq & \bruch{\ln(c)}{\ln(3) - \ln(2)}  \\[0.1cm]
\end{array}
\]
Die letzte Ungleichung m\"usste nun f\"ur beliebig gro{\ss}e nat\"urliche Zahlen $n$ gelten und liefert
damit den gesuchten Widerspruch zu unserer Annahme.

\exercise
\begin{enumerate}
\item Es sei $b \geq 1$. Zeigen Sie $\log_{b}(n) \in \Oh(\ln(n))$.
\item $3 \cdot n^2 + 5 \cdot n + \sqrt{n} \in \Oh(n^2)$
\item $7 \cdot n + \bigl(\log_2(n)\bigr)^2 \in \Oh(n)$
\item $\sqrt{n} + \log_2(n) \in \Oh\left(\sqrt{n}\right)$
\item $n^n \in \mathcal{O}\bigl(2^{2^n}\bigr)$.

      Hinweis:  Die letzte Teilaufgabe ist schwer!
\end{enumerate}

\section{Fallstudie: Effiziente Berechnung der Potenz}
Wir  verdeutlichen die bisher eingef\"uhrten Begriffe an einem Beispiel.  Wir betrachten ein
Programm zur Berechnung der Potenz $m^n$ f\"ur nat\"urliche Zahlen $m$ und $n$.
Abbildung \ref{fig:power-naive.stlx} zeigt ein naives Programm zur Berechnung von $m^n$.
Die diesem Programm zu Grunde liegende Idee ist es, die Berechnung von $m^n$ 
nach der Formel \\[0.1cm]
\hspace*{1.3cm} 
$m^n = \underbrace{m \cdot {\dots} \cdot m}_n$ \\[0.1cm]
durchzuf\"uhren.  

\begin{figure}[!h]
  \centering
\begin{Verbatim}[ frame         = lines, 
                  framesep      = 0.3cm, 
                  labelposition = bottomline,
                  numbers       = left,
                  numbersep     = -0.2cm,
                  xleftmargin   = 0.8cm,
                  xrightmargin  = 0.8cm
                ]
    power := procedure(m, n) {
        r := 1;
        for (i in {1 .. n}) {
            r := r * m;
        }
        return r;
    };
\end{Verbatim}
\vspace*{-0.3cm}
  \caption{Naive Berechnung von $m^n$ f\"ur $m,n \in \N$.}
  \label{fig:power-naive.stlx}
\end{figure} 

Das Programm ist  offenbar korrekt.
Zur Berechnung  von $m^n$ werden f\"ur positive Exponenten $n$ insgesamt $n-1$ Multiplikationen durchgef\"uhrt.
Wir k\"onnen $m^n$ aber wesentlich effizienter berechnen.  Die Grundidee erl\"autern wir an
der Berechnung von $m^4$.  Es gilt \\[0.1cm]
\hspace*{1.3cm} 
$m^4 = (m \cdot m) \cdot (m \cdot m)$.\\[0.1cm]
Wenn wir den Ausdruck $m\cdot m$ nur einmal berechnen, dann kommen wir bei der Berechnung von
$m^4$ nach der obigen Formel mit zwei Multiplikationen aus, w\"ahrend bei einem
naiven Vorgehen 3 Multiplikationen durchgef\"uhrt w\"urden! F\"ur die Berechnung von $m^8$
k\"onnen wir folgende Formel verwenden: \\[0.1cm]
\hspace*{1.3cm} 
$m^8 = \bigl( (m \cdot m) \cdot (m \cdot m) \bigr) \cdot \bigl( (m \cdot m) \cdot (m \cdot m) \bigr)$. \\[0.1cm]
Berechnen wir den Term $(m \cdot m) \cdot (m \cdot m)$ nur einmal, so werden jetzt 3 Multiplikationen
ben\"otigt um $m^8$ auszurechnen.  Ein naives Vorgehen w\"urde 7 Multiplikationen ben\"otigen.
Wir versuchen die oben an Beispielen erl\"auterte Idee in ein Programm umzusetzen.
Abbildung \ref{fig:power.stlx} zeigt das Ergebnis.  Es berechnet die Potenz $m^n$ nicht durch eine
naive $(n-1)$-malige Multiplikation sondern es verwendet das Paradigma \\[0.1cm]
\hspace*{1.3cm} \emph{Teile und Herrsche}. \quad (engl. \emph{divide and conquer})
\\[0.1cm]
Die Grundidee um den Term $m^n$ f\"ur $n \geq 1$ effizient zu berechnen,
l\"asst sich durch folgende Formel beschreiben: \\[0.1cm] 
\hspace*{1.3cm} 
$m^n = 
\left\{\begin{array}{ll}
m^{n/2} \cdot m^{n/2}      & \mbox{falls $n$ gerade ist};    \\
m^{n/2} \cdot m^{n/2} \cdot m  & \mbox{falls $n$ ungerade ist}.
\end{array}
\right.
$


\begin{figure}[!h]
  \centering
\begin{Verbatim}[ frame         = lines, 
                  framesep      = 0.3cm, 
                  labelposition = bottomline,
                  numbers       = left,
                  numbersep     = -0.2cm,
                  xleftmargin   = 0.8cm,
                  xrightmargin  = 0.8cm
                ]
    power := procedure(m, n) {
        if (n == 0) {
            return 1;
        }
        p := power(m, floor(n / 2));
        if (n % 2 == 0) {
            return p * p;
        } else {
            return p * p * m;
        }
    };
\end{Verbatim}
\vspace*{-0.3cm}
  \caption{Berechnung von $m^n$ f\"ur $m,n \in \N$.}
  \label{fig:power.stlx}
\end{figure} 

Da es keineswegs offensichtlich ist, dass das Programm in \ref{fig:power.stlx} 
tats\"achlich die Potenz $m^n$ berechnet,  wollen wir dies nachweisen.  Wir benutzen dazu
die Methode der \emph{Wertverlaufs-Induktion} (engl. \emph{computational induction}).
Die Wertverlaufs-Induktion ist eine Induktion \"uber die Anzahl der rekursiven Aufrufe.
Diese Methode bietet sich immer dann an, wenn die Korrektheit einer rekursiven Prozedur
nachzuweisen ist. Das Verfahren besteht aus zwei Schritten:
\begin{enumerate}
\item \emph{Induktions-Anfang}.

      Beim Induktions-Anfang weisen wir nach, dass die Prozedur in allen den F\"allen korrekt arbeitet,
      in denen sie sich nicht selbst aufruft.  
\item \emph{Induktions-Schritt}

      Im Induktions-Schritt beweisen wir, dass die Prozedur auch in den F\"allen korrekt
      arbeitet, in denen sie sich rekursiv aufruft.   Beim Beweis dieser Tatsache d\"urfen
      wir voraussetzen, dass die Prozedur bei jedem rekursiven Aufruf den korrekten Wert
      produziert. Diese Voraussetzung wird auch als \emph{Induktions-Voraussetzung} bezeichnet.
\end{enumerate}
Wir demonstrieren die Methode, indem wir durch Wertverlaufs-Induktion beweisen, dass 
gilt: 
\\[0.1cm]
\hspace*{1.3cm} $\mathtt{power}(m,n) \leadsto m^n$.
\begin{enumerate}
\item \textbf{Induktions-Anfang}.

      Die Methode ruft sich dann nicht rekursiv auf, wenn $n = 0$  gilt.  In diesem Fall
      haben wir \\[0.1cm]
      \hspace*{1.3cm} 
      $\mathtt{power}(m,0) \leadsto 1 =  m^0$.
\item \textbf{Induktions-Schritt}.

      Der rekursive Aufruf der Prozedur $\mathtt{power}$ hat die Form 
       $\mathtt{power}(m,n/2)$.  Also gilt nach Induktions-Voraussetzung \\[0.1cm]
       \hspace*{1.3cm} $\displaystyle \mathtt{power}(m,n/2) \leadsto m^{n/2}$. \\[0.1cm]
       Danach k\"onnen in der weiteren Rechnung zwei F\"alle auftreten.
       Wir f\"uhren daher eine Fallunterscheidung entsprechend der \texttt{if}-Abfrage in Zeile 6 durch:
      \begin{enumerate}
      \item $n \;\mathtt{\%}\; 2 = 0$, $n$ ist also gerade.

            Dann gibt es ein $k \in \N$ mit $n = 2 \cdot k$ und also ist $n/2 = k$.
            In diesem Fall gilt 
            \[ 
            \begin{array}{lcl}
            \mathtt{power}(m,n) & \leadsto & \mathtt{power}(m,k) \cdot \mathtt{power}(m,k) \\[0.1cm]
                                & \stackrel{I.V.}{\leadsto} & m^k \cdot m^k  \\[0.1cm]
                                & = & m^{2\cdot k} \\[0.1cm]
                                & = & m^{n}.
            \end{array}
            \]            
      \item $n \;\mathtt{\%}\; 2 = 1$, $n$ ist also ungerade.

            Dann gibt es ein $k \in \N$ mit $n = 2 \cdot k + 1$ und wieder ist $n/2 = k$.
            In diesem Fall gilt 
            \[ 
            \begin{array}{lcl}
            \mathtt{power}(m,n) & \leadsto & \mathtt{power}(m,k) \cdot \mathtt{power}(m,k) \cdot m  \\[0.1cm]
                                & \stackrel{I.V.}{\leadsto} & m^k \cdot m^k \cdot m  \\[0.1cm]
                                & = & m^{2\cdot k+1} \\[0.1cm]
                                & = & m^{n}.
            \end{array}
            \]
      \end{enumerate}
      Damit ist der Beweis der Korrektheit abgeschlossen. \hspace*{\fill} $\Box$
\end{enumerate}
Als n\"achstes wollen wir die Komplexit\"at des obigen Programms untersuchen. Dazu berechnen
wir zun\"achst die Anzahl der Multiplikationen, die beim Aufruf $\mathtt{power}(m,n)$
durchgef\"uhrt werden.  Je nach dem, ob der Test in Zeile 6 negativ ausgeht oder nicht, gibt
es mehr oder weniger Multiplikationen.  Wir untersuchen zun\"achst den schlechtesten Fall
(engl. \emph{worst case}).  Der schlechteste Fall tritt dann ein, wenn es
ein $l\in \N$ gibt, so dass \\[0.1cm]
\hspace*{1.3cm} $n = 2^l - 1$ \\[0.1cm]
ist, denn dann gilt \\[0.1cm]
\hspace*{1.3cm} $n/2 = 2^{l-1} - 1$ \quad und \quad $n \,\texttt{\symbol{37}}\, 2 = 1$, \\[0.1cm]
was wir sofort durch die Probe
\\[0.2cm]
\hspace*{1.3cm}
$2 \cdot(n/2) + n \,\texttt{\symbol{37}}\, 2 = 2 \cdot (2^{l-1} - 1) + 1 = 2^l - 1 = n$
\\[0.2cm]
verifizieren.  Folglich ist, wenn $n$ die Form $2^l - 1$ hat, bei jedem rekursiven Aufruf
der Exponent $n$ ungerade. 
Wir nehmen also $n = 2^l - 1$ an und berechnen die Zahl $a_n$ der Multiplikationen, die beim
Aufruf von $\mathtt{power}(m,n)$ durchgef\"uhrt werden. \\[0.1cm]
Zun\"achst gilt $a_0 = 0$, denn wenn $n =0$  ist, wird keine Multiplikation durchgef\"uhrt.
Ansonsten haben wir in Zeile 9 zwei Multiplikationen, die zu den Multiplikationen, die beim
rekursiven Aufruf in Zeile $5$ anfallen, hinzu addiert werden m\"ussen.  Damit erhalten wir
die folgende Rekurrenz-Gleichung: \\[0.1cm]
\hspace*{1.3cm} $a_n = a_{n/2} + 2$ \qquad f\"ur alle $n \in \left\{2^l - 1 \mid l \in \N\right\}$\quad mit $a_0 = 0$. \\[0.1cm]
Wir definieren $b_l := a_{2^l-1}$ und erhalten dann f\"ur die Folge $(b_l)_l$ die
Rekurrenz-Gleichung \\[0.1cm]
\hspace*{1.3cm} 
$b_l = a_{2^l-1} = a_{(2^l-1)/2} + 2 = a_{2^{l-1}-1} + 2 = b_{l-1} +2$ \qquad f\"ur alle $l\in\N$. \\[0.1cm]
Die Anfangs-Bedingung lautet $b_0 = a_{2^0-1} = a_0 = 0$.  
Offenbar lautet die L\"osung der Rekurrenz-Gleichung \\[0.1cm]
\hspace*{1.3cm} $b_l = 2 \cdot l$ \qquad f\"ur alle $l \in \N$. \\[0.1cm] 
Diese Behauptung k\"onnen Sie durch eine triviale Induktion verifizieren. 
F\"ur die Folge $a_n$ haben wir dann: \\[0.1cm]
\hspace*{1.3cm} $a_{2^l-1} = 2 \cdot l$. \\[0.1cm]
Formen wir die Gleichung $n = 2^l - 1$ nach $l$ um, so erhalten wir $l =
\log_2(n+1)$. Setzen wir diesen Wert ein, so sehen wir \\[0.1cm]
\hspace*{1.3cm} $a_n = 2 \cdot \log_2(n+1) \in \Oh\bigl(\log_2(n)\bigr)$.
\vspace*{0.3cm}

Wir betrachten jetzt den g\"unstigsten Fall, der bei der Berechnung von
$\mathtt{power}(m,n)$ auftreten kann. Der g\"unstigste Fall tritt dann ein, wenn 
der Test in Zeile 6 immer gelingt weil $n$ jedesmal eine gerade Zahl ist.  In diesem Fall muss
es ein $l\in \N$ geben, so dass $n$ die Form \\[0.1cm]
\hspace*{1.3cm} $n = 2^l$ \\[0.1cm]
hat. Wir nehmen also $n = 2^l$ an und berechnen die Zahl $a_n$ der Multiplikationen, die
dann beim
Aufruf von $\mathtt{power}(m,n)$ durchgef\"uhrt werden. \\[0.1cm]
Zun\"achst gilt $a_{2^0} = a_1 = 2$, denn wenn $n = 1$  ist, scheitert der Test in Zeile 6 und Zeile 9
liefert 2 Multiplikationen.  Zeile 5 liefert in diesem Fall keine Multiplikation, weil
beim Aufruf $\mathtt{power}(m,0)$ sofort das Ergebnis in Zeile 4 zur\"uck gegeben wird.

Ist $n = 2^l > 1$, so  haben wir in Zeile 7 eine Multiplikation, die zu den Multiplikationen, die beim
rekursiven Aufruf in Zeile $5$ anfallen, hinzu addiert werden muss.  Damit erhalten wir
die folgende Rekurrenz-Gleichung: \\[0.1cm]
\hspace*{1.3cm} $a_n = a_{n/2} + 1$ \qquad f\"ur alle $n \in \left\{2^l \mid l \in \N\right\}$\quad mit $a_1 = 2$. \\[0.1cm]
Wir definieren $b_l := a_{2^l}$ und erhalten dann f\"ur die Folge $(b_l)_l$ die
Rekurrenz-Gleichung \\[0.1cm]
\hspace*{1.3cm} 
$b_l = a_{2^l} = a_{(2^l)/2} + 1 = a_{2^{l-1}} + 1 = b_{l-1} + 1$ \qquad f\"ur alle $l\in\N$, \\[0.1cm]
mit der Anfangs-Bedingungen $b_0 = a_{2^0} = a_1 = 2$.
Also l\"osen wir die Rekurrenz-Gleichung \\[0.1cm]
\hspace*{1.3cm} $b_{l+1} = b_l + 1$ \qquad f\"ur alle $l \in \N$ \quad mit $b_0 = 2$.\\[0.1cm]
Offenbar lautet die L\"osung \\[0.1cm]
\hspace*{1.3cm} $b_l = 2 + l$ \qquad f\"ur alle $l\in\N$.
\\[0.1cm]
Setzen wir hier $b_l = a_{2^l}$, so erhalten wir: \\[0.1cm]
\hspace*{1.3cm} $a_{2^l} = 2 + l$. \\[0.1cm]
Formen wir die Gleichung $n = 2^l$ nach $l$ um, so erhalten wir $l =
\log_2(n)$. Setzen wir diesen Wert 
ein, so sehen wir \\[0.1cm]
\hspace*{1.3cm} $a_n = 2 + \log_2(n) \in \Oh\bigl(\log_2(n)\bigr)$.
\vspace*{0.3cm}

Da wir sowohl im besten als auch im schlechtesten Fall dasselbe Ergebnis bekommen haben,
k\"onnen wir schlie{\ss}en, dass f\"ur die Zahl $a_n$ der Multiplikationen allgemein gilt:\\[0.1cm]
\hspace*{1.3cm} $a_n \in \Oh\bigl(\log_2(n)\bigr)$.
\vspace*{0.3cm}

\noindent
\textbf{Bemerkung}:  Wenn wir nicht die Zahl der Multiplikationen sondern die Rechenzeit
ermitteln wollen, die der obige Algorithmus ben\"otigt, so wird die Rechnung wesentlich
aufwendiger.  Der Grund ist, dass wir dann ber\"ucksichtigen m\"ussen, dass die Rechenzeit bei
der Berechnung der Produkte in den Zeilen 7 und 9 von der Gr\"o{\ss}e der Faktoren abh\"angig ist.
\vspace*{0.3cm}

\exercise
Schreiben Sie eine Prozedur $\mathtt{prod}$ zur Multiplikation zweier Zahlen.
F\"ur zwei nat\"urliche Zahlen $m$ und $n$ soll der Aufruf $\mathtt{prod}(m, n)$  das Produkt
$m\cdot n$ mit Hilfe von Additionen
berechnen.  Benutzen Sie bei der Implementierung das Paradigma ``Teile und Herrsche'' und
beweisen Sie die Korrektheit des Algorithmus mit Hilfe einer Wertverlaufs-Induktion.
Sch\"atzen  Sie die Anzahl der Additionen, die beim Aufruf von $\mathtt{prod}(m,n)$
im schlechtesten Fall durchgef\"uhrt werden, mit Hilfe der $\Oh$-Notation ab. 
\pagebreak

\section{Der Hauptsatz der Laufzeit-Funktionen}
Im letzten Abschnitt haben wir zur Analyse der Rechenzeit der Funktion $\textsl{power}()$
zun\"achst eine Rekurrenz-Gleichung aufgestellt, diese gel\"ost und anschlie{\ss}end das Ergebnis
mit Hilfe der $\Oh$-Notation abgesch\"atzt.  Wenn wir nur an einer Absch\"atzung interessiert
sind, dann ist es in vielen F\"allen nicht notwendig, die zu Grunde liegende
Rekurrenz-Gleichung exakt zu l\"osen, denn der \textsl{Hauptsatz der Laufzeit-Funktionen}
(Englisch:~\textsl{Master Theorem}) \cite{cormen:01} bietet eine Methode zur Gewinnung von Absch\"atzungen,
bei der es nicht notwendig ist, die Rekurrenz-Gleichung zu l\"osen.  
Wir pr\"asentieren eine etwas vereinfachte Form dieses Hauptsatzes.

\begin{Theorem}[Hauptsatz der Laufzeit-Funktionen] 
  Es seien 
  \begin{enumerate}
  \item $\alpha,\beta \in \mathbb{N}$ mit $\alpha \geq 1$ und $\beta > 1$,
  \item $f:\N \rightarrow \R_+$,
  \item die Funktion $g:\N \rightarrow \R_+$ gen\"uge der Rekurrenz-Gleichung 
        \\[0.2cm]
        \hspace*{1.3cm}
        $g(n) = \alpha \cdot g\left(n/\beta\right) + f(n)$,
        \\[0.2cm]
        wobei der Ausdruck $n/\beta$ die ganzzahlige Division von $n$ durch $\beta$ bezeichnet.
  \end{enumerate}
  Dann k\"onnen wir in den gleich genauer beschriebenen Situationen asymptotische
  Absch\"atzungen f\"ur die Funktion $g(n)$ angeben:
  \begin{enumerate}
  \item Falls es eine Konstante $\varepsilon > 0$ gibt, so dass 
        \\[0.2cm]
        \hspace*{1.3cm}
        $f(n) \in \Oh\bigl(n^{\log_\beta(\alpha) - \varepsilon}\bigr)$
        \\[0.2cm]
        gilt, dann haben wir 
        \\[0.2cm]
        \hspace*{1.3cm}
        $g(n) \in \Oh\left(n^{\log_\beta(\alpha)}\right)$.
  \item Falls sowohl $f(n) \in \Oh\bigl(n^{\log_\beta(\alpha)}\bigr)$ als auch $n^{\log_\beta(\alpha)} \in \Oh\bigl(f(n)\bigr)$
        gilt, dann folgt
        \\[0.2cm]
        \hspace*{1.3cm}
        $g(n) \in \Oh\bigl(\log_\beta(n) \cdot n^{\log_\beta(\alpha)}\bigr)$. 
  \item Falls es eine Konstante $\gamma < 1$ und eine Konstante $k \in \mathbb{N}$ gibt, so dass
        f\"ur $n \geq k$
        \\[0.2cm]
        \hspace*{1.3cm}
        $\alpha \cdot f\left(n/\beta\right) \leq \gamma \cdot f(n)$        
        \\[0.2cm]
        gilt, dann folgt 
        \\[0.2cm]
        \hspace*{1.3cm}
        $g(n) \in \Oh\bigl(f(n)\bigr)$. \hspace*{\fill} $\Box$
  \end{enumerate}
\end{Theorem}
\textbf{Erl\"auterung}:
Ein vollst\"andiger Beweis dieses Theorems geht \"uber den Rahmen einer einf\"uhrenden Vorlesung hinaus.
Wir wollen aber erkl\"aren, wie die drei F\"alle zustande kommen.
\begin{enumerate}
\item Wir betrachten zun\"achst den ersten Fall.  In diesem Fall kommt der asymptotisch 
wesentliche Anteil des Wachstums der Funktion $g$ von der Rekursion. 
Um diese Behauptung einzusehen, betrachten wir die homogene Rekurrenz-Gleichung
\[ g(n) = \alpha \cdot g\left(n/\beta\right). \]
Wir beschr\"anken uns auf solche Werte von $n$, die sich als Potenzen von $\beta$ schreiben
lassen, also Werte der Form 
\[ n = \beta^k \quad \mbox{mit $k\in\N$.} \]
Definieren wir f\"ur $k \in \N$ die Folge $\bigl(b_k\bigr)_{k\in\N}$ durch
\[ b_k := g\bigl(\beta^k\bigr), \]
so erhalten wir f\"ur die Folgenglieder $b_k$ die Rekurrenz-Gleichung 
\[
     b_k = g\bigl(\beta^k\bigr) = \alpha \cdot g\left( \beta^k/\beta \right) 
   = \alpha \cdot g\bigl(\beta^{k-1}\bigr) = \alpha \cdot b_{k-1}.
\]
Wir sehen unmittelbar, dass diese Rekurrenz-Gleichung die L\"osung 
\begin{equation}
  \label{eq:master}
  b_k = \alpha^k \cdot b_0   
\end{equation}
hat.  Aus $n = \beta^k$ folgt sofort 
\[  k = \log_\beta(n). \]
Ber\"ucksichtigen wir, dass $b_k = g(n)$ ist, so liefert Gleichung (\ref{eq:master}) also 
\begin{equation}
  \label{eq:master2}
   g(n) = \alpha^{\log_\beta(n)} \cdot b_0.   
\end{equation}
Wir zeigen, dass
\begin{equation}
  \label{eq:master1}
  \alpha^{\log_\beta(n)} = n^{\log_\beta(\alpha)} 
\end{equation}
gilt.  Dazu betrachten wir die folgende Kette von Äquivalenz-Umformungen:
\[ 
\begin{array}[t]{llcll}
                & \alpha^{\log_\beta(n)} & = & n^{\log_\beta(\alpha)} & \mid\; \log_\beta(\cdot)  \\[0.2cm]
\Leftrightarrow & \log_\beta\left(\alpha^{\log_\beta(n)}\right) 
                  & = & \log_\beta\left(n^{\log_\beta(\alpha)}\right) \\[0.2cm]
\Leftrightarrow & \log_\beta(n) \cdot log_\beta(\alpha) 
                  & = & \log_\beta(\alpha) \cdot \log_\beta(n) & 
                  \mbox{wegen $\log_b(x^y) = y \cdot \log_b(x)$}
\end{array}
\]
Da die letzte Gleichung offenbar richtig ist, und wir zwischendurch nur Äquivalenz-Umformungen
durchgef\"uhrt haben, ist auch die erste Gleichung richtig und wir haben Gleichung (\ref{eq:master1}) gezeigt.
Insgesamt haben wir damit 
\[ g(n) = n^{\log_\beta(\alpha)} \cdot b_0 \]
gezeigt.  Also gilt: Vernachl\"assigen wir die Inhomogenit\"at $f$, so erhalten wir
die folgende asymptotische Absch\"atzung:
\[ g(n) \in \Oh\bigl(n^{\log_\beta(\alpha)}\bigr). \]
\item Im zweiten Fall liefert
die Inhomogenit\"at $f$ einen Beitrag, der genau so gro{\ss} ist wie die L\"osung der homogenen
Rekurrenz-Gleichung.  Dies f\"uhrt dazu, dass die L\"osung asymptotisch um einen Faktor
$\log_\beta(n)$ gr\"o{\ss}er wird.  Um das zu verstehen, betrachten wir exemplarisch die Rekurrenz-Gleichung
\[ g(n) = \alpha \cdot g\left(n/\beta\right) + n^{\log_\beta(\alpha)} \]
mit der Anfangs-Bedingung $g(1) = 0$.  Wir betrachten wieder nur Werte 
$n \in \{ \beta^k \mid k \in \N \}$ und setzen daher
\[ n = \beta^k. \]
Wie eben definieren wir
\[ b_k := g(n) = g\bigl(\beta^k\bigr). \]
Das liefert
\[ b_k = \alpha \cdot g\left(\beta^k/\beta\right) + \bigl(\beta^k\bigr)^{\log_\beta(\alpha)}
       = \alpha \cdot g(\beta^{k-1}) + \left(\beta^{\log_\beta(\alpha)}\right)^k
       = \alpha \cdot b_{k-1} + \alpha^k.
 \]
Nun gilt $b_0 = g(1) = 0$.  Um die Rekurrenz-Gleichung $b_k = \alpha \cdot b_{k-1} + \alpha^k$ zu l\"osen,
berechnen wir zun\"achst die Werte f\"ur $k=1,2,3$:
\begin{eqnarray*}
  b_1 & = & \alpha \cdot b_0 + \alpha^1               \\
      & = & \alpha \cdot 0 + \alpha                   \\
      & = & 1 \cdot \alpha^1                          \\[0.2cm]
  b_2 & = & \alpha \cdot b_1 + \alpha^1               \\
      & = & \alpha \cdot 1 \cdot \alpha^1 + \alpha^2  \\
      & = & 2 \cdot \alpha^2                          \\[0.2cm]
  b_3 & = & \alpha \cdot b_2 + \alpha^2               \\
      & = & \alpha \cdot 2 \cdot \alpha^2 + \alpha^3  \\
      & = & 3 \cdot \alpha^3                          
\end{eqnarray*}
Wir vermuten hier, dass die L\"osung dieser Rekurrenz-Gleichung durch die Formel
\[ b_k = k \cdot \alpha^k \]
gegeben wird.  Den Nachweis dieser Vermutung f\"uhren wir durch eine triviale Induktion:
\begin{enumerate}
\item[I.A.:] $k = 0$
             
             Einerseits gilt $b_0 = 0$, andererseits gilt $0 \cdot \alpha^0 = 0$.

\item[I.S.:] $k \mapsto k+1$

            \[
            \begin{array}[t]{lcl}
              b_{k+1} &               =  & \alpha \cdot b_k + \alpha^{k+1}              \\[0.1cm] 
                      & \stackrel{IV}{=} & \alpha \cdot k \cdot \alpha^k + \alpha^{k+1} \\[0.1cm]  
                      &               =  & k \cdot \alpha^{k+1} + \alpha^{k+1}          \\[0.1cm] 
                      &               =  & (k + 1) \cdot \alpha^{k+1}. 
            \end{array}
            \]
\end{enumerate}
Da aus $n = \beta^k$ sofort $k = \log_\beta(n)$ folgt, ergibt sich f\"ur die Funktion
$g(n)$ 
\[ g(n) = b_k = k \cdot \alpha^k = \log_\beta(n) \cdot \alpha^{\log_\beta(n)} = \log_\beta(n) \cdot
n^{\log_\beta(\alpha)} \]
und das ist genau die Form, durch die im zweiten Fall des Hauptsatzes die Funktion
$g(n)$ abgesch\"atzt wird.
\item
Im letzten Fall des Hauptsatzes \"uberwiegt schlie{\ss}lich der Beitrag der Inhomogenit\"at, so dass die L\"osung 
nun asymptotisch durch die Inhomogenit\"at dominiert wird.
Wir machen wieder den Ansatz 
\[ n = \beta^k \quad \mbox{und} \quad b_k = g\bigl(\beta^k\bigr). \]
Wir \"uberlegen uns, wie die Ungleichung 
\[ \alpha \cdot f\left(n/\beta\right) \leq \gamma \cdot f(n) \]
f\"ur $n = \beta^k$ aussieht und erhalten
\begin{equation}
  \label{eq:master_u1}
 \alpha \cdot f\left(\beta^{k-1}\right) \leq \gamma \cdot f\bigl(\beta^k\bigr) 
\end{equation}
Setzen wir hier f\"ur $k$ den Wert $k-1$ ein, so erhalten wir
\begin{equation}
  \label{eq:master_u2}
 \alpha \cdot f\left(\beta^{k-2}\right) \leq \gamma \cdot f\bigl(\beta^{k-1}\bigr) 
\end{equation}
Wir multiplizieren nun die Ungleichung (\ref{eq:master_u2}) mit $\alpha$ und
Ungleichung (\ref{eq:master_u1}) mit $\gamma$ und erhalten die Ungleichungen
\[ \alpha^2 \cdot f\left(\beta^{k-2}\right) \leq \alpha \cdot \gamma \cdot f\bigl(\beta^{k-1}\bigr) 
   \quad \mbox{und} \quad
   \alpha \cdot \gamma \cdot f\left(\beta^{k-1}\right) \leq \gamma^2 \cdot f\bigl(\beta^k\bigr) 
\]
Setzen wir diese Ungleichungen zusammen, so erhalten wir die neue Ungleichung
\[ \alpha^2 \cdot f\left(\beta^{k-2}\right) \leq \gamma^2 \cdot f\bigl(\beta^k\bigr) \]
Iterieren wir diesen Prozess, so sehen wir, dass 

\begin{equation}
  \label{eq:master_u3}
\alpha^i \cdot f\bigl(\beta^{k-i}\bigr) \leq \gamma^i \cdot f(\beta^k) 
   \quad \mbox{f\"ur alle $i \in \{1,\cdots k\}$ gilt.}   
\end{equation}
Wir berechnen nun $g(\beta^k)$ durch Iteration der Rekurrenz-Gleichung:
\[
\begin{array}[t]{lcl}
g(\beta^k) & = & \alpha \cdot g(\beta^{k-1}) + f(\beta^k) \\
           & = & \alpha \cdot \bigl(\alpha \cdot g(\beta^{k-2}) + f(\beta^{k-1})\bigr) + f(\beta^k) \\
           & = & \alpha^2 \cdot g(\beta^{k-2}) + \alpha \cdot f(\beta^{k-1}) + f(\beta^k) \\
           & = & \alpha^3 \cdot g(\beta^{k-3}) + 
                 \alpha^2 \cdot f(\beta^{k-2}) + \alpha \cdot f(\beta^{k-1}) + f(\beta^k) \\
           &   & \vdots \\
           & = & \alpha^k \cdot g(\beta^0) + \alpha^{k-1} \cdot f(\beta^1) + \cdots +
                 \alpha^1 \cdot f(\beta^{k-1}) + \alpha^0 \cdot f(\beta^k) \\
           & = & \alpha^k \cdot g(\beta^0) + \sum\limits_{i=1}^{k} \alpha^{k-i} \cdot f(\beta^i)
\end{array}
\]
Da bei der $\Oh$-Notation die Werte von $f$ f\"ur kleine Argumente keine Rolle spielen,
k\"onnen wir ohne Beschr\"ankung der Allgemeinheit annehmen, dass $g(\beta^0) \leq f(\beta^0)$
ist.  Damit erhalten wir dann die Absch\"atzung
\[ 
\begin{array}{lcl}
g(\beta^k) & \leq & \alpha^k \cdot f(\beta^0) + \sum\limits_{i=1}^{k} \alpha^{k-i} \cdot f(\beta^i) \\[0.3cm]
           & =    & \sum\limits_{i=0}^{k} \alpha^{k-i} \cdot f(\beta^i) \\[0.3cm]
           & =    & \sum\limits_{j=0}^{k} \alpha^{j} \cdot f(\beta^{k-j}) 
\end{array}
\]
wobei wir im letzten Schritt den Index $i$ durch $k-j$ ersetzt haben.  Ber\"ucksichtigen wir
nun die Ungleichung (\ref{eq:master_u3}), so erhalten wir die Ungleichungen
\[
\begin{array}{lcl}
   g(\beta^k) & \leq & \sum\limits_{j=0}^{k} \gamma^j \cdot f(\beta^k) \\[0.3cm]
              & =    & f(\beta^k) \cdot \sum\limits_{j=0}^{k} \gamma^j \\[0.5cm]
              & \leq & f(\beta^k) \cdot \sum\limits_{j=0}^{\infty} \gamma^j \\[0.5cm]
              & =    & f(\beta^k) \cdot \bruch{1}{\;1 - \gamma\;},
\end{array}
\]
wobei wir im letzten Schritt die Formel f\"ur die geometrische Reihe 
\[ \sum\limits_{j=0}^{\infty} q^j = \bruch{1}{1 - q} \]
benutzt haben.  Ersetzen wir nun $\beta^k$ wieder durch $n$, so sehen wir, dass 
\[ g(n) \leq \bruch{1}{\;1 - \gamma\;} \cdot f(n) \]
gilt und daraus folgt sofort
\[ g(n) \in \Oh\bigl(f(n)\bigr). \hspace*{10cm} \Box \]
\end{enumerate}
\vspace*{0.3cm}

\noindent
\textbf{Beispiel}:
Wir untersuchen das asymptotische Wachstum der Folge, die durch die Rekurrenz-Gleichung
\[ a_n = 9 \cdot a_{n/3} + n \]
definiert ist.  Wir haben hier 
\[ g(n) = 9 \cdot g(n/3) + n, \quad \mbox{also} \quad \alpha = 9, 
   \quad \beta = 3, \quad f(n) = n.
\]
Damit gilt 
\[ \log_\beta(\alpha) = \log_3(9) = 2. \]
Wir setzen $\varepsilon := 1 > 0$.  Dann gilt
\[ f(n) = n \in \Oh(n) = \Oh\left(n^{2-1}\right) = \Oh\left(n^{2-\varepsilon}\right). \]
Damit liegt der erste Fall des Hauptsatzes vor und wir k\"onnen schlie{\ss}en, dass
\[ g(n) \in \Oh(n^2) \]
gilt. \qed
\vspace*{0.3cm}

\noindent
\textbf{Beispiel}:
Wir  betrachten die Rekurrenz-Gleichung 
\\[0.2cm]
\hspace*{1.3cm}
$a_n = a_{n/2} + 2$
\\[0.2cm]
und analysieren das asymptotische Wachstum der Funktion $n \mapsto a_n$ mit Hilfe des
Hauptsatzes der Laufzeit-Funktionen.
Wir setzen $g(n) := a_n$ und haben also f\"ur die Funktion $g$ die Rekurrenz-Gleichung
\\[0.2cm]
\hspace*{1.3cm}
$g(n) = 1 \cdot g\left(n/2\right) + 2$
\\[0.2cm]
Wir definieren $\alpha := 1$, $\beta := 2$ und $f(n) = 2$.  Wegen 
\\[0.2cm]
\hspace*{1.3cm}
$\log_\beta(\alpha) = \log_2(1) = 0$ \quad und \quad
$2 \in \Oh(1)= \Oh(n^0)$ \quad sowie \quad $n^0 \in \Oh(2)$
\\[0.2cm]
sind die Voraussetzungen des zweiten Falls erf\"ullt und wir erhalten 
\\[0.2cm]
\hspace*{1.3cm}
$a_n \in \Oh\bigl(\log_2(n)\bigr)$.
\qed
\vspace*{0.3cm}

\noindent
\textbf{Beispiel}:
Diesmal betrachten  wir die Rekurrenz-Gleichung 
\\[0.2cm]
\hspace*{1.3cm}
$a_n = 3 \cdot a_{n/4} + n \cdot \log_2(n)$.
\\[0.2cm]
Es gilt $\alpha = 3$, $\beta = 4$ und $f(n) = n \cdot \log_2(n)$.  Damit gilt
\[ log_\beta(\alpha) = \log_4(3) < 1. \]
Damit ist klar, dass die Funktion $f(n) = n \cdot log_2(n)$ schneller w\"achst als die
Funktion $n^{\log_4(3)}$.  Damit kann h\"ochstens der dritte Fall des Hauptsatzes vorliegen.
Wir suchen also ein $\gamma < 1$, so dass die Ungleichung
\[ \alpha \cdot f(n/\beta) \leq \gamma \cdot f(n) \]
gilt.  Setzen wir hier die Funktion $f(n) = n \cdot \log_2(n)$ und die Werte f\"ur $\alpha$
und $\beta$ ein, so erhalten wir die Ungleichung 
\[ 3 \cdot n/4 \cdot \log_2(n/4) \leq \gamma \cdot n \cdot \log_2(n), \]
die f\"ur durch 4 teilbares $n$ offenbar \"aquivalent ist zu
\[ \frac{3}{4} \cdot \log_2(n/4) \leq \gamma \cdot \log_2(n). \]
Setzen wir $\gamma := \frac{3}{4}$ und k\"urzen, so geht diese Ungleichung \"uber in die
offensichtlich wahre Ungleichung
\[ \log_2(n/4) \leq \log_2(n). \] 
Damit liegt also der dritte Fall des Hauptsatzes vor und wir k\"onnen schlie{\ss}en, dass
\[ a_n \in \Oh\left(n \cdot \log_2(n)\right) \]
gilt. \qed
\vspace*{0.3cm}

\exercise
Benutzen Sie den Hauptsatz der Laufzeit-Funktionen um das asymptotische Wachstum 
der Folgen $\folge{a_n}$, $\folge{b_n}$ und $\folge{c_n}$  abzusch\"atzen, falls diese
Folgen den nachstehenden Rekurrenz-Gleichungen gen\"ugen:
\begin{enumerate}
\item $a_n = 4 \cdot a_{n/2} + 2 \cdot n + 3$.
\item $b_n = 4 \cdot b_{n/2} + n^2$.
\item $c_n = 3 \cdot c_{n/2} + n^3$.
\end{enumerate}

\noindent
\textbf{Bemerkung}: Es ist wichtig zu sehen, dass die drei F\"alle des Theorems nicht vollst\"andig sind:
Es gibt Situationen, in denen der Hauptsatz nicht anwendbar ist.  Beispielsweise l\"asst sich der Hauptsatz nicht
f\"ur die Funktion $g$, die durch die Rekurrenz-Gleichung
\[ g(n) = 2 \cdot g(n/2) + n \cdot \log_2(n) \quad \mbox{mit der Anfangs-Bedingung $g(1) = 0$}\]
definiert ist, anwenden, denn die Inhomogenit\"at w\"achst schneller als im zweiten Fall, aber
nicht so schnell, dass der dritte Fall vorliegen w\"urde.  Dies k\"onnen wir wie folgt sehen.
Es gilt 
\[ \alpha = 2, \quad \beta = 2 \quad \mbox{und damit} \quad \log_\beta(\alpha) = 1. \]
Damit der zweite Fall vorliegt, m\"usste 
\[ n \cdot \log_2(n) \in \Oh(n^1) \]
gelten, was sicher falsch ist.  Da die Inhomogenit\"at $n \cdot \log_2(n)$ offenbar
schneller w\"achst als der Term $n^1$, kann jetzt h\"ochstens noch der dritte Fall vorliegen.
Um diese Vermutung zu \"uberpr\"ufen, nehmen wir an, dass ein $\gamma < 1$ existiert, so dass die Inhomogenit\"at 
\[ f(n) := n \cdot \log_2(n) \]
die Ungleichung 
\[ \alpha \cdot f(n/\beta) \leq \gamma \cdot f(n) \]
erf\"ullt.  Einsetzen von $f$ sowie von $\alpha$ und $\beta$ f\"uhrt auf die Ungleichung
\[ 2 \cdot n/2 \cdot \log_2(n/2) \leq \gamma \cdot n \cdot \log_2(n). \]
Dividieren wir diese Ungleichung durch $n$ und vereinfachen, so erhalten wir
\[ \log_2(n) - \log_2(2) \leq \gamma \cdot \log_2(n). \]
Wegen $\log_2(2) = 1$ addieren wir auf beiden Seiten 1 und subtrahieren $\gamma \cdot \log_2(n)$.
Dann erhalten wir
\[ \log_2(n) \cdot (1 - \gamma) \leq 1, \]
woraus schlie{\ss}lich 
\[ \log_2(n) \leq \bruch{1}{\;1 - \gamma\;} \]
folgt.  Daraus folgt durch Anwenden der Funktion $x \mapsto 2^x$ die Ungleichung
\[ \displaystyle n \leq 2^\frac{1}{\;1 - \gamma\;}, \]
die aber sicher nicht f\"ur beliebige $n$ gelten kann.  Damit haben wir einen Widerspruch
zu der Annahme, dass der dritte Fall des Hauptsatzes vorliegt.
\hspace*{\fill} $\Box$
\vspace*{0.3cm}

\exercise
L\"osen Sie die Rekurrenz-Gleichung
\[ g(n) = 2 \cdot g(n/2) + n \cdot \log_2(n) \quad \mbox{mit der Anfangs-Bedingung $g(1) = 0$} \]
f\"ur den Fall, dass $n$ eine Zweier-Potenz ist.
\hspace*{\fill} $\Box$


%%% Local Variables: 
%%% mode: latex
%%% TeX-master: "algorithmen"
%%% End: 
