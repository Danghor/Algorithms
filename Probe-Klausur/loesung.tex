\documentclass{article}
\usepackage{german}
\usepackage[latin1]{inputenc}
\usepackage{fancyvrb}
\usepackage{a4wide}
\usepackage{amssymb}
\usepackage{epsfig}

\setlength{\textwidth}{15cm}

\newcommand{\bruch}[2]{\frac{\displaystyle \raisebox{1pt}[0pt][-0pt]{$\,#1\,$}}{\displaystyle \raisebox{0pt}[9pt]{$\,#2\,$}}}
\newcommand{\cq}{\symbol{34}}
\newcommand{\Ll}{{\cal L}}
\newcommand{\Rl}{{\cal R}}
\newcommand{\NS}{{\cal N\!S}}
\newcommand{\cl}[1]{{\cal #1}}
\renewcommand{\labelenumi}{(\alph{enumi})}
\renewcommand{\labelenumii}{\arabic{enumii}.}

\begin{document}

\noindent
{\large Aufgaben mit L\"osung zur Vorlesung  ``{\sl Algorithmen und Datenstrukturen}''}
\vspace{0.5cm}


\noindent
\textbf{Aufgabe 1}:
\begin{enumerate}
\item L\"osen Sie die Rekurrenz-Gleichung \\[0.2cm]
      \hspace*{1.3cm} $a_{n+2} = a_n + 2$ \\[0.2cm]
      f\"ur die Anfangs-Bedingungen $a_0 = 2$ und $a_1 = 1$.
      \hspace*{\fill} (10 Punkte)
\item L\"osen Sie die Rekurrenz-Gleichung \\[0.2cm]
      \hspace*{1.3cm} $a_{n+2} = 2 \cdot a_n - a_{n+1}$ \\[0.2cm]
      f\"ur die Anfangs-Bedingungen $a_0 = 0$ und $a_1 = 3$.
      \hspace*{\fill} (10 Punkte)
\end{enumerate}
\vspace{0.1cm}

\noindent
\textbf{L\"osung}:
\begin{enumerate}
\item Es handelt sich um eine lineare, inhomogene Rekurrenz-Gleichung der Ordnung 2.
      Das charakteristische Polynom lautet $\chi(x) = x^2 - 1 = (x-1)\cdot(x+1)$. 
      Es gilt $\chi(1) = 0$. Wegen $\chi'(x) = 2\cdot x$ und $\chi'(1) = 2 \not=0$ erhalten wir
      eine spezielle L\"osung mit der Formel
      \\[0.2cm]
      \hspace*{1.3cm}
      $a_n = \bruch{2}{2} \cdot n = n$.
      \\[0.2cm]
      Wegen $\chi(x) = (x-1)\cdot (x+1)$ lautet
      die allgemeine L\"osung 
      \\[0.2cm]
      \hspace*{1.3cm} $a_{n} = \alpha \cdot  1^n + \beta \cdot  (-1)^n + n$.
      \\[0.2cm]
      Die Koeffizienten $\alpha$ und $\beta$ bestimmen wir durch Einsetzen der
      Anfangsbedingungen. Das f\"uhrt auf das Gleichungs-System 
      \\[0.2cm]
      \hspace*{1.3cm}
      $
      \left\{\begin{array}[c]{lcl}
        2 = \alpha + \beta \\
        1 = \alpha - \beta + 1 \\
      \end{array}\right\} \quad \Leftrightarrow\quad
      \left\{\begin{array}[c]{lcl}
        3 = 2\cdot \alpha + 1 \\
        1 = 2\cdot \beta  - 1 \\
      \end{array}\right\} \quad \Leftrightarrow\quad
      \left\{\begin{array}[c]{lcl}
        1 = \alpha  \\
        1 = \beta \\
      \end{array}\right\} 
      $
      \\[0.2cm]
      Damit lautet die L\"osung 
      \\[0.2cm]
      \hspace*{1.3cm} $a_n = (-1)^n + n + 1$.
\item Es handelt sich um eine lineare, homogene Rekurrenz-Gleichung der Ordnung 2.
      Das charakteristische Polynom lautet $\chi(x) = x^2 + x - 2 = (x-1)\cdot (x+2)$. 
      Damit lautet die allgemeine L\"osung 
      \\[0.2cm]
      \hspace*{1.3cm} $a_n = \alpha \cdot  1^n + \beta \cdot  (-2)^n$. \\[0.2cm]
      Wir bestimmen die Konstanten $\alpha$ und $\beta$ durch Einsetzen
      der Anfangsbedingungen: 
      \\[0.2cm]
      \hspace*{1.3cm}
       $
      \left\{\begin{array}[c]{lcl}
        0 = \alpha + \beta \\
        3 = \alpha - 2\cdot \beta \\
      \end{array}\right\} \quad \Leftrightarrow\quad
      \left\{\begin{array}[c]{lcl}
        \alpha = -\beta \\
        3 = -3\cdot \beta  \\
      \end{array}\right\} \quad \Leftrightarrow\quad
      \left\{\begin{array}[c]{lcl}
        -1 = \beta  \\
        1 = \alpha \\
      \end{array}\right\} 
      $
      \\[0.2cm]
      Damit lautet die L\"osung 
      \\[0.2cm]
      \hspace*{1.3cm} $a_n = 1 - (-2)^n$.
\end{enumerate}

\noindent
\textbf{Aufgabe 2}:
Der geordnete bin\"are Baum $t$ sei durch den folgenden Term definiert,
wobei zur Vereinfachung auf die Angabe der Werte, die mit den Schl\"usseln
assoziiert sind, verzichtet wurde.
\\[0.2cm]
\hspace*{1.3cm}
$t = \textsl{node}(11, \textsl{node}(10, \textsl{nil}, \textsl{nil}), \textsl{node}(15, \textsl{nil}, 
  \textsl{node}(18, \textsl{node}(17, \textsl{nil}, \textsl{nil}), 
  \textsl{node}(24, \textsl{nil}, \textsl{nil}))))
$
%\hspace*{1.3cm}\epsfig{file=aufgabe2,scale=0.5}
\begin{enumerate}
\item F\"ugen Sie in diesem Baum den Schl\"ussel \texttt{16} ein und geben Sie den
      resultierenden Term an.  \\ 
      \hspace*{\fill} (3 Punkte)
\item F\"ugen Sie in dem in Teil (a) berechneten Baum den Schl\"ussel \texttt{13} ein und
      geben Sie den resultierenden Term an.     
      \hspace*{\fill} (3 Punkte)
\item L\"oschen Sie aus dem in Teil (b) berechneten Baum den Schl\"ussel \texttt{15} und
      geben Sie den resultierenden Term an. 
      \hspace*{\fill} (4 Punkte)
\end{enumerate}
\textbf{Hinweis}: Bei der L\"osung dieser und der folgenden Aufgabe sind selbstverst\"andlich
die in der Vorlesung vorgestellten Algorithmen zu verwenden.
\vspace{0.3cm}

\noindent
\textbf{L\"osung}:
Wir wiederholen zun\"achst die Gleichungen, die das Einf\"ugen und L\"oschen beschreiben:
\begin{enumerate}
\renewcommand{\labelenumi}{\arabic{enumi}.}
\item Die Gleichungen f\"ur das Einf\"ugen lauten:
  \begin{enumerate}
  \item $\textsl{nil}\mathtt{.}\textsl{insert}(k,v) = \textsl{node}(k,v, \textsl{nil}, \textsl{nil})$,
  \item $\textsl{node}(k, v_2, l, r)\mathtt{.}\textsl{insert}(k,v_1) = \textsl{node}(k, v_1, l, r)$,
  \item $k_1 < k_2 \rightarrow 
          \textsl{node}(k_2, v_2, l, r)\mathtt{.}\textsl{insert}(k_1, v_1) =
          \textsl{node}\bigl(k_2, v_2, l\mathtt{.}\textsl{insert}(k_1, v_1), r\bigr)$,
  \item $k_1 > k_2 \rightarrow 
         \textsl{node}(k_2, v_2, l, r)\mathtt{.}\textsl{insert}(k_1, v_1) = 
         \textsl{node}\bigl(k_2, v_2, l, r\mathtt{.}\textsl{insert}(k_1, v_1)\bigr)$.
\end{enumerate}
\item Die Gleichungen f\"ur das L\"oschen lauten:
  \begin{enumerate}
  \item $\textsl{nil}\mathtt{.}\textsl{delete}(k) = \textsl{nil}$,
  \item $\textsl{node}(k,v,\textsl{nil},r)\mathtt{.}\textsl{delete}(k) = r$,
  \item $\textsl{node}(k,v,l,\textsl{nil})\mathtt{.}\textsl{delete}(k) = l$,
  \item $l \not= \textsl{nil} \,\wedge\, r \not= \textsl{nil} \,\wedge\, r\mathtt{.}\textsl{delMin}() = [r',k_{min}, v_{min}]  \;\rightarrow$ \\[0.1cm]
      \hspace*{1.3cm}
      $\textsl{node}(k,v,l,r)\mathtt{.}\textsl{delete}(k) = \textsl{node}(k_{min},v_{min},l,r')$,
  \item $k_1 < k_2 \rightarrow \textsl{node}(k_2,v_2,l,r)\mathtt{.}\textsl{delete}(k_1) = 
       \textsl{node}\bigl(k_2,v_2,l\mathtt{.}\textsl{delete}(k_1),r\bigr)$,
  \item $k_1 > k_2 \rightarrow \textsl{node}(k_2,v_2,l,r)\mathtt{.}\textsl{delete}(k_1) = 
       \textsl{node}\bigl(k_2,v_2,l,r\mathtt{.}\textsl{delete}(k_1)\bigr)$.
\end{enumerate}
\end{enumerate}
Damit k\"onnen wir nun die Teilaufgaben l\"osen.  Um die Notation \"ubersichtlich zu halten,
k\"urzen wir $\textsl{node}()$ durch $\mathtt{n}()$ ab und statt $\textsl{nil}$ schreiben wir $\mathtt{*}$
\begin{enumerate}
\item \begin{verbatim}
       t =  n(11, n(10, *, *),
                  n(15, *, 
                        n(18, n(17, n(16, *, *), *), 
                              n(24, *, *))))
      \end{verbatim}
\item \begin{verbatim}
       t =  n(11, n(10, *, *),
                  n(15, node(13, *, *),
                        n(18, n(17, n(16, *, *), *), 
                              n(24, *, *))))
      \end{verbatim}
\item \begin{verbatim}
       t =  n(11, n(10, *, *),
                  n(16, node(13, *, *),
                        n(18, n(17, *, *), 
                              n(24, *, *))))
      \end{verbatim}
\end{enumerate}
\vspace{0.3cm}

\noindent
\textbf{Aufgabe 3}: Der AVL-Baum $t$ sei durch den folgenden Term gegeben,
wobei zur Vereinfachung auf die Angabe der Werte, die mit den Schl\"usseln
assoziiert sind, verzichtet wurde.
\\[0.2cm]
\hspace*{1.3cm}
$t = \textsl{node}(17, 
           \textsl{node}(8, 
               \textsl{node}(2, \textsl{nil}, \textsl{nil}),
               \textsl{node}(10, \textsl{nil}, \textsl{nil})), 
           \textsl{node}(23,\textsl{nil},\textsl{nil}))$
%\epsfig{file=avl1,scale=0.5}
\begin{enumerate}
\item F\"ugen Sie  in diesem Baum den Schl\"ussel \texttt{13} ein und geben Sie den
      resultierenden Baum an.   \\[0.2cm]
      \hspace*{\fill} (6 Punkte)
\item F\"ugen Sie in dem Baum aus Teil (b) den Schl\"ussel \texttt{15} ein und geben Sie den
      resultierenden Baum an.
      \hspace*{\fill} (3 Punkte)
\item Entfernen Sie den Schl\"ussel 2 aus dem unter Teil (b) berechneten Baum und geben Sie
      den resultierenden Baum an.
      \hspace*{\fill} (4 Punkte)
\end{enumerate}
\pagebreak

\noindent
\textbf{L\"osung}:
Das Einf\"ugen und L\"oschen in einem AVL-Baum unterscheidet sich von dem Einf\"ugen und L\"oschen
in einem bin\"aren Baum durch die zus\"atzliche Anwendung der Funktion $\textsl{restore}()$.
Diese Funktion ist durch die folgenden Gleichungen spezifiziert:
\begin{enumerate}
\renewcommand{\labelenumi}{\arabic{enumi}.}
\item $\textsl{nil}.\textsl{restore}() = \textsl{nil}$,
\item $|l.\textsl{height}() - r.\textsl{height}()| \leq 1 \rightarrow \textsl{node}(k,v,l,r).\textsl{restore}() = \textsl{node}(k,v,l,r)$,
\item $\begin{array}[t]{cl}
              & l_1.\textsl{height}() = r_1.\textsl{height}() + 2    \\ 
       \wedge & l_1 = \textsl{node}(k_2,v_2,l_2,r_2)               \\
       \wedge & l_2.\textsl{height}() \geq r_2.\textsl{height}()     \\[0.2cm]
       \rightarrow & \textsl{node}(k_1,v_1,l_1,r_1).\textsl{restore}() = 
                     \textsl{node}\bigl(k_2,v_2,l_2,\textsl{node}(k_1,v_1,r_2,r_1)\bigr)
       \end{array}
      $
\item $\begin{array}[t]{cl}
               & l_1.\textsl{height}() = r_1.\textsl{height}() + 2    \\ 
        \wedge & l_1 = \textsl{node}(k_2,v_2,l_2,r_2)               \\
        \wedge & l_2.\textsl{height}() < r_2.\textsl{height}()     \\
        \wedge & r_2 = \textsl{node}(k_3,v_3,l_3,r_3)               \\
        \rightarrow & \textsl{node}(k_1,v_1,l_1,r_1).\textsl{restore}() = 
                      \textsl{node}\bigl(k_3,v_3,\textsl{node}(k_2,v_2,l_2,l_3),\textsl{node}(k_1,v_1,r_3,r_1) \bigr)
        \end{array}
       $
\item $\begin{array}[t]{cl}
              & r_1.\textsl{height}() = l_1.\textsl{height}() + 2    \\ 
       \wedge & r_1 = \textsl{node}(k_2,v_2,l_2,r_2)               \\
       \wedge & r_2.\textsl{height}() \geq l_2.\textsl{height}()     \\[0.2cm]
       \rightarrow & \textsl{node}(k_1,v_1,l_1,r_1).\textsl{restore}() = 
                     \textsl{node}\bigl(k_2,v_2,\textsl{node}(k_1,v_1,l_1,l_2),r_2\bigr)
       \end{array}
      $
\item $\begin{array}[t]{cl}
               & r_1.\textsl{height}() = l_1.\textsl{height}() + 2    \\ 
        \wedge & r_1 = \textsl{node}(k_2,v_2,l_2,r_2)               \\
        \wedge & r_2.\textsl{height}() < l_2.\textsl{height}()     \\
        \wedge & l_2 = \textsl{node}(k_3,v_3,l_3,r_3)               \\
        \rightarrow & \textsl{restore}\bigl(\textsl{node}(k_1,v_1,l_1,r_1)\bigr) = 
                      \textsl{node}\bigl(k_3,v_3,\textsl{node}(k_1,v_1,l_1,l_3),\textsl{node}(k_2,v_2,r_3,r_2) \bigr)
        \end{array}
       $
\end{enumerate}
Damit lautet die L\"osung der Aufgaben:
\begin{enumerate}
\item Zun\"achst f\"ugen wir den Schl\"ussel 13 ein, ohne auf die Balancierungs-Bedingung zu
      achten.  Wir erhalten den folgenden Term, wobei wir die Knoten noch mit ihren H\"ohen
      annotieren, um sp\"ater die Balancierungs-Bedingung \"uberpr\"ufen zu k\"onnen:
      \\[0.2cm]
      \hspace*{1.3cm}
      $t = n(17, n(8, n(2, *, *):1,
                    n(10, n(13, *, *):1, *):2):3,$ \\
      \hspace*{2.75cm} $
                n(23,*,*):1):4$
      \\[0.2cm]
      Damit sehen wir, dass die Balancierungs-Bedingung an der Wurzel dieses
      Knotens verletzt ist, denn der linke Teilbaum hat eine Tiefe von drei, w\"ahrend der
      rechte Teilbaum eine Tiefe von 1 hat.  Da der rechte Teilbaum des linken Teilbaums
      eine gr\"o{\ss}ere Tiefe hat als der linke Teilbaum, liegt die in Gleichung 4 beschriebene
      Situation vor.  Im Einzelnen gilt:
      \begin{enumerate}
      \item $k_1 = 17$,
      \item $l_1 = n(8, n(2, *, *), n(10, *, n(13, *, *)))$,
      \item $k_2 = 8$,
      \item $l_2 = n(2, *, *)$,
      \item $r_2 = n(10, *, n(13, *, *))$,
      \item $k_3 = 10$,
      \item $l_3 = *$,
      \item $r_3 = n(13, *, *)$,
      \item $r_1 = n(23,*,*))$.
      \end{enumerate}
      Damit erhalten wir den AVL-Baum
      \\[0.2cm]
      \hspace*{1.3cm}
      $t = n(10, 
                   n(8, n(2, *, *), *)$ \\
      \hspace*{2.75cm} $
                n(17, n(13, *, *), n(23, *,*)))$
\item F\"ugen wir den Schl\"ussel 15 ein, so erhalten wir 
      \\[0.2cm]
      \hspace*{1.3cm}
      $t = n(10, 
                n(8, n(2, *, *), *),$ \\
      \hspace*{2.75cm} $
                n(17,
                     n(13, *, n(15, *, *)),
                     n(23, *, *)))$
      \\[0.2cm]
      Dies ist bereits ein AVL-Baum.
\item Nachdem wir den Schl\"ussel $2$ entfernt haben, hat der Baum die Form
      \\[0.2cm]
      \hspace*{1.3cm}
      $t = n(10, 
                n(8, *, *),$ \\
      \hspace*{2.75cm} $
                n(17,
                     n(13, *, n(15, *, *)),
                     n(23, *,*)))$
      \\[0.2cm]
      Das ist kein AVL-Baum mehr, denn an der Wurzel ist die Balancierungs-Bedingung verletzt.
      Wenn wir diesen Baum rebalancieren, erhalten wir:
      \\[0.2cm]
      \hspace*{1.3cm}
      $t = n(13, 
                n(10, n(8,*,*), *),$ \\
      \hspace*{2.75cm} $
                n(17,
                     n(15, *, *),
                     n(23, *,*)))$
      \\[0.2cm]
\end{enumerate}
\vspace{0.3cm}

\noindent
\textbf{Aufgabe 4}:
\begin{enumerate}
\item Zeigen Sie, dass f\"ur die Funktion \textsl{merge}, die wir im Skript definiert haben,
      folgende Gleichung gilt:
      \\
      \hspace*{1.3cm}
      $\textsl{count}(x, \textsl{merge}(L_1, L_2)) = \textsl{count}(x, L_1) + \textsl{count}(x, L_2)$
      \hspace*{\fill} (8 Punkte)
\item Zeigen Sie, dass f\"ur die Funktion \textsl{split}, die wir im Skript definiert haben,
      folgende Gleichung gilt:
      \\
      \hspace*{1.3cm}
      $\textsl{split}(L) = [L_1, L_2] \rightarrow \textsl{count}(x, L) = \textsl{count}(x, L_1) + \textsl{count}(x, L_2)$
      \hspace*{\fill} (7 Punkte)
\end{enumerate}

\noindent
\textbf{L\"osung}:
\begin{enumerate}
\item Wir f\"uhren den Nachweis durch Wertverlaufs-Induktion nach der Definition von $\textsl{merge}()$.
      Um diese Induktion durchf\"uhren zu k\"onnen, schlagen wir zun\"achst die Definition der Funktion 
      $\textsl{merge}()$ im Skript nach.  Wir finden dort:
      \begin{enumerate}
        \renewcommand{\labelenumi}{\arabic{enumi}.}
      \item $\mathtt{merge}([], L_2) = L_2$,
      \item $\mathtt{merge}(L_1, []) = L_1$,
      \item $x \preceq y \rightarrow 
             \mathtt{merge}\bigl([x]+R_1, [y]+R_2\bigr) = [x] + \mathtt{merge}\bigl(R_1,[y]+R_2\bigr)$.
           \item $\neg x \preceq y \rightarrow 
                  \mathtt{merge}\bigl([x]+R_1, [y]+R_2\bigr) = [y] + \mathtt{merge}\bigl([x] + R_1,R_2\bigr)$.
      \end{enumerate}
      Damit k\"onnen wir jetzt die Induktion durchf\"uhren:
      \begin{enumerate}
        \renewcommand{\labelenumii}{\arabic{enumii}.}
      \item $L_1 = []$.  

            Wir formen zun\"achst die linke Seite der Behauptung um.
            \\[0.2cm]
            \hspace*{1.3cm}
            $\textsl{count}(x, \textsl{merge}(L_1, L_2)) = \textsl{count}(x, \textsl{merge}([], L_2))
             = \textsl{count}(x, L_2)$
            \\[0.2cm]
            Jetzt formen wir die rechte Seite der Behauptung um.
            \begin{eqnarray*}
                  \textsl{count}(x, L_1) + \textsl{count}(x, L_2) 
            & = & \textsl{count}(x, []) + \textsl{count}(x, L_2)   \\
            & = & 0 + \textsl{count}(x, L_2)                       \\
            & = & \textsl{count}(x, L_2)
            \end{eqnarray*}
            Da wir dasselbe Ergebnis erhalten wie bei der Umformung der linken Seite der Gleichung,
            ist der erste Fall damit abgeschlossen.
      \item $L_2 = []$.  Dieser Fall ist analog zum ersten Fall.
      \item $L_1 = [u] + R_1 \wedge L_2 = [v] + R_2 \wedge u \preceq v$.

            Wir formen zun\"achst die linke Seite der Behauptung um.
            \begin{eqnarray*}
                  \textsl{count}(x, \textsl{merge}([u] + R_1, [v] + R_2)) 
            & = & \textsl{count}(x, [u] + \textsl{merge}(R_1, [v] + R_2))  \\
            & = & \textsl{eq}(x,u) + \textsl{count}(x, \textsl{merge}(R_1, [v] + R_2))  \\
            & \stackrel{IV}{=} & \textsl{eq}(x,u) + \textsl{count}(x, R_1) + \textsl{count}(x, [v] + R_2)  
            \end{eqnarray*}
            Jetzt formen wir die rechte Seite der Behauptung um.
            \begin{eqnarray*}
                  \textsl{count}(x, [u] + R_1) + \textsl{count}(x, [v] + R_2) 
            & = & \textsl{eq}(x, u) + \textsl{count}(x, R_1) + \textsl{count}(x, [v] + R_2) 
          \end{eqnarray*}
          Da wir dasselbe Ergebnis erhalten wie bei der Umformung der linken Seite der Gleichung,
          ist der dritte Fall damit abgeschlossen.
      \item $L_1 = [u] + R_1 \wedge L_2 = [v] + R_2 \wedge \neg(u \preceq v)$.

            Dieser Fall ist analog zum dritten Fall.
      \end{enumerate}
\item Wir f\"uhren den Nachweis durch Wertverlaufs-Induktion nach der Definition von $\textsl{split}()$.
      Um diese Induktion durchf\"uhren zu k\"onnen, schlagen wir zun\"achst die Definition der Funktion 
      $\textsl{split}()$ im Skript nach.  Wir finden dort:
      \begin{enumerate}
        \renewcommand{\labelenumii}{\arabic{enumii}.}
      \item $\mathtt{split}(\texttt{[]}) = \mathtt{[} \texttt{[]}, \texttt{[]} \mathtt{]}$.
      \item $\mathtt{split}(\mathtt{[}x\mathtt{]}) = \mathtt{[} \texttt{[}x\texttt{]}, \texttt{[]} \mathtt{]}$.
      \item $\mathtt{split}(R) = \mathtt{[}R_1, R_2\mathtt{]} \rightarrow
      \mathtt{split}\bigl(\mathtt{[}x, y\mathtt{]} + R\bigr) = \bigl[ [x] + R_1, [y] + R_2 \bigr]$.
      \end{enumerate}
      Damit k\"onnen wir die Induktion durchf\"uhren:
      \begin{enumerate}
        \renewcommand{\labelenumii}{\arabic{enumii}.}
      \item $L=[]$.  Dann gilt $L_1 = []$ und $L_2 = []$. Wir formen die linke Seite der zu beweisenden
            Gleichung $\textsl{count}(x,L) = \textsl{count}(x,L_1) + \textsl{count}(x,L_2)$ um:
            \\[0.2cm]
            \hspace*{1.3cm}
            $\textsl{count}(x,L) = \textsl{count}(x,[]) = 0$.
            \\[0.2cm]
            f\"ur die rechte Seite gilt:
            \\[0.2cm]
            \hspace*{1.3cm}
            $\textsl{count}(x,L_1) + \textsl{count}(x,L_2) = 
             \textsl{count}(x,[]) + \textsl{count}(x,[]) = 0 + 0 = 0$.
      \item $L = [u]$.  Dann gilt $L_1 = [u]$ und $L_2 = []$. Wir formen die linke Seite der zu beweisenden
            Gleichung  um:
            \\[0.2cm]
            \hspace*{1.3cm}
            $\textsl{count}(x,L) = \textsl{count}(x,[u]) = \textsl{eq}(x,u)$.
            \\[0.2cm]
            f\"ur die rechte Seite gilt:
            \begin{eqnarray*}
                  \textsl{count}(x,L_1) + \textsl{count}(x,L_2) 
            & = & \textsl{count}(x,[u]) + \textsl{count}(x,[]) \\
            & = & \textsl{eq}(x,u) + 0 \\
            & = & \textsl{eq}(x,u).
            \end{eqnarray*}
      \item $L = [u,v] + R$.  f\"ur den rekursiven Aufruf von $\textsl{split}()$ gelte
            \\[0.2cm]
            \hspace*{1.3cm}
            $\textsl{split}(R) = [R_1, R_2]$.
            \\[0.2cm]
            Nach der Induktions-Voraussetzung gilt dann zun\"achst
            \\[0.2cm]
            \hspace*{1.3cm}
            $\textsl{count}(x,R) = \textsl{count}(x, R_1) + \textsl{count}(x, R_2)$.
            \\[0.2cm]
            Jetzt formen wir die linke Seite der zu beweisenden Gleichung um:
            \begin{eqnarray*}
                    \textsl{count}(x,L)
              & = & \textsl{count}(x,[u,v] + R) \\
              & = & \textsl{eq}(x,u) + \textsl{eq}(x,v) + \textsl{count}(x,R) \\
              & \stackrel{IV}{=} & \textsl{eq}(x,u) + \textsl{eq}(x,v) + 
                \textsl{count}(x,R_1) + \textsl{count}(x,R_2) 
            \end{eqnarray*}
            Nun formen wir die rechte Seite der zu beweisenden Gleichung um:
            \begin{eqnarray*}
                    \textsl{count}(x,L_1) + \textsl{count}(x,L_2)
              & = & \textsl{count}(x,[u] + R_1) + \textsl{count}(x, [v] + R_2) \\
              & = & \textsl{eq}(x,u) + \textsl{count}(x,R_1) + \textsl{eq}(x,v) + \textsl{count}(x, R_2) 
            \end{eqnarray*}
            Da dieser Ausdruck bis auf die Reihenfolge der Summanden mit dem f\"ur die linke Seite
            gefundenen Ausdruck \"ubereinstimmt, ist der Beweis damit abgeschlossen.
      \end{enumerate}
\end{enumerate}
\vspace{0.3cm}

\noindent
\textbf{Aufgabe 5}:  Betrachten Sie das folgende Programm:
\begin{verbatim}
    double sum(unsigned n) {
        unsigned i = 0;
        unsigned s = 0;
        while (i <= n) {
            s = i + s;
            i = i + 1;
        }
        return s;
    }
\end{verbatim}
Die Funktion $\textsl{sum}()$ soll die folgende Spezifikation erf\"ullen:
\\[0.2cm]
\hspace*{1.3cm} $\textsl{sum}(n) = \frac{1}{2} \cdot n \cdot (n + 1)$
\begin{enumerate}
\item Weisen Sie mit Hilfe des Hoare-Kalk\"uls nach, dass das Programm korrekt ist.
\item Beweisen Sie mit Hilfe der Methode der symbolischen Programm-Ausf\"uhrung nach,
      dass das Programm korrekt ist.
\end{enumerate}
\vspace{0.3cm}

\noindent
\textbf{L\"osung}:
\begin{enumerate}
\item zun\"achst der Hoare-Kalk\"ul:
\begin{enumerate}
      \renewcommand{\labelenumii}{\arabic{enumii}.}
\item Wir zeigen als erstes, dass die \texttt{while}-Schleife der Invariante 
      \\[0.2cm]
      \hspace*{1.3cm}
      $I := \bigl(s = \frac{1}{2} \cdot i \cdot (i - 1) \wedge i \leq n + 1\bigr)$ 
      \\[0.2cm]
      gen\"ugt.  (Diese Invariante ist etwas st\"arker als in der Vorlesung.
      Dadurch l\"asst sich dann sp\"ater zeigen, dass beim Abbruch der Schleife
      die Gleichung $i = n+1$ gilt.  In der Vorlesung hatten wir an dieser Stelle nur
      anschaulich argumentiert.)

      f\"ur die erste Zuweisung in der Schleife gilt
      \\[0.2cm]
      \hspace*{1.3cm}
      $\{ I \wedge i \leq n \}$ \quad \texttt{s = s + i;} \quad $\{ (I \wedge i \leq n)[s \mapsto s - i] \}$
      \\[0.2cm]
      Wir formen den Ausdruck $\bigl(I \wedge i \leq n\bigr)[s \mapsto s - i]$ um:
      \begin{eqnarray*}
        \bigl(I \wedge i \leq n\bigr)[s \mapsto s - i] 
        & \leftrightarrow & \bigl(s = \frac{1}{2} \cdot i \cdot (i-1) \wedge i \leq n+1 \wedge i \leq n\bigr)[s \mapsto s - i] \\
        & \leftrightarrow & s - i = \frac{1}{2} \cdot i \cdot (i-1) \wedge i \leq n \\
        & \leftrightarrow & s = \frac{1}{2} \cdot i \cdot (i-1) + i \wedge i \leq n \\
        & \leftrightarrow & s = \frac{1}{2} \cdot i \cdot (i+1) \wedge i \leq n \\
      \end{eqnarray*}
      Als n\"achstes betrachten wir die Zuweisung \texttt{i = i + 1;}:
      \\[0.2cm]
      \hspace*{1.3cm}
      $\{ s = \frac{1}{2} \cdot i \cdot (i+1) \wedge i \leq n \}$ \quad \texttt{i = i + 1;} \quad
      $\{ \bigl(s = \frac{1}{2} \cdot i \cdot (i+1)\wedge i \leq n\bigr)[i \mapsto i - 1] \}$ 
      \\[0.2cm]
      Es gilt 
      \begin{eqnarray*}        
      &                 & \bigl(s = \frac{1}{2} \cdot i \cdot (i+1)\wedge i \leq n\bigr)[i \mapsto i - 1] \\
      & \leftrightarrow & s = \frac{1}{2} \cdot (i-1) \cdot i \wedge i - 1 \leq n                         \\
      & \leftrightarrow & s = \frac{1}{2} \cdot i \cdot (i-1) \wedge i \leq n + 1                         
      \end{eqnarray*}
      und damit haben wir die Invariante nachgewiesen.
\item Die Invariante ist zu Beginn der Schleife erf\"ullt, denn zu Beginn der
      Schleife gilt $s = 0$ und $i = 0$ und offenbar gilt
      \\[0.2cm]
      \hspace*{1.3cm}
      $i = 0 \wedge s = 0 \rightarrow s = \frac{1}{2} \cdot (i-1) \cdot i \wedge i \leq n + 1$
\item Nach Beendigung der Schleife gilt dann
      \\[0.2cm]
      \hspace*{1.3cm}
      $s = \frac{1}{2} \cdot (i-1) \cdot i \wedge i \leq n + 1 \wedge \neg (i \leq n)$
      \\[0.2cm]
      Es gilt 
      \\
      \hspace*{1.3cm}
      $i \leq n + 1 \wedge \neg (i \leq n) \rightarrow i = n + 1$
      \\[0.2cm]
      Damit erf\"ullt das Programm die Spezifikation
      \\[0.2cm]
      \hspace*{1.3cm}
      $s = \frac{1}{2} \cdot n \cdot (n+1)$.
\end{enumerate}
\item Jetzt die symbolische Programm-Ausf\"uhrung:
      \begin{Verbatim}[ frame         = lines, 
                  framesep      = 0.3cm, 
                  labelposition = bottomline,
                  numbers       = left,
                  numbersep     = -0.2cm,
                  xleftmargin   = 1.3cm,
                  xrightmargin  = 1.3cm,
                  codes         = {\catcode`_=8\catcode`$=3},
                  commandchars  = \\\{\},
                ]
    double sum(unsigned n) \{
        unsigned i$_0$ = 0;
        unsigned s$_0$ = 0;
        while (i$_0$ <= n) \{
            s$_{k+1}$ = i$_k$ + s$_k$;
            i$_{k+1}$ = i$_k$ + 1;
        \}
        return s$_{K}$;
    \}
    \end{Verbatim} 
    %\$
      Wir zeigen nun, dass die \texttt{while}-Schleife die folgende Invariante hat: 
      \\[0.2cm]
      \hspace*{1.3cm}
      $s_k = \frac{1}{2} \cdot (i_{k}-1) \cdot i_k$.
      \renewcommand{\labelenumii}{\arabic{enumii}.}
      \begin{enumerate}
      \item[I.A.:] $k=0$.

            Es gilt $s_0 = 0$ und $i_0 = 0$ und damit folgt sofort
            \\[0.2cm]
            \hspace*{1.3cm} $s_0 = \frac{1}{2} \cdot (i_{0}-1) \cdot i_0$.
      \item[I.S.:] $k \mapsto k+1$ 

            Offenbar gilt $i_{k+1} = i_k + 1$ und damit haben wir 
            \begin{eqnarray*}
              s_{k+1} & = & i_k + s_k \\
            & \stackrel{IV}{=} & i_k + \frac{1}{2} \cdot (i_{k}-1) \cdot i_k \\
            & = & \frac{1}{2} \cdot i_{k} \cdot (i_k + 1) \\
            & = & \frac{1}{2} \cdot (i_{k+1} - 1) \cdot i_{k+1}.
            \end{eqnarray*}
      \end{enumerate}
      Die Schleife wird offenbar $n+1$ mal durchlaufen und es gilt $i_K = n+1$.  Daraus folgt
      \\[0.2cm]
      \hspace*{1.3cm}
      $s_K = s_{n+1} = \frac{1}{2} \cdot (i_{n+1} - 1) \cdot i_{n+1} = \frac{1}{2} \cdot n \cdot (n+1)$.
\end{enumerate}
\vspace{0.3cm}

\noindent
\textbf{Aufgabe 6}:  Im Abschnitt 7.2 des Skriptes
werden Gleichungen angegeben, die das Einf\"ugen und L\"oschen in einem Heap beschreiben.
In diesem Zusammenhang sollen Sie in dieser Aufgabe  einige zus\"atzliche Methoden auf
bin\"aren B\"aumen durch bedingte Gleichungen spezifizieren.
\begin{enumerate}
\item Spezifizieren Sie eine Methode \textsl{isHeap}, so
      dass f\"ur einen bin\"aren Baum $b \in \mathcal{B}$ der Ausdruck 
      $b.\mathtt{isHeap}()$ genau dann den Wert $\mathtt{true}$ hat, wenn $b$ die
      \emph{Heap-Bedingung} erf\"ullt.  \hspace*{\fill} (10 Punkte)
\item Implementieren Sie eine Methode \textsl{isBalanced}, so
      dass f\"ur einen bin\"aren Baum $b \in \mathcal{B}$ der Ausdruck 
      $b.\mathtt{isBalanced}()$ genau dann den Wert $\mathtt{true}$ hat, wenn $b$ die
      \emph{Balancierungs-Bedingung} f\"ur \emph{Heaps} erf\"ullt.  
      \hspace*{\fill} (5 Punkte)
\end{enumerate}
\vspace{0.3cm}

\noindent
\textbf{L\"osung}:
\begin{enumerate}
\item Um die Methode \textsl{isHeap} leicht spezifizieren zu k\"onnen, definieren wir zun\"achst
      eine Hilfsfunktion
      \\[0.2cm]
      \hspace*{1.3cm}
      $\leq : \textsl{Key} \times \mathcal{B} \rightarrow \mathbb{B}$.
      \\[0.2cm]
      f\"ur einen Schl\"ussel $k$ und einen bin\"aren Baum $b$ soll $k \leq b$ genau dann gelten,
      wenn $k$ kleiner-gleich als alle Schl\"ussel sind, die in $b$ auftreten.
      Diese Funktion wird durch Gleichungen spezifiziert:
      \begin{enumerate}
      \item $k \leq \textsl{nil}$,
      \item $k \leq \textsl{node}(k_1,v_1,l,r) = \bigl(k \leq k_1 \wedge k \leq l \wedge k \leq r\bigr)$.
      \end{enumerate}
      Damit l\"asst sich jetzt die Methode $\textsl{isHeap}()$ durch Gleichungen spezifizieren:
      \begin{enumerate}
      \item $\textsl{nil}.\textsl{isHeap}() = \mathtt{true}$,
      \item $\textsl{node}(k,v,l,r).\textsl{isHeap}() = 
             \bigl(k \leq l \wedge k \leq r \wedge l.\textsl{isHeap}() \wedge r.\textsl{isHeap}()\bigr)$.
      \end{enumerate}
\item Wir definieren die Methode $\textsl{isBalanced}()$ induktiv.
      \begin{enumerate}
      \item $\textsl{nil}.\textsl{isBalanced}() = \mathtt{true}$,
      \item $\textsl{node}(k,v,l,r).\textsl{isBalanced}() =$ \\[0.1cm]
            \hspace*{\fill}
            $\bigl(|l.\textsl{count}() - r.\textsl{count}()| \leq 1 
             \wedge l.\textsl{isBalanced}()\wedge r.\textsl{isBalanced}()\bigr)$.
      \end{enumerate}
\end{enumerate}
\vspace{0.3cm}

\noindent
\textbf{Aufgabe 7}:
\begin{enumerate}
\item Zeigen Sie $\log_2(n) \in \mathcal{O}\bigl(\ln(\sqrt{n})\bigr)$. \hspace*{\fill} (4  Punkte)
\item Zeigen Sie $\bigl|\sin(n)\bigr| \in \mathcal{O}\bigl(1\bigr)$. \hspace*{\fill} (4 Punkte)
\item Es sei 
      $f(n) := \biggl(\sum\limits_{i=1}^n \frac{1}{i}\biggr) - \ln(n)$.
      Zeigen Sie $f(n)\in \mathcal{O}\bigl(1\bigr)$. \hspace*{\fill} (12 Punkte)
      
      \textbf{Hinweis}:  Zeigen Sie 
      \\[0.2cm]
      \hspace*{1.3cm}
      $0 \leq \biggl(\sum\limits_{i=1}^n \frac{1}{i}\biggr) - \ln(n) \leq 1$.
\end{enumerate}
\pagebreak

\textbf{L\"osung}:
\begin{enumerate}
\item zun\"achst bemerken wir, dass gilt:
      \\[0.2cm]
      \hspace*{1.3cm}
      $\log_2(n) = \bruch{\ln(n)}{\ln(2)}$ \quad und \quad
      $\ln(\sqrt{n}) = \ln\bigl(n^\frac{1}{2}\bigr) = \frac{1}{2} \cdot \ln(n)$.
      \\[0.2cm]
      Wir zeigen die Behauptung nun mit dem Grenzwertsatz:
      \begin{eqnarray*}
              \lim\limits_{n \rightarrow \infty} \bruch{\log_2(n)}{\ln(\sqrt{n})}
        & = & \lim\limits_{n \rightarrow \infty} \bruch{\bruch{\ln(n)}{\ln(2)}}{\frac{1}{2}\ln(n)} \\[0.3cm]
        & = & \lim\limits_{n \rightarrow \infty} \bruch{\bruch{1}{\ln(2)}}{\frac{1}{2}} \\[0.3cm]
        & = & \lim\limits_{x \rightarrow \infty} \bruch{2}{\ln(2)} \\[0.2cm]
        & = & \bruch{2}{\ln(2)} 
      \end{eqnarray*}
      Da der Grenzwert existiert, folgt die Behauptung aus dem Grenzwertsatz.
\item Das ist eigentlich eine Scherzaufgabe.  Aus der Geometrie wissen wir, dass 
      \\[0.2cm]
      \hspace*{1.3cm}
      $0 \leq | \sin(n) | \leq 1$ \quad f\"ur alle $n \in \mathbb{N}$ 
      \\[0.2cm]
      gilt, woraus die Behauptung unmittelbar folgt.  Die Aufgabe ist ein Beispiel, bei dem
      der Grenzwertsatz versagt.
\item f\"ur alle $x \in [i-1,i]$ gilt:
      \begin{eqnarray*}
      &             & x \leq i \\[0.3cm]
      & \Rightarrow & \bruch{1}{x} \geq \bruch{1}{i} \\[0.2cm]
      & \Rightarrow & \int_{i-1}^i \bruch{1}{x} dx \geq \int_{i-1}^i \bruch{1}{i} dx \\[0.2cm]
      & \Rightarrow & \int_{i-1}^i \bruch{1}{x} dx \geq \bruch{1}{i} \\[0.2cm]
      & \Rightarrow & \sum\limits_{i=2}^{n} \int_{i-1}^i \bruch{1}{x} dx \geq 
                      \sum\limits_{i=2}^{n} \bruch{1}{i}                      \\[0.2cm]
      & \Rightarrow & \int_{1}^n \bruch{1}{x} dx \geq \sum\limits_{i=2}^{n} \bruch{1}{i} \\[0.2cm]
      & \Rightarrow & \ln(n) - \ln(1) \geq \sum\limits_{i=2}^{n} \bruch{1}{i} \\[0.2cm]
      & \Rightarrow & \ln(n) + 1 \geq \sum\limits_{i=1}^{n} \bruch{1}{i} \\[0.2cm]
      & \Rightarrow & \sum\limits_{i=1}^{n} \bruch{1}{i} - \ln(n) \leq 1 
      \end{eqnarray*}
      Analog gilt f\"ur alle $x \in [i-1,i]$:
      \begin{eqnarray*}
      &             & x \geq i - 1 \\[0.3cm]
      & \Rightarrow & \bruch{1}{x} \leq \bruch{1}{i-1} \\[0.2cm]
      & \Rightarrow & \int_{i-1}^i \bruch{1}{x} dx \leq \int_{i-1}^i \bruch{1}{i-1} dx \\[0.2cm]
      & \Rightarrow & \int_{i-1}^i \bruch{1}{x} dx \leq \bruch{1}{i-1} \\[0.2cm]
      & \Rightarrow & \sum\limits_{i=2}^{n} \int_{i-1}^i \bruch{1}{x} dx \leq 
                      \sum\limits_{i=2}^{n} \bruch{1}{i-1}                      \\[0.2cm]
      & \Rightarrow & \int_{1}^n \bruch{1}{x} dx \leq \sum\limits_{i=2}^{n} \bruch{1}{i-1} \\[0.2cm]
      & \Rightarrow & \ln(n) - \ln(1) \leq \sum\limits_{i=2}^{n} \bruch{1}{i-1} \\[0.2cm]
      & \Rightarrow & \ln(n) \leq \sum\limits_{i=1}^{n-1} \bruch{1}{i} \\[0.2cm]
      & \Rightarrow & 0 \leq \sum\limits_{i=1}^{n-1} \bruch{1}{i} - \ln(n) \\[0.2cm]
      & \Rightarrow & 0 \leq \sum\limits_{i=1}^{n} \bruch{1}{i} - \ln(n) 
      \end{eqnarray*}
      Insgesamt haben wir damit die Ungleichung
      \\[0.2cm]
      \hspace*{1.3cm}
      $0 \leq \biggl(\sum\limits_{i=1}^n \frac{1}{i}\biggr) - \ln(n) \leq 1$
      \\[0.2cm]
      bewiesen und daraus folgt die Behauptung unmittelbar.
\end{enumerate}

\vspace{0.3cm}

\noindent
\textbf{Aufgabe 8}:
Es sei $\mathcal{B}$ die Menge der bin\"aren B\"aume, die im Kapitel 6 des
Skriptes definiert wird.  
\begin{enumerate}
\item Spezifizieren Sie eine Methode \\[0.2cm]
      \hspace*{1.3cm}
      $\textsl{isOrdered}: \mathcal{B} \rightarrow \mathbb{B}$
      \\[0.2cm]
      durch bedingte Gleichungen.  f\"ur einen bin\"aren Baum $b$ soll der Aufruf
      $b.\textsl{isOrdered}()$ genau dann \texttt{true} zur\"uck liefern, wenn $b\in \mathcal{B}_<$
      gilt.
      \hspace*{\fill} (8 Punkte)

      \textbf{Hinweis}: Definieren Sie sich geeignete Hilfsfunktionen.
\item Es sei $\textsl{insert}()$ die in Abschnitt 6.1 definierte Methode.
      Nehmen Sie an, dass Sie f\"ur alle $b\in \mathcal{B}_<$, alle Schl\"ussel $k$ und alle Werte
      $v$ die Gleichung
      \\[0.2cm]
      \hspace*{1.3cm} $b.\textsl{insert}(k,v).\textsl{isOrdered}() = \texttt{true}$
      \\[0.2cm]
      beweisen sollen.  Geben Sie an, welche Lemmata \"uber die in Teil (a) definierten
      Hilfsfunktionen zu einem solchen Beweis ben\"otigt werden.
      \hspace*{\fill} (4 Punkte)
\item Zeigen Sie nun f\"ur geordnete bin\"are B\"aume $b$ die Gleichung
      \\[0.2cm]
      \hspace*{1.3cm} $b.\textsl{insert}(k,v).\textsl{isOrdered}() = \texttt{true}$
      \hspace*{\fill} (12 Punkte)
\end{enumerate}

\noindent
\textbf{L\"osung}:
\begin{enumerate}
\item zun\"achst spezifizieren wir zwei Methoden
      \\[0.2cm]
      \hspace*{1.3cm} 
      $\leq: \textsl{Key} \times \mathcal{B} \rightarrow \mathbb{B}$ \quad und \quad
      $\leq: \textsl{Key} \times \mathcal{B} \rightarrow \mathbb{B}$
      \\[0.2cm]
      durch bedingte Gleichungen.  Die Beziegung $k \leq b$ soll f\"ur einen Schl\"ussel $k$ und einen Baum
      $b$ genau dann gelten, wenn $k$ kleiner-gleich alle Schl\"ussel aus $b$ ist und
      $b \leq k$ gilt genau dann, wenn alle Schl\"ussel aus $b$ $k$ kleiner-gleich  $k$ sind.
      Die Beziegung $k \leq b$ haben wir bereits in Aufgabe 6 angegeben.  Analog definieren wir jetzt
      $b \leq k$:
      \begin{enumerate}
      \item $\textsl{nil} \leq k$,
      \item $\textsl{node}(k_1,v_1,l,r) \leq k = \bigl(k_1 \leq k \wedge l \leq k \wedge r \leq k \bigr)$.
      \end{enumerate}
      Damit k\"onnen wir nun Gleichungen f\"ur $\textsl{isOrdered}()$ angeben.
      \begin{enumerate}
      \item $\textsl{nil}.\textsl{isOrdered}() = \mathtt{true}$,
      \item $\textsl{node}(k,v,l,r).\textsl{isOrdered}() = 
             \bigl(l \leq k \wedge k \leq r \wedge l.\textsl{isOrdered}() \wedge r.\textsl{isOrdered}()\bigr)$.
      \end{enumerate}
\item Wir ben\"otigen die folgenden beiden Lemmata:
      \begin{enumerate}
      \item $k \leq b.\textsl{insert}(k_0,v_0) = k \leq k_0 \wedge k \leq b$,
      \item $b.\textsl{insert}(k_0,v_0) \leq k = k_0 \leq k \wedge b \leq k$.
      \end{enumerate}
\item Wir beweisen die Behauptung durch Wertverlaufs-Induktion nach der Definition von $\textsl{insert}()$. 
      Dazu betrachten wir die folgenden Gleichungen aus dem Skript:
      \begin{enumerate}
      \item $\textsl{nil}\mathtt{.}\textsl{insert}(k,v) = \textsl{node}(k,v, \textsl{nil}, \textsl{nil})$,
      \item $\textsl{node}(k, v_2, l, r)\mathtt{.}\textsl{insert}(k,v_1) = \textsl{node}(k, v_1, l, r)$,
      \item $k_1 < k_2 \rightarrow 
             \textsl{node}(k_2, v_2, l, r)\mathtt{.}\textsl{insert}(k_1, v_1) =
             \textsl{node}\bigl(k_2, v_2, l\mathtt{.}\textsl{insert}(k_1, v_1), r\bigr)$,
      \item $k_1 > k_2 \rightarrow 
             \textsl{node}(k_2, v_2, l, r)\mathtt{.}\textsl{insert}(k_1, v_1) = 
             \textsl{node}\bigl(k_2, v_2, l, r\mathtt{.}\textsl{insert}(k_1, v_1)\bigr)$.
      \end{enumerate}
      Jetzt f\"uhren wir die Induktion durch:
      \begin{enumerate}
      \item $b = \textsl{nil}$
            \begin{eqnarray*}
                  \textsl{nil}.\textsl{insert}(k,v).\textsl{isOrdered}()
            & = & \textsl{node}(k,v,\textsl{nil},\textsl{nil}).\textsl{isOrdered}() \\
            & = & \textsl{nil} \leq k \wedge k \leq \textsl{nil} 
                  \wedge \textsl{nil}.\textsl{isOrdered}()\wedge \textsl{nil}.\textsl{isOrdered}() \\
            & = & \mathtt{true} \wedge \mathtt{true} \wedge \mathtt{true} \wedge \mathtt{true}  \\
            & = & \mathtt{true}
            \end{eqnarray*}
      \item $b = \textsl{node}(k,v_0,l,r)$

            Da $b\in \mathcal{B}_<$ ist, wissen wir, dass $b.\textsl{isOrdered}() = \mathtt{true}$
            ist.  Daraus folgt sofort
            \\[0.2cm]
            \hspace*{1.3cm}
            $l \leq k \wedge k \leq r \wedge l.\textsl{isOrdered}() \wedge r.\textsl{isOrdered}()$
            \\[0.2cm]
            Damit k\"onnen wir nun den Nachweis f\"uhren:
            \begin{eqnarray*}
            &   & \textsl{node}(k,v_0,l,r).\textsl{insert}(k,v).\textsl{isOrdered}() \\
            & = & \textsl{node}(k,v,l,r).\textsl{isOrdered}() \\
            & = & l \leq k \wedge k \leq r \wedge l.\textsl{isOrdered}() \wedge r.\textsl{isOrdered}() 
            \end{eqnarray*}
            und die letzte Formel haben wir oben schon aus der Tatsache gefolgert,
            dass $b$ ein geordneter bin\"arer Baum ist.
      \item $b = \textsl{node}(k_0,v_0,l,r)$ und $k \leq k_0$.  Dann gilt
            \begin{eqnarray*}
            &   & \textsl{node}(k_0,v_0,l,r).\textsl{insert}(k,v).\textsl{isOrdered}() \\
            & = & \textsl{node}(k_0,v_0,l.\textsl{insert}(k,v),r).\textsl{isOrdered}() \\
            & = & l.\textsl{insert}(k,v) \leq k_0 \wedge k_0 \leq r
                  \wedge l.\textsl{insert}(k,v).\textsl{isOrdered}() \wedge r.\textsl{isOrdered}() \\
            & \stackrel{IV}{=} & l.\textsl{insert}(k,v) \leq k_0 \wedge k_0 \leq r \wedge
                  \mathtt{true} \wedge r.\textsl{isOrdered}() \\
            & = & l \leq k_0 \wedge  k \leq k_0 \wedge k_0 \leq r \wedge
                  \mathtt{true} \wedge r.\textsl{isOrdered}() \quad\mbox{nach Lemma aus (b)}\\
            & = & \mathtt{true} \qquad\mbox{wegen $b.\textsl{isOrdered}()$}
            \end{eqnarray*}
      \item $b = \textsl{node}(k_0,v_0,l,r)$ und $\neg(k \leq k_0)$. 

            Dieser Fall ist analog zum 3.~Fall.
      \end{enumerate}
\end{enumerate}

\noindent
\textbf{Aufgabe 9}:
Es gelte $\Sigma = \{ \mathtt{a},\,\mathtt{b},\,\mathtt{c},\,\mathtt{d},\,\mathtt{e},\,\mathtt{f} \}$.
Die H\"aufigkeit, mit der diese Buchstaben in dem zu kodierenden String $s$ auftreten, sei durch die
folgende Tabelle gegeben:

\begin{center}
\begin{tabular}[t]{|l|r|r|r|r|r|r|}
\hline
Buchstabe  & \texttt{a} & \texttt{b} & \texttt{c} & \texttt{d} & \texttt{e} & \texttt{f} \\
\hline
H\"aufigkeit &          8 &          9 &         10 &         11 &         12 &         13 \\
\hline
\end{tabular}
\end{center}
\begin{enumerate}
\item Berechnen sie einen optimalen Kodierungs-Baum f\"ur die angegebenen H\"aufigkeiten.
\item Geben die Kodierung der einzelnen Buchstaben an, die sich aus diesem Baum ergibt.
\end{enumerate}

\noindent
\textbf{L\"osung}:
\begin{enumerate}
\item Wir wenden den Huffman-Algorithmus an und erhalten die folgenden Mengen.  Zur Abk\"urzung
      schreiben wir dort $l(a,f)$ statt $\textsl{leaf}(a,f)$ und $n(l,r)$ statt $\textsl{node}(l,r)$.
      \begin{enumerate}
      \item $\{ l(a,8), l(b,9), l(c,10), l(d,11), l(e,12), l(f,13) \}$
      \item $\{ n(l(a,8), l(b,9)):17, l(c,10), l(d,11), l(e,12), l(f,13) \}$
      \item $\{ n(l(a,8), l(b,9)):17, n(l(c,10), l(d,11)):21, l(e,12), l(f,13) \}$
      \item $\{ n(l(a,8), l(b,9)):17, n(l(c,10), l(d,11)):21, n(l(e,12), l(f,13)):25 \}$
      \item $\{ n(l(e,12), l(f,13)):25, n(n(l(a,8), l(b,9)):17, n(l(c,10), l(d,11))):38 \}$
      \item $\{ n(n(l(e,12), l(f,13)):25, n(n(l(a,8), l(b,9)):17, n(l(c,10), l(d,11)))):63 \}$
      \end{enumerate}
      Damit ist 
      \\[0.2cm]
      \hspace*{1.3cm} $n(n(l(e,12), l(f,13)), n(n(l(a,8), l(b,9)), n(l(c,10), l(d,11))))$
      \\[0.2cm]
      der gesuchte Kodierungsbaum.
\item Damit ergibt sich die folgende Kodierung f\"ur die einzelnen Buchstaben:
      \begin{verbatim}
      e = 00, f = 01, a = 100, b = 101, c = 110, d = 111.
      \end{verbatim}
\end{enumerate}

\end{document}

%%% Local Variables: 
%%% mode: latex
%%% TeX-master: t
%%% End: 
