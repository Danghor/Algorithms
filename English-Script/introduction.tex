\chapter{Introduction}
\section{Motivation}
The previous lecture in the winter term has shown us how interesting problems can be solved with the
help of sets and relations.  However, we did not discuss how sets and relations can be represented
and how the operations on sets can be implemented in an efficient way.  This course will answer
these questions:  We will see data structure that can be used to represent sets in an efficient way.
Furthermore, we will discuss a number of other data structures and algorithms that should be in the
toolbox of every computer scientist.

While the class in the last term has introduced the students to the theoretical foundations of
computer science, this class is more application oriented.  Indeed, it may be one of the most 
important classes for your future career: Stanford University regularly asks their former students 
to rank those classes that were the most useful for their professional career.  Together with
programming and databases, the class on algorithms consistently ranks highest.  The practical
importance of the topic of this class can also be seen by the availability of book titles like 
 ``Algorithms for Interviews'' \cite{aziz:10} or the
\href{http://www.youtube.com/watch?v=k4RRi_ntQc8}{Google job interview questions}.
 
 
\section{Overview}
This lecture covers the design and the analysis of algorithms.  We will discuss the following topics.
\begin{enumerate}
\item Undecidability of the \href{http://en.wikipedia.org/wiki/Halting_problem}{halting problem}.

      At the beginning of the lecture we discuss the limits of computability:  We will show that
      there is no \textsc{SetlX} function \texttt{doesTerminate} such that for a given function $f$
      of one argument and a string $s$ the expression
      \\[0.2cm]
      \hspace*{1.3cm}
      $\texttt{doesTerminate}(f, s)$ 
      \\[0.2cm]
      yields \texttt{true} if the evaluation of $f(s)$ terminates and yields \texttt{false} otherwise.
\item \href{http://en.wikipedia.org/wiki/Computational_complexity_theory}{Complexity} of algorithms.

      In general, in order to solve a given problem it is not enough to develop an algorithm that
      implements a function $f$ that computes values $f(x)$ for some argument $x$.  If the size of
      the argument $x$ is big, then it is also important  
      that the computation of $f(x)$ does not take too much time.  Therefore, we want to have
      \emph{efficient} algorithms.
      In order to be able to discuss the efficiency of algorithms we have to introduce two 
      mathematical notions.
      \begin{enumerate}
      \item \href{http://en.wikipedia.org/wiki/Recurrence_relation}{\emph{Recurrence relations}}
            are discrete analogues of differential equations.  Recurrence relations occur naturally
            when analyzing the runtime of algorithms.
      \item \href{http://en.wikipedia.org/wiki/Big_Oh}{\emph{Big O notation}} offers a convenient way to
            discuss the growth rate of functions.  This notation is useful to abstract from
            unimportant details when discussing the runtime of algorithms.  
      \end{enumerate}
\item \href{http://en.wikipedia.org/wiki/Abstract_data_types}{Abstract data types}.

      Abstract data types are a means to describe the behavior of an algorithm in a concise way.
\item \href{http://en.wikipedia.org/wiki/Sorting_algorithm}{Sorting algorithms}.

      Sorting algorithm are the algorithms that are most frequently used in practice.  As these
      algorithms are, furthermore, quite easy to understand they serve best for an introduction to
      the design of algorithms.  We discuss the following sorting algorithms:
      \begin{enumerate}
      \item \href{http://en.wikipedia.org/wiki/Insertion_sort}{insertion sort},
      \item \href{http://en.wikipedia.org/wiki/Selection_sort}{selection sort},
      \item \href{http://en.wikipedia.org/wiki/Merge_sort}{merge sort}, and
      \item \href{http://en.wikipedia.org/wiki/Quicksort}{quicksort}.
      \end{enumerate}
\item \href{http://en.wikipedia.org/wiki/Hoare_logic}{Hoare logic}.
  
      The most important property of an algorithm is its correctness.  The \emph{Hoare calculus}
      is a method to investigate the correctness of an algorithm.  

\item \href{http://en.wikipedia.org/wiki/Map_(computer_science)}{Associative arrays}.
  
      Associative arrays are a means to represent a function.  We discuss
      various data structures that can be used to implement associative arrays efficiently.
\item \href{http://en.wikipedia.org/wiki/Priority_queue}{Priority queues}.
  
      Many graph theoretical algorithms use priority queues as a basic building block.
      Furthermore, priority queues have important applications in the theory of operating systems
      and in simulation.
\item \href{http://en.wikipedia.org/wiki/Graph_theory}{Graph theory}.
  
      There are many applications of graphs in computer science.  The topic of graph theory is very
      rich and can easily fill a class of its own.  Therefore, we can only cover a small subset of this topic.
      In particular, we will discuss
      \href{http://en.wikipedia.org/wiki/Dijkstra%27s_algorithm}{Dijkstra's algorithm}
      for computing the shortest path.
\item \href{http://en.wikipedia.org/wiki/Monte_Carlo_method}{Monte Carlo Method} 
 
      Many important problems either do not have an exact solution at all or the computation of an
      exact solution would be prohibitively expensive.  In these cases it is often possible to use 
      simulation in order to get an approximate solution.  As a concrete example we will show
      how certain probabilities in \href{http://en.wikipedia.org/wiki/Texas_hold_%27em}{Texas hold 'em} 
      poker can be determined approximately with the help of the Monte
      Carlo method.
\end{enumerate}
The primary goal of these lectures on algorithms is not to teach as many algorithms as possible.
Instead, I want to teach how to think algorithmically:  At the end of these
lectures, you should be able to develop your own algorithms on yourself.  This is a process that
requires a lot of creativity on your side, as there are no cookbook recipes that can be followed.
However, once you are acquainted with a fair number of algorithms, you should be able to discover
algorithms on your own.

\section{Algorithms and Programs}
This is a lecture on algorithms, not on programming.  It is important that you do not mix up
programs and algorithms.  An algorithm is an \emph{abstract} concept to solve a given problem.  In
contrast, a program is a \emph{concrete} implementation of an algorithm.  In order to implement an
algorithm by a program we have to cover every detail, be it trivial or not.  On the other hand, 
to specify an algorithm it is often sufficient to describe the interesting aspects.  It is
quite possible for an algorithm to leave a number of questions open.

In the literature, algorithms are mostly presented as pseudo code.  Syntactically, pseudo code looks
similar to a program, but in contrast to a program, pseudo code can also contain parts that are only
described in natural language.   However, it is important to realize that a piece of pseudo code is
\underline{not} an algorithm but is only a \emph{ representation} of an algorithm.  However, the
advantage of pseudo code is that we are not confined by the randomness of the syntax of a
programming language.

Conceptually, the difference between an algorithm and a program is similar to the difference between
an \emph{idea} and a \emph{text} that describes the idea.  If you have an idea, you can write it down to
make it concrete.  As you can write down the idea in English or French or any other language, the
textual descriptions of the idea might be quite different.  This is the same with an algorithm:  We
can code it in \textsl{Java} or \textsc{Cobol} or any other language.  The programs will be very
different but the algorithm will be the same.

Having discussed the difference between algorithms and programs, let us now decide how to present
algorithms in this lecture.  
\begin{enumerate}
\item We can describe algorithms using natural language.  While natural language certainly is
      expressive enough, it also suffers from ambiguities.  Furthermore, natural language
      descriptions of complex algorithms tend to be very hard to understand.
\item Instead, we can describe an algorithm by implementing it.  There is certainly no ambiguity
      in a program, but on the other hand this approach would require us to implement every aspect
      of an  algorithm and our descriptions of algorithms would therefore get longer than we want.
\item Finally, we can try to describe algorithms in the language of mathematics.  This language is 
      concise, unambiguous, and easy to understand, once you are accustomed to it.  This is
      therefore our method of choice.

      However, after having presented an algorithm in the language of mathematics, it is often very
      simple to implement this algorithm in the programming language
      \href{http://randoom.org/Software/SetlX}{\textsc{SetlX}}.  The reason is 
      that \textsc{SetlX} is based on set theory, which is the language of mathematics.  We will see
      that \textsc{SetlX} enables us to present and implement algorithms on a very high abstraction level.
\end{enumerate}

\section{Desirable Properties of Algorithms}
Before we start with our discussion of algorithms we should think about our goals when designing
algorithms.  
\begin{enumerate}
\item Algorithms have to be \emph{correct}.
\item Algorithms should be as \emph{efficient} as possible.
\item Algorithms should be \emph{simple}.
\end{enumerate}
The first goal in this list is so self-evident that it is often overlooked.  The
importance of the last goal might not be as obvious as the other goals.
However, the reasons for the last goal are economical:  If it takes too long to code an algorithm, the
cost of the implementation might well be unaffordable.  Furthermore, even if the budget is unlimited there 
is another reasons to strife for simple algorithms:
If the conceptual complexity of an algorithm is too high, it may become impossible to check
the correctness of the implementation.  Therefore, the third goal is strongly related to the first
goal.  

\section{Literature}
These lecture notes are available online at
\\[0.2cm]
\hspace*{1.3cm}
\href{https://github.com/karlstroetmann/Algorithms/blob/master/English-Script/algorithms.pdf}{https://github.com/karlstroetmann/Algorithms/blob/master/English-Script/algorithms.pdf}.  
\\[0.2cm]
They are intended to be the main source for my lecture.  Additionally, I want
to mention those books that have inspired me most.
\begin{enumerate}
\item \textsl{Alfred V.~Aho}, \textsl{John E.~Hopcraft}, and \textsl{Jeffrey D.~Ullman}:
      \emph{Data Structures and Algorithms}, Addison-Wesley, 1987, \cite{aho:87}.
      
      This book is quite dated now but it is one of the classics on algorithms.  It discusses algorithms at an
      advanced level.
\item \textsl{Thomas H.~Cormen}, \textsl{Charles E.~Leiserson}, 
      \textsl{Ronald L.~Rivest}, and \textsl{Clifford Stein}:
      \emph{Introduction to Algorithms}, 
      third edition, MIT Press, 2009, \cite{cormen:09}

      Due to the level of detail and the number of algorithms given, this book is well suited as a
      lookup library.
\item \textsl{Robert Sedgewick}: \emph{Algorithms in \textsl{Java}}, 
      fourth edition, Pearson, 2011, \cite{sedgewick:11}.
    
      This book has a nice \href{http://algs4.cs.princeton.edu/home/}{booksite} containing a wealth
      of additional material.  This book seems to be the best choice for the working practitioner.
      Furthermore, \href{http://www.cs.princeton.edu/~rs/}{Professor Sedgewick} teaches a 
      \href{https://www.coursera.org/course/algs4partI}{course} on algorithms on 
      \href{https://www.coursera.org/}{coursera} that is based on this book.
\item \emph{Einf�hrung in die Informatik},
      written by \textsl{Heinz-Peter Gumm} and \texttt{Manfred Sommer} \cite{gumm:2008}.
      
      This German book is a very readable introduction to computer science and it has a comprehensive chapter
      on algorithms.  Furthermore, this book is
      \href{http://www.oldenbourg-link.com/doi/book/10.1524/9783486595390}{available} electronically
      in our library. 
\item Furthermore, there is a set of 
      \href{https://class.coursera.org/algo-003/class/index}{video lectures} 
      from \href{http://theory.stanford.edu/~tim/}{Professor Roughgarden}
      at \href{https://www.coursera.org/}{coursera}.
\item \href{http://www.cs.berkeley.edu/~vazirani/algorithms/all.pdf}{\emph{Algorithms}},
      written by \textsl{Sanjoy Dasgupta, Christos H.~Papadimitriou, and Unmesh V.~Vazirani} \cite{dasgupta:2008}
      is a short text book on Algorithms that is available online.
\item \href{http://www.mpi-inf.mpg.de/~mehlhorn/Toolbox.html}{\emph{Data Structures and Algorithms}}
      written by \emph{Kurt Mehlhorn and Peter Sanders} \cite{mehlhorn:2008} is another good text
      book on algorithms that is available online.
\end{enumerate}
As this is the first English edition of this set of lecture notes, it will undoubtedly contain some
errors.  If you spot any of these errors, I would like you to send me an 
\href{mailto:stroetmann@dhbw-stuttgart.de}{email}.  It is of no use if you tell me any of these
errors after the lecture because by the time I am back at my desk I have surely forgotten them.
Furthermore,  when time permits, I will gladly answer all mailed questions concerning the theory
explained in this script or in my lecture.
\vspace*{0.3cm}

\section{A Final Remark}
There is one final remark I would like to make at this point:  Frequently, I get questions from
students concerning the exams.  While I will answer most of them, I should warn you that, 50\% 
of the time, my answers will be lies.  The other 50\%,
my answer will be either dirty lies or accidentally true.

\section{A Request}
Computer science is a very active field of research.  Therefore, these lecture notes are constantly
evolving and hence might contain typos or bugs.  If you find a bug, please take the time and send me
an email.  My email address is
\\[0.2cm]
\hspace*{1.3cm}
\href{mailto:karl. stroetmann@ dhbw- mannheim. de}{karl.stroetmann@dhbw-mannheim.de}.
\\[0.2cm]
If you are familiar with \href{http://github.com}{\texttt{github}}, you might even consider
sending me a \href{https://help.github.com/articles/using-pull-requests}{pull request}.


%%% local Variables: 
%%% mode: latex
%%% TeX-master: "algorithms"
%%% End: 
