\chapter{Hoare Logic}
In this chapter we introduce \href{http://en.wikipedia.org/wiki/Hoare_logic}{\emph{Hoare logic}}.
This is a formal system that is used to prove the correctness of imperative computer programs.  
Hoare logic has been introduced 1969 by  
\href{http://en.wikipedia.org/wiki/C._A._R._Hoare}{Sir Charles Antony Richard Hoare}.
 

\section{Preconditions and Postconditions}
Hoare logic is based on preconditions and postconditions.  If \texttt{P}
 is a program fragment and if $F$ and $G$ are logical formul\ae, then we call
$F$ a precondition and $G$ a postcondition for the program fragment \texttt{P}
if the following holds:  If \texttt{P} is executed in a state $s$ such that the formula $F$ holds in
$s$, then the execution of \texttt{P} will change the state $s$ into a new state $s'$ such that 
$G$ holds in $s'$,  This is written as
\\[0.2cm]
\hspace*{1.3cm}
$ \hoare{F}{P}{G} $.
\\[0.2cm]
We will read this notation as ``\emph{executing $P$ changes $F$ into $G$}''.
The formula
\\[0.2cm]
\hspace*{1.3cm}
$ \hoare{F}{P}{G} $
\\[0.2cm]
is called a \emph{Hoare tripel}.
\vspace*{0.3cm}


\examples
\begin{enumerate}
\item The assignment ``\texttt{x := 1;}'' satisfies the specification
      \\[0.2cm]
      \hspace*{1.3cm}
      $ \hoare{\mathtt{true}}{x := 1;}{x = 1}. $
      \\[0.2cm]
      Here, the precondition is the trivial condition ``\texttt{true}'', since
      the postcondition ``$x = 1$'' will always be satisfied after this assignment.
\item The assignment  ``\texttt{x = x + 1;}'' satisfies the specification
      \\[0.2cm]
      \hspace*{1.3cm}
      $ \hoare{x=1}{x := x + 1;}{x=2}. $
      \\[0.2cm]
      If the precondition is ``$x = 1$'', then it is obvious that the postcondition has to be  
      ``$x = 2$''.
\item Let us consider the assignment ``\texttt{x = x + 1;}'' again.  However, this time
      the precondition is given as ``$\textsl{prime}(x)$'', which is only true if $x$ is a
      prime number.  This time, the Hoare tripel is given as
      \\[0.2cm]
      \hspace*{1.3cm}
      $ \hoare{\textsl{prime}(x)}{x := x + 1;}{\textsl{prime}(x-1)}$.
      \\[0.2cm]
      This might look strange at first.  Many students think that this Hoare triple should rather
      be written as
      \\[0.2cm]
      \hspace*{1.3cm}
      $ \hoare{\textsl{prime}(x)}{x := x + 1;}{\textsl{prime}(x+1)} $.
      \\[0.2cm]
      However, this can easily be refuted by taking $x$ to have the value $2$.  Then, the
      precondition $\textsl{prime}(x)$ is satisfied since $2$ is a prime number.  After the
      assignment, $x$ has the value $3$ and
      \\[0.2cm]
      \hspace*{1.3cm}
      $x - 1 = 3 - 1 = 2$
      \\[0.2cm]
      still is a prime number.  However, we also have
      \\[0.2cm]
      \hspace*{1.3cm}
      $x + 1 = 3 + 1 = 4$ 
      \\[0.2cm]
      and as $4 = 2 \cdot 2$ we see that $x + 1$ is not a prime number!
\end{enumerate}
Let us proceed to show how the different parts of a program can be specified using Hoare tripels.
We start with the analysis of assignments.
 
\subsection{Assignments}
Let us generalize the previous example.  Let us therefore assume that we have an assignment of the
form 
\\[0.2cm]
\hspace*{1.3cm}
$ \texttt{x := h(x);} $
\\[0.2cm]
and we want to investigate how the postcondition $G$ of this assignment is related to the
precondition $F$.  To simplify matters, let us assume that the function
$h$ is invertible, i.~e.~we assume that there is a function $h^{-1}$ such that we have
\\[0.2cm]
\hspace*{1.3cm}
$ h^{-1}\bigl(h(x)\bigr) = x \quad \mathtt{and} \quad h\bigl(h^{-1}(x)\bigr) = x $
\\[0.2cm]
for all $x$.  In order to understand the problem at hand, let us consider an example.
The assignment
\\[0.2cm]
\hspace*{1.3cm}
$ \texttt{x := x + 1;} $
\\[0.2cm]
can be written as
\\[0.2cm]
\hspace*{1.3cm}
$ \texttt{x := h(x);} $
\\[0.2cm]
where the function $h$ is given as
\\[0.2cm]
\hspace*{1.3cm}
$ h(x) = x + 1 $
\\[0.2cm]
and the inverse function $h^{-1}$ is 
\\[0.2cm]
\hspace*{1.3cm}
$h^{-1}(x) = x - 1$. 
\\[0.2cm]
Now we are able to compute the postcondition of the assignment ``\texttt{x := h(x);}'' from the
precondition.  We have
\\[0.2cm]
\hspace*{1.3cm}
$\hoare{F}{x := h(x);}{F\sigma}$ \quad where \quad 
$\sigma = \bigl[x \mapsto h^{-1}(x)\bigr]$.
\\[0.2cm]
Here, $F\sigma$ denotes the application of the substitution $\sigma$ to the formula $F$.  The
expression  $F\sigma$ is computed from the expression $F$ by replacing every occurrence of the variable
$x$ by the term $h^{-1}(x)$.  In order to understand why this is the correct way to compute the 
postcondition, we consider the assignment
``\texttt{x := x + 1}'' again and choose the formula $x = 7$ as precondition.  
Since $h^{-1}(x) = x - 1$, the substitution $\sigma$ is given as
$\sigma = [ x \mapsto x - 1 ]$.  Therefore, $F\sigma$ has the form
\\[0.2cm]
\hspace*{1.3cm}
$ (x = 7)[x \mapsto x - 1] \;\equiv\; (x - 1 = 7). $
\\[0.2cm]
I have used the symbol ``$\equiv$'' here in order to express that these formul\ae\ are
syntactically identical.  
Therefore, we have
\\[0.2cm]
\hspace*{1.3cm}
$ \hoare{x = 7}{x := x + 1;}{x - 1 = 7}. $
\\[0.2cm]
Since the formula $x - 1 = 7$ is equivalent to the formula $x = 8$ the Hoare tripel above can be
rewritten as  
\\[0.2cm]
\hspace*{1.3cm}
$ \hoare{x = 7}{x := x + 1;}{x = 8} $
\\[0.2cm]
and this is obviously correct:  If the value of $x$ is $7$ before the assignment
\\[0.2cm]
\hspace*{1.3cm}
``\texttt{x := x + 1;}'' 
\\[0.2cm]
is executed, then after the assignment is executed, $x$ will have the value $8$.

Let us try to understand why
\\[0.2cm]
\hspace*{1.3cm}
$\hoare{F}{x := h(x);}{F\sigma}$ \quad where \quad 
$\sigma = \bigl[x \mapsto h^{-1}(x)\bigr] $
\\[0.2cm]
is, indeed, correct:   Before the assignment ``x \texttt{:=} h(x);'' is executed,
the variable $x$ has some fixed value $x_0$.  The precondition $F$ is valid for $x_0$.  Therefore,
the formula $F[x \mapsto x_0]$ is valid before the assignment is executed.  However,
the variable $x$ does not occur in the formula $F[x \mapsto x_0]$ because it has been replaced by
the fixed value $x_0$.  Therefore, the formula
\\[0.2cm]
\hspace*{1.3cm}
$ F[x \mapsto x_0] $
\\[0.2cm]
remains valid after the assignment  ``\texttt{x = h(x);}'' is executed.  After this assignment,
the variable $x$ is set to $h(x_0)$. Therefore, we have
\\[0.2cm]
\hspace*{1.3cm}
$x = h(x_0)$.
\\[0.2cm]  
Let us solve this equation for $x_0$.  We find
\\[0.2cm]
\hspace*{1.3cm}
$h^{-1}(x) = x_0$.
\\[0.2cm]
Therefore, after the assignment the formula 
\\[0.2cm]
\hspace*{1.3cm}
$ F[x \mapsto x_0] \equiv  F[x \mapsto h^{-1}(x)]$ 
\\[0.2cm]
is valid and  this is the formula that we had written as $F\sigma$ above.

We conclude this discussion with another example.  The unary predicate \textsl{prime} checks whether
its argument is a prime number.  Therefore, $\textsl{prime}(x)$ is true if $x$ is a prime number.
Then we have
\\[0.2cm]
\hspace*{1.3cm}
$\hoare{\textsl{prime}(x)}{x := x + 1;}{\textsl{prime}(x-1)}$.
\\[0.2cm]
The correctness of this Hoare triple should be obvious: If $x$ is a prime and if $x$ is then
incremented by $1$, then afterwards $x-1$ is prime.

\paragraph{Different Forms of Assignments}
Not all assignments can be written in the form ``\texttt{x := h(x);}'' where the function $h$ is invertible.
Often, a constant $c$ is assigned to some variable $x$.  If $x$ does not occur in the precondition
$F$, then we have
\\[0.2cm]
\hspace*{1.3cm}
$ \hoare{F}{x := c;}{F \wedge x = c}. $
\\[0.2cm]
The formula $F$ can be used to restrict the values of other variables occurring in the program under
consideration.


\paragraph{General Form of the Assignment Rule}
In the literature the rule for specifying an assignment is given as
\\[0.2cm]
\hspace*{1.3cm}
$ \hoare{F[x \mapsto t]}{x := t;}{F}. $
\\[0.2cm]
Here, $t$ is an arbitrary term that can contain the variable $x$.  This rule can be read as follows:
\\[0.2cm]
\hspace*{1.3cm}
\begin{minipage}[c]{0.8\linewidth}
``\emph{If the formula $F(t)$ is valid in some state and $t$ is assigned to $x$,
        then after this assignment we have $F(x)$.}'' 
\end{minipage}
\\[0.2cm]
This rule is obviously correct.  However, it is not very useful because in order to apply this rule 
we first have to rewrite the precondition as $F(t)$.  If $t$ is some complex term, this can be impossible.


\subsection{The Weakening Rule}
If a program fragment $P$ satisfies the specification
\\[0.2cm]
\hspace*{1.3cm}
$ \hoare{F}{P}{G} $
\\[0.2cm]
and if, furthermore, the formula $G$ implies the validity of the formula $H$, that is if
\\[0.2cm]
\hspace*{1.3cm}
$G \rightarrow H$
\\[0.2cm]
holds, then the program fragment $P$ satisfies
\\[0.2cm]
\hspace*{1.3cm}
$\hoare{F}{P}{H}$.
\\[0.2cm]
The reasoning is as follows:  If after executing $P$ we know that $G$
is valid, then, since $G$ implies $H$,  the formula $H$ has to be
valid, too.
Therefore, the  following \emph{verification rule}, which is known as the \emph{weakening rule}, is valid:
\\[0.4cm]
$\bruch{\quad \hoare{F}{P}{G}, \qquad G \rightarrow H \quad}{\hoare{F}{P}{H}}$ 
\\[0.2cm]
The formul\ae\ written over the fraction line are called the \emph{premisses} and the formula under the
fraction line is called the \emph{conclusion}.   The conclusion and the first premiss  are 
Hoare tripels, the second premiss is a formula of first order logic.
The interpretation of this rule is that the conclusion is true if the premisses are true.


\subsection{Compound Statements}
If the program fragments $\texttt{P}$ and $\texttt{Q}$ have the specifications
\\[0.2cm]
\hspace*{1.3cm}
$ \hoare{F_1}{P}{G_1}$  \quad and \quad $\hoare{F_2}{Q}{G_2}$
\\[0.2cm]
and if, furthermore, the postcondition $G_1$ implies the precondition $F_2$,
then the composition $\texttt{P;Q}$ of $\texttt{P}$ and $\texttt{Q}$ satisfies the specification
\\[0.2cm]
\hspace*{1.3cm}
$\hoare{F_1}{P;Q}{G_2}$.
\\[0.2cm]
The reasoning is as follows:  If, initially, $F_1$ is satisfied and we execute $P$ then we have
$G_1$ afterwards.  Therefore we also have $F_2$ and if we now execute $Q$ then afterwards we will have
$G_2$.  This chain of thoughts is combined in the following verification rule:
\\[0.4cm]
\hspace*{1.3cm}
$\bruch{\quad\hoare{F_1}{P}{G_1}, \qquad G_1 \rightarrow F_2, \qquad \hoare{F_2}{Q}{G_2}\quad}{\hoare{F_1}{P;Q}{G_2}} 
$
\\[0.2cm] 
If the formul\ae\ $G_1$ and $F_2$ are identical, then this rule can be simplified as follows:
\\[0.4cm]
\hspace*{1.3cm}
$\bruch{\quad\hoare{F_1}{P}{G_1}, \qquad \hoare{G_1}{Q}{G_2}\quad}{ \hoare{F_1}{P;Q}{G_2}}$


\example
Let us analyse the program fragment shown in Figure \ref{fig:swap}.  
We start our analysis by using the precondition
\\[0.2cm]
\hspace*{1.3cm}
$ \texttt{x} = a \wedge \texttt{y} = b. $
\\[0.2cm]
Here, $a$ and $b$ are two variables that we use to store the initial values of \texttt{x} and
\texttt{y}.  The first assignment yields the Hoare triple
\\[0.2cm]
\hspace*{1.3cm}
$\hoare{\texttt{x} = a \wedge \texttt{y} = b}{x := x - y;}{(\texttt{x} = a \wedge \texttt{y} =  b)\sigma}$
\\[0.2cm] 
where $\sigma = [x \mapsto x + y]$. The form of $\sigma$ follows from the fact that the function $x \mapsto x + y$ is the inverse of the
function $x \mapsto x - y$.  If we apply $\sigma$ to the formula $x = a \wedge y = b$
we get
\begin{equation}
  \label{eq:swap1}
 \hoare{\texttt{x} = a \wedge \texttt{y} = b}{x := x - y;}{\texttt{x + y} = a \wedge \texttt{y} = b}.   
\end{equation}
The second assignment yields the Hoare triple
\\[0.2cm]
\hspace*{1.3cm}
$ \hoare{\texttt{x + y} = a \wedge \texttt{y} = b}{y := y + x;}{(\texttt{x + y} = a \wedge  \texttt{y} = b)\sigma}$
\\[0.2cm] 
where $\sigma = [y \mapsto y - x]$.  The reason is that the function $y \mapsto y - x$ is the inverse of the function $y \mapsto y + x$.
This time, we get
\[ \hoare{\texttt{x + y} = a \wedge \texttt{y} = b}{y := y + x;}{
          \texttt{x + y - x} = a \wedge \texttt{y - x} = b}.
\]
Simplifying the postcondition yields
\begin{equation}
  \label{eq:swap2}
 \hoare{\texttt{x + y} = a \wedge \texttt{y} = b}{y := y + x;}{ \texttt{y} = a \wedge \texttt{y - x} = b}.
\end{equation}
Let us consider the last assignment.  We have
\\[0.2cm]
\hspace*{1.3cm}
$\hoare{\texttt{y} = a \wedge \texttt{y - x} = b}{x := y - x;}{ (\texttt{y} = a \wedge \texttt{y - x} = b)\sigma}$ 
\\[0.2cm] 
where $\sigma = [x \mapsto y - x]$,
since the function $x \mapsto y - x$ is the inverse of the function $x \mapsto y - x$.
This yields
\[ \hoare{\texttt{y} = a \wedge \texttt{y - x} = b}{x := y - x;}{
          \texttt{y} = a \wedge \texttt{y - (y - x)} = b} 
\]
Simplifying the postcondition gives
\begin{equation}
  \label{eq:swap3}
  \hoare{\texttt{y} = a \wedge \texttt{y - x} = b}{x := y - x;}{ \texttt{y} = a \wedge \texttt{x} = b}.   
\end{equation}
Combining the Hoare tripels (\ref{eq:swap1}), (\ref{eq:swap2}) and (\ref{eq:swap3})
we get
\begin{equation}
  \label{eq:swap}
  \hoare{\texttt{x} = a \wedge \texttt{y} = b}{x:=x+y; y:=y+x; x:=y-x;}{ 
         \texttt{y} = a \wedge \texttt{x} = b}.   
\end{equation}
The Hoare tripel (\ref{eq:swap}) shows that the program fragment shown in Figure
\ref{fig:swap} swaps the values of the variables $x$ and $y$: If the value of 
$x$ is $a$ and $y$ has the value $b$ before the program is executed, then afterwards
$y$ has the value $a$ and $x$ has the value $b$.  The trick shown in Figure
\ref{fig:swap} 
can be used to swap variables without using an auxiliary variable.  This is useful because when this
code is compiled into machine language, the resulting code will only use two registers.


\begin{figure}[!ht]
\centering
\begin{Verbatim}[ frame         = lines, 
                  framesep      = 0.3cm, 
                  labelposition = bottomline,
                  numbers       = left,
                  numbersep     = -0.2cm,
                  xleftmargin   = 0.8cm,
                  xrightmargin  = 0.8cm,
                ]
    x := x - y;
    y := y + x;
    x := y - x;
\end{Verbatim}
\vspace*{-0.3cm}
\caption{A tricky way to swap variables.}
\label{fig:swap}
\end{figure}



\subsection{Conditional Statements}
In order to compute the effect of a conditional of the form
\\[0.2cm]
\hspace*{1.3cm}
\texttt{if ($B$) \{ $P$ \} else \{  Q \}}
\\[0.2cm]
let us assume that before the conditional statement is executed, the
precondition $F$ is satisfied.   We have to analyse the effect of the program fragments $P$ and $Q$.
The program fragment $P$ is only executed when $B$ is true.  Therefore, the precondition for $P$ is 
$F \wedge B$.  On the other hand, the precondition for the program fragment $Q$ is $F \wedge \neg B$,
since $Q$ is only executed if $B$ is false.
Hence, we have the following verification rule:
\begin{equation}
  \label{eq:hoareIf}
  \bruch{\quad\hoare{F \wedge B}{P}{G}, \qquad \hoare{F \wedge \neg B}{Q}{G}\quad}{
              \hoare{F}{if ($B$) P else Q}{G}}  
\end{equation}
In this form, the rule is not always applicable.  The reason is that the analysis of the program
fragments \texttt{P} and \texttt{Q} yields Hoare tripel of the form
\begin{equation}
  \label{eq:hoareI}
 \hoare{F \wedge B}{P}{G_1} \qquad \mathrm{and} \qquad \hoare{F \wedge \neg B}{Q}{G_2},   
\end{equation}
and in general $G_1$ and $G_2$ will be different from each other.  In order to be able to apply the
rule for conditionals we have to find a formula $G$ that is a consequence of $G_1$ and also a
consequence of $G_2$, i.~e.~we want to have
\[ G_1 \rightarrow G \qquad \mathrm{and} \qquad G_2 \rightarrow G. \]
If we find $G$, then the weakening rule  can be applied to conclude the validity of
\[ \hoare{F \wedge B}{P}{G} \qquad \mathrm{and} \qquad \hoare{F \wedge \neg B}{Q}{G},    \] 
and this gives us the premisses that are needed for the rule
(\ref{eq:hoareIf}).

\example
Let us analyze the following program fragment:
\\[0.2cm]
\hspace*{1.3cm}
\texttt{if (x < y) \{ z := x; \} else \{ z := y; \}}
\\[0.2cm]
We start with the precondition
\\[0.2cm]
\hspace*{1.3cm}
$F = \bigl(\texttt{x} = a \wedge \texttt{y} = b\bigr)$
\\[0.2cm]
and want to show that the execution of the conditional establishes the postcondition
\\[0.2cm]
\hspace*{1.3cm}
$G = \bigl(z = \textsl{min}(a, b)\bigr)$.
\\[0.2cm]
The first assignment ``\texttt{z := x;}'' gives the Hoare tripel 
\\[0.2cm]
\hspace*{1.3cm}
$\hoare{\texttt{x} = a \wedge \texttt{y} = b \wedge \texttt{x} < \texttt{y}}{z := x}{
          \texttt{x} = a \wedge \texttt{y} = b \wedge \texttt{x} < \texttt{y} \wedge \texttt{z} = \texttt{x}}
$.
\\[0.2cm]
In the same way, the second assignment ``\texttt{z := y}'' yields
\\[0.2cm]
\hspace*{1.3cm}
$\hoare{\texttt{x} = a \wedge \texttt{y} = b \wedge \texttt{x} \geq \texttt{y}}{z := y}{
          \texttt{x} = a \wedge \texttt{y} = b \wedge \texttt{x} \geq \texttt{y} \wedge
          \texttt{z} = \texttt{y}}$.
\\[0.2cm]
Since we have
\\[0.2cm]
\hspace*{1.3cm}
$\texttt{x} = a \wedge \texttt{y} = b \wedge \texttt{x} < \texttt{y} \wedge \texttt{z} = \texttt{x}
   \rightarrow \texttt{z} = \min(a,b)$
\\[0.2cm]
and also
\\[0.2cm]
\hspace*{1.3cm}
$ \texttt{x} = a \wedge \texttt{y} = b \wedge \texttt{x} \geq \texttt{y} \wedge \texttt{z} = \texttt{y} 
   \rightarrow \texttt{z} = \min(a,b)
$.
\\[0.2cm]
Using the weakening rule we conclude that 
\begin{eqnarray*}
\hoare{\texttt{x} = a \wedge \texttt{y} = b \wedge \texttt{x} < \texttt{y}}{z := x;}{
       \texttt{z} = \min(a,b)} & & \mathrm{and} \\
\hoare{\texttt{x} = a \wedge \texttt{y} = b \wedge \texttt{x} \geq \texttt{y}}{z := y;}{
          \texttt{z} = \min(a,b)}
\end{eqnarray*}
holds.  Now we can apply the rule for the conditional and conclude
that 
\\[0.2cm]
$ \hoare{\texttt{x} = a \wedge \texttt{y} = b}{if (x < y) \{ z := x; \} else \{ z := y; \}}{ \texttt{z} = \min(a,b)} $
\\[0.2cm]
holds.  Thus we have shown that the program frgament above computes
the minimum of the numbers $a$ and $b$.

\subsection{Loops}
Finally, let us analyze the effect of a loop of the form
\\[0.2cm]
\hspace*{1.3cm}
\texttt{while ($B$) \{ P \}} 
\\[0.2cm]
The important point here is that the postcondition of the $n$-th
execution of the body of the loop $P$ is the precondition of the $(n\!+\!1)$-th
execution of $P$.  Basically this means that the precondition and the
postcondition of $P$ have to be more or less the same.
Hence, this condition is called the \emph{loop invariant}.  Therefore,
the details of the verification rule for \texttt{while} loops are as follows:
\\[0.2cm]
\hspace*{1.3cm}
$\bruch{\hoare{I \wedge B}{P}{I}}{\quad \hoare{I}{while ($B$) \{ P \}}{I \wedge \neg B}\quad}$
\\[0.2cm]
The premiss of this rules expresses the fact that the invariant $I$ remains valid on execution of $P$.
However, since $P$ is only executed as long as $B$ is \texttt{true}, the precondition for $P$ is actually 
the formula $I \wedge B$.  The conclusion of the rule says that if the invariant $I$ is true before
the loop is executed, then $I$ will be true after the loop has finished.  This result is intuitive
since every time $P$ is executed $I$ remains valid.  Furthermore, the loop only terminates once $B$
gets \texttt{false}.  Therefore, the postcondition of the loop can be strengthened by adding $\neg B$.



\section{The Euclidean  Algorithm}
In this section we show how the verification rules of the last section can be used to prove the
correctness of a non-trivial program.
We will show that the algorithm shown in Figure \ref{fig:ggt.stlx} on page \pageref{fig:ggt.stlx} is correct.
The procedure shown in this figure implements the
\href{http://en.wikipedia.org/wiki/Euclidean_algorithm}{\emph{Euclidean algorithm}}
to compute the greatest common divisor of two natural numbers.  Our proof is based on the following
property of the function \texttt{ggt}:
\\[0.2cm]
\hspace*{1.3cm}
$\texttt{ggt}(x + y, y) = \texttt{ggt}(x,y) \quad \mbox{for all $x, y \in \mathbb{N}$}$.


\begin{figure}[!ht]
\centering
\begin{Verbatim}[ frame         = lines, 
                  framesep      = 0.3cm, 
                  labelposition = bottomline,
                  numbers       = left,
                  numbersep     = -0.2cm,
                  xleftmargin   = 0.8cm,
                  xrightmargin  = 0.8cm,
                ]
    ggt := procedure(x, y) {
        while (x != y) {
            if (x < y) {
                y := y - x;
            } else {
                x := x - y;
            }
        }
        return x;
    };
\end{Verbatim}
\vspace*{-0.3cm}
\caption{The Euclidean Algorithm to compute the greatest common divisor.}
\label{fig:ggt.stlx}
\end{figure}

\subsection{Correctness Proof of the Euclidean Algorithm} 
To start our correctness proof we formulate the invariant of the \texttt{while} loop.
Let us define
\\[0.2cm]
\hspace*{1.3cm}
$ I := \bigl(x > 0 \wedge y > 0 \wedge \texttt{ggt}(x,y) = \texttt{ggt}(a,b) \bigr)$
\\[0.2cm]
In this formula we have defined the initial values of $x$ and $y$ as $a$ and $b$.
In order to establish the invariant at the beginning we have to ensure that the function
\texttt{ggt} is only called with positive natural numbers.  If we denote these numbers as 
$a$ and $b$, then the invariant $I$ is valid initially.  The reason is that $x = a$ and $y = b$
implies $\texttt{ggt}(x,y) = \texttt{ggt}(a,b)$.


In order to prove that the invariant $I$ is maintained in the loop we formulate the Hoare triples
for both alternatives of the conditional.  For the first conditional we know that
\\[0.2cm]
\hspace*{1.3cm}
$\hoare{I \wedge x \not= y \wedge x < y}{y := y - x;}{(I \wedge x \not= y \wedge x < y)\sigma}$
\\[0.2cm]
holds, where $\sigma$ is defined as $\sigma = [y \mapsto y + x]$.  Here, the condition $x \not= y$
is the condition controlling the execution of the  
\texttt{while} loop and the condition $x < y$ is the condition of the \texttt{if} conditional.
We rewrite the formula $(I \wedge x \not= y \wedge x < y)\sigma$:
\begin{eqnarray*}
 &                 & \bigl(I \wedge x \not= y \wedge x < y\bigr)\sigma \\
 & \leftrightarrow & \bigl(I \wedge x < y\bigr)\sigma 
                     \qquad\qquad \mbox{because $x < y$ implies $x \not= y$} \\
 & \leftrightarrow & \bigl(x > 0 \wedge y > 0 \wedge \texttt{ggt}(x,y) = \texttt{ggt}(a,b) \wedge 
                     x < y\bigr)[y \mapsto y + x] \\
 & \leftrightarrow & x > 0 \wedge y + x > 0 \wedge \texttt{ggt}(x,y+x) = \texttt{ggt}(a,b)
                     \wedge x < y + x \\
 & \leftrightarrow & x > 0 \wedge y + x > 0 \wedge \texttt{ggt}(x,y) = \texttt{ggt}(a,b) \wedge 0 < y
\end{eqnarray*}
In the last step we have used the formula
\\[0.2cm]
\hspace*{1.3cm}
$ \texttt{ggt}(x,y+x) = \texttt{ggt}(x,y) $
\\[0.2cm]
and we have simplified the inequality $x < y + x$ as $0 < y$.
The last formula implies
\\[0.2cm]
\hspace*{1.3cm}
$ x > 0 \wedge y > 0 \wedge \texttt{ggt}(x,y) = \texttt{ggt}(a,b) $.
\\[0.2cm]
However, this is precisely the invariant $I$.  Therefore we have shown that
\begin{equation}
  \label{eq:if1}
  \hoare{I \wedge x \not= y \wedge x < y}{y := y - x;}{I}
\end{equation}
holds.  Next, let us consider the second alternative of the \texttt{if} conditional.  We have
\\[0.2cm]
\hspace*{1.3cm}
$\hoare{I \wedge x \not= y \wedge x \geq y}{x := x - y;}{(I \wedge x \not= y \wedge x \geq y)\sigma}$ 
\\[0.2cm]
where $\sigma = [x \mapsto x + y]$. The expression $(I \wedge x \not= y \wedge x \geq y)\sigma$ is rewritten as follows:
\begin{eqnarray*}
 &   & \bigl(I \wedge x \not= y \wedge x \geq y\bigr)\sigma \\
 & \leftrightarrow & \bigl(I \wedge x > y \bigr)\sigma \\
 & \leftrightarrow & \bigl(x > 0 \wedge y > 0 \wedge \texttt{ggt}(x,y) = \texttt{ggt}(a,b) \wedge 
             x > y \bigr)[x \mapsto x + y] \\
 & \leftrightarrow & x + y > 0 \wedge y > 0 \wedge \texttt{ggt}(x+y,y) = \texttt{ggt}(a,b) 
       \wedge x + y > y  \\
 & \leftrightarrow & x + y > 0 \wedge y > 0 \wedge \texttt{ggt}(x,y) = \texttt{ggt}(a,b) 
       \wedge x > 0 
\end{eqnarray*}
The last formula implies that
\\[0.2cm]
\hspace*{1.3cm}
$ x > 0 \wedge y > 0 \wedge \texttt{ggt}(x,y) = \texttt{ggt}(a,b). $
\\[0.2cm]
holds.  Again, this is our invariant $I$.  Therefore we have shown that
\begin{equation}
  \label{eq:if2}
  \hoare{I \wedge x \not= y \wedge x \geq y}{x := x - y;}{I} 
\end{equation}
holds.  If we use the Hoare triples (\ref{eq:if1}) and (\ref{eq:if2}) as premisses
for the rule for conditionals we have shown that
\\[0.2cm]
\hspace*{1.3cm}
$  \hoare{I \wedge x \not= y}{if (x < y) \{ y := y - x; \} else \{  x := x - y; \}}{I} $
\\[0.2cm]
holds.  Now the verification rule for \texttt{while} loops yields
\pagebreak

\noindent
\hspace*{1.3cm} 
$\{ I \}$
\\[0.1cm]
\hspace*{2.2cm}
\texttt{while (x != y ) \{} \\[0.1cm]
\hspace*{3.2cm}
         \texttt{if (x < y) \{ x := x - y; \} else \{ y := y - x;\}}
\\[0.1cm]
\hspace*{2.2cm}
\texttt{\}} \quad 
\\[0.1cm]
\hspace*{1.3cm}
$\{ I \wedge x = y \}$. 
\\[0.2cm]
Expanding the invariant $I$ in the formula $I \wedge x = y$ shows that the postcondition of the 
\texttt{while} loop is given as
\\[0.2cm]
\hspace*{1.3cm}
$x > 0 \wedge y > 0 \wedge \texttt{ggt}(x,y) = \texttt{ggt}(a,b) \wedge x = y$.
\\[0.2cm]
Now the correctness of the Euclidean algorithm can be established as follows:
\begin{eqnarray*}
&             &  x > 0 \wedge y > 0 \wedge \texttt{ggt}(x,y) = \texttt{ggt}(a,b) \wedge x = y \\
& \Rightarrow & \texttt{ggt}(x,y) = \texttt{ggt}(a,b) \wedge x = y \\
& \Rightarrow & \texttt{ggt}(x,x) = \texttt{ggt}(a,b)  \\
& \Rightarrow & x = \texttt{ggt}(a,b) \qquad\qquad \mathrm{because} \quad \texttt{ggt}(x,x) = x.
\end{eqnarray*}
All in all we have shown the following: If the \texttt{while} loop terminates, then
the variable $x$ will be set to the greatest common divisor of $a$ and $b$, where $a$ and $b$ are
the initial values of the variables $x$ and $y$.  In order to finish our correctness proof we have
to show that the while does indeed terminate for all choices of $a$ and $b$.
To this end let us define the variable $s$ as follows:
\\[0.2cm]
\hspace*{1.3cm}
$ s := x + y. $
\\[0.2cm]
The variables $x$ and $y$ are natural numbers.  Therefore $s$ is a natural number, too.
Every iteration of the loop reduces the number $s$: either $x$ is subtracted from $s$
or $y$ is subtracted from  $s$ and the invariant $I$ shows that both $x$ and $y$ are
positive.  Therefore, if the \texttt{while} loop would run forever, at some point $s$ would get
negative.  Since $s$ can not be negative, the loop must terminate.
Hence we have shown the correctness of the Euclidean  algorithm.
\pagebreak

\exercise
Show that the function $\texttt{power}(x,y)$ that is defined in Figure
\ref{fig:power-iterative.stlx} does compute $x^y$, i.~e.~show that $\texttt{power}(x,y) = x^y$ 
for all natural numbers $x$ and $y$.


\begin{figure}[!h]
\centering
\begin{Verbatim}[ frame         = lines, 
                  framesep      = 0.3cm, 
                  labelposition = bottomline,
                  numbers       = left,
                  numbersep     = -0.2cm,
                  xleftmargin   = 1.3cm,
                  xrightmargin  = 1.3cm,
                ]
    power := procedure(x, y) {
        r := 1;
        while (y > 0) {
            if (y % 2 == 1) {
                r := r * x;
            }
            x := x * x;
            y := y \ 2;
        }
        return r;
    };
\end{Verbatim}
\vspace*{-0.3cm}
\caption{A program to compute $x^y$ iteratively.}
\label{fig:power-iterative.stlx}
\end{figure}


\noindent
\textbf{Hints}: 
\begin{enumerate}
\item If the initial values of $x$ and $y$ are called $a$ and $b$,
      then an invariant for the \texttt{while} loop is given as 
      \\[0.2cm]
      \hspace*{1.3cm}
      $I := \bigl(r \cdot x^y = a^b\bigr)$.
\item The verification rule for the conditional without \texttt{else} is given as
      \\[0.4cm]
      \hspace*{1.3cm}
      $\bruch{\quad\hoare{F \wedge B}{P}{G}, \qquad F \wedge \neg B \rightarrow G\quad}{
                     \hoare{F}{if ($B$) \{ P \}}{G}}
      $
      \\[0.2cm]
      This rule is interpreted as follows:
      \begin{enumerate}
      \item If both the precondition $F$ and the condition $B$ is valid, then execution of the
            program fragment $P$ has to establish the validity of the postcondition $G$.
      \item If the precondition $F$ is valid but we have $\neg B$, then this must imply the postcondition
            $G$.
      \end{enumerate}
\end{enumerate}

\remark
Proving the correctness of a nontrivial program is very tedious.
Therefore, various attempts have been made to automate the task.  For
example, \href{http://www.key-project.org/download/hoare/}{\emph{KeY Hoare}}
is a tool that can be used to verify the correctness of programs.  It is based on
Hoare calculus.


\section{Symbolic  Program Execution}
The last section has shown that using Hoare logic to verify a program can be very tedious.
There is another method to prove the correctness of imperative programs.  This method is called
\emph{symbolic program execution}.  Let us demonstrate this method.  Consider the program
shown in Figure \ref{fig:power-iterative-annotated.stlx}.


\begin{figure}[!h]
\centering
\begin{Verbatim}[ frame         = lines, 
                  framesep      = 0.3cm, 
                  labelposition = bottomline,
                  numbers       = left,
                  numbersep     = -0.2cm,
                  xleftmargin   = 1.3cm,
                  xrightmargin  = 1.3cm,
                  codes         = {\catcode`_=8\catcode`$=3},
                  commandchars  = \\\{\},
                ]
    power := procedure(x$_0$, y$_0$) \{
        r$_0$ := 1;
        while (y$_n$ > 0) \{
            if (y$_n$ % 2 == 1) \{
                r$_{n+1}$ := r$_n$ * x$_n$;
            \} 
            x$_{n+1}$ := x$_n$ * x$_n$;
            y$_{n+1}$ := y$_n$ \symbol{92} 2;            
        \} 
        return r$_N$;
    \};
\end{Verbatim}
\vspace*{-0.3cm}
\caption{An annotated programm to compute powers.}
\label{fig:power-iterative-annotated.stlx}
\end{figure} % $

The main difference between a mathematical formula and a program is that in a formula all
occurrences of a variable refer to the same value.   This is different in a program because the
variables change their values dynamically.  In order to deal with this property of program variables
we have to be able to distinguish the different occurrences of a variable.  To this end,  we 
index the program variables. 
When doing this we have to be aware of the fact that the same occurrence of a program variable can
still denote different values if the variable occurs inside  a loop.  In this case we have to index
the variables in a way that the index includes a counter that counts the number of loop iterations.
For concreteness, consider the  program shown in 
Figure \ref{fig:power-iterative-annotated.stlx}.  
Here, in line 5 the variable \texttt{r} has the index $n$ on the right side of the assignment,
while it has the index $r_{n+1}$ on the left side of the assignment in line 5.  Here, $n$ denotes 
the number of times the \texttt{while} loop has been iterated.
After the loop in line 10 the variable is indexed as
$\texttt{r}_N$, where $N$ denotes the total number of loop iterations.
We show the correctness of the given program next.  Let us define
\\[0.2cm]
\hspace*{1.3cm}
$ a := x_0, \quad b := y_0$.
\\[0.2cm]
We show, that the \texttt{while} loop satisfies the invariant
\begin{equation}
  \label{eq:powerInv}
   r_n \cdot x_n^{y_n} = a^b.
\end{equation}
This claim is proven by induction on the number of loop iterations.
\begin{enumerate}
\item[I.A.] $n=0$: Since we have $r_0 = 1$, $x_0 = a$, and $y_0 = b$ we have 
            \\[0.2cm]
            \hspace*{1.3cm}
            $r_n \cdot x_n^{y_n} = r_0 \cdot x_0^{y_0} = 1 \cdot a^{b} = a^b$.
\item[I.S.] $n \mapsto n + 1$:  We need a case distinction with respect to $y \mod 2$:
            \begin{enumerate}
            \item $y_n \mod 2 = 1$.  Then we have $y_{n} = 2 \cdot (y_n\symbol{92}2) + 1$.
                  This implies
                  \begin{eqnarray*}
                      &   & r_{n+1} \cdot x_{n+1}^{y_{n+1}} \\[0.2cm] 
                      & = & (r_{n} \cdot x_n) \cdot (x_{n} \cdot x_{n})^{y_{n}\symbol{92}2} \\[0.2cm] 
                      & = & r_{n} \cdot x_n \cdot x_{n}^{y_{n}\symbol{92}2} \cdot x_{n}^{y_{n}\symbol{92}2} \\[0.2cm] 
                      & = & r_{n} \cdot x_n^{1 + y_{n}\symbol{92}2 + y_{n}\symbol{92}2} \\[0.2cm] 
                      & = & r_{n} \cdot x_n^{2 \cdot (y_{n}\symbol{92}2) + 1} \\[0.2cm] 
                      & = & r_{n} \cdot x_n^{y_n} \\
                      & \stackrel{IH}{=} & a^{b} 
                  \end{eqnarray*}
\pagebreak

            \item $y_n \mod 2 = 0$.  Then we have $y_{n} = 2 \cdot (y_n\symbol{92}2)$.
                  This implies
                  \begin{eqnarray*}
                      &   & r_{n+1} \cdot x_{n+1}^{y_{n+1}} \\[0.2cm] 
                      & = & r_{n} \cdot (x_{n} \cdot x_{n})^{y_{n}\symbol{92}2} \\[0.2cm] 
                      & = & r_{n} \cdot x_{n}^{y_{n}\symbol{92}2} \cdot x_{n}^{y_{n}\symbol{92}2} \\[0.2cm] 
                      & = & r_{n} \cdot x_n^{y_{n}\symbol{92}2 + y_{n}\symbol{92}2} \\[0.2cm] 
                      & = & r_{n} \cdot x_n^{2 \cdot (y_{n} \symbol{92}  2)} \\[0.2cm] 
                      & = & r_{n} \cdot x_n^{y_n} \\
                      & \stackrel{IH}{=} & a^{b} 
                  \end{eqnarray*}
            \end{enumerate}
\end{enumerate}
This shows the validity of the equation (\ref{eq:powerInv}).   If the \texttt{while} loop
terminates, we must have $y_N = 0$.  If $n=N$ equation (\ref{eq:powerInv}) yields:
\[ r_N \cdot x_N^{y_N} = x_0^{y_0} 
  \leftrightarrow  r_N \cdot x_N^{0}   = a^b
  \leftrightarrow  r_N \cdot 1         = a^b
  \leftrightarrow  r_N                 = a^b
\]
This shows $r_N = a^b$ and since we already know that the
\texttt{while} loop terminates we have proven that
$\texttt{power}(a,b) =a^b$.

\exercise
Use the method of symbolic program execution to prove the correctness of the implementation of the
Euclidean algorithm that is shown in Figure \ref{fig:gcd.stlx}.  During the proof you should make
use of the fact that for all positive natural numbers $a$ and $b$ the equation
\\[0.2cm]
\hspace*{1.3cm}
$\mathtt{ggt}(a, b) = \mathtt{ggt}(a \,\texttt{\%}\, b, b)$
\\[0.2cm]
is valid.  

\begin{figure}[!ht]
\centering
\begin{Verbatim}[ frame         = lines, 
                  framesep      = 0.3cm, 
                  firstnumber   = 1,
                  labelposition = bottomline,
                  numbers       = left,
                  numbersep     = -0.2cm,
                  xleftmargin   = 0.8cm,
                  xrightmargin  = 0.8cm,
                ]
    ggt := procedure(a, b) {
        while (b != 0) {
            r := a % b;
            a := b;
            b := r;
        }
        return a;
    };
\end{Verbatim}
\vspace*{-0.3cm}
\caption{An efficient version of the Euclidean algorithm.}
\label{fig:gcd.stlx}
\end{figure}


%%% Local Variables: 
%%% mode: latex
%%% TeX-master: "algorithms"
%%% End: 
