\documentclass{article}
\usepackage{a4wide}
\usepackage{german}
\usepackage{amssymb}

\newtheorem{Definition}{Definition}
\newtheorem{Notation}[Definition]{Notation}
\newtheorem{Korollar}[Definition]{Korollar}
\newtheorem{Lemma}[Definition]{Lemma}
\newtheorem{Satz}[Definition]{Satz}
\newtheorem{Theorem}[Definition]{Theorem}


\title{Beweis der Korrektheit von \texttt{minSort}}
\begin{document}
\maketitle

\section{Die C--Implementierung}
\noindent
Der Algorithmus \texttt{minSort} kann wie folgt in \texttt{C} implementiert werden:
\begin{verbatim}
NodePtr getMin(List l) {
    assert(l != 0);
    Data first  = l->data;
    if (l->nextPtr == 0) {
        return l;
    }
    NodePtr min_rest = getMin(l->nextPtr);
    Data    rest = min_rest->data;
    int cmp = compareData(first, rest);
    if (cmp <= 0) {
        return l;
    } else {
        return min_rest;
    }
}
List delete(List l, NodePtr ptr) {
    if (l == ptr) {
        return ptr->nextPtr;
    }
    l->nextPtr = delete(l->nextPtr, ptr);
    return l;
}
List cons(NodePtr ptr, List l) {
    ptr->nextPtr = l;
    return ptr;
}
List minSort(List l1) {
    List l2, l3, l4;
    if (l1 == 0)
        return l1;
    NodePtr m = getMin(l1);
    l2 = delete(l1, m);
    l3 = minSort(l2);
    l4 = cons(m, l3);
    return l4;
}
\end{verbatim}

\section{Überf\"uhrung in bedingte Gleichungen}
Um den Algorithmus verifizieren zu k\"onnen, \"uberf\"uhren wir ihn zun\"achst in bedingte
Gleichungen:

\begin{enumerate}
\item Transformation von \texttt{minSort}:
    \begin{enumerate}
    \item $\mathtt{minSort}([]) = []$
    \item $l \not=[] \rightarrow \mathtt{minSort}(l) \;=\; [\; \mathtt{getMin}(l) \;|\; \mathtt{minSort}(\,\mathtt{delete}(l, \mathtt{getMin}(l))\,)\;]$
    \end{enumerate}
\item Transformation von \texttt{delete}:
    \begin{enumerate}
    \item $\texttt{delete}([],d) = []$
    \item $d_1 = d_2 \rightarrow \mathtt{delete}([d_1|r_1], d_2) = r_1$
    \item $d_1 \not= d_2 \rightarrow \mathtt{delete}([d_1|r_1], d_2) = [\;d_1 \;|\; \mathtt{delete}(r_1, d_2)\;]$
    \end{enumerate}
\item Transformation von \texttt{getMin}:
    \begin{enumerate}
    \item $\mathtt{getMin}([d]) = d$
    \item $d \leq \mathtt{getMin}(r) \rightarrow \mathtt{getMin}([d|r]) = d$
    \item $d >    \mathtt{getMin}(r) \rightarrow \mathtt{getMin}([d|r]) = \mathtt{getMin}(r)$
    \end{enumerate}
\item Definition: $\leq : \textsl{Data} \times \textsl{List} \rightarrow \mathbb{B}$
    \begin{enumerate}
    \item $x \leq []$
    \item $x \leq [d|r] \;\leftrightarrow\; x \leq d \wedge x \leq r$
    \end{enumerate}
\item Definition von $\texttt{ordered}: \textsl{List} \rightarrow \mathbb{B}$
\begin{enumerate}
\item $\mathtt{ordered}([]) \leftrightarrow \mathtt{true}$
\item $\mathtt{ordered}([d|r]) \leftrightarrow d \leq r \wedge \mathtt{ordered}(r)$
\end{enumerate}
\end{enumerate}

\section{Nachweis der Korrektheit}
Um den Nachweis der Korrektheit erbringen zu k\"onnen, brauchen wir eine Reihe von
Hilfss\"atzen.
\begin{Lemma}[Transitivit\"at] \label{l1}
F\"ur beliebige Daten $x$, $y$ und Listen $l$ gilt: \\[0.1cm]
\hspace*{1.3cm} $x \leq y \wedge y \leq l \rightarrow x \leq l$.
\end{Lemma}
\textbf{Beweis}:  Wir f\"uhren den Beweis durch Induktion \"uber die L\"ange der Liste $l$.
\begin{enumerate}
\item Fall: $l = []$

      Zu zeigen ist $x \leq []$. Dies folgt unmittelbar aus der Definition von $\leq$.
\item Fall: $l = [d|r]$

      Als Voraussetzung haben wir \\[0.1cm]
      \hspace*{1.3cm} $x \leq y$ \quad und \hspace*{\fill} (1) \\[0.1cm]
      \hspace*{1.3cm} $y \leq [\,d\,|\,r\,]$. \hspace*{\fill} (2) \\[0.1cm]
      Aus (2) folgt nach Definition von $\leq$ \\[0.1cm]
      \hspace*{1.3cm} $y \leq d$ \quad und \hspace*{\fill} (3) \\[0.1cm]
      \hspace*{1.3cm} $y \leq r$. \hspace*{\fill} (4) \\[0.1cm]
      Aus (1) und (3) folgt \\[0.1cm]
      \hspace*{1.3cm} $x \leq d$ \hspace*{\fill} (5) \\[0.1cm]
      Aus (1) und (4) folgt nach Induktions--Voraussetzung \\[0.1cm]
      \hspace*{1.3cm} $x \leq r$. \hspace*{\fill} (6) \\[0.1cm]
      Aus (5) und (6) folgt nach Definition von $\leq$ \\[0.1cm]
      \hspace*{1.3cm} $y \leq [\,d\,|\,r\,]$. \hspace*{\fill} $_\Box$ \\[0.1cm]
\end{enumerate}

\begin{Lemma}[Vertr\"aglichkeit von \texttt{getMin} und $\leq$] \label{l2}
F\"ur beliebige Daten $x$ und Listen $l \not= []$ gilt: \\[0.1cm]
\hspace*{1.3cm} $x \leq l \leftrightarrow x \leq \mathtt{getMin}(l)$
\end{Lemma}
\textbf{Beweis}:  Wir f\"uhren den Beweis durch Induktion \"uber die L\"ange der Liste $l$.
\begin{enumerate}
\item Fall: $l = [d]$
    
      Wir m\"ussen zeigen \\[0.1cm]
      \hspace*{1.3cm} $x \leq [d] \leftrightarrow x \leq \mathtt{getMin}([d])$ \\[0.1cm]
      Wegen $\mathtt{getMin}([d]) = d$ ist dies \"aquivalent zu \\[0.1cm]
      \hspace*{1.3cm} $x \leq [d] \leftrightarrow x \leq d$ \\[0.1cm]
      Dies folgt unmittelbar aus der Definition von $\leq$.
\item Fall: $l = [\,d\,|\,r\,]$ mit $r \not = []$ und $d \leq \mathtt{getMin}(r)$

      Zu zeigen ist $x \leq [\,d\,|\,r\,] \leftrightarrow x \leq \mathtt{getMin}([\,d\,|\,r\,])$.

      Wir nehmen zun\"achst an, dass $x \leq [\,d\,|\,r\,]$ gilt und zeigen, dass daraus \\[0.1cm]
      \hspace*{1.3cm} $x \leq \mathtt{getMin}([\,d\,|\,r\,])$ \\[0.1cm]
      folgt.  Aus $x \leq [\,d\,|\,r\,]$ folgt \\[0.1cm]
      \hspace*{1.3cm} $x \leq d$.   \hspace*{\fill} (1) \\[0.1cm]
      Wegen $d \leq \mathtt{getMin}(r)$ gilt $\mathtt{getMin}([\,d\,|\,r\,]) = d$ und wegen (1)
      folgt daraus \\[0.1cm]
      \hspace*{1.3cm} $x \leq \mathtt{getMin}([\,d\,|\,r\,])$.
     
      Wir nehmen nun an, dass  $x \leq \mathtt{getMin}([\,d\,|\,r\,])$ gilt und zeigen,
      dass daraus \\[0.1cm]
      \hspace*{1.3cm}  $x \leq [\,d\,|\,r\,]$ \\[0.1cm]
      folgt.  Aus $\mathtt{getMin}([\,d\,|\,r\,]) = d$ und der Annahme
        $x \leq \mathtt{getMin}([\,d\,|\,r\,])$ folgt \\[0.1cm]
       \hspace*{1.3cm} $x \leq d$. \hspace*{\fill} (2) \\[0.1cm]
      Aus $x\leq d$ und $d \leq \mathtt{getMin}(r)$  folgt       \\[0.1cm] 
      \hspace*{1.3cm} $x \leq \mathtt{getMin}(r)$. \hspace*{\fill} (3) \\[0.1cm]
      Aus $(3)$ folgt mit der Induktions--Voraussetzung dann \\[0.1cm]
      \hspace*{1.3cm} $x \leq r$. \hspace*{\fill} (4) \\[0.1cm]
      Aus (2) und (4) folgt nun \\[0.1cm]
      \hspace*{1.3cm} $x \leq [\,d\,|\,r\,]$.

\item Fall: $l = [\,d\,|\,r\,]$ mit $r \not = []$ und $d > \mathtt{getMin}(r)$

      Zu zeigen ist $x \leq [\,d\,|\,r\,] \leftrightarrow x \leq \mathtt{getMin}([\,d\,|\,r\,])$.

      Wir nehmen zun\"achst an, dass $x \leq [\,d\,|\,r\,]$ gilt und zeigen, dass daraus \\[0.1cm]
      \hspace*{1.3cm} $x \leq \mathtt{getMin}([\,d\,|\,r\,])$ \\[0.1cm]
      folgt.  Aus $x \leq [\,d\,|\,r\,]$ folgt nach Definition von $\leq$ \\[0.1cm]
      \hspace*{1.3cm} $x \leq r$.   \hspace*{\fill} (1) \\[0.1cm]
      Wegen $r\not=[]$ folgt daraus mit Induktions--Voraussetzung \\[0.1cm]
      \hspace*{1.3cm} $x \leq \mathtt{getMin}(r)$. \hspace*{\fill} (2) \\[0.1cm]
      Wegen $d > \mathtt{getMin}(r)$ gilt $\mathtt{getMin}([\,d\,|\,r\,]) = \mathtt{getMin}(r)$ und wegen (2)
      folgt daraus \\[0.1cm]
      \hspace*{1.3cm} $x \leq \mathtt{getMin}([\,d\,|\,r\,])$.
     
      Wir nehmen nun an, dass  $x \leq \mathtt{getMin}([\,d\,|\,r\,])$ gilt und zeigen,
      dass daraus \\[0.1cm]
      \hspace*{1.3cm}  $x \leq [\,d\,|\,r\,]$  \\[0.1cm]
      folgt.  Aus $\mathtt{getMin}([\,d\,|\,r\,]) = \mathtt{getMin}(r)$ und der Annahme
        $x \leq \mathtt{getMin}([\,d\,|\,r\,])$ folgt \\[0.1cm]
       \hspace*{1.3cm} $x \leq \mathtt{getMin}(r)$. \hspace*{\fill} (2) \\[0.1cm]
      Aus  $x \leq \mathtt{getMin}(r)$ und $r \not= []$ folgt wegen der
      Induktions--Voraussetzung \\[0.1cm]
      \hspace*{1.3cm} $x \leq r$. \hspace*{\fill} (3) \\[0.1cm]
      Aus der Voraussetzung $d > \mathtt{getMin}(r)$ folgt mit (2) sofort \\[0.1cm]
      \hspace*{1.3cm} $x \leq d$. \hspace*{\fill} (4) \\[0.1cm]
      Aus (4) und (3) folgt nun \\[0.1cm]
      \hspace*{1.3cm} $x \leq [\,d\,|\,r\,]$. \hspace*{\fill} $_\Box$
\end{enumerate}

\begin{Lemma}[Minimalit\"at von \texttt{getMin}] \label{l3}
Falls $l \not= []$ ist, so gilt $\mathtt{getMin}(l) \leq l$.
\end{Lemma}
\textbf{Beweis}:  Wir f\"uhren den Beweis durch Induktion \"uber die L\"ange der Liste $l$.
\begin{enumerate}
\item $l = [d]$

      Wegen $\mathtt{getMin}([d]) = d$ ist die Behauptung dann \"aquivalent zu \\[0.1cm]
      \hspace*{1.3cm} $d \leq [d]$ \\[0.1cm]
      und das folgt sofort aus der Definition von $\leq$.
\item $l = [\,d\,|\,r\,]$, $r \not= []$ und $d \leq \mathtt{getMin}(r)$

      Aus $d \leq \mathtt{getMin}(r)$ folgt mit Lemma \ref{l2} \\[0.1cm]
      \hspace*{1.3cm} $d \leq r$. \hspace*{\fill} (1) \\[0.1cm]
      Wegen $d = \mathtt{getMin}([\,d\,|\,r\,])$ und $d \leq d$ folgt daraus sofort \\[0.1cm]
      \hspace*{1.3cm} $\mathtt{getMin}([\,d\,|\,r\,]) \leq [\,d\,|\,r\,]$.
\item $l = [\,d\,|\,r\,]$, $r \not= []$ und $d > \mathtt{getMin}(r)$

     Nach Induktions--Voraussetzung gilt \\[0.1cm]
     \hspace*{1.3cm} $\mathtt{getMin}(r) \leq r$. \hspace*{\fill} (1) \\[0.1cm]
     Wegen $d > \mathtt{getMin}(r)$ gilt erst recht \\[0.1cm]
     \hspace*{1.3cm} $\mathtt{getMin}(r) \leq d$. \hspace*{\fill} (2) \\[0.1cm]
     Aus (1) und (2) folgt \\[0.1cm]
     \hspace*{1.3cm} $\mathtt{getMin}(r) \leq [\,d\,|\,r\,]$. \hspace*{\fill}  \\[0.1cm]
     Wegen $\mathtt{getMin}([\,d\,|\,r\,]) = \mathtt{getMin}(r)$ ist das die Behauptung. \hspace*{\fill}  $_\Box$
\end{enumerate}

\begin{Lemma}[Vertr\"aglichkeit von \texttt{delete} und $\leq$] 
\label{l4}
F\"ur beliebige Daten $x$, $y$   und Listen $l$ gilt: \\[0.1cm]
\hspace*{1.3cm} $x \leq l \rightarrow x \leq \mathtt{delete}(l,y)$
\end{Lemma}
\textbf{Beweis} durch Induktion \"uber die L\"ange der Liste $l$:
\begin{enumerate}
\item Fall: $l = []$

      Wegen $\mathtt{delete}([],y) = []$ folgt die Behauptung unmittelbar aus der
      Definition von $[]$.
\item Fall: $l = [\,d\,|\,r\,]$ und $y = d$

      Aus $x \leq l$ folgt sofort \\[0.1cm]
      \hspace*{1.3cm} $x \leq r$. \hspace*{\fill} (1) \\[0.1cm]
      Wegen $\mathtt{delete}([\,d\,|\,r\,],d) = r$ ist das die Behauptung.
\item Fall:  $l = [\,d\,|\,r\,]$ und $y \not= d$

      Aus $x \leq l$ folgt sofort \\[0.1cm]
      \hspace*{1.3cm} $x \leq d$ \quad und \hspace*{\fill} (1) \\[0.1cm]
      \hspace*{1.3cm} $x \leq r$. \hspace*{\fill} (2) \\[0.1cm]
      Nach Induktions--Voraussetzung folgt aus (2) \\[0.1cm]
      \hspace*{1.3cm} $x \leq \mathtt{delete}(r,y)$. \hspace*{\fill} (3) \\[0.1cm]
      Aus (1) und (3) folgt nun \\[0.1cm]
      \hspace*{1.3cm} $x \leq [\;d\;|\;\mathtt{delete}(r,y)\;]$ \\[0.1cm]
      Wegen $\mathtt{delete}([\,d\,|\,r\,],y) = [\;d\;|\;\mathtt{delete}(r,y)\;]$ ist das die Behauptung.
      \hspace*{\fill} $_\Box$
\end{enumerate}

\begin{Lemma}[Vertr\"aglichkeit von \texttt{minSort} und $\leq$] \hspace*{\fill} \\
\label{l6}
Sei ein Datum $x$ und eine Liste $l$ mit $x \leq l$ gegeben. Dann gilt \\[0.1cm]
\hspace*{1.3cm} $x \leq \mathtt{minSort}(l)$
\end{Lemma}
\textbf{Beweis} durch Induktion \"uber die Anzahl der Rechenschritte in der Berechnung
von $\texttt{minSort}(l)$:
\begin{enumerate}
\item Fall: $l = []$

      Aus $\mathtt{minSort}([]) = []$ und der Definition von $\leq$ folgt die Behauptung
      unmittelbar.
\item Fall: $l \not= []$

     Aus der Voraussetzung $x \leq l$ folgt mit Lemma \ref{l4} \\[0.1cm]
     \hspace*{1.3cm} $x \leq  \mathtt{delete}(l, \mathtt{getMin}(l))$ \hspace*{\fill} (1) \\[0.1cm]
     Weil zur Berechnung von $\mathtt{minSort}(l)$ rekursiv die Berechnung von \\[0.1cm]
     \hspace*{1.3cm} $\mathtt{minSort}(\mathtt{delete}(l, \mathtt{getMin}(l)))$ \\[0.1cm]
     durchgef\"uhrt wird,
     werden zur Durchf\"uhrung der Berechnung von  $\mathtt{getMin}(\mathtt{delete}(l, \mathtt{getMin}(l)))$
     weniger Rechenschritte ben\"otigt als f\"ur die Berechnung von $\mathtt{minSort}(l)$.
     Wegen (1) k\"onnen wir also die Induktions--Voraussetzung auf die Liste \\
     $\mathtt{delete}(l, \mathtt{getMin}(l))$ anwenden und finden \\[0.1cm]
     \hspace*{1.3cm}  $x \leq  \mathtt{minSort}(\mathtt{delete}(l, \mathtt{getMin}(l)))$ \hspace*{\fill} (2) \\[0.1cm]
     Aus der Voraussetzung $x \leq l$ folgt mit Lemma \ref{l2} \\[0.1cm]
     \hspace*{1.3cm} $x \leq \mathtt{getMin}(l)$  \hspace*{\fill} (3) \\[0.1cm]
     Aus (2) und (3) folgt mit der Definition von $\leq$ \\[0.1cm]
     \hspace*{1.3cm}      $x \leq [\;\mathtt{getMin}(l)\;|\;\mathtt{minSort}(\mathtt{delete}(l,\mathtt{getMin}(l)))\;]$   \\[0.1cm]
     und wegen $\mathtt{minSort}(l) =  [\;\mathtt{getMin}(l)\;|\;\mathtt{minSort}(\mathtt{delete}(l,\mathtt{getMin}(l)))\;]$ 
     ist das die Behauptung. \hspace*{\fill} $_\Box$
\end{enumerate}

\noindent
\textbf{Bemerkung}: Einen Beweis, der durch Induktion \"uber die Anzahl der Rechenschritte in der Berechnung
einer Funktion $f$ durchgef\"uhrt wird, bezeichnen wir auch als Beweis durch 
\emph{Wertverlaufs--Induktion} nach der Definition der Funktion $f$.

\begin{Satz}[1. Korrektheits--Eigenschaft von \texttt{minSort}] 
F\"ur beliebige Listen $l$ gilt: \\[0.1cm]
\hspace*{1.3cm} $\mathtt{ordered}(\mathtt{minSort}(l))$
\end{Satz}
\textbf{Beweis} durch Wertverlaufs--Induktion nach der Definition der Funktion \texttt{minSort}:
\begin{enumerate}
\item Fall: $l=[]$

      Es gilt $\mathtt{minSort}([]) = []$ und wegen $\mathtt{ordered}([])$ folgt daraus die Behauptung.
\item Fall: $l \not= []$

      Es gilt \\[0.1cm]
      \hspace*{1.3cm}  $\mathtt{minSort}(l) = [\;\mathtt{getMin}(l)\;|\;\mathtt{minSort}(\mathtt{delete}(l,\mathtt{getMin}(l)))\;]$. \hspace*{\fill} (1) \\[0.1cm]
      Daher folgt aus der Induktions--Voraussetzung \\[0.1cm]
      \hspace*{1.3cm} $\mathtt{ordered}(\mathtt{minSort}(\mathtt{delete}(l,\mathtt{getMin}(l)))$ \hspace*{\fill} (2) \\[0.1cm]
      Nach Lemma \ref{l3} gilt \\[0.1cm]
      \hspace*{1.3cm} $\mathtt{getMin}(l) \leq l$ \\[0.1cm]
      Nach Lemma \ref{l4} folgt daraus \\[0.1cm]
      \hspace*{1.3cm} $\mathtt{getMin}(l) \leq \mathtt{delete}(l,\mathtt{getMin}(l))$ \hspace*{\fill} (3) \\[0.1cm]
      Wenn wir in Lemma \ref{l6} jetzt f\"ur $x$ den Wert $\mathtt{getMin}(l)$ einsetzen, so folgt aus (3) \\[0.1cm]
      \hspace*{1.3cm} $\mathtt{getMin}(l) \leq \mathtt{minSort}(\mathtt{delete}(l,\mathtt{getMin}(l)))$ \hspace*{\fill} (4)\\[0.1cm]
      Nach der Definition von \texttt{ordered} folgt aus (2) und (4) jetzt \\[0.1cm]
       \hspace*{1.3cm} $\mathtt{ordered}([\;\mathtt{getMin}(l)\;|\;\mathtt{minSort}(\mathtt{delete}(l,\mathtt{getMin}(l)))\;]$ \\[0.1cm]
      und wegen (1) ist das ist die Behauptung. \hspace*{\fill} $_\Box$
\end{enumerate}

\begin{Lemma}[Vertr\"aglichkeit von \texttt{delete} und $\in$] 
\label{l8}
F\"ur beliebige Daten $x$, $y$ und Listen $l$, 
f\"ur die $x\in l$ ist, gilt \\[0.1cm]
\hspace*{1.3cm} $x = y \vee x \in \mathtt{delete}(l,y)$
\end{Lemma}
\textbf{Beweis} durch Wertverlaufs--Induktion nach der Definition der Funktion \texttt{delete}:
\begin{enumerate}
\item Fall: $l = []$

      Wegen der Voraussetzung $x \in l$ kann dieser Fall nicht auftreten.
\item Fall: $l = [\,d\,|\,r\,]$ und $y = d$

      Aus der Voraussetzung $x \in l$ folgt \\[0.1cm]
      \hspace*{1.3cm} $x = d \vee x \in r$.  \\[0.1cm]
      Wir f\"uhren eine entsprechende Fallunterscheidung durch:
      \begin{enumerate}
      \item $x = d$.

            Wegen $y = d$ folgt dann $x = y$, also die Behauptung.
      \item $x \in r$

            Da $\mathtt{delete}([\,d\,|\,r\,],d) = r$, folgt auch hier die Behauptung.
      \end{enumerate}
\item Fall: $l = [\,d\,|\,r\,]$ und $y \not= d$

      Jetzt gilt $\mathtt{delete}([\,d\,|\,r\,],y) = [\;d\;|\;\mathtt{delete}(r,y)\;]$. 
      Aus der Voraussetzung $x \in l$ folgt wieder \\[0.1cm]
      \hspace*{1.3cm} $x = d \vee x \in r$.  \\[0.1cm]
      Wir f\"uhren eine entsprechende Fallunterscheidung durch:
      \begin{enumerate}
      \item $x= d$

            Dann folgt $x\in \mathtt{delete}([\,d\,|\,r\,],y)$ aus $\mathtt{delete}([\,d\,|\,r\,],y) = [\;d\;|\;\mathtt{delete}(r,y)\;]$. 
      \item $x \in r$

            Dann gilt nach Induktions--Voraussetzung \\[0.1cm]
            \hspace*{1.3cm} $x = y \vee x \in \mathtt{delete}(r,y)$ \\[0.1cm]
            Falls $x = y$ gilt, ist der Beweis abgeschlossen.  Sonst gilt also $x \in \mathtt{delete}(r,y)$
            und damit na\"urlich auch \\[0.1cm]
            \hspace*{1.3cm} $x \in [\;d\;|\;\mathtt{delete}(r,y)\;]$ \\[0.1cm]
            was wegen  $\mathtt{delete}([\,d\,|\,r\,],y) = [\;d\;|\;\mathtt{delete}(r,y)\;]$ die Behauptung ist. 
            \hspace*{\fill} $_\Box$
      \end{enumerate}
      \hspace*{1.3cm} 
\end{enumerate}

\begin{Lemma}[Vertr\"aglichkeit von \texttt{delete} und $\in$] 
\label{l9}
F\"ur beliebige Daten $x$, $y$ und Listen $l$, f\"ur die $x \in \mathtt{delete}(l,y)$ gilt, haben wir \\[0.1cm]
\hspace*{1.3cm} $x \in l$.
\end{Lemma}
\textbf{Beweis} durch Wertverlaufs--Induktion nach der Definition der Funktion \texttt{delete}:
\begin{enumerate}
\item Fall: $l = []$

      Wegen  $\mathtt{delete}([],y) = []$ und der Voraussetzung $x \in \mathtt{delete}(l,y)$ 
      kann dieser Fall nicht auftreten.
\item Fall: $l = [\,d\,|\,r\,]$ und $y = d$

      Da $\mathtt{delete}([\,d\,|\,r\,],d) = r$ ist, folgt aus  $x \in \mathtt{delete}(l,y)$ sofort
       $x \in r$ und das liefert wegen $l = [\,d\,|\,r\,]$ die Behauptung.
\item Fall: $l = [\,d\,|\,r\,]$ und $y \not= d$

      Jetzt gilt $\mathtt{delete}([\,d\,|\,r\,],y) = [\;d\;|\;\mathtt{delete}(r,y)\;]$. 
      Aus der Voraussetzung $x \in \mathtt{delete}([\,d\,|\,r\,],y)$ folgt also \\[0.1cm]
      \hspace*{1.3cm} $x = d \vee x \in \mathtt{delete}(r,y)$. 
      \begin{enumerate}
      \item $x= d$

            Dann folgt sofort $x\in [\,d\,|\,r\,]$ und das ist die Behauptung.
      \item $x \in \mathtt{delete}(r,y)$

            Dann gilt nach Induktions--Voraussetzung \\[0.1cm]
            \hspace*{1.3cm} $x \in r$ \\[0.1cm]
            und also auch $x \in [\,d\,|\,r\,]$.
            \hspace*{\fill} $_\Box$
      \end{enumerate}
\end{enumerate}

\begin{Lemma}[Vertr\"aglichkeit von \texttt{getMin} und $\in$]
\label{l10}
F\"ur alle Listen $l$ mit $l\not = []$ gilt \\[0.1cm]
\hspace*{1.3cm} $\mathtt{getMin}(l) \in l$
\end{Lemma}
\textbf{Beweis} durch Wertverlaufs--Induktion nach der Definition der Funktion \texttt{getMin}:
\begin{enumerate}
\item $l = [d]$

      In diesem Fall gilt die Behauptung wegen $\mathtt{getMin}([d]) = d$.
\item $l = [\,d\,|\,r\,]$ und $d \leq \mathtt{getMin}(r)$

      Dann gilt $\mathtt{getMin}([\,d\,|\,r\,]) = d$ und die Behauptung ist offensichtlich.
\item $l = [\,d\,|\,r\,]$ und $d > \mathtt{getMin}(r)$

      Dann gilt $\mathtt{getMin}([\,d\,|\,r\,]) = \mathtt{getMin}(r)$.  Aus der
      Induktions--Voraussetzung folgt \\[0.1cm]
      \hspace*{1.3cm} $\mathtt{getMin}(r) \in r$ \\[0.1cm]
      Daraus folgt aber sofort $\mathtt{getMin}(r) \in [\,d\,|\,r\,]$ und wegen
       $\mathtt{getMin}([\,d\,|\,r\,]) = \mathtt{getMin}(r)$ ist das die Behauptung.
\end{enumerate}


\begin{Satz}[2. Korrektheits--Eigenschaft von \texttt{minSort}] 
F\"ur beliebige Daten $x$ und Listen $l$ gilt \\[0.1cm]
\hspace*{1.3cm} $x \in l \leftrightarrow x \in \mathtt{minSort}(l)$
\end{Satz}
\textbf{Beweis} durch Wertverlaufs--Induktion nach der Definition der Funktion \texttt{minSort}:
\begin{enumerate}
\item $l = []$

      Zu zeigen ist $x \in \mathtt{minSort}([]) \leftrightarrow x \in []$ und das folgt wegen $\mathtt{minSort}([]) = []$.
\item  $l \not= []$

       Wir spalten den Beweis in zwei Teile auf und zeigen zun\"achst, dass aus  $x \in l$ \\[0.1cm]
       \hspace*{1.3cm} $x \in \mathtt{minSort}(l)$ \\[0.1cm]
       folgt.  Nach Lemma \ref{l8} folgt aus $x \in l$ \\[0.1cm]
       \hspace*{1.3cm} $x = \mathtt{getMin}(l) \vee x \in \mathtt{delete}(l,\mathtt{getMin}(l))$ \\[0.1cm]
       Wir f\"uhren eine Fallunterscheidung durch:
       \begin{enumerate}
       \item $x = \mathtt{getMin}(l)$

             Wegen $\mathtt{minSort}(l) = [\;\mathtt{getMin}(l)\;|\;\mathtt{minSort}(\mathtt{delete}(l,\mathtt{getMin}(l)))\;]$ 
             folgt dann die Behauptung sofort.
       \item $x \in \mathtt{delete}(l,\mathtt{getMin}(l))$

             Dann gilt nach Induktions--Voraussetzung \\[0.1cm]
             \hspace*{1.3cm} $x \in \mathtt{minSort}(\mathtt{delete}(l,\mathtt{getMin}(l)))$ \\[0.1cm]
             und die Behauptung folgt wieder aus $\mathtt{minSort}(l) = [\;\mathtt{getMin}(l)\;|\;\mathtt{minSort}(\mathtt{delete}(l,\mathtt{getMin}(l)))\;]$. 
       \end{enumerate}
       Wir f\"uhren nun den zweiten Teil des Beweises und zeigen, dass aus $x \in \mathtt{minSort}(l)$
       auf $x\in l$ geschlossen werden kann.  Wegen \\[0.1cm]
       \hspace*{1.3cm}  $\mathtt{minSort}(l) = [\;\mathtt{getMin}(l)\;|\;\mathtt{minSort}(\mathtt{delete}(l,\mathtt{getMin}(l)))\;]$  \\[0.1cm]
       folgt aus der Voraussetzung $x \in \mathtt{minSort} (l)$ \\[0.1cm]
       \hspace*{1.3cm} $x = \mathtt{getMin}(l) \vee \mathtt{minSort}(\mathtt{delete}(l,\mathtt{getMin}(l)))]$ \\[0.1cm]
       Wir f\"uhren eine Fallunterscheidung durch.
       \begin{enumerate}
       \item $x = \mathtt{getMin}(l)$

             In diesem Fall folgt die Behauptung aus Lemma \ref{l10}.
       \item $x \in \mathtt{minSort}(\mathtt{delete}(l,\mathtt{getMin}(l)))$

             Dann gilt nach Induktions--Voraussetzung \\[0.1cm]
             \hspace*{1.3cm} $x \in \mathtt{delete}(l,\mathtt{getMin}(l))$ \\[0.1cm]
             Daraus folgt mit Lemma \ref{l9} aber sofort $x \in l$. \hspace*{\fill} $_\Box$
       \end{enumerate}
\end{enumerate}

\end{document}

%%% Local Variables: 
%%% mode: latex
%%% TeX-master: "min-sort"
%%% End: 
