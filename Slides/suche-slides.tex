%%%%%%%%%%%%%%%%%%%%%%%%%%%%%%%%%%%%%%%%%%%%%%%%%%%%%%%%%%%%%%%%%%%%%%%%

\begin{slide}{}
\normalsize

\begin{center}
Listen--Implementierung von \textsl{Dictionary}
\end{center}
\vspace*{0.5cm}

\footnotesize
\begin{verbatim}
typedef struct Node* NodePtr;
struct Node {
    Key      key;
    Value    val;
    NodePtr  nextPtr;
};
typedef NodePtr List;
typedef List    Table;

Table insert(Table t, Key k, Value v) {
    NodePtr nodePtr  = malloc( sizeof(struct Node) );
    nodePtr->key     = k;
    nodePtr->val     = v;
    nodePtr->nextPtr = t;
    return nodePtr;
}
Value* search(Table t, Key k) {
    if (t == 0)       
        return 0;
    if (t->key == k) 
        return &(t->val);
    else 
        return search(t->nextPtr, k);
}
Table delete(Table t, Key k) {
    if (t->key == k) {
        Table res = t->nextPtr;
        free(t);
        return res;
    } else {
        t->nextPtr = delete(t->nextPtr, k);
        return t;
    }
}
\end{verbatim}

\vspace*{\fill}
\tiny \addtocounter{mypage}{1}
\rule{17cm}{1mm}
Dictionaries  \hspace*{\fill} Seite \arabic{mypage}
\end{slide}

%%%%%%%%%%%%%%%%%%%%%%%%%%%%%%%%%%%%%%%%%%%%%%%%%%%%%%%%%%%%%%%%%%%%%%%%

\begin{slide}{}
\normalsize

\begin{center}
Baum--Implementierung von \textsl{Dictionary}
\end{center}
\vspace*{0.5cm}

\footnotesize
Komplexit\"at der Listen--Implementierung: 

Ist $n$ Zahl der Elemente der Liste, so gilt im \emph{worst case}:
\begin{enumerate}
\item Komplexit\"at \texttt{insert(t, k, v)} $\in \Oh(1)$
\item Komplexit\"at \texttt{search(t, k)} $\in \Oh(n)$
\item Komplexit\"at \texttt{delete(t, k)} $\in \Oh(n)$
\end{enumerate}
Es geht schneller!

Idee: Daten in bin\"arem Suchbaum organisieren

Jeder Knoten enth\"alt:
\begin{enumerate}
\item Schl\"ussel
\item Wert
\item Zeiger auf linken Teilbaum
\item Zeiger auf rechten Teilbaum
\end{enumerate}

Ordnungs--Prinzip: F\"ur jeden Knoten $n$ gilt:
\begin{enumerate}
\item Alle Schl\"ussel im linken Teilbaum sind kleiner als \\
      Schl\"ussel am Knoten $n$
\item Alle Schl\"ussel im rechten Teilbaum sind gr\"o\3er als
      Schl\"ussel am Knoten $n$
\end{enumerate}


\vspace*{\fill}
\tiny \addtocounter{mypage}{1}
\rule{17cm}{1mm}
Dictionaries  \hspace*{\fill} Seite \arabic{mypage}
\end{slide}

%%%%%%%%%%%%%%%%%%%%%%%%%%%%%%%%%%%%%%%%%%%%%%%%%%%%%%%%%%%%%%%%%%%%%%%%

\begin{slide}{}
\normalsize

\begin{center}
Baum--Implementierung von \textsl{Dictionary}
\end{center}
\vspace*{0.5cm}

\footnotesize
Daten--Struktur
\begin{verbatim}
typedef struct BinNode* BinNodePtr;
struct BinNode {
    Key        key;
    Value      val;
    BinNodePtr left;
    BinNodePtr right;
};
typedef BinNodePtr BinTree;
typedef BinTree    Table;
\end{verbatim}
Einf\"ugen von Knoten
\begin{verbatim}
Table insert(Table t, Key key, Value val) {
    if (t == 0) {
        BinTree nodePtr = malloc( sizeof(struct BinNode) );
        nodePtr->key    = key;
        nodePtr->val    = val;
        nodePtr->left   = 0;
        nodePtr->right  = 0;
        return nodePtr;
    }
    int cmp = compareKey(key, t->key);
    if (cmp == -1) {
        t->left = insert(t->left, key, val);
    } else if (cmp == 1) {
        t->right = insert(t->right, key, val);
    } else {
        t->val = val;
    }
    return t;
}
\end{verbatim}


\vspace*{\fill}
\tiny \addtocounter{mypage}{1}
\rule{17cm}{1mm}
Dictionaries  \hspace*{\fill} Seite \arabic{mypage}
\end{slide}

%%%%%%%%%%%%%%%%%%%%%%%%%%%%%%%%%%%%%%%%%%%%%%%%%%%%%%%%%%%%%%%%%%%%%%%%

\begin{slide}{}
\normalsize

\begin{center}
Baum--Implementierung von Dictionaries
\end{center}
\vspace*{0.5cm}

\footnotesize
Suchen nach einem Schl\"ussel
\begin{verbatim}
Value* search(BinTree t, Key key) 
{
    if (t == 0)
        return 0;
    int cmp = compareKey(key, t->key);
    if (cmp == -1)
        return search(t->left, key);
    if (cmp == 1)
        return search(t->right, key);
    return &(t->val);
}
\end{verbatim}

Komplexit\"at der Baum--Implementierung: 

Ist $l$ die L\"ange des l\"angsten Pfades im Baum, so gilt im \emph{worst case}:
\begin{enumerate}
\item Komplexit\"at \texttt{insert(t, k, v)} $\in \Oh(l)$
\item Komplexit\"at \texttt{search(t, k)} $\in \Oh(l)$
\item Komplexit\"at \texttt{delete(t, k)} $\in \Oh(l)$
\end{enumerate}

Ist $n$ die Anzahl der Elemente des Baums, so gilt:
\begin{enumerate}
\item Im besten Fall ist $n = 2^{l+1} - 1$, also \\[0.3cm]
      \hspace*{1.3cm} $l = \mathtt{log}_2(n+1) - 1$

      Dann ist die Komplexit\"at der Implementierung logarithmisch!
\item Im schlechtesten Fall ist $n = l + 1$.

      Dann ist die Komplexit\"at linear!
\end{enumerate}

\vspace*{\fill}
\tiny \addtocounter{mypage}{1}
\rule{17cm}{1mm}
Dictionaries  \hspace*{\fill} Seite \arabic{mypage}
\end{slide}

%%%%%%%%%%%%%%%%%%%%%%%%%%%%%%%%%%%%%%%%%%%%%%%%%%%%%%%%%%%%%%%%%%%%%%%%

\begin{slide}{}
\normalsize

\begin{center}
Entfernen eines Schl\"ussels
\end{center}
\vspace*{0.5cm}

\footnotesize
Beim L\"oschen eines Knotens sind 3 F\"alle m\"oglich:
\begin{enumerate}
\item Knoten hat keine Kinder

      Knoten l\"oschen und 0 zur\"uckgeben
\item Knoten hat nur ein Kind

      Knoten l\"oschen, Kind zur\"uckgeben
\item Knoten hat zwei Kinder
  \begin{enumerate}
  \item Finde \emph{Nachfolger} des Knotens im rechten Teilbaum
  \item Ersetze Daten des Knotens durch Daten des Nachfolgers
  \item L\"osche Nachfolger
  \end{enumerate}
\end{enumerate}

\textbf{Definition}: Knoten $n$ ist Nachfolger des Knotens $m$ g.d.w.
\begin{enumerate}
\item $m\rightarrow\mathtt{key} < n\rightarrow\mathtt{key}$
\item Es gibt keinen Knoten $x$, dessen Schl\"ussel $x\rightarrow\mathtt{key}$ zwischen  
      $m\rightarrow\mathtt{key}$ und $n\rightarrow\mathtt{key}$ liegt: \\[0.3cm]
     \hspace*{-1.3cm}
 $\neg \exists x \in \mathtt{BinNode}: m\rightarrow\mathtt{key} < x\rightarrow\mathtt{key} \;\wedge\; x\rightarrow\mathtt{key} < n\rightarrow\mathtt{key}$
\end{enumerate}
Implementierung: 
\begin{enumerate}
\item Gehe vom rechten Kind von $m$ aus
\item Finde rekursiv Knoten, dessen linkes Kind 0 ist.
\end{enumerate}

\vspace*{\fill}
\tiny \addtocounter{mypage}{1}
\rule{17cm}{1mm}
Dictionaries  \hspace*{\fill} Seite \arabic{mypage}
\end{slide}

%%%%%%%%%%%%%%%%%%%%%%%%%%%%%%%%%%%%%%%%%%%%%%%%%%%%%%%%%%%%%%%%%%%%%%%%

\begin{slide}{}
\normalsize

\begin{center}
Implementierung \texttt{delete}
\end{center}
\vspace*{0.5cm}

\footnotesize

\begin{verbatim}
Table delete(Table t, Key key) {
    assert(t != 0);
    int cmp = compareKey(key, t->key);
    if (cmp == -1)
        t->left = delete(t->left, key);
    if (cmp == 1)
        t->right = delete(t->right, key);
    if (cmp != 0)
        return t;
    // If we ever get here, we must have key == t->key.
    if (t->left == 0 && t->right == 0) {
        free(t);
        return 0;
    }
    if (t->left == 0) {
        Table res = t->right;
        free(t);
        return res;
    }
    if (t->right == 0) {
        Table res = t->left;
        free(t);
        return res;
    }
    BinNodePtr ptr = findRightSucc(t->right);
    t->key = ptr->key;
    t->val = ptr->val;
    t->right = delete(t->right, ptr->key);
    return t;   
}
\end{verbatim}

\vspace*{\fill}
\tiny \addtocounter{mypage}{1}
\rule{17cm}{1mm}
Dictionaries  \hspace*{\fill} Seite \arabic{mypage}
\end{slide}

%%%%%%%%%%%%%%%%%%%%%%%%%%%%%%%%%%%%%%%%%%%%%%%%%%%%%%%%%%%%%%%%%%%%%%%%

\begin{slide}{}
\normalsize

\begin{center}
Bin\"are B\"aume
\end{center}
\vspace*{0.5cm}

\footnotesize
Formalisierung der Theorie der bin\"aren B\"aume 
\begin{enumerate}
\item Typen: \textsl{Key}, \textsl{Value}, \textsl{Tree}
\item $\mathtt{nil}: \textsl{Tree}$
\item $\mathtt{bin}: \textsl{Key} \times \textsl{Value} \times \textsl{Tree} \times \textsl{Tree} \rightarrow \textsl{Tree}$
\end{enumerate}
Wir definieren folgende Pr\"adikate 
\begin{enumerate}
\item $<: \textsl{Key} \times \textsl{Tree} \rightarrow \mathbb{B}$ 
\item $>: \textsl{Key} \times \textsl{Tree} \rightarrow \mathbb{B}$ 
\item $\textsl{ordered}: \textsl{Tree} \rightarrow \mathbb{B}$ 
\end{enumerate}
\begin{enumerate}
\item $k < \mathtt{nil}$
\item $k_1 < \mathtt{bin}(k_2, v, l, r) \;\leftrightarrow\; k_1 < k_2 \;\wedge\; k_1 < l \;\wedge\; k_1 < r$
\end{enumerate}
\begin{enumerate}
\item $k > \mathtt{nil}$
\item $k_1 > \mathtt{bin}(k_2, v, l, r) \;\leftrightarrow\; k_1 > k_2 \;\wedge\; k_1 > l \;\wedge\; k_1 > r$
\end{enumerate}
\begin{enumerate}
\item $\textsl{ordered}(\mathtt{nil}) \leftrightarrow \mathtt{true}$
\item $\textsl{ordered}\bigg(\mathtt{bin}(k,v,l,r)\bigg) \;\;\leftrightarrow\;$ \\[0.3cm]
      \hspace*{1.3cm} $k > l \;\wedge\; k < r \;\wedge\; \textsl{ordered}(l) \;\wedge\; \textsl{ordered}(r)$
\end{enumerate}
\textbf{Definition}: $t \in \textsl{Tree}$ hei\3t \emph{geordneter} Baum g.d.w. \\[0.3cm]
\hspace*{1.3cm} $\textsl{ordered}(t) = \mathtt{true}$

\vspace*{\fill}
\tiny \addtocounter{mypage}{1}
\rule{17cm}{1mm}
Dictionaries  \hspace*{\fill} Seite \arabic{mypage}
\end{slide}

%%%%%%%%%%%%%%%%%%%%%%%%%%%%%%%%%%%%%%%%%%%%%%%%%%%%%%%%%%%%%%%%%%%%%%%%

\begin{slide}{}
\normalsize

\begin{center}
Bin\"are B\"aume (Fortsetzung)
\end{center}
\vspace*{0.5cm}

\footnotesize
Berechnung der \textsl{H\"ohe} eines Baums \\[0.3cm]
\hspace*{1.3cm} $\textsl{height}: \textsl{Tree} \rightarrow \mathbb{N}$
\begin{enumerate}
\item $\textsl{height}(\mathtt{nil}) = 0$
\item $\textsl{height}(\mathtt{bin}(k,v,l,r)) = 1 + \max\bigg( \textsl{height}(l),\, height(r)\bigg)$
\end{enumerate}
Berechnung der Zahl der Knoten eines Baumes \\[0.3cm]
\hspace*{1.3cm} $\textsl{count}: \textsl{Tree} \rightarrow \mathbb{N}$
\begin{enumerate}
\item $\textsl{count}(\mathtt{nil}) = 0$
\item $\textsl{count}\bigg(\mathtt{bin}(k,v,l,r)\bigg) = 1 + \textsl{count}(l) + \textsl{count}(r)$
\end{enumerate}

Berechnung der gesamten Pfadl\"ange im Baum: \\[0.3cm]
anderer Begriff \emph{innere Pfadl\"ange}
\begin{enumerate}
\item $\textsl{totalLength}(\mathtt{nil}) = 0$
\item $\textsl{totalLength}(\mathtt{bin}(k,v,l,r) \;= $\\[0.3cm]
\hspace*{1.3cm} $
\begin{array}{llclc}
     & \textsl{count}(l) & + & \textsl{totalLength}(l)  \\
   + & \textsl{count}(r) & + & \textsl{totalLength}(r) \\
\end{array}$
\end{enumerate}
Durchschnittliche Pfadl\"ange: \\[0.3cm]
\hspace*{1.3cm} gesamte Pfadl\"ange / Anzahl Knoten


\vspace*{\fill}
\tiny \addtocounter{mypage}{1}
\rule{17cm}{1mm}
Dictionaries  \hspace*{\fill} Seite \arabic{mypage}
\end{slide}

%%%%%%%%%%%%%%%%%%%%%%%%%%%%%%%%%%%%%%%%%%%%%%%%%%%%%%%%%%%%%%%%%%%%%%%%

\begin{slide}{}
\normalsize

\begin{center}
Volle bin\"are B\"aume
\end{center}
\vspace*{0.5cm}

\footnotesize
\textbf{Definition}: Ein bin\"arer Baum ist \emph{voll} wenn er die bei gegebener H\"ohe
maximale Anzahl an Knoten hat.

$c_h$: Zahl Knoten des vollen bin\"aren Baums der H\"ohe $h$.
\begin{enumerate}
\item $c_0 = 0$
\item $c_{h+1} = c_h + c_h + 1$
\end{enumerate}
Daraus folgt: $c_h = 2^{h} - 1$

$g_h$: gesamte Pfadl\"ange des vollen bin\"are Baums \\[0.3cm]
$h$: H\"ohe des vollen bin\"aren Baum
\begin{enumerate}
\item $g_1 = 0$
\item $g_{h+1} = c_h + c_h + g_h + g_h$, also \\[0.3cm]
      \hspace*{1.3cm} $g_{h+1} = 2 * g_h + 2 * (2^h - 1)$
\end{enumerate}
Homogenisierung dieser Rekursions--Gleichung liefert \\[0.3cm]
\hspace*{1.3cm} $g_{h+3} = 5 * g_{h+2} - 8 * g_{h+1} + 4 * g_h$ \\[0.3cm]
Charakteristisches Polynom: \\[0.3cm]
\hspace*{1.3cm} $\chi(x) = x^3 - 5 * x^2 + 8 * x - 4 = (x - 1) * (x - 2)^2$ \\[0.3cm]
Allgemeine L\"osung: \\[0.3cm]
\hspace*{1.3cm} $g_h = \alpha + \beta * 2^h + \gamma * h * 2^h$ \\[0.3cm]


\vspace*{\fill}
\tiny \addtocounter{mypage}{1}
\rule{17cm}{1mm}
Dictionaries  \hspace*{\fill} Seite \arabic{mypage}
\end{slide}
%%%%%%%%%%%%%%%%%%%%%%%%%%%%%%%%%%%%%%%%%%%%%%%%%%%%%%%%%%%%%%%%%%%%%%%%

\begin{slide}{}
\normalsize

\footnotesize
Allgemeine L\"osung: \\[0.3cm]
\hspace*{1.3cm} $g_h = \alpha + \beta * 2^h + \gamma * h * 2^h$ \\[0.3cm]
Berechnung der Anfangs--Bedingungen
\begin{enumerate}
\item $g_1 = 0$
\item $g_2 = 2 * g_1 + 2 * (2^1 - 1) = 2$
\item $g_3 = 2 * g_2 + 2 * (2^2 - 1) = 2 * 2 + 2 * 3 = 10$
\end{enumerate}
Berechnung der Koeffizienten $\alpha$, $\beta$, $\gamma$:
$$
\begin{array}{rcl}
  0 & = & \alpha + 2 * \beta + \;\;2 * \gamma \\
  2 & = & \alpha + 4 * \beta + \;\;8 * \gamma \\
 10 & = & \alpha + 8 * \beta + 24 * \gamma \\
\end{array}
$$
L\"osung: \\[0.3cm]
\hspace*{1.3cm}  $\alpha = 2$, $\beta = -2$, $\gamma = 1$.

Tool zur Online--L\"osung im Internet unter \\[0.3cm]
\hspace*{0.3cm} \texttt{mss.math.vanderbilt.edu/\symbol{126}pscrooke/MSS/solvesystem.html}



L\"osung der Rekursions--Gleichung: \\[0.3cm]
\hspace*{1.3cm} $g_h = 2 + (h-2) * 2^{h}$ 

Erwartungswert f\"ur den Aufwand bei der Suche nach einem Schl\"ussel ist die durchschnittliche Pfadl\"ange + 1.
F\"ur den vollen Baum, also im Optimal--Fall, gilt: \\[0.3cm]
\hspace*{1.3cm} $\bruch{g_h}{c_h} + 1 = \bruch{2 + (h-2) * 2^{h}}{2^{h} -1} + 1\;\; _{\displaystyle \longrightarrow \atop {\scriptstyle h \rightarrow \infty}} \;\; h - 1$

Ausgedr\"uckt durch Anzahl $n$ der Knoten: \\[0.3cm]
\hspace*{1.3cm} $n = c_h = 2^h -1 \;\Leftrightarrow\; h = \log_2(n + 1)$

\hspace*{1.3cm} $\bruch{g(n)}{n}\;  _{\displaystyle \longrightarrow \atop {\scriptstyle n \rightarrow \infty}} \;\; \log_2(n + 1) - 1$


\vspace*{\fill}
\tiny \addtocounter{mypage}{1}
\rule{17cm}{1mm}
Dictionaries  \hspace*{\fill} Seite \arabic{mypage}
\end{slide}
%%%%%%%%%%%%%%%%%%%%%%%%%%%%%%%%%%%%%%%%%%%%%%%%%%%%%%%%%%%%%%%%%%%%%%%%

\begin{slide}{}
\normalsize

\begin{center}
Berechnung der durchschnittlichen gesamten Pfadl\"ange f\"ur zuf\"allig erzeugte bin\"are B\"aume
mit $n$ Knoten
\end{center}
\vspace*{0.5cm}

\footnotesize
Gegeben: Menge von $n$ Knoten der Form \\[0.3cm]
\hspace*{1.3cm} $M = \{1,2,3,\cdots, n\}$ \\[0.3cm]
Ist $k \in M$ erster Knoten, so hat Baum die Gestalt:

\setlength{\unitlength}{4144sp}
\begingroup\makeatletter\ifx\SetFigFont\undefined
\gdef\SetFigFont#1#2#3#4#5{
  \reset@font\fontsize{#1}{#2pt}
  \fontfamily{#3}\fontseries{#4}\fontshape{#5}
  \selectfont}
\fi\endgroup
\begin{picture}(5470,2007)(2775,-3315)
{\thinlines
\put(5581,-1591){\circle{550}}
}
{\put(7198,-2088){\line(-1,-1){1192.500}}
\put(6028,-3303){\line( 1, 0){2205}}
\put(8233,-3303){\line(-5, 6){1043.852}}
\put(7198,-2043){\line( 0,-1){ 45}}
}
{\put(3912,-2036){\line(-5,-4){1125}}
\put(2787,-2936){\line( 1, 0){2070}}
\put(4857,-2936){\line(-6, 5){1026.885}}
\put(3867,-2036){\line( 1, 0){ 45}}
}
{\put(5851,-1681){\vector( 3,-1){1336.500}}
}
{\put(5311,-1636){\vector(-3,-1){1336.500}}
}
\put(5181,-1681){\makebox(0,0)[lb]{\smash{\SetFigFont{17}{20.4}{\rmdefault}{\mddefault}{\updefault}{k}}}}
\put(6196,-3141){\makebox(0,0)[lb]{\smash{\SetFigFont{17}{20.4}{\rmdefault}{\mddefault}{\updefault}{$k+1,...,n$}}}}
\put(2911,-2871){\makebox(0,0)[lb]{\smash{\SetFigFont{17}{20.4}{\rmdefault}{\mddefault}{\updefault}{$1,...,k\!-\!1$}}}}
\end{picture}
\begin{enumerate}
\item Linker Teilbaum: Knoten $\{1,\cdots,k-1\}$
\item Rechter Teilbaum: Knoten $\{k+1,\cdots,n\}$
\end{enumerate}
Die Wahrscheinlichkeit daf\"ur, dass gerade Knoten $k$ als erstes eingef\"ugt wird, betr\"agt: \\[0.3cm]
\hspace*{1.3cm} $\bruch{1}{n}$


$g_n$: statistischer Mittelwert f\"ur gesamte Pfadl\"ange des \\
\hspace*{0.8cm}  zuf\"allig erzeugten bin\"aren Baums mit $n$ Knoten \\[0.5cm]
\hspace*{1.3cm} $g_n = \bruch{1}{n} \sum\limits_{k=1}^n \Bigg(\; (k - 1) + g_{k-1} \;+\; (n-k) + g_{n-k} \;\Bigg)$

\vspace*{\fill}
\tiny \addtocounter{mypage}{1}
\rule{17cm}{1mm}
Dictionaries  \hspace*{\fill} Seite \arabic{mypage}
\end{slide}

%%%%%%%%%%%%%%%%%%%%%%%%%%%%%%%%%%%%%%%%%%%%%%%%%%%%%%%%%%%%%%%%%%%%%%%%

\begin{slide}{}
\normalsize

\begin{center}
Pfadl\"ange zuf\"aliger B\"aume (Fortsetzung)
\end{center}
\vspace*{0.5cm}

\footnotesize
\hspace*{1.3cm} $g_n = \bruch{1}{n} \sum\limits_{k=1}^n \Bigg(\; (k - 1) + g_{k-1} \;+\; (n-k) + g_{n-k} \;\Bigg)$
\textbf{Erkl\"arung}:
\begin{enumerate}
\item $\textsl{totalLength}(\mathtt{bin}(k,v,l,r) \;= $\\[0.3cm]
\hspace*{1.3cm} $
\begin{array}{llclc}
     & \textsl{count}(l) & + & \textsl{totalLength}(l)  \\
   + & \textsl{count}(r) & + & \textsl{totalLength}(r) \\
\end{array}$
\item $k - 1$: Anzahl Knoten im linken Teilbaum
\item $g_{k-1}$: gesamte Pfadl\"ange im linken Teilbaum
\item $n - k$: Anzahl Knoten im rechten Teilbaum
\item $g_{n-k}$: gesamte Pfadl\"ange im rechten Teilbaum
\item $\bruch{1}{n}$: Wahrscheinlichkeit f\"ur jeden der $n$ F\"alle
\end{enumerate}
Vereinfachungen: 
\begin{enumerate}
\item  $\bruch{1}{n} \sum\limits_{k=1}^n \Bigg(\; (k-1) + (n-k) \;\Bigg) = \bruch{1}{n} \sum\limits_{k=1}^n (n-1) = n - 1$
\item $\sum\limits_{k=1}^n g_{n-k} = \sum\limits_{i=0}^{n-1} g_i$
\item $\sum\limits_{k=1}^n g_{k-1} = \sum\limits_{i=0}^{n-1} g_i$
\item $g_n = n - 1 + \bruch{2}{n} \sum\limits_{i=0}^{n-1} g_i$
\end{enumerate}


\vspace*{\fill}
\tiny \addtocounter{mypage}{1}
\rule{17cm}{1mm}
Dictionaries  \hspace*{\fill} Seite \arabic{mypage}
\end{slide}

%%%%%%%%%%%%%%%%%%%%%%%%%%%%%%%%%%%%%%%%%%%%%%%%%%%%%%%%%%%%%%%%%%%%%%%%

\begin{slide}{}
\normalsize

\begin{center}
L\"osung von $g_n = n - 1 + \bruch{2}{n} \sum\limits_{i=0}^{n-1} g_i$
\end{center}
\vspace*{0.5cm}

\footnotesize
\hspace*{1.3cm} $g_n = n - 1 + \bruch{2}{n} \sum\limits_{i=0}^{n-1} g_i$
    \hspace*{\fill} (A) \\[0.3cm]
Setze $n \mapsto n + 1$ \\[0.3cm]
\hspace*{1.3cm} $g_{n+1} = n + \bruch{2}{n+1} \sum\limits_{i=0}^{n} g_i$
    \hspace*{\fill} (B) \\[0.3cm]
Bilde $(n+1)*(B) - n * (A)$: \\[0.5cm]
\hspace*{1.3cm} $(n+1) * g_{n+1} - n * g_n = (n+1)*n - n * (n-1) + 2 * g_n$ \\[0.5cm]
Vereinfachen: \\[0.3cm]
\hspace*{1.3cm} $(n+1) * g_{n+1} = (n+2)*g_n + 2*n$ \\[0.3cm]
Division durch $(n+1)*(n+2)$:\\[0.5cm]
\hspace*{1.3cm} $\bruch{g_{n+1}}{n+2} = \bruch{g_n}{n+1} + 2*\bruch{n}{(n+1)*(n+2)}$ \\[0.3cm]
Definiere: \\[0.3cm]
\hspace*{1.3cm} $a_n := \bruch{g_n}{n+1}$ \quad und \quad $b_n = \bruch{2*n}{(n+1)*(n+2)}$ \\[0.5cm]
Wegen $a_1 = g_1/2 = 0/2 = 0$ gilt: \\[0.3cm]
\hspace*{1.3cm} $a_{n+1} = a_1 + \sum\limits_{i=1}^n b_n = \sum\limits_{i=1}^n b_n$ \\[0.3cm]
Also \\[0.3cm]
  \hspace*{1.3cm} $g_{n+1} = (n+2) * \sum\limits_{i=1}^n \bruch{2*i}{(i+1)*(i+2)}$

\vspace*{\fill}
\tiny \addtocounter{mypage}{1}
\rule{17cm}{1mm}
Dictionaries  \hspace*{\fill} Seite \arabic{mypage}
\end{slide}

%%%%%%%%%%%%%%%%%%%%%%%%%%%%%%%%%%%%%%%%%%%%%%%%%%%%%%%%%%%%%%%%%%%%%%%%

\begin{slide}{}
\normalsize

\begin{center}
Berechnung von $g_n$
\end{center}
\vspace*{0.5cm}

\footnotesize
 \hspace*{1.3cm} $g_{n+1} = (n+2) * \sum\limits_{i=1}^n \bruch{2*i}{(i+1)*(i+2)}$ \\[0.3cm]
Es gilt \\[0.3cm]
\hspace*{1.3cm} $\bruch{i}{(i+1)*(i+2)} = \bruch{2}{i+2} - \bruch{1}{i+1}$ \\[0.3cm]
Also \\[0.3cm]
\hspace*{1.3cm} $g_{n+1} = (n+2) * 2 * \left(\sum\limits_{i=1}^n \bruch{2}{i+2} - \sum\limits_{i=1}^n \bruch{1}{i+1}\right)$ \\[0.3cm]
Definition der harmonischen Funktion $H(n)$ \\[0.3cm]
\hspace*{1.3cm} $H(n) = \sum\limits_{i=1}^n \bruch{1}{i}$ \\[0.3cm]
Es gilt \\[0.3cm]
\hspace*{1.3cm} $
\begin{array}{lcl}
\sum\limits_{i=1}^n \bruch{2}{i+2} & = & 2 * \sum\limits_{i=3}^{n+2} \bruch{1}{i} \\
                                   & = & 2 * \left( H(n+1) - \bruch{1}{1} - \bruch{1}{2} +  \bruch{1}{n+2} \right)\\
                                   & = & 2 * H(n+1) - 3 + \bruch{2}{n+2} \\[0.3cm]
\sum\limits_{i=1}^n \bruch{1}{i+1} & = &  \sum\limits_{i=2}^{n+1} \bruch{1}{i} = H(n+1) - 1 \\[0.3cm]
\end{array}
$ \\[0.5cm]
$\sum\limits_{i=1}^n \bruch{2}{i+2} - \sum\limits_{i=1}^n \bruch{1}{i+1} =  H(n+1) - 2 + \bruch{2}{n+2}$

Also gilt \\[0.3cm]
\hspace*{1.3cm} $g_{n+1} = 2 * (n+2) * H(n+1) - 4 *(n+2) + 4$

\vspace*{\fill}
\tiny \addtocounter{mypage}{1}
\rule{17cm}{1mm}
Dictionaries  \hspace*{\fill} Seite \arabic{mypage}
\end{slide}

%%%%%%%%%%%%%%%%%%%%%%%%%%%%%%%%%%%%%%%%%%%%%%%%%%%%%%%%%%%%%%%%%%%%%%%%

\begin{slide}{}
\normalsize

\begin{center}
Approximation von $g_n$
\end{center}
\vspace*{0.5cm}

\footnotesize
F\"ur gro\3e $n$ gilt \\[0.3cm]
\hspace*{1.3cm} $H(n) = \sum\limits_{i=1}^n \bruch{1}{n} \approx \int\limits_1^n \bruch{dx}{x} = \ln(x)$ \\[0.3cm]
Mathematisch pr\"azise ist \\[0.3cm]
\hspace*{1.3cm} $\lim\limits_{n\rightarrow\infty} \bruch{H(n)}{\ln(n)} = 1$ \\[0.3cm]
Also \\[0.3cm]
\hspace*{1.3cm} $g_n = 2 * n * \ln(n) + \Oh(n)$ \\[0.3cm]
Es ist \\[0.3cm]
\hspace*{1.3cm} $\ln(n) = \ln(2) * \log_2(n)$ \\[0.3cm]
Also \\[0.3cm]
\hspace*{1.3cm} $g_n = 2 * \ln(2) n * \log_2(n) + \Oh(n)$ \\[0.3cm]
Vergleich: im optimalen Fall gilt: \\[0.3cm]
\hspace*{1.3cm} $g_n = n * (\log_2(n+1)-1)= n * \log_2(n) + \Oh(n)$ \\[0.3cm]
Also gilt f\"ur das Verh\"altnis \\[0.3cm]
\hspace*{1.3cm} $\bruch{g_n\;\; \mbox{zuf\"allig}}{g_n\;\; \mbox{optimal}} \;\;_{\displaystyle \longrightarrow \atop {\scriptstyle n \rightarrow \infty}}\;\; 2 * \ln(2) \approx 1.386$

\textbf{Konklusion} f\"ur Suche: \\[0.9cm]
\hspace*{0.3cm} \framebox{zuf\"alliger Baum  im Durchschnitt $39\%$ schlechter}


\vspace*{0.3cm}

\vspace*{\fill}
\tiny \addtocounter{mypage}{1}
\rule{17cm}{1mm}
Dictionaries  \hspace*{\fill} Seite \arabic{mypage}
\end{slide}

%%%%%%%%%%%%%%%%%%%%%%%%%%%%%%%%%%%%%%%%%%%%%%%%%%%%%%%%%%%%%%%%%%%%%%%%

\begin{slide}{}
\normalsize

\begin{center}
Erzeugung balancierter B\"aume
\end{center}
\vspace*{0.5cm}

\footnotesize
\textbf{Gegeben}: Liste $l$ von $\langle\textsl{Key}, \textsl{Value}\rangle$ Paaren \\[0.3cm]
\textbf{Gesucht}: Balancierter Baum, der alle Elemente aus $l$ \\
\hspace*{2.88cm} enth\"alt \\[0.3cm]
\textbf{Vorgehen}: Divide--and--Conquer Algorithmus
\begin{enumerate}
\item Sortieren der Liste $l$: Sortierte Liste $s$
\item Aufteilen von $s$ in der Mitte in drei Teile:
  \begin{enumerate}
  \item mittleres Element $m$
  \item linke Liste: Elemente kleiner als $m$
  \item rechte Liste: Elemente gr\"o\3er als $m$
  \end{enumerate}
\item Bilde rekursiv balancierten Baum $l$ aus linker Liste
\item Bilde rekursiv balancierten Baum $r$ aus rechter Liste
\item Bilde aus $l$, $m$ und $r$ balancierten Baum mit $m$ an der Wurzel
\end{enumerate}

\begin{verbatim}
BinTree list2tree(List l, unsigned length) {
    if (length == 0)
        return 0;
    unsigned middle = length / 2 + length % 2;
    NodePtr  node   = ithNode(l, middle);
    BinTree  left   = list2tree(l, middle - 1);
    BinTree  right  = 
        list2tree(node->nextPtr, length - middle);
    return createBinNode(node->key, 
                         node->val, left, right);    
}
\end{verbatim}

\vspace*{\fill}
\tiny \addtocounter{mypage}{1}
\rule{17cm}{1mm}
Dictionaries  \hspace*{\fill} Seite \arabic{mypage}
\end{slide}

%%%%%%%%%%%%%%%%%%%%%%%%%%%%%%%%%%%%%%%%%%%%%%%%%%%%%%%%%%%%%%%%%%%%%%%%

\begin{slide}{}
\normalsize

\begin{center}
Komplexit\"at des Verfahrens
\end{center}
\vspace*{0.5cm}

\footnotesize
\begin{verbatim}
NodePtr ithNode(List l, unsigned i) {
    if (i == 1)
        return l;
    return ithNode(l->nextPtr, i - 1);
}
\end{verbatim}
Analyse des Algorithmus ergibt: 

Sei $n = 2^k -1$ Zahl der Knoten der Liste. Sei $b_k$ ``Aufwand'' f\"ur Berechnung des Teilbaums.
Dann gilt n\"aherungsweise 
\begin{enumerate}
\item $b_0 = 1$
\item $b_1 = 2^k + 2 * b_k$
\end{enumerate}
Der Summand $2^k$ kommt von \\[0.3cm]
\hspace*{1.3cm} \texttt{ithNode(l, middle);}

Ergebnis:
Ist $n$ L\"ange der anfangs gegebenen Liste, so betr\"agt der Aufwand der Berechnung des balancierten Baums\\[0.3cm]
\hspace*{1.3cm} $\Oh(n * \log(n))$

Aufwand zum Sortieren betr\"agt ebenfalls \\[0.3cm]
\hspace*{1.3cm} $\Oh(n * \log(n))$

Gesamte Komplexit\"at:\\[0.3cm]
\hspace*{1.3cm} $\Oh(n * \log(n))$

Verfahren ist geeignet falls Tabelle, in der gesucht werden soll,
von Anfang an vollst\"andig bekannt ist.


\vspace*{\fill}
\tiny \addtocounter{mypage}{1}
\rule{17cm}{1mm}
Dictionaries  \hspace*{\fill} Seite \arabic{mypage}
\end{slide}



%%% Local Variables: 
%%% mode: latex
%%% TeX-master: "suche.tex"
%%% TeX-master: "suche"
%%% End: 
