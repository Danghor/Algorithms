\documentclass[11pt]{article}

    \usepackage[breakable]{tcolorbox}
    \usepackage{parskip} % Stop auto-indenting (to mimic markdown behaviour)
    

    % Basic figure setup, for now with no caption control since it's done
    % automatically by Pandoc (which extracts ![](path) syntax from Markdown).
    \usepackage{graphicx}
    % Maintain compatibility with old templates. Remove in nbconvert 6.0
    \let\Oldincludegraphics\includegraphics
    % Ensure that by default, figures have no caption (until we provide a
    % proper Figure object with a Caption API and a way to capture that
    % in the conversion process - todo).
    \usepackage{caption}
    \DeclareCaptionFormat{nocaption}{}
    \captionsetup{format=nocaption,aboveskip=0pt,belowskip=0pt}

    \usepackage{float}
    \floatplacement{figure}{H} % forces figures to be placed at the correct location
    \usepackage{xcolor} % Allow colors to be defined
    \usepackage{enumerate} % Needed for markdown enumerations to work
    \usepackage{geometry} % Used to adjust the document margins
    \usepackage{amsmath} % Equations
    \usepackage{amssymb} % Equations
    \usepackage{textcomp} % defines textquotesingle
    % Hack from http://tex.stackexchange.com/a/47451/13684:
    \AtBeginDocument{%
        \def\PYZsq{\textquotesingle}% Upright quotes in Pygmentized code
    }
    \usepackage{upquote} % Upright quotes for verbatim code
    \usepackage{eurosym} % defines \euro

    \usepackage{iftex}
    \ifPDFTeX
        \usepackage[T1]{fontenc}
        \IfFileExists{alphabeta.sty}{
              \usepackage{alphabeta}
          }{
              \usepackage[mathletters]{ucs}
              \usepackage[utf8x]{inputenc}
          }
    \else
        \usepackage{fontspec}
        \usepackage{unicode-math}
    \fi

    \usepackage{fancyvrb} % verbatim replacement that allows latex
    \usepackage{grffile} % extends the file name processing of package graphics 
                         % to support a larger range
    \makeatletter % fix for old versions of grffile with XeLaTeX
    \@ifpackagelater{grffile}{2019/11/01}
    {
      % Do nothing on new versions
    }
    {
      \def\Gread@@xetex#1{%
        \IfFileExists{"\Gin@base".bb}%
        {\Gread@eps{\Gin@base.bb}}%
        {\Gread@@xetex@aux#1}%
      }
    }
    \makeatother
    \usepackage[Export]{adjustbox} % Used to constrain images to a maximum size
    \adjustboxset{max size={0.9\linewidth}{0.9\paperheight}}

    % The hyperref package gives us a pdf with properly built
    % internal navigation ('pdf bookmarks' for the table of contents,
    % internal cross-reference links, web links for URLs, etc.)
    \usepackage{hyperref}
    % The default LaTeX title has an obnoxious amount of whitespace. By default,
    % titling removes some of it. It also provides customization options.
    \usepackage{titling}
    \usepackage{longtable} % longtable support required by pandoc >1.10
    \usepackage{booktabs}  % table support for pandoc > 1.12.2
    \usepackage{array}     % table support for pandoc >= 2.11.3
    \usepackage{calc}      % table minipage width calculation for pandoc >= 2.11.1
    \usepackage[inline]{enumitem} % IRkernel/repr support (it uses the enumerate* environment)
    \usepackage[normalem]{ulem} % ulem is needed to support strikethroughs (\sout)
                                % normalem makes italics be italics, not underlines
    \usepackage{mathrsfs}
    

    
    % Colors for the hyperref package
    \definecolor{urlcolor}{rgb}{0,.145,.698}
    \definecolor{linkcolor}{rgb}{.71,0.21,0.01}
    \definecolor{citecolor}{rgb}{.12,.54,.11}

    % ANSI colors
    \definecolor{ansi-black}{HTML}{3E424D}
    \definecolor{ansi-black-intense}{HTML}{282C36}
    \definecolor{ansi-red}{HTML}{E75C58}
    \definecolor{ansi-red-intense}{HTML}{B22B31}
    \definecolor{ansi-green}{HTML}{00A250}
    \definecolor{ansi-green-intense}{HTML}{007427}
    \definecolor{ansi-yellow}{HTML}{DDB62B}
    \definecolor{ansi-yellow-intense}{HTML}{B27D12}
    \definecolor{ansi-blue}{HTML}{208FFB}
    \definecolor{ansi-blue-intense}{HTML}{0065CA}
    \definecolor{ansi-magenta}{HTML}{D160C4}
    \definecolor{ansi-magenta-intense}{HTML}{A03196}
    \definecolor{ansi-cyan}{HTML}{60C6C8}
    \definecolor{ansi-cyan-intense}{HTML}{258F8F}
    \definecolor{ansi-white}{HTML}{C5C1B4}
    \definecolor{ansi-white-intense}{HTML}{A1A6B2}
    \definecolor{ansi-default-inverse-fg}{HTML}{FFFFFF}
    \definecolor{ansi-default-inverse-bg}{HTML}{000000}

    % common color for the border for error outputs.
    \definecolor{outerrorbackground}{HTML}{FFDFDF}

    % commands and environments needed by pandoc snippets
    % extracted from the output of `pandoc -s`
    \providecommand{\tightlist}{%
      \setlength{\itemsep}{0pt}\setlength{\parskip}{0pt}}
    \DefineVerbatimEnvironment{Highlighting}{Verbatim}{commandchars=\\\{\}}
    % Add ',fontsize=\small' for more characters per line
    \newenvironment{Shaded}{}{}
    \newcommand{\KeywordTok}[1]{\textcolor[rgb]{0.00,0.44,0.13}{\textbf{{#1}}}}
    \newcommand{\DataTypeTok}[1]{\textcolor[rgb]{0.56,0.13,0.00}{{#1}}}
    \newcommand{\DecValTok}[1]{\textcolor[rgb]{0.25,0.63,0.44}{{#1}}}
    \newcommand{\BaseNTok}[1]{\textcolor[rgb]{0.25,0.63,0.44}{{#1}}}
    \newcommand{\FloatTok}[1]{\textcolor[rgb]{0.25,0.63,0.44}{{#1}}}
    \newcommand{\CharTok}[1]{\textcolor[rgb]{0.25,0.44,0.63}{{#1}}}
    \newcommand{\StringTok}[1]{\textcolor[rgb]{0.25,0.44,0.63}{{#1}}}
    \newcommand{\CommentTok}[1]{\textcolor[rgb]{0.38,0.63,0.69}{\textit{{#1}}}}
    \newcommand{\OtherTok}[1]{\textcolor[rgb]{0.00,0.44,0.13}{{#1}}}
    \newcommand{\AlertTok}[1]{\textcolor[rgb]{1.00,0.00,0.00}{\textbf{{#1}}}}
    \newcommand{\FunctionTok}[1]{\textcolor[rgb]{0.02,0.16,0.49}{{#1}}}
    \newcommand{\RegionMarkerTok}[1]{{#1}}
    \newcommand{\ErrorTok}[1]{\textcolor[rgb]{1.00,0.00,0.00}{\textbf{{#1}}}}
    \newcommand{\NormalTok}[1]{{#1}}
    
    % Additional commands for more recent versions of Pandoc
    \newcommand{\ConstantTok}[1]{\textcolor[rgb]{0.53,0.00,0.00}{{#1}}}
    \newcommand{\SpecialCharTok}[1]{\textcolor[rgb]{0.25,0.44,0.63}{{#1}}}
    \newcommand{\VerbatimStringTok}[1]{\textcolor[rgb]{0.25,0.44,0.63}{{#1}}}
    \newcommand{\SpecialStringTok}[1]{\textcolor[rgb]{0.73,0.40,0.53}{{#1}}}
    \newcommand{\ImportTok}[1]{{#1}}
    \newcommand{\DocumentationTok}[1]{\textcolor[rgb]{0.73,0.13,0.13}{\textit{{#1}}}}
    \newcommand{\AnnotationTok}[1]{\textcolor[rgb]{0.38,0.63,0.69}{\textbf{\textit{{#1}}}}}
    \newcommand{\CommentVarTok}[1]{\textcolor[rgb]{0.38,0.63,0.69}{\textbf{\textit{{#1}}}}}
    \newcommand{\VariableTok}[1]{\textcolor[rgb]{0.10,0.09,0.49}{{#1}}}
    \newcommand{\ControlFlowTok}[1]{\textcolor[rgb]{0.00,0.44,0.13}{\textbf{{#1}}}}
    \newcommand{\OperatorTok}[1]{\textcolor[rgb]{0.40,0.40,0.40}{{#1}}}
    \newcommand{\BuiltInTok}[1]{{#1}}
    \newcommand{\ExtensionTok}[1]{{#1}}
    \newcommand{\PreprocessorTok}[1]{\textcolor[rgb]{0.74,0.48,0.00}{{#1}}}
    \newcommand{\AttributeTok}[1]{\textcolor[rgb]{0.49,0.56,0.16}{{#1}}}
    \newcommand{\InformationTok}[1]{\textcolor[rgb]{0.38,0.63,0.69}{\textbf{\textit{{#1}}}}}
    \newcommand{\WarningTok}[1]{\textcolor[rgb]{0.38,0.63,0.69}{\textbf{\textit{{#1}}}}}
    
    
    % Define a nice break command that doesn't care if a line doesn't already
    % exist.
    \def\br{\hspace*{\fill} \\* }
    % Math Jax compatibility definitions
    \def\gt{>}
    \def\lt{<}
    \let\Oldtex\TeX
    \let\Oldlatex\LaTeX
    \renewcommand{\TeX}{\textrm{\Oldtex}}
    \renewcommand{\LaTeX}{\textrm{\Oldlatex}}
    % Document parameters
    % Document title
    \title{An Introduction to \textsl{Python} via a \textsl{Juyter} Notebook}
    
% Pygments definitions
\makeatletter
\def\PY@reset{\let\PY@it=\relax \let\PY@bf=\relax%
    \let\PY@ul=\relax \let\PY@tc=\relax%
    \let\PY@bc=\relax \let\PY@ff=\relax}
\def\PY@tok#1{\csname PY@tok@#1\endcsname}
\def\PY@toks#1+{\ifx\relax#1\empty\else%
    \PY@tok{#1}\expandafter\PY@toks\fi}
\def\PY@do#1{\PY@bc{\PY@tc{\PY@ul{%
    \PY@it{\PY@bf{\PY@ff{#1}}}}}}}
\def\PY#1#2{\PY@reset\PY@toks#1+\relax+\PY@do{#2}}

\@namedef{PY@tok@w}{\def\PY@tc##1{\textcolor[rgb]{0.73,0.73,0.73}{##1}}}
\@namedef{PY@tok@c}{\let\PY@it=\textit\def\PY@tc##1{\textcolor[rgb]{0.24,0.48,0.48}{##1}}}
\@namedef{PY@tok@cp}{\def\PY@tc##1{\textcolor[rgb]{0.61,0.40,0.00}{##1}}}
\@namedef{PY@tok@k}{\let\PY@bf=\textbf\def\PY@tc##1{\textcolor[rgb]{0.00,0.50,0.00}{##1}}}
\@namedef{PY@tok@kp}{\def\PY@tc##1{\textcolor[rgb]{0.00,0.50,0.00}{##1}}}
\@namedef{PY@tok@kt}{\def\PY@tc##1{\textcolor[rgb]{0.69,0.00,0.25}{##1}}}
\@namedef{PY@tok@o}{\def\PY@tc##1{\textcolor[rgb]{0.40,0.40,0.40}{##1}}}
\@namedef{PY@tok@ow}{\let\PY@bf=\textbf\def\PY@tc##1{\textcolor[rgb]{0.67,0.13,1.00}{##1}}}
\@namedef{PY@tok@nb}{\def\PY@tc##1{\textcolor[rgb]{0.00,0.50,0.00}{##1}}}
\@namedef{PY@tok@nf}{\def\PY@tc##1{\textcolor[rgb]{0.00,0.00,1.00}{##1}}}
\@namedef{PY@tok@nc}{\let\PY@bf=\textbf\def\PY@tc##1{\textcolor[rgb]{0.00,0.00,1.00}{##1}}}
\@namedef{PY@tok@nn}{\let\PY@bf=\textbf\def\PY@tc##1{\textcolor[rgb]{0.00,0.00,1.00}{##1}}}
\@namedef{PY@tok@ne}{\let\PY@bf=\textbf\def\PY@tc##1{\textcolor[rgb]{0.80,0.25,0.22}{##1}}}
\@namedef{PY@tok@nv}{\def\PY@tc##1{\textcolor[rgb]{0.10,0.09,0.49}{##1}}}
\@namedef{PY@tok@no}{\def\PY@tc##1{\textcolor[rgb]{0.53,0.00,0.00}{##1}}}
\@namedef{PY@tok@nl}{\def\PY@tc##1{\textcolor[rgb]{0.46,0.46,0.00}{##1}}}
\@namedef{PY@tok@ni}{\let\PY@bf=\textbf\def\PY@tc##1{\textcolor[rgb]{0.44,0.44,0.44}{##1}}}
\@namedef{PY@tok@na}{\def\PY@tc##1{\textcolor[rgb]{0.41,0.47,0.13}{##1}}}
\@namedef{PY@tok@nt}{\let\PY@bf=\textbf\def\PY@tc##1{\textcolor[rgb]{0.00,0.50,0.00}{##1}}}
\@namedef{PY@tok@nd}{\def\PY@tc##1{\textcolor[rgb]{0.67,0.13,1.00}{##1}}}
\@namedef{PY@tok@s}{\def\PY@tc##1{\textcolor[rgb]{0.73,0.13,0.13}{##1}}}
\@namedef{PY@tok@sd}{\let\PY@it=\textit\def\PY@tc##1{\textcolor[rgb]{0.73,0.13,0.13}{##1}}}
\@namedef{PY@tok@si}{\let\PY@bf=\textbf\def\PY@tc##1{\textcolor[rgb]{0.64,0.35,0.47}{##1}}}
\@namedef{PY@tok@se}{\let\PY@bf=\textbf\def\PY@tc##1{\textcolor[rgb]{0.67,0.36,0.12}{##1}}}
\@namedef{PY@tok@sr}{\def\PY@tc##1{\textcolor[rgb]{0.64,0.35,0.47}{##1}}}
\@namedef{PY@tok@ss}{\def\PY@tc##1{\textcolor[rgb]{0.10,0.09,0.49}{##1}}}
\@namedef{PY@tok@sx}{\def\PY@tc##1{\textcolor[rgb]{0.00,0.50,0.00}{##1}}}
\@namedef{PY@tok@m}{\def\PY@tc##1{\textcolor[rgb]{0.40,0.40,0.40}{##1}}}
\@namedef{PY@tok@gh}{\let\PY@bf=\textbf\def\PY@tc##1{\textcolor[rgb]{0.00,0.00,0.50}{##1}}}
\@namedef{PY@tok@gu}{\let\PY@bf=\textbf\def\PY@tc##1{\textcolor[rgb]{0.50,0.00,0.50}{##1}}}
\@namedef{PY@tok@gd}{\def\PY@tc##1{\textcolor[rgb]{0.63,0.00,0.00}{##1}}}
\@namedef{PY@tok@gi}{\def\PY@tc##1{\textcolor[rgb]{0.00,0.52,0.00}{##1}}}
\@namedef{PY@tok@gr}{\def\PY@tc##1{\textcolor[rgb]{0.89,0.00,0.00}{##1}}}
\@namedef{PY@tok@ge}{\let\PY@it=\textit}
\@namedef{PY@tok@gs}{\let\PY@bf=\textbf}
\@namedef{PY@tok@gp}{\let\PY@bf=\textbf\def\PY@tc##1{\textcolor[rgb]{0.00,0.00,0.50}{##1}}}
\@namedef{PY@tok@go}{\def\PY@tc##1{\textcolor[rgb]{0.44,0.44,0.44}{##1}}}
\@namedef{PY@tok@gt}{\def\PY@tc##1{\textcolor[rgb]{0.00,0.27,0.87}{##1}}}
\@namedef{PY@tok@err}{\def\PY@bc##1{{\setlength{\fboxsep}{\string -\fboxrule}\fcolorbox[rgb]{1.00,0.00,0.00}{1,1,1}{\strut ##1}}}}
\@namedef{PY@tok@kc}{\let\PY@bf=\textbf\def\PY@tc##1{\textcolor[rgb]{0.00,0.50,0.00}{##1}}}
\@namedef{PY@tok@kd}{\let\PY@bf=\textbf\def\PY@tc##1{\textcolor[rgb]{0.00,0.50,0.00}{##1}}}
\@namedef{PY@tok@kn}{\let\PY@bf=\textbf\def\PY@tc##1{\textcolor[rgb]{0.00,0.50,0.00}{##1}}}
\@namedef{PY@tok@kr}{\let\PY@bf=\textbf\def\PY@tc##1{\textcolor[rgb]{0.00,0.50,0.00}{##1}}}
\@namedef{PY@tok@bp}{\def\PY@tc##1{\textcolor[rgb]{0.00,0.50,0.00}{##1}}}
\@namedef{PY@tok@fm}{\def\PY@tc##1{\textcolor[rgb]{0.00,0.00,1.00}{##1}}}
\@namedef{PY@tok@vc}{\def\PY@tc##1{\textcolor[rgb]{0.10,0.09,0.49}{##1}}}
\@namedef{PY@tok@vg}{\def\PY@tc##1{\textcolor[rgb]{0.10,0.09,0.49}{##1}}}
\@namedef{PY@tok@vi}{\def\PY@tc##1{\textcolor[rgb]{0.10,0.09,0.49}{##1}}}
\@namedef{PY@tok@vm}{\def\PY@tc##1{\textcolor[rgb]{0.10,0.09,0.49}{##1}}}
\@namedef{PY@tok@sa}{\def\PY@tc##1{\textcolor[rgb]{0.73,0.13,0.13}{##1}}}
\@namedef{PY@tok@sb}{\def\PY@tc##1{\textcolor[rgb]{0.73,0.13,0.13}{##1}}}
\@namedef{PY@tok@sc}{\def\PY@tc##1{\textcolor[rgb]{0.73,0.13,0.13}{##1}}}
\@namedef{PY@tok@dl}{\def\PY@tc##1{\textcolor[rgb]{0.73,0.13,0.13}{##1}}}
\@namedef{PY@tok@s2}{\def\PY@tc##1{\textcolor[rgb]{0.73,0.13,0.13}{##1}}}
\@namedef{PY@tok@sh}{\def\PY@tc##1{\textcolor[rgb]{0.73,0.13,0.13}{##1}}}
\@namedef{PY@tok@s1}{\def\PY@tc##1{\textcolor[rgb]{0.73,0.13,0.13}{##1}}}
\@namedef{PY@tok@mb}{\def\PY@tc##1{\textcolor[rgb]{0.40,0.40,0.40}{##1}}}
\@namedef{PY@tok@mf}{\def\PY@tc##1{\textcolor[rgb]{0.40,0.40,0.40}{##1}}}
\@namedef{PY@tok@mh}{\def\PY@tc##1{\textcolor[rgb]{0.40,0.40,0.40}{##1}}}
\@namedef{PY@tok@mi}{\def\PY@tc##1{\textcolor[rgb]{0.40,0.40,0.40}{##1}}}
\@namedef{PY@tok@il}{\def\PY@tc##1{\textcolor[rgb]{0.40,0.40,0.40}{##1}}}
\@namedef{PY@tok@mo}{\def\PY@tc##1{\textcolor[rgb]{0.40,0.40,0.40}{##1}}}
\@namedef{PY@tok@ch}{\let\PY@it=\textit\def\PY@tc##1{\textcolor[rgb]{0.24,0.48,0.48}{##1}}}
\@namedef{PY@tok@cm}{\let\PY@it=\textit\def\PY@tc##1{\textcolor[rgb]{0.24,0.48,0.48}{##1}}}
\@namedef{PY@tok@cpf}{\let\PY@it=\textit\def\PY@tc##1{\textcolor[rgb]{0.24,0.48,0.48}{##1}}}
\@namedef{PY@tok@c1}{\let\PY@it=\textit\def\PY@tc##1{\textcolor[rgb]{0.24,0.48,0.48}{##1}}}
\@namedef{PY@tok@cs}{\let\PY@it=\textit\def\PY@tc##1{\textcolor[rgb]{0.24,0.48,0.48}{##1}}}

\def\PYZbs{\char`\\}
\def\PYZus{\char`\_}
\def\PYZob{\char`\{}
\def\PYZcb{\char`\}}
\def\PYZca{\char`\^}
\def\PYZam{\char`\&}
\def\PYZlt{\char`\<}
\def\PYZgt{\char`\>}
\def\PYZsh{\char`\#}
\def\PYZpc{\char`\%}
\def\PYZdl{\char`\$}
\def\PYZhy{\char`\-}
\def\PYZsq{\char`\'}
\def\PYZdq{\char`\"}
\def\PYZti{\char`\~}
% for compatibility with earlier versions
\def\PYZat{@}
\def\PYZlb{[}
\def\PYZrb{]}
\makeatother


    % For linebreaks inside Verbatim environment from package fancyvrb. 
    \makeatletter
        \newbox\Wrappedcontinuationbox 
        \newbox\Wrappedvisiblespacebox 
        \newcommand*\Wrappedvisiblespace {\textcolor{red}{\textvisiblespace}} 
        \newcommand*\Wrappedcontinuationsymbol {\textcolor{red}{\llap{\tiny$\m@th\hookrightarrow$}}} 
        \newcommand*\Wrappedcontinuationindent {3ex } 
        \newcommand*\Wrappedafterbreak {\kern\Wrappedcontinuationindent\copy\Wrappedcontinuationbox} 
        % Take advantage of the already applied Pygments mark-up to insert 
        % potential linebreaks for TeX processing. 
        %        {, <, #, %, $, ' and ": go to next line. 
        %        _, }, ^, &, >, - and ~: stay at end of broken line. 
        % Use of \textquotesingle for straight quote. 
        \newcommand*\Wrappedbreaksatspecials {% 
            \def\PYGZus{\discretionary{\char`\_}{\Wrappedafterbreak}{\char`\_}}% 
            \def\PYGZob{\discretionary{}{\Wrappedafterbreak\char`\{}{\char`\{}}% 
            \def\PYGZcb{\discretionary{\char`\}}{\Wrappedafterbreak}{\char`\}}}% 
            \def\PYGZca{\discretionary{\char`\^}{\Wrappedafterbreak}{\char`\^}}% 
            \def\PYGZam{\discretionary{\char`\&}{\Wrappedafterbreak}{\char`\&}}% 
            \def\PYGZlt{\discretionary{}{\Wrappedafterbreak\char`\<}{\char`\<}}% 
            \def\PYGZgt{\discretionary{\char`\>}{\Wrappedafterbreak}{\char`\>}}% 
            \def\PYGZsh{\discretionary{}{\Wrappedafterbreak\char`\#}{\char`\#}}% 
            \def\PYGZpc{\discretionary{}{\Wrappedafterbreak\char`\%}{\char`\%}}% 
            \def\PYGZdl{\discretionary{}{\Wrappedafterbreak\char`\$}{\char`\$}}% 
            \def\PYGZhy{\discretionary{\char`\-}{\Wrappedafterbreak}{\char`\-}}% 
            \def\PYGZsq{\discretionary{}{\Wrappedafterbreak\textquotesingle}{\textquotesingle}}% 
            \def\PYGZdq{\discretionary{}{\Wrappedafterbreak\char`\"}{\char`\"}}% 
            \def\PYGZti{\discretionary{\char`\~}{\Wrappedafterbreak}{\char`\~}}% 
        } 
        % Some characters . , ; ? ! / are not pygmentized. 
        % This macro makes them "active" and they will insert potential linebreaks 
        \newcommand*\Wrappedbreaksatpunct {% 
            \lccode`\~`\.\lowercase{\def~}{\discretionary{\hbox{\char`\.}}{\Wrappedafterbreak}{\hbox{\char`\.}}}% 
            \lccode`\~`\,\lowercase{\def~}{\discretionary{\hbox{\char`\,}}{\Wrappedafterbreak}{\hbox{\char`\,}}}% 
            \lccode`\~`\;\lowercase{\def~}{\discretionary{\hbox{\char`\;}}{\Wrappedafterbreak}{\hbox{\char`\;}}}% 
            \lccode`\~`\:\lowercase{\def~}{\discretionary{\hbox{\char`\:}}{\Wrappedafterbreak}{\hbox{\char`\:}}}% 
            \lccode`\~`\?\lowercase{\def~}{\discretionary{\hbox{\char`\?}}{\Wrappedafterbreak}{\hbox{\char`\?}}}% 
            \lccode`\~`\!\lowercase{\def~}{\discretionary{\hbox{\char`\!}}{\Wrappedafterbreak}{\hbox{\char`\!}}}% 
            \lccode`\~`\/\lowercase{\def~}{\discretionary{\hbox{\char`\/}}{\Wrappedafterbreak}{\hbox{\char`\/}}}% 
            \catcode`\.\active
            \catcode`\,\active 
            \catcode`\;\active
            \catcode`\:\active
            \catcode`\?\active
            \catcode`\!\active
            \catcode`\/\active 
            \lccode`\~`\~ 	
        }
    \makeatother

    \let\OriginalVerbatim=\Verbatim
    \makeatletter
    \renewcommand{\Verbatim}[1][1]{%
        %\parskip\z@skip
        \sbox\Wrappedcontinuationbox {\Wrappedcontinuationsymbol}%
        \sbox\Wrappedvisiblespacebox {\FV@SetupFont\Wrappedvisiblespace}%
        \def\FancyVerbFormatLine ##1{\hsize\linewidth
            \vtop{\raggedright\hyphenpenalty\z@\exhyphenpenalty\z@
                \doublehyphendemerits\z@\finalhyphendemerits\z@
                \strut ##1\strut}%
        }%
        % If the linebreak is at a space, the latter will be displayed as visible
        % space at end of first line, and a continuation symbol starts next line.
        % Stretch/shrink are however usually zero for typewriter font.
        \def\FV@Space {%
            \nobreak\hskip\z@ plus\fontdimen3\font minus\fontdimen4\font
            \discretionary{\copy\Wrappedvisiblespacebox}{\Wrappedafterbreak}
            {\kern\fontdimen2\font}%
        }%
        
        % Allow breaks at special characters using \PYG... macros.
        \Wrappedbreaksatspecials
        % Breaks at punctuation characters . , ; ? ! and / need catcode=\active 	
        \OriginalVerbatim[#1,codes*=\Wrappedbreaksatpunct]%
    }
    \makeatother

    % Exact colors from NB
    \definecolor{incolor}{HTML}{303F9F}
    \definecolor{outcolor}{HTML}{D84315}
    \definecolor{cellborder}{HTML}{CFCFCF}
    \definecolor{cellbackground}{HTML}{F7F7F7}
    
    % prompt
    \makeatletter
    \newcommand{\boxspacing}{\kern\kvtcb@left@rule\kern\kvtcb@boxsep}
    \makeatother
    \newcommand{\prompt}[4]{
        {\ttfamily\llap{{\color{#2}[#3]:\hspace{3pt}#4}}\vspace{-\baselineskip}}
    }
    

    
    % Prevent overflowing lines due to hard-to-break entities
    \sloppy 
    % Setup hyperref package
    \hypersetup{
      breaklinks=true,  % so long urls are correctly broken across lines
      colorlinks=true,
      urlcolor=urlcolor,
      linkcolor=linkcolor,
      citecolor=citecolor,
      }
    % Slightly bigger margins than the latex defaults
    
    \geometry{verbose,tmargin=1in,bmargin=1in,lmargin=1in,rmargin=1in}
    
    

\begin{document}
    
    \maketitle
    
    

    
    A Jupyter notebook consists of two different types of cells: -
\emph{Code cells} contain Python code, while - \emph{Markdown cells} are
used to write comments.

The current cell is a \emph{markdown cell}, the next cell is a
\emph{code cell}. This code cell is only needed to format this notebook.

    \begin{tcolorbox}[breakable, size=fbox, boxrule=1pt, pad at break*=1mm,colback=cellbackground, colframe=cellborder]
\prompt{In}{incolor}{1}{\boxspacing}
\begin{Verbatim}[commandchars=\\\{\}]
\PY{k+kn}{from} \PY{n+nn}{IPython}\PY{n+nn}{.}\PY{n+nn}{core}\PY{n+nn}{.}\PY{n+nn}{display} \PY{k+kn}{import} \PY{n}{HTML}
\PY{k}{with} \PY{n+nb}{open}\PY{p}{(}\PY{l+s+s1}{\PYZsq{}}\PY{l+s+s1}{../style.css}\PY{l+s+s1}{\PYZsq{}}\PY{p}{,} \PY{l+s+s1}{\PYZsq{}}\PY{l+s+s1}{r}\PY{l+s+s1}{\PYZsq{}}\PY{p}{)} \PY{k}{as} \PY{n}{file}\PY{p}{:}
    \PY{n}{css} \PY{o}{=} \PY{n}{file}\PY{o}{.}\PY{n}{read}\PY{p}{(}\PY{p}{)}
\PY{n}{HTML}\PY{p}{(}\PY{n}{css}\PY{p}{)}
\end{Verbatim}
\end{tcolorbox}

            \begin{tcolorbox}[breakable, size=fbox, boxrule=.5pt, pad at break*=1mm, opacityfill=0]
\prompt{Out}{outcolor}{1}{\boxspacing}
\begin{Verbatim}[commandchars=\\\{\}]
<IPython.core.display.HTML object>
\end{Verbatim}
\end{tcolorbox}
        
    \hypertarget{an-introduction-to-python}{%
\section{\texorpdfstring{An Introduction to
\emph{Python}}{An Introduction to Python}}\label{an-introduction-to-python}}

Let us begin by checking the version of our Python interpreter. The
package \texttt{sys} contains the variable \texttt{version} that stores
this information.

    \begin{tcolorbox}[breakable, size=fbox, boxrule=1pt, pad at break*=1mm,colback=cellbackground, colframe=cellborder]
\prompt{In}{incolor}{2}{\boxspacing}
\begin{Verbatim}[commandchars=\\\{\}]
\PY{k+kn}{import} \PY{n+nn}{sys}
\PY{n}{sys}\PY{o}{.}\PY{n}{version}
\end{Verbatim}
\end{tcolorbox}

            \begin{tcolorbox}[breakable, size=fbox, boxrule=.5pt, pad at break*=1mm, opacityfill=0]
\prompt{Out}{outcolor}{2}{\boxspacing}
\begin{Verbatim}[commandchars=\\\{\}]
'3.10.4 (main, Mar 31 2022, 03:38:35) [Clang 12.0.0 ]'
\end{Verbatim}
\end{tcolorbox}
        
    This notebook gives a short introduction to \emph{Python}. We will just
present the basics. More interesting features will be presented later.

    \hypertarget{evaluating-expressions}{%
\subsection{Evaluating expressions}\label{evaluating-expressions}}

As Python is an interactive language, expressions can be evaluated
directly. In a \emph{Jupyter} notebook we just have to type Ctrl-Enter
in the cell containing the expression. Instead of Ctrl-Enter we can also
use Shift-Enter.

    \begin{tcolorbox}[breakable, size=fbox, boxrule=1pt, pad at break*=1mm,colback=cellbackground, colframe=cellborder]
\prompt{In}{incolor}{3}{\boxspacing}
\begin{Verbatim}[commandchars=\\\{\}]
\PY{l+m+mi}{1} \PY{o}{+} \PY{l+m+mi}{2}
\end{Verbatim}
\end{tcolorbox}

            \begin{tcolorbox}[breakable, size=fbox, boxrule=.5pt, pad at break*=1mm, opacityfill=0]
\prompt{Out}{outcolor}{3}{\boxspacing}
\begin{Verbatim}[commandchars=\\\{\}]
3
\end{Verbatim}
\end{tcolorbox}
        
    In \emph{Python}, the precision of integers is not bounded. Hence, the
following expression does \textbf{not} cause an integer overflow.

    \begin{tcolorbox}[breakable, size=fbox, boxrule=1pt, pad at break*=1mm,colback=cellbackground, colframe=cellborder]
\prompt{In}{incolor}{4}{\boxspacing}
\begin{Verbatim}[commandchars=\\\{\}]
\PY{l+m+mi}{1} \PY{o}{*} \PY{l+m+mi}{2} \PY{o}{*} \PY{l+m+mi}{3} \PY{o}{*} \PY{l+m+mi}{4} \PY{o}{*} \PY{l+m+mi}{5} \PY{o}{*} \PY{l+m+mi}{6} \PY{o}{*} \PY{l+m+mi}{7} \PY{o}{*} \PY{l+m+mi}{8} \PY{o}{*} \PY{l+m+mi}{9} \PY{o}{*} \PY{l+m+mi}{10} \PY{o}{*} \PY{l+m+mi}{11} \PY{o}{*} \PY{l+m+mi}{12} \PY{o}{*} \PY{l+m+mi}{13} \PY{o}{*} \PY{l+m+mi}{14} \PY{o}{*} \PY{l+m+mi}{15} \PY{o}{*} \PY{l+m+mi}{16} \PY{o}{*} \PY{l+m+mi}{17} \PY{o}{*} \PY{l+m+mi}{18} \PY{o}{*} \PY{l+m+mi}{19} \PY{o}{*} \PY{l+m+mi}{20} \PY{o}{*} \PY{l+m+mi}{21} \PY{o}{*} \PY{l+m+mi}{22}
\end{Verbatim}
\end{tcolorbox}

            \begin{tcolorbox}[breakable, size=fbox, boxrule=.5pt, pad at break*=1mm, opacityfill=0]
\prompt{Out}{outcolor}{4}{\boxspacing}
\begin{Verbatim}[commandchars=\\\{\}]
1124000727777607680000
\end{Verbatim}
\end{tcolorbox}
        
    Here is another example that show that \emph{Python} is able to compute
with huge numbers.\\
In this example, the operator \texttt{**} is used for exponentiation,
i.e.~the expression

\begin{verbatim}
    7 ** 125
\end{verbatim}

is interpreted as \(7^{125}\).

    \begin{tcolorbox}[breakable, size=fbox, boxrule=1pt, pad at break*=1mm,colback=cellbackground, colframe=cellborder]
\prompt{In}{incolor}{5}{\boxspacing}
\begin{Verbatim}[commandchars=\\\{\}]
\PY{l+m+mi}{7} \PY{o}{*}\PY{o}{*} \PY{l+m+mi}{125}
\end{Verbatim}
\end{tcolorbox}

            \begin{tcolorbox}[breakable, size=fbox, boxrule=.5pt, pad at break*=1mm, opacityfill=0]
\prompt{Out}{outcolor}{5}{\boxspacing}
\begin{Verbatim}[commandchars=\\\{\}]
43376549480979932825373547572631882516978329946204051017448930177445694327209941
68089672192211758909320807
\end{Verbatim}
\end{tcolorbox}
        
    In order to print numbers or strings, we can use the function
\texttt{print}. This function can print objects of any type. In the
following example, this function prints a \emph{string}. In
\emph{Python} any character sequence enclosed in single quotes is
interpreted as a \emph{string}.

    \begin{tcolorbox}[breakable, size=fbox, boxrule=1pt, pad at break*=1mm,colback=cellbackground, colframe=cellborder]
\prompt{In}{incolor}{6}{\boxspacing}
\begin{Verbatim}[commandchars=\\\{\}]
\PY{n+nb}{print}\PY{p}{(}\PY{l+s+s1}{\PYZsq{}}\PY{l+s+s1}{Hello, World!}\PY{l+s+s1}{\PYZsq{}}\PY{p}{)}
\end{Verbatim}
\end{tcolorbox}

    \begin{Verbatim}[commandchars=\\\{\}]
Hello, World!
    \end{Verbatim}

    The last expression in every notebook cell is automatically printed.
Therefore, printing
\texttt{\textquotesingle{}Hello,\ World!\textquotesingle{}} can also be
done by just executing the following cell.

    \begin{tcolorbox}[breakable, size=fbox, boxrule=1pt, pad at break*=1mm,colback=cellbackground, colframe=cellborder]
\prompt{In}{incolor}{7}{\boxspacing}
\begin{Verbatim}[commandchars=\\\{\}]
\PY{l+s+s1}{\PYZsq{}}\PY{l+s+s1}{Hello, World!}\PY{l+s+s1}{\PYZsq{}}
\end{Verbatim}
\end{tcolorbox}

            \begin{tcolorbox}[breakable, size=fbox, boxrule=.5pt, pad at break*=1mm, opacityfill=0]
\prompt{Out}{outcolor}{7}{\boxspacing}
\begin{Verbatim}[commandchars=\\\{\}]
'Hello, World!'
\end{Verbatim}
\end{tcolorbox}
        
    If we want to supress the last expression from being printed, we can end
the expression with a semicolon:

    \begin{tcolorbox}[breakable, size=fbox, boxrule=1pt, pad at break*=1mm,colback=cellbackground, colframe=cellborder]
\prompt{In}{incolor}{8}{\boxspacing}
\begin{Verbatim}[commandchars=\\\{\}]
\PY{l+s+s1}{\PYZsq{}}\PY{l+s+s1}{Hello, World!}\PY{l+s+s1}{\PYZsq{}}\PY{p}{;}
\end{Verbatim}
\end{tcolorbox}

    Instead of using single quotes we can also use double quotes as seen in
the next example. However, the convention is to use single quotes,
unless the string itself contains a single quote.

    \begin{tcolorbox}[breakable, size=fbox, boxrule=1pt, pad at break*=1mm,colback=cellbackground, colframe=cellborder]
\prompt{In}{incolor}{9}{\boxspacing}
\begin{Verbatim}[commandchars=\\\{\}]
\PY{l+s+s2}{\PYZdq{}}\PY{l+s+s2}{Hello, World!}\PY{l+s+s2}{\PYZdq{}}
\end{Verbatim}
\end{tcolorbox}

            \begin{tcolorbox}[breakable, size=fbox, boxrule=.5pt, pad at break*=1mm, opacityfill=0]
\prompt{Out}{outcolor}{9}{\boxspacing}
\begin{Verbatim}[commandchars=\\\{\}]
'Hello, World!'
\end{Verbatim}
\end{tcolorbox}
        
    \begin{tcolorbox}[breakable, size=fbox, boxrule=1pt, pad at break*=1mm,colback=cellbackground, colframe=cellborder]
\prompt{In}{incolor}{10}{\boxspacing}
\begin{Verbatim}[commandchars=\\\{\}]
\PY{l+s+s2}{\PYZdq{}}\PY{l+s+s2}{This can}\PY{l+s+s2}{\PYZsq{}}\PY{l+s+s2}{t be the case!}\PY{l+s+s2}{\PYZdq{}}
\end{Verbatim}
\end{tcolorbox}

            \begin{tcolorbox}[breakable, size=fbox, boxrule=.5pt, pad at break*=1mm, opacityfill=0]
\prompt{Out}{outcolor}{10}{\boxspacing}
\begin{Verbatim}[commandchars=\\\{\}]
"This can't be the case!"
\end{Verbatim}
\end{tcolorbox}
        
    The function print accepts any number of arguments. For example, to
print the string ``36 * 37 / 2 ='' followed by the value of the
expression \(36 \cdot 37 / 2\) we can use the following print statement:

    \begin{tcolorbox}[breakable, size=fbox, boxrule=1pt, pad at break*=1mm,colback=cellbackground, colframe=cellborder]
\prompt{In}{incolor}{11}{\boxspacing}
\begin{Verbatim}[commandchars=\\\{\}]
\PY{n+nb}{print}\PY{p}{(}\PY{l+s+s1}{\PYZsq{}}\PY{l+s+s1}{36 * 37 / 2 =}\PY{l+s+s1}{\PYZsq{}}\PY{p}{,} \PY{l+m+mi}{36} \PY{o}{*} \PY{l+m+mi}{37} \PY{o}{/}\PY{o}{/} \PY{l+m+mi}{2}\PY{p}{)}
\end{Verbatim}
\end{tcolorbox}

    \begin{Verbatim}[commandchars=\\\{\}]
36 * 37 / 2 = 666
    \end{Verbatim}

    In the expression ``36 * 37 // 2'' we have used the operator ``//'' in
order to enforce \emph{integer division}, which is also known as
\emph{Euclidean division}. If we had used the operator ``/'' instead,
\emph{Python} would have used \emph{floating point division} and
therefore it would have printed the floating point number 666.0 instead
of the integer 666.

    \begin{tcolorbox}[breakable, size=fbox, boxrule=1pt, pad at break*=1mm,colback=cellbackground, colframe=cellborder]
\prompt{In}{incolor}{12}{\boxspacing}
\begin{Verbatim}[commandchars=\\\{\}]
\PY{n+nb}{print}\PY{p}{(}\PY{l+s+s1}{\PYZsq{}}\PY{l+s+s1}{36 * 37 / 2 =}\PY{l+s+s1}{\PYZsq{}}\PY{p}{,} \PY{l+m+mi}{36} \PY{o}{*} \PY{l+m+mi}{37} \PY{o}{/} \PY{l+m+mi}{2}\PY{p}{)}
\end{Verbatim}
\end{tcolorbox}

    \begin{Verbatim}[commandchars=\\\{\}]
36 * 37 / 2 = 666.0
    \end{Verbatim}

    The next cell demonstrates
\href{https://en.wikipedia.org/wiki/Euclidean_division}{Euclidean
division}.

    \begin{tcolorbox}[breakable, size=fbox, boxrule=1pt, pad at break*=1mm,colback=cellbackground, colframe=cellborder]
\prompt{In}{incolor}{13}{\boxspacing}
\begin{Verbatim}[commandchars=\\\{\}]
\PY{l+m+mi}{7} \PY{o}{/}\PY{o}{/} \PY{l+m+mi}{3}
\end{Verbatim}
\end{tcolorbox}

            \begin{tcolorbox}[breakable, size=fbox, boxrule=.5pt, pad at break*=1mm, opacityfill=0]
\prompt{Out}{outcolor}{13}{\boxspacing}
\begin{Verbatim}[commandchars=\\\{\}]
2
\end{Verbatim}
\end{tcolorbox}
        
    The \emph{remainder} that is left in an integer division is computed
with the operator \texttt{\%}.

    \begin{tcolorbox}[breakable, size=fbox, boxrule=1pt, pad at break*=1mm,colback=cellbackground, colframe=cellborder]
\prompt{In}{incolor}{14}{\boxspacing}
\begin{Verbatim}[commandchars=\\\{\}]
\PY{l+m+mi}{7} \PY{o}{\PYZpc{}} \PY{l+m+mi}{3}
\end{Verbatim}
\end{tcolorbox}

            \begin{tcolorbox}[breakable, size=fbox, boxrule=.5pt, pad at break*=1mm, opacityfill=0]
\prompt{Out}{outcolor}{14}{\boxspacing}
\begin{Verbatim}[commandchars=\\\{\}]
1
\end{Verbatim}
\end{tcolorbox}
        
    The result of a computation can be stored in a \emph{variable}. For
example, to store the quotient resulting from dividing \(7\) by \(3\)
into the variable with name \texttt{q} and to store the remainder into a
variable with name \texttt{r} we can write:

    \begin{tcolorbox}[breakable, size=fbox, boxrule=1pt, pad at break*=1mm,colback=cellbackground, colframe=cellborder]
\prompt{In}{incolor}{15}{\boxspacing}
\begin{Verbatim}[commandchars=\\\{\}]
\PY{n}{q} \PY{o}{=} \PY{l+m+mi}{7} \PY{o}{/}\PY{o}{/} \PY{l+m+mi}{3}
\PY{n}{r} \PY{o}{=} \PY{l+m+mi}{7} \PY{o}{\PYZpc{}} \PY{l+m+mi}{3}
\end{Verbatim}
\end{tcolorbox}

    Now \texttt{q} and \texttt{r} contain the quotient and the remainder of
the integer division of \texttt{7} by \texttt{3}as can be seen below:

    \begin{tcolorbox}[breakable, size=fbox, boxrule=1pt, pad at break*=1mm,colback=cellbackground, colframe=cellborder]
\prompt{In}{incolor}{16}{\boxspacing}
\begin{Verbatim}[commandchars=\\\{\}]
\PY{n}{q}
\end{Verbatim}
\end{tcolorbox}

            \begin{tcolorbox}[breakable, size=fbox, boxrule=.5pt, pad at break*=1mm, opacityfill=0]
\prompt{Out}{outcolor}{16}{\boxspacing}
\begin{Verbatim}[commandchars=\\\{\}]
2
\end{Verbatim}
\end{tcolorbox}
        
    \begin{tcolorbox}[breakable, size=fbox, boxrule=1pt, pad at break*=1mm,colback=cellbackground, colframe=cellborder]
\prompt{In}{incolor}{17}{\boxspacing}
\begin{Verbatim}[commandchars=\\\{\}]
\PY{n}{r}
\end{Verbatim}
\end{tcolorbox}

            \begin{tcolorbox}[breakable, size=fbox, boxrule=.5pt, pad at break*=1mm, opacityfill=0]
\prompt{Out}{outcolor}{17}{\boxspacing}
\begin{Verbatim}[commandchars=\\\{\}]
1
\end{Verbatim}
\end{tcolorbox}
        
    \hypertarget{lists}{%
\subsection{Lists}\label{lists}}

In order to present more interesting examples we next show how
\emph{Python} supports \emph{lists}. A list is created using the square
brackets \texttt{{[}} and \texttt{{]}}. For example, the empty list,
that is the list containing no elements, is written as \texttt{{[}{]}}.

    \begin{tcolorbox}[breakable, size=fbox, boxrule=1pt, pad at break*=1mm,colback=cellbackground, colframe=cellborder]
\prompt{In}{incolor}{18}{\boxspacing}
\begin{Verbatim}[commandchars=\\\{\}]
\PY{p}{[}\PY{p}{]}
\end{Verbatim}
\end{tcolorbox}

            \begin{tcolorbox}[breakable, size=fbox, boxrule=.5pt, pad at break*=1mm, opacityfill=0]
\prompt{Out}{outcolor}{18}{\boxspacing}
\begin{Verbatim}[commandchars=\\\{\}]
[]
\end{Verbatim}
\end{tcolorbox}
        
    The list containing the elements \texttt{1}, \texttt{2}, and \texttt{3}
is written by seperating the elements with a \texttt{,} as shown below:

    \begin{tcolorbox}[breakable, size=fbox, boxrule=1pt, pad at break*=1mm,colback=cellbackground, colframe=cellborder]
\prompt{In}{incolor}{19}{\boxspacing}
\begin{Verbatim}[commandchars=\\\{\}]
\PY{p}{[}\PY{l+m+mi}{1}\PY{p}{,} \PY{l+m+mi}{2}\PY{p}{,} \PY{l+m+mi}{3}\PY{p}{]}
\end{Verbatim}
\end{tcolorbox}

            \begin{tcolorbox}[breakable, size=fbox, boxrule=.5pt, pad at break*=1mm, opacityfill=0]
\prompt{Out}{outcolor}{19}{\boxspacing}
\begin{Verbatim}[commandchars=\\\{\}]
[1, 2, 3]
\end{Verbatim}
\end{tcolorbox}
        
    The spaces that I have used in the previous example are optional but
recommended.

    There are a number of predefined functions that take lists as arguments.
For example, the function \texttt{sum} takes a list of numbers and adds
these numbers.

    \begin{tcolorbox}[breakable, size=fbox, boxrule=1pt, pad at break*=1mm,colback=cellbackground, colframe=cellborder]
\prompt{In}{incolor}{20}{\boxspacing}
\begin{Verbatim}[commandchars=\\\{\}]
\PY{n+nb}{sum}\PY{p}{(}\PY{p}{[}\PY{l+m+mi}{1}\PY{p}{,} \PY{l+m+mi}{2}\PY{p}{,} \PY{l+m+mi}{3}\PY{p}{]}\PY{p}{)}
\end{Verbatim}
\end{tcolorbox}

            \begin{tcolorbox}[breakable, size=fbox, boxrule=.5pt, pad at break*=1mm, opacityfill=0]
\prompt{Out}{outcolor}{20}{\boxspacing}
\begin{Verbatim}[commandchars=\\\{\}]
6
\end{Verbatim}
\end{tcolorbox}
        
    To create a list containing all numbers starting from a given number
\(a\) up to but excluding the number \(b\) we can use \emph{list
comprehension} together with the function \texttt{range}. For example,
the list of the first 10 positive natural numbers can be written as:

    \begin{tcolorbox}[breakable, size=fbox, boxrule=1pt, pad at break*=1mm,colback=cellbackground, colframe=cellborder]
\prompt{In}{incolor}{21}{\boxspacing}
\begin{Verbatim}[commandchars=\\\{\}]
\PY{n}{L} \PY{o}{=} \PY{p}{[}\PY{n}{n} \PY{k}{for} \PY{n}{n} \PY{o+ow}{in} \PY{n+nb}{range}\PY{p}{(}\PY{l+m+mi}{1}\PY{p}{,} \PY{l+m+mi}{10}\PY{o}{+}\PY{l+m+mi}{1}\PY{p}{)}\PY{p}{]}
\PY{n}{L}
\end{Verbatim}
\end{tcolorbox}

            \begin{tcolorbox}[breakable, size=fbox, boxrule=.5pt, pad at break*=1mm, opacityfill=0]
\prompt{Out}{outcolor}{21}{\boxspacing}
\begin{Verbatim}[commandchars=\\\{\}]
[1, 2, 3, 4, 5, 6, 7, 8, 9, 10]
\end{Verbatim}
\end{tcolorbox}
        
    The function call \texttt{range(a,\ b)} takes two integers \texttt{a}
and \texttt{b} and returns the sequence of numbers from \texttt{a} upto
\texttt{b-1}.

    The general syntax for a list comprehension expression is
\[ [x \quad \texttt{for}\; x \;\texttt{in}\; e]\] where \texttt{for} and
\texttt{in} are keywords, \(x\) is a variable used for iterating while
\(e\) is an expression that evaluates to a so called \emph{iterable}.
For example, a list is an \emph{iterable} because we can \emph{iterate}
over its elements.

    We can also combine \emph{list selection} with \emph{list
comprehension}. This is done by using the keyword \texttt{if} followed
by an expression that evaluates as either \texttt{True} of
\texttt{False} for all elements that are generated by the comprehension
expression. For example, the following expression computes the list of
all positive even numbers less or equal than \(10\).

    \begin{tcolorbox}[breakable, size=fbox, boxrule=1pt, pad at break*=1mm,colback=cellbackground, colframe=cellborder]
\prompt{In}{incolor}{22}{\boxspacing}
\begin{Verbatim}[commandchars=\\\{\}]
\PY{n}{E} \PY{o}{=} \PY{p}{[}\PY{n}{n} \PY{k}{for} \PY{n}{n} \PY{o+ow}{in} \PY{n}{L} \PY{k}{if} \PY{n}{n} \PY{o}{\PYZpc{}} \PY{l+m+mi}{2} \PY{o}{==} \PY{l+m+mi}{0}\PY{p}{]}
\PY{n}{E}
\end{Verbatim}
\end{tcolorbox}

            \begin{tcolorbox}[breakable, size=fbox, boxrule=.5pt, pad at break*=1mm, opacityfill=0]
\prompt{Out}{outcolor}{22}{\boxspacing}
\begin{Verbatim}[commandchars=\\\{\}]
[2, 4, 6, 8, 10]
\end{Verbatim}
\end{tcolorbox}
        
    Here, the expression \texttt{n\ \%\ 2\ ==\ 0} tests whether the
remainder that is left when \texttt{n} is divided by \texttt{2} is equal
to \texttt{0}. Note that we have to use the operator \texttt{==} to
compare to numbers for equality, since the string \texttt{=} is used as
the \emph{assignment operator} in \emph{Python}.

    The \emph{index operator} \([\cdot]\) is a postfix operator that is used
to return the element that is stored at a given \emph{index} in a list.
For example, If \texttt{L} is a list, then \texttt{L{[}0{]}} returns the
first element of \texttt{L}. (Yes, indexing is zero-based in
\emph{Python}, so the first element has index \texttt{0}.)

    \begin{tcolorbox}[breakable, size=fbox, boxrule=1pt, pad at break*=1mm,colback=cellbackground, colframe=cellborder]
\prompt{In}{incolor}{23}{\boxspacing}
\begin{Verbatim}[commandchars=\\\{\}]
\PY{n}{L}
\end{Verbatim}
\end{tcolorbox}

            \begin{tcolorbox}[breakable, size=fbox, boxrule=.5pt, pad at break*=1mm, opacityfill=0]
\prompt{Out}{outcolor}{23}{\boxspacing}
\begin{Verbatim}[commandchars=\\\{\}]
[1, 2, 3, 4, 5, 6, 7, 8, 9, 10]
\end{Verbatim}
\end{tcolorbox}
        
    \begin{tcolorbox}[breakable, size=fbox, boxrule=1pt, pad at break*=1mm,colback=cellbackground, colframe=cellborder]
\prompt{In}{incolor}{24}{\boxspacing}
\begin{Verbatim}[commandchars=\\\{\}]
\PY{n}{L}\PY{p}{[}\PY{l+m+mi}{0}\PY{p}{]}
\end{Verbatim}
\end{tcolorbox}

            \begin{tcolorbox}[breakable, size=fbox, boxrule=.5pt, pad at break*=1mm, opacityfill=0]
\prompt{Out}{outcolor}{24}{\boxspacing}
\begin{Verbatim}[commandchars=\\\{\}]
1
\end{Verbatim}
\end{tcolorbox}
        
    To change the first element of a list, we can use the \emph{index
operator} on the left hand side of an \emph{assignment}:

    \begin{tcolorbox}[breakable, size=fbox, boxrule=1pt, pad at break*=1mm,colback=cellbackground, colframe=cellborder]
\prompt{In}{incolor}{25}{\boxspacing}
\begin{Verbatim}[commandchars=\\\{\}]
\PY{n}{L}\PY{p}{[}\PY{l+m+mi}{0}\PY{p}{]} \PY{o}{=} \PY{l+m+mi}{7}
\PY{n}{L}
\end{Verbatim}
\end{tcolorbox}

            \begin{tcolorbox}[breakable, size=fbox, boxrule=.5pt, pad at break*=1mm, opacityfill=0]
\prompt{Out}{outcolor}{25}{\boxspacing}
\begin{Verbatim}[commandchars=\\\{\}]
[7, 2, 3, 4, 5, 6, 7, 8, 9, 10]
\end{Verbatim}
\end{tcolorbox}
        
    Lists can be \emph{concatenated} with the operator \texttt{+}:

    \begin{tcolorbox}[breakable, size=fbox, boxrule=1pt, pad at break*=1mm,colback=cellbackground, colframe=cellborder]
\prompt{In}{incolor}{26}{\boxspacing}
\begin{Verbatim}[commandchars=\\\{\}]
\PY{p}{[}\PY{l+m+mi}{1}\PY{p}{,}\PY{l+m+mi}{2}\PY{p}{,}\PY{l+m+mi}{3}\PY{p}{]} \PY{o}{+} \PY{p}{[}\PY{l+m+mi}{4}\PY{p}{,}\PY{l+m+mi}{5}\PY{p}{,}\PY{l+m+mi}{6}\PY{p}{]}
\end{Verbatim}
\end{tcolorbox}

            \begin{tcolorbox}[breakable, size=fbox, boxrule=.5pt, pad at break*=1mm, opacityfill=0]
\prompt{Out}{outcolor}{26}{\boxspacing}
\begin{Verbatim}[commandchars=\\\{\}]
[1, 2, 3, 4, 5, 6]
\end{Verbatim}
\end{tcolorbox}
        
    The function \texttt{len} computes the \emph{length} of a list. The
length of a list \(L\) is defined as the number of elements of \(L\):

    \begin{tcolorbox}[breakable, size=fbox, boxrule=1pt, pad at break*=1mm,colback=cellbackground, colframe=cellborder]
\prompt{In}{incolor}{27}{\boxspacing}
\begin{Verbatim}[commandchars=\\\{\}]
\PY{n+nb}{len}\PY{p}{(}\PY{p}{[}\PY{l+m+mi}{4}\PY{p}{,} \PY{l+m+mi}{5}\PY{p}{,} \PY{l+m+mi}{4}\PY{p}{]}\PY{p}{)}
\end{Verbatim}
\end{tcolorbox}

            \begin{tcolorbox}[breakable, size=fbox, boxrule=.5pt, pad at break*=1mm, opacityfill=0]
\prompt{Out}{outcolor}{27}{\boxspacing}
\begin{Verbatim}[commandchars=\\\{\}]
3
\end{Verbatim}
\end{tcolorbox}
        
    Lists support the functions max and min. The expression
\(\texttt{max}(L)\) computes the maximum of all the elements of the list
(or tuple) \(L\), while \(\texttt{min}(L)\) computes the smallest
element of \(L\). These functions also operate on tuples or sets.

    \begin{tcolorbox}[breakable, size=fbox, boxrule=1pt, pad at break*=1mm,colback=cellbackground, colframe=cellborder]
\prompt{In}{incolor}{28}{\boxspacing}
\begin{Verbatim}[commandchars=\\\{\}]
\PY{n+nb}{max}\PY{p}{(}\PY{p}{[}\PY{l+m+mi}{1}\PY{p}{,} \PY{l+m+mi}{2}\PY{p}{,} \PY{l+m+mi}{3}\PY{p}{]}\PY{p}{)}
\end{Verbatim}
\end{tcolorbox}

            \begin{tcolorbox}[breakable, size=fbox, boxrule=.5pt, pad at break*=1mm, opacityfill=0]
\prompt{Out}{outcolor}{28}{\boxspacing}
\begin{Verbatim}[commandchars=\\\{\}]
3
\end{Verbatim}
\end{tcolorbox}
        
    \begin{tcolorbox}[breakable, size=fbox, boxrule=1pt, pad at break*=1mm,colback=cellbackground, colframe=cellborder]
\prompt{In}{incolor}{29}{\boxspacing}
\begin{Verbatim}[commandchars=\\\{\}]
\PY{n+nb}{min}\PY{p}{(}\PY{p}{[}\PY{l+m+mi}{1}\PY{p}{,} \PY{l+m+mi}{2}\PY{p}{,} \PY{l+m+mi}{3}\PY{p}{]}\PY{p}{)}
\end{Verbatim}
\end{tcolorbox}

            \begin{tcolorbox}[breakable, size=fbox, boxrule=.5pt, pad at break*=1mm, opacityfill=0]
\prompt{Out}{outcolor}{29}{\boxspacing}
\begin{Verbatim}[commandchars=\\\{\}]
1
\end{Verbatim}
\end{tcolorbox}
        
    \hypertarget{scripts-and-functions}{%
\subsection{Scripts and Functions}\label{scripts-and-functions}}

Next, we show how it is possible to combine multiple expressions in a
\emph{script} and how to create \emph{user defined functions}.

    To begin with we present a script that asks the user to input a natural
number \(n\) and then proceeds to compute the sum of all natural numbers
from \(1\) upto and including \(n\), that is it computes the sum
\[ \sum\limits_{i=1}^n i = 1 + 2 + 3 + \cdots + (n-2) + (n-1) + n. \]
The script contained in the cell below works as follows: * The function
input prompts the user to enter a string. * This string is then
converted into an integer using the function int. * Next, the list
\texttt{L} is created such that
\[\texttt{L} = [1, 2, 3,\cdots, (n-2), (n-1), n]. \]\\
* The \texttt{print} statement uses the function \texttt{sum} to compute
the sum of all the elements of the List \texttt{L}. The effect of the
last argument \texttt{sep=\textquotesingle{}\textquotesingle{}} is to
prevent \texttt{print} from separating its arguments by spaces.

    \begin{tcolorbox}[breakable, size=fbox, boxrule=1pt, pad at break*=1mm,colback=cellbackground, colframe=cellborder]
\prompt{In}{incolor}{30}{\boxspacing}
\begin{Verbatim}[commandchars=\\\{\}]
\PY{n}{n} \PY{o}{=} \PY{n+nb}{input}\PY{p}{(}\PY{l+s+s1}{\PYZsq{}}\PY{l+s+s1}{Type a natural number and press return: }\PY{l+s+s1}{\PYZsq{}}\PY{p}{)}
\PY{n}{n} \PY{o}{=} \PY{n+nb}{int}\PY{p}{(}\PY{n}{n}\PY{p}{)}
\PY{n}{L} \PY{o}{=} \PY{p}{[}\PY{n}{i} \PY{k}{for} \PY{n}{i} \PY{o+ow}{in} \PY{n+nb}{range}\PY{p}{(}\PY{l+m+mi}{1}\PY{p}{,} \PY{n}{n}\PY{o}{+}\PY{l+m+mi}{1}\PY{p}{)}\PY{p}{]}
\PY{n+nb}{print}\PY{p}{(}\PY{l+s+s1}{\PYZsq{}}\PY{l+s+s1}{The sum 1 + 2 + ... + }\PY{l+s+s1}{\PYZsq{}}\PY{p}{,} \PY{n}{n}\PY{p}{,} \PY{l+s+s1}{\PYZsq{}}\PY{l+s+s1}{ is equal to }\PY{l+s+s1}{\PYZsq{}}\PY{p}{,} \PY{n+nb}{sum}\PY{p}{(}\PY{n}{L}\PY{p}{)}\PY{p}{,} \PY{l+s+s1}{\PYZsq{}}\PY{l+s+s1}{.}\PY{l+s+s1}{\PYZsq{}}\PY{p}{,} \PY{n}{sep}\PY{o}{=}\PY{l+s+s1}{\PYZsq{}}\PY{l+s+s1}{\PYZsq{}}\PY{p}{)}
\end{Verbatim}
\end{tcolorbox}

    \begin{Verbatim}[commandchars=\\\{\}]
Type a natural number and press return: 36
The sum 1 + 2 + {\ldots} + 36 is equal to 666.
    \end{Verbatim}

    If we input the number \(36\), which happens to be equal to \(6^2\),
then we discover the following remarkable identity:
\[ \sum\limits_{i=1}^{6^2} i = 666. \]

    The next example shows how \emph{functions} can be defined in
\emph{Python}. The function \(\texttt{sum}(n)\) is supposed to compute
the sum of all the numbers in the set \(\{1, \cdots, n\}\). Therefore,
we have \[\texttt{sum}(n) = \sum\limits_{i=1}^n i. \]\\
The function \texttt{sum} is defined \emph{recursively}. The recursive
implementation of the function \texttt{sum} can best by understood if we
observe that it satisfies the following two equations: *
\(\texttt{sum}(0) = 0\), *
\(\texttt{sum}(n) = \texttt{sum}(n-1) + n \quad\) provided that
\(n > 0\).

The keyword \texttt{def} is used to signal that we define a function.

    \begin{tcolorbox}[breakable, size=fbox, boxrule=1pt, pad at break*=1mm,colback=cellbackground, colframe=cellborder]
\prompt{In}{incolor}{31}{\boxspacing}
\begin{Verbatim}[commandchars=\\\{\}]
\PY{k}{def} \PY{n+nf}{sum}\PY{p}{(}\PY{n}{n}\PY{p}{)}\PY{p}{:}
    \PY{k}{if} \PY{n}{n} \PY{o}{==} \PY{l+m+mi}{0}\PY{p}{:}
        \PY{k}{return} \PY{l+m+mi}{0}
    \PY{k}{return} \PY{n+nb}{sum}\PY{p}{(}\PY{n}{n}\PY{o}{\PYZhy{}}\PY{l+m+mi}{1}\PY{p}{)} \PY{o}{+} \PY{n}{n}
\end{Verbatim}
\end{tcolorbox}

    Lets test this function.

    \begin{tcolorbox}[breakable, size=fbox, boxrule=1pt, pad at break*=1mm,colback=cellbackground, colframe=cellborder]
\prompt{In}{incolor}{32}{\boxspacing}
\begin{Verbatim}[commandchars=\\\{\}]
\PY{n+nb}{sum}\PY{p}{(}\PY{l+m+mi}{3}\PY{p}{)}
\end{Verbatim}
\end{tcolorbox}

            \begin{tcolorbox}[breakable, size=fbox, boxrule=.5pt, pad at break*=1mm, opacityfill=0]
\prompt{Out}{outcolor}{32}{\boxspacing}
\begin{Verbatim}[commandchars=\\\{\}]
6
\end{Verbatim}
\end{tcolorbox}
        
    Let us discuss the implementation of the function \texttt{sum} line by
line: 1. The keyword \texttt{def} starts the definition of the function.
It is followed by the name of the function that is defined. The name is
followed by the list of the parameters of the function. This list is
enclosed in parentheses. If there had been more than one parameter, the
parameters would have been separated by commas. Finally, there needs to
be a colon at the end of the first line. 2. The body of the function is
\emph{indented}. \textbf{Contrary} to most other programming languages,
Python is space sensitive. The first statement of the body is a
\emph{conditional statement}, which starts with the keyword \texttt{if}.
The keyword is followed by a test. In this case we test whether the
variable \(n\) is equal to the number \(0\). Note that this test is
followed by a colon \texttt{:}. 3. The next line contains a
\texttt{return} statement. Note that this statement is again indented.
All statements indented by the same amount that follow an
\texttt{if}-statement are considered as the \emph{body} of the
\texttt{if}-statement. In this case the body contains only a single
statement. 4. The last line of the function definition contains the
recursive invocation of the function \texttt{sum} that computes the
value \texttt{sum(n)} by reducing it to the value \texttt{sum(n-1)}.

Using the function \texttt{sum}, we can compute the sum
\(\sum\limits_{i=1}^n i\) as follows:

    \begin{tcolorbox}[breakable, size=fbox, boxrule=1pt, pad at break*=1mm,colback=cellbackground, colframe=cellborder]
\prompt{In}{incolor}{33}{\boxspacing}
\begin{Verbatim}[commandchars=\\\{\}]
\PY{n}{n}     \PY{o}{=} \PY{n+nb}{int}\PY{p}{(}\PY{n+nb}{input}\PY{p}{(}\PY{l+s+s2}{\PYZdq{}}\PY{l+s+s2}{Enter a natural number: }\PY{l+s+s2}{\PYZdq{}}\PY{p}{)}\PY{p}{)}
\PY{n}{total} \PY{o}{=} \PY{n+nb}{sum}\PY{p}{(}\PY{n}{n}\PY{p}{)}
\PY{k}{if} \PY{n}{n} \PY{o}{\PYZgt{}} \PY{l+m+mi}{2}\PY{p}{:}
    \PY{n+nb}{print}\PY{p}{(}\PY{l+s+s2}{\PYZdq{}}\PY{l+s+s2}{0 + 1 + 2 + ... + }\PY{l+s+s2}{\PYZdq{}}\PY{p}{,} \PY{n}{n}\PY{p}{,} \PY{l+s+s2}{\PYZdq{}}\PY{l+s+s2}{ = }\PY{l+s+s2}{\PYZdq{}}\PY{p}{,} \PY{n}{total}\PY{p}{,} \PY{n}{sep}\PY{o}{=}\PY{l+s+s1}{\PYZsq{}}\PY{l+s+s1}{\PYZsq{}}\PY{p}{)}
\PY{k}{else}\PY{p}{:} 
    \PY{n+nb}{print}\PY{p}{(}\PY{n}{total}\PY{p}{)}
\end{Verbatim}
\end{tcolorbox}

    \begin{Verbatim}[commandchars=\\\{\}]
Enter a natural number: 10
0 + 1 + 2 + {\ldots} + 10 = 55
    \end{Verbatim}

    \hypertarget{boolean-values-and-boolean-operators}{%
\subsection{Boolean Values and Boolean
Operators}\label{boolean-values-and-boolean-operators}}

    In \emph{Python}, the \emph{truth values}, also known as \emph{Boolean
values}, are written as True and False.

    \begin{tcolorbox}[breakable, size=fbox, boxrule=1pt, pad at break*=1mm,colback=cellbackground, colframe=cellborder]
\prompt{In}{incolor}{34}{\boxspacing}
\begin{Verbatim}[commandchars=\\\{\}]
\PY{k+kc}{True}
\end{Verbatim}
\end{tcolorbox}

            \begin{tcolorbox}[breakable, size=fbox, boxrule=.5pt, pad at break*=1mm, opacityfill=0]
\prompt{Out}{outcolor}{34}{\boxspacing}
\begin{Verbatim}[commandchars=\\\{\}]
True
\end{Verbatim}
\end{tcolorbox}
        
    \begin{tcolorbox}[breakable, size=fbox, boxrule=1pt, pad at break*=1mm,colback=cellbackground, colframe=cellborder]
\prompt{In}{incolor}{35}{\boxspacing}
\begin{Verbatim}[commandchars=\\\{\}]
\PY{k+kc}{False}
\end{Verbatim}
\end{tcolorbox}

            \begin{tcolorbox}[breakable, size=fbox, boxrule=.5pt, pad at break*=1mm, opacityfill=0]
\prompt{Out}{outcolor}{35}{\boxspacing}
\begin{Verbatim}[commandchars=\\\{\}]
False
\end{Verbatim}
\end{tcolorbox}
        
    The following function is needed for pretty-printing. It assumes that
the argument val is a truth value. This truth value is then turned into
a string that has a size of exactly 5 characters.

    \begin{tcolorbox}[breakable, size=fbox, boxrule=1pt, pad at break*=1mm,colback=cellbackground, colframe=cellborder]
\prompt{In}{incolor}{36}{\boxspacing}
\begin{Verbatim}[commandchars=\\\{\}]
\PY{k}{def} \PY{n+nf}{toStr}\PY{p}{(}\PY{n}{val}\PY{p}{)}\PY{p}{:}
    \PY{k}{if} \PY{n}{val}\PY{p}{:}
        \PY{k}{return} \PY{l+s+s1}{\PYZsq{}}\PY{l+s+s1}{True }\PY{l+s+s1}{\PYZsq{}}
    \PY{k}{return} \PY{l+s+s1}{\PYZsq{}}\PY{l+s+s1}{False}\PY{l+s+s1}{\PYZsq{}}
\end{Verbatim}
\end{tcolorbox}

    These values can be combined using the \emph{Boolean operators}
\(\wedge\), \(\vee\), and \(\neg\). In \emph{Python}, these operators
are denoted as \texttt{and}, \texttt{or}, and \texttt{not}. The
following table shows how the operator \texttt{and} is defined:

    \begin{tcolorbox}[breakable, size=fbox, boxrule=1pt, pad at break*=1mm,colback=cellbackground, colframe=cellborder]
\prompt{In}{incolor}{37}{\boxspacing}
\begin{Verbatim}[commandchars=\\\{\}]
\PY{n}{B} \PY{o}{=} \PY{p}{[}\PY{k+kc}{True}\PY{p}{,} \PY{k+kc}{False}\PY{p}{]}
\PY{k}{for} \PY{n}{x} \PY{o+ow}{in} \PY{n}{B}\PY{p}{:}
    \PY{k}{for} \PY{n}{y} \PY{o+ow}{in} \PY{n}{B}\PY{p}{:}
        \PY{n+nb}{print}\PY{p}{(}\PY{n}{toStr}\PY{p}{(}\PY{n}{x}\PY{p}{)}\PY{p}{,} \PY{l+s+s1}{\PYZsq{}}\PY{l+s+s1}{and}\PY{l+s+s1}{\PYZsq{}}\PY{p}{,} \PY{n}{toStr}\PY{p}{(}\PY{n}{y}\PY{p}{)}\PY{p}{,} \PY{l+s+s1}{\PYZsq{}}\PY{l+s+s1}{=}\PY{l+s+s1}{\PYZsq{}}\PY{p}{,} \PY{n}{x} \PY{o+ow}{and} \PY{n}{y}\PY{p}{)}
\end{Verbatim}
\end{tcolorbox}

    \begin{Verbatim}[commandchars=\\\{\}]
True  and True  = True
True  and False = False
False and True  = False
False and False = False
    \end{Verbatim}

    The \emph{disjunction} of two Boolean values is only \emph{False} if
both values are \emph{False}. In \emph{Python}, the disjunction operator
is written as \texttt{or}:

    \begin{tcolorbox}[breakable, size=fbox, boxrule=1pt, pad at break*=1mm,colback=cellbackground, colframe=cellborder]
\prompt{In}{incolor}{38}{\boxspacing}
\begin{Verbatim}[commandchars=\\\{\}]
\PY{k}{for} \PY{n}{x} \PY{o+ow}{in} \PY{n}{B}\PY{p}{:}
    \PY{k}{for} \PY{n}{y} \PY{o+ow}{in} \PY{n}{B}\PY{p}{:}
        \PY{n+nb}{print}\PY{p}{(}\PY{n}{toStr}\PY{p}{(}\PY{n}{x}\PY{p}{)}\PY{p}{,} \PY{l+s+s1}{\PYZsq{}}\PY{l+s+s1}{or}\PY{l+s+s1}{\PYZsq{}}\PY{p}{,} \PY{n}{toStr}\PY{p}{(}\PY{n}{y}\PY{p}{)}\PY{p}{,} \PY{l+s+s1}{\PYZsq{}}\PY{l+s+s1}{=}\PY{l+s+s1}{\PYZsq{}}\PY{p}{,} \PY{n}{x} \PY{o+ow}{or} \PY{n}{y}\PY{p}{)}
\end{Verbatim}
\end{tcolorbox}

    \begin{Verbatim}[commandchars=\\\{\}]
True  or True  = True
True  or False = True
False or True  = True
False or False = False
    \end{Verbatim}

    Finally, the negation operator \texttt{not} works as expected:

    \begin{tcolorbox}[breakable, size=fbox, boxrule=1pt, pad at break*=1mm,colback=cellbackground, colframe=cellborder]
\prompt{In}{incolor}{39}{\boxspacing}
\begin{Verbatim}[commandchars=\\\{\}]
\PY{k}{for} \PY{n}{x} \PY{o+ow}{in} \PY{n}{B}\PY{p}{:}
    \PY{n+nb}{print}\PY{p}{(}\PY{l+s+s1}{\PYZsq{}}\PY{l+s+s1}{not}\PY{l+s+s1}{\PYZsq{}}\PY{p}{,} \PY{n}{toStr}\PY{p}{(}\PY{n}{x}\PY{p}{)}\PY{p}{,} \PY{l+s+s1}{\PYZsq{}}\PY{l+s+s1}{=}\PY{l+s+s1}{\PYZsq{}}\PY{p}{,} \PY{o+ow}{not} \PY{n}{x}\PY{p}{)}
\end{Verbatim}
\end{tcolorbox}

    \begin{Verbatim}[commandchars=\\\{\}]
not True  = False
not False = True
    \end{Verbatim}

    Boolean values are created by comparing numbers using the comparison
operators \texttt{==}, \texttt{!=}, \texttt{\textless{}},
\texttt{\textgreater{}}, \texttt{\textless{}=}, and
\texttt{\textgreater{}=}: * \(a\;\texttt{==}\;b\) is true iff \(a\) is
equal to \(b\). * \(a\;\texttt{!=}\;b\) is true iff \(a\) is different
from \(b\). * \(a\;\texttt{<}\;b\) is true iff \(a\) is less than \(b\).
* \(a\;\texttt{<=}\;b\) is true iff \(a\) is less than or equal to
\(b\). * \(a\;\texttt{>=}\;b\) is true iff \(a\) is bigger than or equal
to \(b\). * \(a\;\texttt{>}\;b\) is true iff \(a\) is bigger than \(b\).

    \begin{tcolorbox}[breakable, size=fbox, boxrule=1pt, pad at break*=1mm,colback=cellbackground, colframe=cellborder]
\prompt{In}{incolor}{40}{\boxspacing}
\begin{Verbatim}[commandchars=\\\{\}]
\PY{l+m+mi}{1} \PY{o}{==} \PY{l+m+mi}{2}
\end{Verbatim}
\end{tcolorbox}

            \begin{tcolorbox}[breakable, size=fbox, boxrule=.5pt, pad at break*=1mm, opacityfill=0]
\prompt{Out}{outcolor}{40}{\boxspacing}
\begin{Verbatim}[commandchars=\\\{\}]
False
\end{Verbatim}
\end{tcolorbox}
        
    \begin{tcolorbox}[breakable, size=fbox, boxrule=1pt, pad at break*=1mm,colback=cellbackground, colframe=cellborder]
\prompt{In}{incolor}{41}{\boxspacing}
\begin{Verbatim}[commandchars=\\\{\}]
\PY{l+m+mi}{1} \PY{o}{!=} \PY{l+m+mi}{2}
\end{Verbatim}
\end{tcolorbox}

            \begin{tcolorbox}[breakable, size=fbox, boxrule=.5pt, pad at break*=1mm, opacityfill=0]
\prompt{Out}{outcolor}{41}{\boxspacing}
\begin{Verbatim}[commandchars=\\\{\}]
True
\end{Verbatim}
\end{tcolorbox}
        
    \begin{tcolorbox}[breakable, size=fbox, boxrule=1pt, pad at break*=1mm,colback=cellbackground, colframe=cellborder]
\prompt{In}{incolor}{42}{\boxspacing}
\begin{Verbatim}[commandchars=\\\{\}]
\PY{l+m+mi}{1} \PY{o}{\PYZlt{}} \PY{l+m+mi}{2}
\end{Verbatim}
\end{tcolorbox}

            \begin{tcolorbox}[breakable, size=fbox, boxrule=.5pt, pad at break*=1mm, opacityfill=0]
\prompt{Out}{outcolor}{42}{\boxspacing}
\begin{Verbatim}[commandchars=\\\{\}]
True
\end{Verbatim}
\end{tcolorbox}
        
    \begin{tcolorbox}[breakable, size=fbox, boxrule=1pt, pad at break*=1mm,colback=cellbackground, colframe=cellborder]
\prompt{In}{incolor}{43}{\boxspacing}
\begin{Verbatim}[commandchars=\\\{\}]
\PY{l+m+mi}{1} \PY{o}{\PYZlt{}}\PY{o}{=} \PY{l+m+mi}{2}
\end{Verbatim}
\end{tcolorbox}

            \begin{tcolorbox}[breakable, size=fbox, boxrule=.5pt, pad at break*=1mm, opacityfill=0]
\prompt{Out}{outcolor}{43}{\boxspacing}
\begin{Verbatim}[commandchars=\\\{\}]
True
\end{Verbatim}
\end{tcolorbox}
        
    \begin{tcolorbox}[breakable, size=fbox, boxrule=1pt, pad at break*=1mm,colback=cellbackground, colframe=cellborder]
\prompt{In}{incolor}{44}{\boxspacing}
\begin{Verbatim}[commandchars=\\\{\}]
\PY{l+m+mi}{1} \PY{o}{\PYZgt{}} \PY{l+m+mi}{2}
\end{Verbatim}
\end{tcolorbox}

            \begin{tcolorbox}[breakable, size=fbox, boxrule=.5pt, pad at break*=1mm, opacityfill=0]
\prompt{Out}{outcolor}{44}{\boxspacing}
\begin{Verbatim}[commandchars=\\\{\}]
False
\end{Verbatim}
\end{tcolorbox}
        
    \begin{tcolorbox}[breakable, size=fbox, boxrule=1pt, pad at break*=1mm,colback=cellbackground, colframe=cellborder]
\prompt{In}{incolor}{45}{\boxspacing}
\begin{Verbatim}[commandchars=\\\{\}]
\PY{l+m+mi}{1} \PY{o}{\PYZgt{}}\PY{o}{=} \PY{l+m+mi}{2}
\end{Verbatim}
\end{tcolorbox}

            \begin{tcolorbox}[breakable, size=fbox, boxrule=.5pt, pad at break*=1mm, opacityfill=0]
\prompt{Out}{outcolor}{45}{\boxspacing}
\begin{Verbatim}[commandchars=\\\{\}]
False
\end{Verbatim}
\end{tcolorbox}
        
    Comparison operators can be chained as shown in the following example:

    \begin{tcolorbox}[breakable, size=fbox, boxrule=1pt, pad at break*=1mm,colback=cellbackground, colframe=cellborder]
\prompt{In}{incolor}{46}{\boxspacing}
\begin{Verbatim}[commandchars=\\\{\}]
\PY{l+m+mi}{1} \PY{o}{\PYZlt{}} \PY{l+m+mi}{2} \PY{o}{\PYZlt{}} \PY{l+m+mi}{3}
\end{Verbatim}
\end{tcolorbox}

            \begin{tcolorbox}[breakable, size=fbox, boxrule=.5pt, pad at break*=1mm, opacityfill=0]
\prompt{Out}{outcolor}{46}{\boxspacing}
\begin{Verbatim}[commandchars=\\\{\}]
True
\end{Verbatim}
\end{tcolorbox}
        
    \emph{Python} supports the universal quantifier \(\forall\). If \(L\) is
a list of Boolean values, then we can check whether all elements of
\(L\) are True by writing \[ \texttt{all}(L). \] For example, to check
whether all elements of a list \(L\) are even we can write the
following:

    \begin{tcolorbox}[breakable, size=fbox, boxrule=1pt, pad at break*=1mm,colback=cellbackground, colframe=cellborder]
\prompt{In}{incolor}{47}{\boxspacing}
\begin{Verbatim}[commandchars=\\\{\}]
\PY{n}{L} \PY{o}{=} \PY{p}{[}\PY{l+m+mi}{2}\PY{p}{,} \PY{l+m+mi}{4}\PY{p}{,} \PY{l+m+mi}{6}\PY{p}{,} \PY{l+m+mi}{7}\PY{p}{]}
\PY{n+nb}{all}\PY{p}{(}\PY{p}{[}\PY{n}{x} \PY{o}{\PYZpc{}} \PY{l+m+mi}{2} \PY{o}{==} \PY{l+m+mi}{0} \PY{k}{for} \PY{n}{x} \PY{o+ow}{in} \PY{n}{L}\PY{p}{]}\PY{p}{)}
\end{Verbatim}
\end{tcolorbox}

            \begin{tcolorbox}[breakable, size=fbox, boxrule=.5pt, pad at break*=1mm, opacityfill=0]
\prompt{Out}{outcolor}{47}{\boxspacing}
\begin{Verbatim}[commandchars=\\\{\}]
False
\end{Verbatim}
\end{tcolorbox}
        
    \emph{Python} also supports the existential quantifier \(\exists\). If
\(L\) is a list of Boolean values, the expression \[ \texttt{any}(L) \]
is true iff there exists an element \(x \in L\) such that \(x\) is true.

    \begin{tcolorbox}[breakable, size=fbox, boxrule=1pt, pad at break*=1mm,colback=cellbackground, colframe=cellborder]
\prompt{In}{incolor}{48}{\boxspacing}
\begin{Verbatim}[commandchars=\\\{\}]
\PY{n}{L} \PY{o}{=} \PY{p}{[}\PY{n}{n} \PY{k}{for} \PY{n}{n} \PY{o+ow}{in} \PY{n+nb}{range}\PY{p}{(}\PY{l+m+mi}{1}\PY{p}{,} \PY{l+m+mi}{10}\PY{p}{)}\PY{p}{]}
\PY{n+nb}{any}\PY{p}{(}\PY{p}{[}\PY{n}{n} \PY{o}{*}\PY{o}{*} \PY{l+m+mi}{2} \PY{o}{\PYZgt{}} \PY{l+m+mi}{2} \PY{o}{*}\PY{o}{*} \PY{n}{n} \PY{k}{for} \PY{n}{n} \PY{o+ow}{in} \PY{n}{L}\PY{p}{]}\PY{p}{)}
\end{Verbatim}
\end{tcolorbox}

            \begin{tcolorbox}[breakable, size=fbox, boxrule=.5pt, pad at break*=1mm, opacityfill=0]
\prompt{Out}{outcolor}{48}{\boxspacing}
\begin{Verbatim}[commandchars=\\\{\}]
True
\end{Verbatim}
\end{tcolorbox}
        
    \hypertarget{control-structures}{%
\subsection{Control Structures}\label{control-structures}}

    First of all, \emph{Python} supports \emph{branching} via the keywords
\texttt{if}, \texttt{elif}, and \texttt{else}. The following example is
taken from the \emph{Python} tutorial at
\href{https://docs.python.org/3/tutorial/controlflow.html}{https://python.org}:

    \begin{tcolorbox}[breakable, size=fbox, boxrule=1pt, pad at break*=1mm,colback=cellbackground, colframe=cellborder]
\prompt{In}{incolor}{49}{\boxspacing}
\begin{Verbatim}[commandchars=\\\{\}]
\PY{n}{s} \PY{o}{=} \PY{n+nb}{input}\PY{p}{(}\PY{l+s+s2}{\PYZdq{}}\PY{l+s+s2}{Please enter an integer: }\PY{l+s+s2}{\PYZdq{}}\PY{p}{)}
\PY{n}{x} \PY{o}{=} \PY{n+nb}{int}\PY{p}{(}\PY{n}{s}\PY{p}{)}
\PY{k}{if} \PY{n}{x} \PY{o}{\PYZlt{}} \PY{l+m+mi}{0}\PY{p}{:}
    \PY{n+nb}{print}\PY{p}{(}\PY{l+s+s1}{\PYZsq{}}\PY{l+s+s1}{The number is negative!}\PY{l+s+s1}{\PYZsq{}}\PY{p}{)}
\PY{k}{elif} \PY{n}{x} \PY{o}{==} \PY{l+m+mi}{0}\PY{p}{:}
    \PY{n+nb}{print}\PY{p}{(}\PY{l+s+s1}{\PYZsq{}}\PY{l+s+s1}{The number is zero.}\PY{l+s+s1}{\PYZsq{}}\PY{p}{)}
\PY{k}{elif} \PY{n}{x} \PY{o}{==} \PY{l+m+mi}{1}\PY{p}{:}
    \PY{n+nb}{print}\PY{p}{(}\PY{l+s+s2}{\PYZdq{}}\PY{l+s+s2}{It}\PY{l+s+s2}{\PYZsq{}}\PY{l+s+s2}{s a one.}\PY{l+s+s2}{\PYZdq{}}\PY{p}{)}
\PY{k}{else}\PY{p}{:}
    \PY{n+nb}{print}\PY{p}{(}\PY{l+s+s2}{\PYZdq{}}\PY{l+s+s2}{It}\PY{l+s+s2}{\PYZsq{}}\PY{l+s+s2}{s more than one.}\PY{l+s+s2}{\PYZdq{}}\PY{p}{)}
\end{Verbatim}
\end{tcolorbox}

    \begin{Verbatim}[commandchars=\\\{\}]
Please enter an integer: 1
It's a one.
    \end{Verbatim}

    \emph{Loops} can be created with the keyword \texttt{for}. They can be
used to iterate over lists.\\
The syntax of a \texttt{for}-loop is:

\begin{verbatim}
   for x in L:
       S1
       ...
       Sn
\end{verbatim}

Here, \texttt{x} is the name of a variable that is used to iterate over
the elements of the list \texttt{L}. The statements \texttt{S1},
\(\cdots\), \texttt{Sn} make up the \emph{body} of the loop. These
statements are executed for every element of the list \texttt{L}.

The next examples shows a simple loop that prints all the elements from
the list \texttt{L}:

    \begin{tcolorbox}[breakable, size=fbox, boxrule=1pt, pad at break*=1mm,colback=cellbackground, colframe=cellborder]
\prompt{In}{incolor}{50}{\boxspacing}
\begin{Verbatim}[commandchars=\\\{\}]
\PY{k}{for} \PY{n}{x} \PY{o+ow}{in} \PY{n}{L}\PY{p}{:}
    \PY{n+nb}{print}\PY{p}{(}\PY{n}{x}\PY{p}{)}
\end{Verbatim}
\end{tcolorbox}

    \begin{Verbatim}[commandchars=\\\{\}]
1
2
3
4
5
6
7
8
9
    \end{Verbatim}

    Another way to write a \emph{loop} is to use the keyword \texttt{while}
as in the following example:

    \begin{tcolorbox}[breakable, size=fbox, boxrule=1pt, pad at break*=1mm,colback=cellbackground, colframe=cellborder]
\prompt{In}{incolor}{51}{\boxspacing}
\begin{Verbatim}[commandchars=\\\{\}]
\PY{n}{x} \PY{o}{=} \PY{l+m+mi}{1}
\PY{k}{while} \PY{n}{x} \PY{o}{\PYZlt{}} \PY{l+m+mi}{10}\PY{p}{:}
    \PY{n+nb}{print}\PY{p}{(}\PY{n}{x}\PY{p}{)}
    \PY{n}{x} \PY{o}{+}\PY{o}{=} \PY{l+m+mi}{1}
\end{Verbatim}
\end{tcolorbox}

    \begin{Verbatim}[commandchars=\\\{\}]
1
2
3
4
5
6
7
8
9
    \end{Verbatim}

    The general syntax of a \texttt{while}-loop is as follows:

\begin{verbatim}
    while e:
        S1
        ...
        Sn
\end{verbatim}

Here, \texttt{while} is the keyword starting the \texttt{while}-loop,
while \texttt{e} is an expression that contains some variable \texttt{x}
which is presumably updated in the statements \texttt{S1}, \(\cdots\),
\texttt{Sn}. These statements are executed as long as the expression
\texttt{e} is \texttt{True}.

    The following program computes the prime numbers according to an
\href{https://en.wikipedia.org/wiki/Sieve_of_Eratosthenes}{algorithm
given by Eratosthenes}. 1. We set \(n\) equal to 100 as we want to
compute the set all prime numbers less or equal that 100. 2.
\texttt{primes} is the list of numbers from 0 upto \(n\), i.e.~we have
initially \[ \texttt{primes} = [0,1,2,\cdots,n] \] Therefore, we have
initially
\[ \texttt{primes}[i] = i \quad \mbox{for all $i \in \{0,1,\cdots,n\}$.} \]
The idea is to set \(\texttt{primes}[i]\) to zero iff \(i\) is a proper
product of two numbers. 3. To this end we iterate over all \(i\) and
\(j\) from the list \([2,\cdots,n]\) and set the product
\(\texttt{primes}[i\cdot j]\) to zero.\\
This is achieved by the two \texttt{for} loops in line 4 and line 5
below.

\begin{verbatim}
Note that we have to check that the product $i * j$ is not bigger than $n$ 
for otherwise we would get an *out of range error* when trying to assign $\texttt{primes}[i*j]$.     
\end{verbatim}

\begin{enumerate}
\def\labelenumi{\arabic{enumi}.}
\setcounter{enumi}{3}
\tightlist
\item
  After the iteration, all non-prime elements greater than one of the
  list primes have been set to zero.
\item
  Finally, we compute the list of all prime numbers by collecting those
  elements from the list \texttt{primes} that have not been set to
  \(0\).
\end{enumerate}

    \begin{tcolorbox}[breakable, size=fbox, boxrule=1pt, pad at break*=1mm,colback=cellbackground, colframe=cellborder]
\prompt{In}{incolor}{52}{\boxspacing}
\begin{Verbatim}[commandchars=\\\{\}]
\PY{n}{n}         \PY{o}{=} \PY{l+m+mi}{100}
\PY{n}{primes}    \PY{o}{=} \PY{p}{[}\PY{n}{i} \PY{k}{for} \PY{n}{i} \PY{o+ow}{in} \PY{n+nb}{range}\PY{p}{(}\PY{l+m+mi}{0}\PY{p}{,} \PY{n}{n}\PY{o}{+}\PY{l+m+mi}{1}\PY{p}{)}\PY{p}{]}
\PY{n}{primes}\PY{p}{[}\PY{l+m+mi}{1}\PY{p}{]} \PY{o}{=} \PY{l+m+mi}{0}
\PY{k}{for} \PY{n}{i} \PY{o+ow}{in} \PY{n+nb}{range}\PY{p}{(}\PY{l+m+mi}{2}\PY{p}{,} \PY{n}{n}\PY{o}{+}\PY{l+m+mi}{1}\PY{p}{)}\PY{p}{:}
    \PY{k}{for} \PY{n}{j} \PY{o+ow}{in} \PY{n+nb}{range}\PY{p}{(}\PY{l+m+mi}{2}\PY{p}{,} \PY{n}{n}\PY{o}{+}\PY{l+m+mi}{1}\PY{p}{)}\PY{p}{:}
        \PY{k}{if} \PY{n}{i} \PY{o}{*} \PY{n}{j} \PY{o}{\PYZlt{}}\PY{o}{=} \PY{n}{n}\PY{p}{:}
            \PY{n}{primes}\PY{p}{[}\PY{n}{i} \PY{o}{*} \PY{n}{j}\PY{p}{]} \PY{o}{=} \PY{l+m+mi}{0}
\PY{n+nb}{print}\PY{p}{(}\PY{n}{primes}\PY{p}{)}
\PY{n+nb}{print}\PY{p}{(}\PY{p}{[}\PY{n}{i} \PY{k}{for} \PY{n}{i} \PY{o+ow}{in} \PY{n+nb}{range}\PY{p}{(}\PY{l+m+mi}{2}\PY{p}{,} \PY{n}{n}\PY{o}{+}\PY{l+m+mi}{1}\PY{p}{)} \PY{k}{if} \PY{n}{primes}\PY{p}{[}\PY{n}{i}\PY{p}{]} \PY{o}{!=} \PY{l+m+mi}{0}\PY{p}{]}\PY{p}{)}
\end{Verbatim}
\end{tcolorbox}

    \begin{Verbatim}[commandchars=\\\{\}]
[0, 0, 2, 3, 0, 5, 0, 7, 0, 0, 0, 11, 0, 13, 0, 0, 0, 17, 0, 19, 0, 0, 0, 23, 0,
0, 0, 0, 0, 29, 0, 31, 0, 0, 0, 0, 0, 37, 0, 0, 0, 41, 0, 43, 0, 0, 0, 47, 0, 0,
0, 0, 0, 53, 0, 0, 0, 0, 0, 59, 0, 61, 0, 0, 0, 0, 0, 67, 0, 0, 0, 71, 0, 73, 0,
0, 0, 0, 0, 79, 0, 0, 0, 83, 0, 0, 0, 0, 0, 89, 0, 0, 0, 0, 0, 0, 0, 97, 0, 0,
0]
[2, 3, 5, 7, 11, 13, 17, 19, 23, 29, 31, 37, 41, 43, 47, 53, 59, 61, 67, 71, 73,
79, 83, 89, 97]
    \end{Verbatim}

    The algorithm given above can be improved by using the following
observations: 1. If a number \(x\) can be written as a product
\(a \cdot b\), then at least one of the numbers \(a\) or \(b\) has to be
less than \(\sqrt{x}\). Therefore, the \texttt{for} loop in line 5 below
iterates as long as \(i \leq \sqrt{x}\). The function \texttt{ceil} is
needed to cast the square root of \(x\) to a natural number. In order to
use the functions \texttt{sqrt} and \texttt{ceil} we have to import them
from the module \texttt{math}. This is done in line 1 of the program
shown below. 2. When we iterate over \(j\) in the inner loop, it is
sufficient if we start with \(j = i\) since all products of the form
\(i \cdot j\) where \(j < i\) have already been eliminated at the time
when the multiples of \(i\) had been eliminated. 3. If
\(\texttt{primes}[i] = 0\), then \(i\) is not a prime and hence it has
to be a product of two numbers \(a\) and \(b\) both of which are smaller
than \(i\). However, since all the multiples of \(a\) and \(b\) have
already been eliminated, there is no point in eliminating the multiples
of \(i\) since these are also multiples of both \(a\) and \(b\) and
hence they have already been eliminated.\\
Therefore, if \(\texttt{primes}[i] = 0\) we can immediately jump to the
next value of \(i\).\\
This is achieved by the \emph{continue} statement in line 8 below.

The program shown below is easily capable of computing all prime numbers
less than a million. On my desktop computer, which is a 2017 iMac with a
Quad-Core Intel i5 processor, this takes less than a second.

    \begin{tcolorbox}[breakable, size=fbox, boxrule=1pt, pad at break*=1mm,colback=cellbackground, colframe=cellborder]
\prompt{In}{incolor}{53}{\boxspacing}
\begin{Verbatim}[commandchars=\\\{\}]
\PY{o}{\PYZpc{}\PYZpc{}time}
\PY{k+kn}{import} \PY{n+nn}{math}

\PY{n}{n} \PY{o}{=} \PY{l+m+mi}{1000000}
\PY{n}{primes} \PY{o}{=} \PY{n+nb}{list}\PY{p}{(}\PY{n+nb}{range}\PY{p}{(}\PY{n}{n}\PY{o}{+}\PY{l+m+mi}{1}\PY{p}{)}\PY{p}{)}
\PY{k}{for} \PY{n}{i} \PY{o+ow}{in} \PY{n+nb}{range}\PY{p}{(}\PY{l+m+mi}{2}\PY{p}{,} \PY{n}{math}\PY{o}{.}\PY{n}{ceil}\PY{p}{(}\PY{n}{math}\PY{o}{.}\PY{n}{sqrt}\PY{p}{(}\PY{n}{n}\PY{p}{)}\PY{p}{)}\PY{p}{)}\PY{p}{:}
    \PY{k}{if} \PY{n}{primes}\PY{p}{[}\PY{n}{i}\PY{p}{]} \PY{o}{==} \PY{l+m+mi}{0}\PY{p}{:}
        \PY{k}{continue}
    \PY{n}{j} \PY{o}{=} \PY{n}{i}
    \PY{k}{while} \PY{n}{i} \PY{o}{*} \PY{n}{j} \PY{o}{\PYZlt{}}\PY{o}{=} \PY{n}{n}\PY{p}{:}
        \PY{n}{primes}\PY{p}{[}\PY{n}{i} \PY{o}{*} \PY{n}{j}\PY{p}{]} \PY{o}{=} \PY{l+m+mi}{0}
        \PY{n}{j} \PY{o}{+}\PY{o}{=} \PY{l+m+mi}{1}
\PY{n}{P} \PY{o}{=} \PY{p}{[}\PY{n}{i} \PY{k}{for} \PY{n}{i} \PY{o+ow}{in} \PY{n+nb}{range}\PY{p}{(}\PY{l+m+mi}{2}\PY{p}{,} \PY{n}{n}\PY{o}{+}\PY{l+m+mi}{1}\PY{p}{)} \PY{k}{if} \PY{n}{primes}\PY{p}{[}\PY{n}{i}\PY{p}{]} \PY{o}{!=} \PY{l+m+mi}{0}\PY{p}{]}
\PY{n+nb}{print}\PY{p}{(}\PY{n}{P}\PY{p}{)}
\end{Verbatim}
\end{tcolorbox}

    \begin{Verbatim}[commandchars=\\\{\}]
[2, 3, 5, 7, 11, 13, 17, 19, 23, 29, 31, 37, 41, 43, 47, 53, 59, 61, 67, 71, 73,
79, 83, 89, 97, 101, 103, 107, 109, 113, 127, 131, 137, 139, 149, 151, 157, 163,
...
999749, 999763, 999769, 999773, 999809, 999853, 999863, 999883, 999907, 999917,
999931, 999953, 999959, 999961, 999979, 999983]
CPU times: user 728 ms, sys: 13 ms, total: 741 ms
Wall time: 745 ms
    \end{Verbatim}

    \hypertarget{numerical-functions}{%
\subsection{Numerical Functions}\label{numerical-functions}}

    \emph{Python} provides all of the mathematical functions that you have
learned about at school. A detailed listing of these functions can be
found at \url{https://docs.python.org/3.6/library/math.html}. We just
show the most important functions and constants. In order to make the
module \texttt{math} available, we use the following \texttt{import}
statement:

    \begin{tcolorbox}[breakable, size=fbox, boxrule=1pt, pad at break*=1mm,colback=cellbackground, colframe=cellborder]
\prompt{In}{incolor}{54}{\boxspacing}
\begin{Verbatim}[commandchars=\\\{\}]
\PY{k+kn}{import} \PY{n+nn}{math}
\end{Verbatim}
\end{tcolorbox}

    The mathematical constant \href{https://en.wikipedia.org/wiki/Pi}{Pi},
which in mathematics is written as \(\pi\), is available as
\texttt{math.pi}.

    \begin{tcolorbox}[breakable, size=fbox, boxrule=1pt, pad at break*=1mm,colback=cellbackground, colframe=cellborder]
\prompt{In}{incolor}{55}{\boxspacing}
\begin{Verbatim}[commandchars=\\\{\}]
\PY{n}{math}\PY{o}{.}\PY{n}{pi}
\end{Verbatim}
\end{tcolorbox}

            \begin{tcolorbox}[breakable, size=fbox, boxrule=.5pt, pad at break*=1mm, opacityfill=0]
\prompt{Out}{outcolor}{55}{\boxspacing}
\begin{Verbatim}[commandchars=\\\{\}]
3.141592653589793
\end{Verbatim}
\end{tcolorbox}
        
    The sine function is called as follows:

    \begin{tcolorbox}[breakable, size=fbox, boxrule=1pt, pad at break*=1mm,colback=cellbackground, colframe=cellborder]
\prompt{In}{incolor}{56}{\boxspacing}
\begin{Verbatim}[commandchars=\\\{\}]
\PY{n}{math}\PY{o}{.}\PY{n}{sin}\PY{p}{(}\PY{n}{math}\PY{o}{.}\PY{n}{pi}\PY{o}{/}\PY{l+m+mi}{6}\PY{p}{)}
\end{Verbatim}
\end{tcolorbox}

            \begin{tcolorbox}[breakable, size=fbox, boxrule=.5pt, pad at break*=1mm, opacityfill=0]
\prompt{Out}{outcolor}{56}{\boxspacing}
\begin{Verbatim}[commandchars=\\\{\}]
0.49999999999999994
\end{Verbatim}
\end{tcolorbox}
        
    The cosine function is called as follows:

    \begin{tcolorbox}[breakable, size=fbox, boxrule=1pt, pad at break*=1mm,colback=cellbackground, colframe=cellborder]
\prompt{In}{incolor}{57}{\boxspacing}
\begin{Verbatim}[commandchars=\\\{\}]
\PY{n}{math}\PY{o}{.}\PY{n}{cos}\PY{p}{(}\PY{l+m+mf}{0.0}\PY{p}{)}
\end{Verbatim}
\end{tcolorbox}

            \begin{tcolorbox}[breakable, size=fbox, boxrule=.5pt, pad at break*=1mm, opacityfill=0]
\prompt{Out}{outcolor}{57}{\boxspacing}
\begin{Verbatim}[commandchars=\\\{\}]
1.0
\end{Verbatim}
\end{tcolorbox}
        
    The tangent function is called as follows:

    \begin{tcolorbox}[breakable, size=fbox, boxrule=1pt, pad at break*=1mm,colback=cellbackground, colframe=cellborder]
\prompt{In}{incolor}{58}{\boxspacing}
\begin{Verbatim}[commandchars=\\\{\}]
\PY{n}{math}\PY{o}{.}\PY{n}{tan}\PY{p}{(}\PY{n}{math}\PY{o}{.}\PY{n}{pi}\PY{o}{/}\PY{l+m+mi}{4}\PY{p}{)}
\end{Verbatim}
\end{tcolorbox}

            \begin{tcolorbox}[breakable, size=fbox, boxrule=.5pt, pad at break*=1mm, opacityfill=0]
\prompt{Out}{outcolor}{58}{\boxspacing}
\begin{Verbatim}[commandchars=\\\{\}]
0.9999999999999999
\end{Verbatim}
\end{tcolorbox}
        
    The \emph{arc sine}, \emph{arc cosine}, and \emph{arc tangent} are
called by prefixing the character
\texttt{\textquotesingle{}a\textquotesingle{}} to the name of the
function as seen below:

    \begin{tcolorbox}[breakable, size=fbox, boxrule=1pt, pad at break*=1mm,colback=cellbackground, colframe=cellborder]
\prompt{In}{incolor}{59}{\boxspacing}
\begin{Verbatim}[commandchars=\\\{\}]
\PY{n}{math}\PY{o}{.}\PY{n}{asin}\PY{p}{(}\PY{l+m+mf}{1.0}\PY{p}{)}
\end{Verbatim}
\end{tcolorbox}

            \begin{tcolorbox}[breakable, size=fbox, boxrule=.5pt, pad at break*=1mm, opacityfill=0]
\prompt{Out}{outcolor}{59}{\boxspacing}
\begin{Verbatim}[commandchars=\\\{\}]
1.5707963267948966
\end{Verbatim}
\end{tcolorbox}
        
    \begin{tcolorbox}[breakable, size=fbox, boxrule=1pt, pad at break*=1mm,colback=cellbackground, colframe=cellborder]
\prompt{In}{incolor}{60}{\boxspacing}
\begin{Verbatim}[commandchars=\\\{\}]
\PY{n}{math}\PY{o}{.}\PY{n}{acos}\PY{p}{(}\PY{l+m+mf}{1.0}\PY{p}{)}
\end{Verbatim}
\end{tcolorbox}

            \begin{tcolorbox}[breakable, size=fbox, boxrule=.5pt, pad at break*=1mm, opacityfill=0]
\prompt{Out}{outcolor}{60}{\boxspacing}
\begin{Verbatim}[commandchars=\\\{\}]
0.0
\end{Verbatim}
\end{tcolorbox}
        
    \begin{tcolorbox}[breakable, size=fbox, boxrule=1pt, pad at break*=1mm,colback=cellbackground, colframe=cellborder]
\prompt{In}{incolor}{61}{\boxspacing}
\begin{Verbatim}[commandchars=\\\{\}]
\PY{n}{math}\PY{o}{.}\PY{n}{atan}\PY{p}{(}\PY{l+m+mf}{1.0}\PY{p}{)}
\end{Verbatim}
\end{tcolorbox}

            \begin{tcolorbox}[breakable, size=fbox, boxrule=.5pt, pad at break*=1mm, opacityfill=0]
\prompt{Out}{outcolor}{61}{\boxspacing}
\begin{Verbatim}[commandchars=\\\{\}]
0.7853981633974483
\end{Verbatim}
\end{tcolorbox}
        
    Euler's number \(e\) is stored as the constant \texttt{math.e}:

    \begin{tcolorbox}[breakable, size=fbox, boxrule=1pt, pad at break*=1mm,colback=cellbackground, colframe=cellborder]
\prompt{In}{incolor}{62}{\boxspacing}
\begin{Verbatim}[commandchars=\\\{\}]
\PY{n}{math}\PY{o}{.}\PY{n}{e}
\end{Verbatim}
\end{tcolorbox}

            \begin{tcolorbox}[breakable, size=fbox, boxrule=.5pt, pad at break*=1mm, opacityfill=0]
\prompt{Out}{outcolor}{62}{\boxspacing}
\begin{Verbatim}[commandchars=\\\{\}]
2.718281828459045
\end{Verbatim}
\end{tcolorbox}
        
    The exponential function \(\mathrm{exp}(x) := e^x\) is computed as
follows:

    \begin{tcolorbox}[breakable, size=fbox, boxrule=1pt, pad at break*=1mm,colback=cellbackground, colframe=cellborder]
\prompt{In}{incolor}{63}{\boxspacing}
\begin{Verbatim}[commandchars=\\\{\}]
\PY{n}{math}\PY{o}{.}\PY{n}{exp}\PY{p}{(}\PY{l+m+mi}{1}\PY{p}{)}
\end{Verbatim}
\end{tcolorbox}

            \begin{tcolorbox}[breakable, size=fbox, boxrule=.5pt, pad at break*=1mm, opacityfill=0]
\prompt{Out}{outcolor}{63}{\boxspacing}
\begin{Verbatim}[commandchars=\\\{\}]
2.718281828459045
\end{Verbatim}
\end{tcolorbox}
        
    The natural logarithm \(\ln(x)\), which is defined as the inverse of the
function \(\exp(x)\), is called \texttt{log} (instead of \texttt{ln}):

    \begin{tcolorbox}[breakable, size=fbox, boxrule=1pt, pad at break*=1mm,colback=cellbackground, colframe=cellborder]
\prompt{In}{incolor}{64}{\boxspacing}
\begin{Verbatim}[commandchars=\\\{\}]
\PY{n}{math}\PY{o}{.}\PY{n}{log}\PY{p}{(}\PY{n}{math}\PY{o}{.}\PY{n}{e} \PY{o}{*} \PY{n}{math}\PY{o}{.}\PY{n}{e}\PY{p}{)}
\end{Verbatim}
\end{tcolorbox}

            \begin{tcolorbox}[breakable, size=fbox, boxrule=.5pt, pad at break*=1mm, opacityfill=0]
\prompt{Out}{outcolor}{64}{\boxspacing}
\begin{Verbatim}[commandchars=\\\{\}]
2.0
\end{Verbatim}
\end{tcolorbox}
        
    The square root \(\sqrt{x}\) of a number \(x\) is computed using the
function \texttt{sqrt}:

    \begin{tcolorbox}[breakable, size=fbox, boxrule=1pt, pad at break*=1mm,colback=cellbackground, colframe=cellborder]
\prompt{In}{incolor}{65}{\boxspacing}
\begin{Verbatim}[commandchars=\\\{\}]
\PY{n}{math}\PY{o}{.}\PY{n}{sqrt}\PY{p}{(}\PY{l+m+mi}{2}\PY{p}{)}
\end{Verbatim}
\end{tcolorbox}

            \begin{tcolorbox}[breakable, size=fbox, boxrule=.5pt, pad at break*=1mm, opacityfill=0]
\prompt{Out}{outcolor}{65}{\boxspacing}
\begin{Verbatim}[commandchars=\\\{\}]
1.4142135623730951
\end{Verbatim}
\end{tcolorbox}
        
    The flooring function \(\texttt{floor}(x)\) truncates a floating point
number \(x\) down to the biggest integer number less or equal to \(x\):
\[ \texttt{floor}(x) = \max(\{ n \in \mathbb{Z} \mid n \leq x \} \]

    \begin{tcolorbox}[breakable, size=fbox, boxrule=1pt, pad at break*=1mm,colback=cellbackground, colframe=cellborder]
\prompt{In}{incolor}{66}{\boxspacing}
\begin{Verbatim}[commandchars=\\\{\}]
\PY{n}{math}\PY{o}{.}\PY{n}{floor}\PY{p}{(}\PY{l+m+mf}{1.999}\PY{p}{)}
\end{Verbatim}
\end{tcolorbox}

            \begin{tcolorbox}[breakable, size=fbox, boxrule=.5pt, pad at break*=1mm, opacityfill=0]
\prompt{Out}{outcolor}{66}{\boxspacing}
\begin{Verbatim}[commandchars=\\\{\}]
1
\end{Verbatim}
\end{tcolorbox}
        
    The ceiling function \(\texttt{ceil}(x)\) rounds a floating point number
\(x\) up to the next integer number bigger or equal to \(x\):
\[ \texttt{ceil}(x) = \min(\{ n \in \mathbb{Z} \mid x \leq n \} \]

    \begin{tcolorbox}[breakable, size=fbox, boxrule=1pt, pad at break*=1mm,colback=cellbackground, colframe=cellborder]
\prompt{In}{incolor}{67}{\boxspacing}
\begin{Verbatim}[commandchars=\\\{\}]
\PY{n}{math}\PY{o}{.}\PY{n}{ceil}\PY{p}{(}\PY{l+m+mf}{1.001}\PY{p}{)}
\end{Verbatim}
\end{tcolorbox}

            \begin{tcolorbox}[breakable, size=fbox, boxrule=.5pt, pad at break*=1mm, opacityfill=0]
\prompt{Out}{outcolor}{67}{\boxspacing}
\begin{Verbatim}[commandchars=\\\{\}]
2
\end{Verbatim}
\end{tcolorbox}
        
    The function \texttt{round} rounds its input to the nearest integer. For
an integer number \(n\), the floating point number \(n.5\) is rounded to
\(n+1\).

    \begin{tcolorbox}[breakable, size=fbox, boxrule=1pt, pad at break*=1mm,colback=cellbackground, colframe=cellborder]
\prompt{In}{incolor}{68}{\boxspacing}
\begin{Verbatim}[commandchars=\\\{\}]
\PY{n+nb}{round}\PY{p}{(}\PY{l+m+mf}{1.5}\PY{p}{)}\PY{p}{,} \PY{n+nb}{round}\PY{p}{(}\PY{l+m+mf}{1.39}\PY{p}{)}\PY{p}{,} \PY{n+nb}{round}\PY{p}{(}\PY{l+m+mf}{1.51}\PY{p}{)}
\end{Verbatim}
\end{tcolorbox}

            \begin{tcolorbox}[breakable, size=fbox, boxrule=.5pt, pad at break*=1mm, opacityfill=0]
\prompt{Out}{outcolor}{68}{\boxspacing}
\begin{Verbatim}[commandchars=\\\{\}]
(2, 1, 2)
\end{Verbatim}
\end{tcolorbox}
        
    The function \texttt{abs} computes the absolute value. Note that this
function is not part of the math library but rather is a predefined
function. Therefore the name \texttt{abs} is not prefixed with the
package name \texttt{math}.

    \begin{tcolorbox}[breakable, size=fbox, boxrule=1pt, pad at break*=1mm,colback=cellbackground, colframe=cellborder]
\prompt{In}{incolor}{69}{\boxspacing}
\begin{Verbatim}[commandchars=\\\{\}]
\PY{n+nb}{abs}\PY{p}{(}\PY{o}{\PYZhy{}}\PY{l+m+mi}{1}\PY{p}{)}
\end{Verbatim}
\end{tcolorbox}

            \begin{tcolorbox}[breakable, size=fbox, boxrule=.5pt, pad at break*=1mm, opacityfill=0]
\prompt{Out}{outcolor}{69}{\boxspacing}
\begin{Verbatim}[commandchars=\\\{\}]
1
\end{Verbatim}
\end{tcolorbox}
        
    \hypertarget{the-help-system}{%
\subsection{The Help System}\label{the-help-system}}

    Typing a single question mark `?' starts the help system of the
\emph{Jupyter} notebook.

    \begin{tcolorbox}[breakable, size=fbox, boxrule=1pt, pad at break*=1mm,colback=cellbackground, colframe=cellborder]
\prompt{In}{incolor}{70}{\boxspacing}
\begin{Verbatim}[commandchars=\\\{\}]
\PY{o}{?}
\end{Verbatim}
\end{tcolorbox}

    If the name of a module is followed by a question mark, a description of
the module is printed.

    \begin{tcolorbox}[breakable, size=fbox, boxrule=1pt, pad at break*=1mm,colback=cellbackground, colframe=cellborder]
\prompt{In}{incolor}{71}{\boxspacing}
\begin{Verbatim}[commandchars=\\\{\}]
math\PY{o}{?}
\end{Verbatim}
\end{tcolorbox}

    This also works for functions defined in a module.

    \begin{tcolorbox}[breakable, size=fbox, boxrule=1pt, pad at break*=1mm,colback=cellbackground, colframe=cellborder]
\prompt{In}{incolor}{72}{\boxspacing}
\begin{Verbatim}[commandchars=\\\{\}]
math.sin\PY{o}{?}
\end{Verbatim}
\end{tcolorbox}

    The question mark operator only works inside a Jupyter notebook.

Furthermore, there is the function \texttt{help}, which is part of
\emph{Python}.

    \begin{tcolorbox}[breakable, size=fbox, boxrule=1pt, pad at break*=1mm,colback=cellbackground, colframe=cellborder]
\prompt{In}{incolor}{73}{\boxspacing}
\begin{Verbatim}[commandchars=\\\{\}]
\PY{n}{help}\PY{p}{(}\PY{n}{math}\PY{p}{)}
\end{Verbatim}
\end{tcolorbox}

    \begin{Verbatim}[commandchars=\\\{\}]
Help on module math:

NAME
    math

MODULE REFERENCE
    https://docs.python.org/3.10/library/math.html

    The following documentation is automatically generated from the Python
    source files.  It may be incomplete, incorrect or include features that
    are considered implementation detail and may vary between Python
    implementations.  When in doubt, consult the module reference at the
    location listed above.

DESCRIPTION
    This module provides access to the mathematical functions
    defined by the C standard.

FUNCTIONS
    acos(x, /)
        Return the arc cosine (measured in radians) of x.

        The result is between 0 and pi.

    acosh(x, /)
        Return the inverse hyperbolic cosine of x.

    asin(x, /)
        Return the arc sine (measured in radians) of x.

        The result is between -pi/2 and pi/2.

    asinh(x, /)
        Return the inverse hyperbolic sine of x.

    atan(x, /)
        Return the arc tangent (measured in radians) of x.

        The result is between -pi/2 and pi/2.

        ...
        
DATA
    e = 2.718281828459045
    inf = inf
    nan = nan
    pi = 3.141592653589793
    tau = 6.283185307179586

FILE
    /opt/anaconda3/envs/algo/lib/python3.10/lib-
dynload/math.cpython-310-darwin.so


    \end{Verbatim}

    \begin{tcolorbox}[breakable, size=fbox, boxrule=1pt, pad at break*=1mm,colback=cellbackground, colframe=cellborder]
\prompt{In}{incolor}{74}{\boxspacing}
\begin{Verbatim}[commandchars=\\\{\}]
\PY{n}{help}\PY{p}{(}\PY{n}{math}\PY{o}{.}\PY{n}{sin}\PY{p}{)}
\end{Verbatim}
\end{tcolorbox}

    \begin{Verbatim}[commandchars=\\\{\}]
Help on built-in function sin in module math:

sin(x, /)
    Return the sine of x (measured in radians).

    \end{Verbatim}

    \begin{tcolorbox}[breakable, size=fbox, boxrule=1pt, pad at break*=1mm,colback=cellbackground, colframe=cellborder]
\prompt{In}{incolor}{75}{\boxspacing}
\begin{Verbatim}[commandchars=\\\{\}]
\PY{n}{help}\PY{p}{(}\PY{n+nb}{dir}\PY{p}{)}
\end{Verbatim}
\end{tcolorbox}

    \begin{Verbatim}[commandchars=\\\{\}]
Help on built-in function dir in module builtins:

dir({\ldots})
    dir([object]) -> list of strings

    If called without an argument, return the names in the current scope.
    Else, return an alphabetized list of names comprising (some of) the
attributes
    of the given object, and of attributes reachable from it.
    If the object supplies a method named \_\_dir\_\_, it will be used; otherwise
    the default dir() logic is used and returns:
      for a module object: the module's attributes.
      for a class object:  its attributes, and recursively the attributes
        of its bases.
      for any other object: its attributes, its class's attributes, and
        recursively the attributes of its class's base classes.

    \end{Verbatim}

    We can use the function \texttt{dir()} to print the names of the variables that
have been defined.

    \begin{tcolorbox}[breakable, size=fbox, boxrule=1pt, pad at break*=1mm,colback=cellbackground, colframe=cellborder]
\prompt{In}{incolor}{76}{\boxspacing}
\begin{Verbatim}[commandchars=\\\{\}]
\PY{n+nb}{dir}\PY{p}{(}\PY{p}{)}
\end{Verbatim}
\end{tcolorbox}

            \begin{tcolorbox}[breakable, size=fbox, boxrule=.5pt, pad at break*=1mm, opacityfill=0]
\prompt{Out}{outcolor}{76}{\boxspacing}
\begin{Verbatim}[commandchars=\\\{\}]
['B',
 'E',
 'HTML',
 'In',
 'L',
 'Out',
 'P',
 '\_',
 '\_1',
 '\_2',
 ...
 '\_i',
 '\_i1',
 '\_i2',
 ...
 'x',
 'y']
\end{Verbatim}
\end{tcolorbox}
        
    The names of the form \(\textrm{\_i}n\) where \(n\) is a number refer to
the \emph{input} of the \(n^{\textrm{th}}\) cell of this notebook. Let
us check the input of the \(55^{\textrm{th}}\) cell of this notebook.

    \begin{tcolorbox}[breakable, size=fbox, boxrule=1pt, pad at break*=1mm,colback=cellbackground, colframe=cellborder]
\prompt{In}{incolor}{77}{\boxspacing}
\begin{Verbatim}[commandchars=\\\{\}]
\PY{n}{\PYZus{}i55}
\end{Verbatim}
\end{tcolorbox}

            \begin{tcolorbox}[breakable, size=fbox, boxrule=.5pt, pad at break*=1mm, opacityfill=0]
\prompt{Out}{outcolor}{77}{\boxspacing}
\begin{Verbatim}[commandchars=\\\{\}]
'math.pi'
\end{Verbatim}
\end{tcolorbox}
        
    The names of the form \(\textrm{\_}n\) where \(n\) is a number refer to
the \emph{output} produced by the \(n^{\textrm{th}}\) cell of this
notebook.

    \begin{tcolorbox}[breakable, size=fbox, boxrule=1pt, pad at break*=1mm,colback=cellbackground, colframe=cellborder]
\prompt{In}{incolor}{78}{\boxspacing}
\begin{Verbatim}[commandchars=\\\{\}]
\PY{n}{\PYZus{}55}
\end{Verbatim}
\end{tcolorbox}

            \begin{tcolorbox}[breakable, size=fbox, boxrule=.5pt, pad at break*=1mm, opacityfill=0]
\prompt{Out}{outcolor}{78}{\boxspacing}
\begin{Verbatim}[commandchars=\\\{\}]
3.141592653589793
\end{Verbatim}
\end{tcolorbox}
        
    \begin{tcolorbox}[breakable, size=fbox, boxrule=1pt, pad at break*=1mm,colback=cellbackground, colframe=cellborder]
\prompt{In}{incolor}{79}{\boxspacing}
\begin{Verbatim}[commandchars=\\\{\}]
\PY{l+m+mi}{2} \PY{o}{*} \PY{l+m+mi}{3}
\end{Verbatim}
\end{tcolorbox}

            \begin{tcolorbox}[breakable, size=fbox, boxrule=.5pt, pad at break*=1mm, opacityfill=0]
\prompt{Out}{outcolor}{79}{\boxspacing}
\begin{Verbatim}[commandchars=\\\{\}]
6
\end{Verbatim}
\end{tcolorbox}
        
    Inside a \emph{Jupyter} notebook, the variable \texttt{\_} refers to the
value of the last expression that has been executed.

    \begin{tcolorbox}[breakable, size=fbox, boxrule=1pt, pad at break*=1mm,colback=cellbackground, colframe=cellborder]
\prompt{In}{incolor}{80}{\boxspacing}
\begin{Verbatim}[commandchars=\\\{\}]
\PY{n}{\PYZus{}}
\end{Verbatim}
\end{tcolorbox}

            \begin{tcolorbox}[breakable, size=fbox, boxrule=.5pt, pad at break*=1mm, opacityfill=0]
\prompt{Out}{outcolor}{80}{\boxspacing}
\begin{Verbatim}[commandchars=\\\{\}]
6
\end{Verbatim}
\end{tcolorbox}
        
    The magic command \%quickref prints an overview of the so called magic
commands that are available in Jupyter notebooks.

    \begin{tcolorbox}[breakable, size=fbox, boxrule=1pt, pad at break*=1mm,colback=cellbackground, colframe=cellborder]
\prompt{In}{incolor}{81}{\boxspacing}
\begin{Verbatim}[commandchars=\\\{\}]
\PY{o}{\PYZpc{}}\PY{k}{quickref}
\end{Verbatim}
\end{tcolorbox}

    We can use the command \texttt{ls} to list the files in the current
directory.

    \begin{tcolorbox}[breakable, size=fbox, boxrule=1pt, pad at break*=1mm,colback=cellbackground, colframe=cellborder]
\prompt{In}{incolor}{82}{\boxspacing}
\begin{Verbatim}[commandchars=\\\{\}]
\PY{n}{ls} \PY{o}{\PYZhy{}}\PY{n}{al}
\end{Verbatim}
\end{tcolorbox}

    \begin{Verbatim}[commandchars=\\\{\}]
total 1352
drwxr-xr-x@  4 stroetmann  staff     128 Sep 30 16:10 \textcolor{ansi-blue}{.}/
drwxr-xr-x@ 24 stroetmann  staff     768 Sep 20 13:59 \textcolor{ansi-blue}{..}/
drwxr-xr-x@  3 stroetmann  staff      96 Sep 13 16:58
\textcolor{ansi-blue}{.ipynb\_checkpoints}/
-rw-r--r--@  1 stroetmann  staff  691296 Sep 30 16:10 Introduction.ipynb
    \end{Verbatim}

    Prefixing a shell command with a \texttt{!} executes this command in a
shell. On the Windows operating system, the system command to list the
files in the current directory is called \texttt{dir}, while on Linux
and MacOS, the corresponding command is known as \texttt{ls}.

    \begin{tcolorbox}[breakable, size=fbox, boxrule=1pt, pad at break*=1mm,colback=cellbackground, colframe=cellborder]
\prompt{In}{incolor}{83}{\boxspacing}
\begin{Verbatim}[commandchars=\\\{\}]
\PY{o}{!}ls
\end{Verbatim}
\end{tcolorbox}

    \begin{Verbatim}[commandchars=\\\{\}]
Introduction.ipynb
    \end{Verbatim}
    
\end{document}
